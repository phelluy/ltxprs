%% LyX 1.1 created this file.  For more info, see http://www.lyx.org/.
%% Do not edit unless you really know what you are doing.
\documentclass[a4paper,english]{amsart}
\usepackage[T1]{fontenc}
\usepackage[utf8]{inputenc}
\usepackage{babel}
\usepackage{ae}
\usepackage{graphics}
\usepackage{subfigure}
\usepackage{hyperref}

\makeatletter

%%%%%%%%%%%%%%%%%%%%%%%%%%%%%% LyX specific LaTeX commands.
\providecommand{\LyX}{L\kern-.1667em\lower.25em\hbox{Y}\kern-.125emX\@}
%% Special footnote code from the package 'stblftnt.sty'
%% Author: Robin Fairbairns -- Last revised Dec 13 1996
\let\SF@@footnote\footnote
\def\footnote{\ifx\protect\@typeset@protect
    \expandafter\SF@@footnote
  \else
    \expandafter\SF@gobble@opt
  \fi
}
\expandafter\def\csname SF@gobble@opt \endcsname{\@ifnextchar[%]
  \SF@gobble@twobracket
  \@gobble
}
\edef\SF@gobble@opt{\noexpand\protect
  \expandafter\noexpand\csname SF@gobble@opt \endcsname}
\def\SF@gobble@twobracket[#1]#2{}

%%%%%%%%%%%%%%%%%%%%%%%%%%%%%% Textclass specific LaTeX commands.
 \theoremstyle{plain}
 \newtheorem{thm}{Theorem}[section]
 \numberwithin{equation}{section} %% Comment out for sequentially-numbered
 \numberwithin{figure}{section} %% Comment out for sequentially-numbered
 \theoremstyle{remark}
 \newtheorem{rem}[thm]{Remark}
 \newcommand{\lyxaddress}[1]{
   \par {\raggedright #1
   \vspace{1.4em}
   \noindent\par}
 }

\makeatother


\begin{document}

\title{Practical computation of axisymmetrical multifluid flows}


\author{Thomas BARBERON, Philippe HELLUY, Sandra ROUY}


\lyxaddress{ISITV, Laboratoire ANAM/MNC, BP56, 83162 La Valette
CEDEX. France. \texttt{helluy@univ-tln.fr}}


\thanks{This work has been supported by a joint research grant DCN/ING Toulon
\& Région PACA.}
\begin{abstract}
We adapt the Saurel-Abgrall front capturing finite volumes method
for an industrial simulation of compressible multifluid flows. We
then apply the method to the case of air-water flow in the cooling
chamber of an axisymmetrical gas generator. We describe
successively how to deal with exact and global Riemann solvers,
pressure oscillations, unstructured meshes, axisymmetry, boundary
conditions and overly restrictive CFL conditions. The resulting
algorithm is efficient and robust.
\end{abstract}

\keywords{compressible multifluid, front capturing,
nonconservative scheme, axisymmetry.}
\maketitle \date \tableofcontents

\section{Introduction}

This work is devoted to the application of recent finite volumes
schemes, and particularly the one proposed by Saurel and Abgrall
in \cite{saurel99}, to the simulation of an axisymmetrical
multiphase flow in a complex geometry. Because of the complexity
of the application, we have to specify or adapt the original
Saurel-Abgrall idea to: \nolinebreak global Riemann solver,
unstructured
meshes, axisymmetry, boundary conditions, multi time steps... \\
We base our simulation on an effective mathematical model for
compressible multifluid flows and especially air-water flows. For
our application, the pressure law for the air is a classical
perfect gas law. Because we have in mind flows with high and low
pressures, we have also to take into account the compressibility
effects in water. In such applications, it is classical to observe
\textsl{cavitation} zones in the liquid phase. Cavitation is a
phenomenon that appears in a region of the flow where the pressure
drops below the saturation pressure of water. In a first and very
short stage, the liquid stays in a metastable state. It can also
happen in this stage that the pressure becomes negative (it is
then called a tension). In a second stage, a phase transition
(vaporization) occurs. Thus the original two-phase flow, made of
air and water, becomes a three-phase flow made of air, liquid and
vapor. We restrict ourself to the two-phase case, where the phase
transition has not started (or is not taken into account...).
Then, as for the air, a pressure law which expresses the pressure
of the water as a function of density and internal energy has to
be supplied. We use a very simple generalization of the perfect
gas law, the stiffened gas equation of state (see
(\ref{presslaw})), which allows
negative values for the pressure. \\
Compressible single fluids have been extensively studied and the
main difficulty here is the representation of the interface
between the fluids. There are two main approaches for treating
interfaces:
\begin{itemize}
  \item One can favor the Lagrangian interpretation of the fluid
  equations. The interface then receives a particular treatment in
  the numerical method. This approach leads to the family of
  the \textsl{front-tracking methods}. We will not
  consider this kind of solution here.
  \item If the Eulerian point of view is preferred, the resulting
  scheme belongs to the family of the \textsl{front-capturing
  methods}. No special treatment is applied to the numerical cells that are
  crossed by the interface. We prefer the front-capturing methods
  because they are more general and easier to implement than the
  front-tracking methods.
\end{itemize}
A now classical and simple approach, proposed for example in
\cite{osher92}, \cite{karni96}, is to locate the interface by
means of the level-set of a function which is convected with the
flow. In this approach only an advection equation has to be added
to the classical compressible Euler model. One switches from one
law (perfect gas) to the other (stiffened gas) according to the
value of the level-set function. Because the stiffened gas law is
a generalization of the perfect gas law, the level-set model is
also equivalent, in our case, to a model where the two
coefficients of the stiffened gas law are convected with the flow.
The interface is then located by the discontinuities of the
pressure law coefficients. Our model is finally made up of the
Euler system (conservation of mass, momentum and energy, equations
(\ref{euler})), two additional transport equations (equations
(\ref{convcoef})) and the stiffened gas law (equation
(\ref{presslaw})).
This multifluid model presents a supplementary
transport equation when compared to the level-set
model but it is generally easier to discretize. \\
%trsltx-begin-ignoble

Despite its simplicity and its mathematical perfection, this model
leads to numerous numerical difficulties as shown in many works
among which we can cite \cite{abgrall88}, \cite{karni94},
\cite{karni96}, \cite{saurel99}, \cite{karni01}, \cite{fedkiw99},
\cite{rouy00}... These difficulties have to be overcome before
envisaging practical applications. Of greatest concern are the
spurious pressure oscillations that appear near the material
interface when the model is approximated by any conservative
finite volumes scheme. The same kind of oscillations occurs in the
simulation of mixtures of perfect gases \cite{abgrall88} or across
numerically diffused shear interfaces \cite{nkonga02}. Actually,
it appears that classical conservative schemes (such as Godunov's,
Roe's, HLL, Rusanov's, etc.) suffer from a very slow convergence
and a very bad precision on standard meshes when applied to the
previous multifluid model. Mulder, Osher and Sethian did not
mention this fact in \cite{osher92} although they use in their
work a classical conservative Roe scheme. Their trick to obtain
acceptable numerical results is not detailed.
Karni, in
\cite{karni94}, points out this difficulty and proposes a simple
hybrid scheme to remove the pressure oscillations. The idea is to
solve the classical conservative equations far from the interface
between the two fluids and a nonconservative pressure equation
near the interface. The resulting nonconservative scheme, which is
built in order to preserve constant pressure and velocity states,
gives good results. But it is not clear, in the paper of Karni,
whether the conservation error tends to zero with the step of the
mesh. Indeed, as it is proved in the work of Hou and LeFloch
\cite{hou94}, nonconservative schemes generally converge to wrong
solutions. Then Abgrall in \cite{abgrall96}, Saurel and Abgrall in
\cite{saurel99} propose a simpler approach based on the
nonconservative convection of well chosen pressure law parameters.
As in the work of Karni, the main idea in order to choose the good
transported quantities, is to construct a scheme which preserves
the moving contact discontinuities. In a one dimensional framework
this is expressed by the fact that the pressure and the velocity
should not change if they are constant at the first time step, but
a numerical diffusion of the density is possible. The resulting
scheme is quasi-conservative. More precisely, the numerical fluxes
for mass, momentum and energy are conservative whereas the
numerical fluxes for the gas law coefficients are not
conservative.

Hence, the resulting scheme allows slight mass
transfers between the gas and the liquid. According to numerical
experiments, this scheme seems to be converging. This is not in
contradiction with the previously cited result of Hou and LeFloch.
Indeed, according to the Rankine-Hugoniot jump relations, the
velocity of the flow and the gas law parameters cannot present a
simultaneous jump. Thus the nonconservative products in the
nonconservative transport equations are perfectly defined. Another
interesting fix to the spurious oscillations is proposed by
Fedkiw, Aslam, Merriman and Osher in \cite{fedkiw99}, under the
name of the Ghost Fluid Method. With the aid of a ghost fluid,
these authors propose a nonconservative Godunov scheme where only
one-fluid Riemann problems have to be solved. This method has been
simplified by Abgrall and Karni in \cite{karni01}, under the name
of the "Two-Flux Method". The Ghost Fluid and the Two-Flux methods
are completely nonconservative at the interface. Numerical
experiments (see \cite{barberon02}) seem to indicate that they
converge, but there is still no complete theoretical justification
of this
"miracle".\\


Once the pressure oscillations have been corrected by an adequate
scheme, still other difficulties remain. The remedies have more to
do with numerical engineering than with mathematics but must be
carefully assembled. In this way, we have to deal with
axisymmetry, unstructured meshes, implementation of
the Riemann solver, CFL condition... \\

This paper is thus organized as follows. \\


Following this Introduction,  the second part is devoted to a
presentation of the mathematical model. We describe its
mathematical properties. The main feature is that the Riemann
problem is globally well posed. This fact is important for the
numerical simulation when a Godunov scheme is used. On the other
hand the model allows negative values for the pressure. As we have
already said, this can be justified
in some physical configurations.\\
The third part begins with an exposition of the pressure
oscillations that appear at the material interface in two-fluid
simulations. We illustrate the pressure oscillations thanks to a
simple test case. We recall the bases of the construction of the
Saurel-Abgrall scheme. In the original Saurel and Abgrall paper
\cite{saurel99}, the construction is achieved with the help of the
approximate Riemann solver of Rusanov. As detailed in
\cite{rouy00}, we use here instead an exact Riemann solver. In
order to solve the convection of the pressure law coefficients, we
use a nonconservative numerical flux based on the velocity of the
contact in the Riemann solution between two cells.
The Abgrall-Saurel reasoning is only valid for a multifluid flow where
each fluid satisfies the stiffened gas law. For the sake of
completeness, we also present another finite volumes scheme which
preserves constant pressure-velocity states, and this for any gas
law. This scheme is a Lagrange plus projection scheme. In the
Lagrangian step, the contacts are perfectly resolved. The
projection step is thus constructed in order to preserve this
property. We propose to project the pressure back on the Eulerian
grid instead of other conservative variables. The resulting scheme
is valid for any pressure law and can be generalized to higher
dimensions. This scheme is not conservative for the mass fraction
it is thus precise only for moderate shocks. It is generally not
convergent. For strong shocks a hybrid scheme should be used as in
the papers of Karni \cite{karni94}, \cite{karni96}: the idea would
be to project the conservative variables near the shocks and the
pressure near the contacts. Thus the Lagrange-projection scheme is
not used in the sequel of the paper. It is slightly more diffusive
and complex in its hybrid version than the Saurel-Abgrall scheme.
Furthermore in our application,
 the validity of the stiffened gas law is sufficient. The
 Lagrange-projection approach would be necessary to take into account the
 vaporization of the liquid in cavitation zones (see
 \cite{barberon02}).
 \\
 
The fourth part is devoted to an exposition of several practical
difficulties that have to be solved before the application of the
previous theory to an industrial problem:
\begin{itemize}
  \item The first adaptation concerns the construction of a 2D
  axisymmetrical scheme based on the 1D scheme of Saurel and
  Abgrall that preserves moving contacts. It is not trivial to
  extend the idea of Saurel and Abgrall
  to higher dimensions. This is due to the fact that in higher
  dimensions the velocity is generally not continuous through a
  contact discontinuity - only the normal component is. Nkonga
  recently proposed a 2D scheme for resolving shear interfaces
  in \cite{nkonga02}. His scheme perfectly resolves contact
  discontinuities aligned with the mesh, but because it is
  not conservative for the momentum, it is probably not convergent.
  In this paper we restrict ourselves to a scheme that preserves
  constant pressure-velocity states.
  First, we write a 3D Godunov scheme
  using the rotationnal invariance of the Euler equations, as
  usual. Some pressure law coefficients are convected in a nonconservative
  way, as in the Saurel-Abgrall scheme. Then we restrict this scheme to an axisymmetrical mesh.
  This leads to a 2D axisymmetrical scheme. The interest of this
  approach is to avoid complicated source terms that arise from
  the axisymmetrical Euler equations. It must be pointed out that
  in the literature many authors (as \cite{butler99}, \cite{kuszla98}, etc.) do
  not follow this (in our sense) correct approach and have to deal
  with singular source terms on the axis of rotation.
  \item The second necessary adaptation deals with the boundary
  conditions. We shall use in the application a classical
  treatment of the boundary conditions by defining artificial
  cells on the boundary. In order to define the physical values of
  the artificial cells, we follow the approach of partial Riemann
  problems as in the works of Dubois and LeFloch \cite{dubois88}, \cite{dubois01}.
  \item In our application, the geometry of the mesh is quite
  complex. This imposes the use of unstructured meshes.
  We thus have to develop a special technique of multiple time
  steps because the CFL condition given by the small cells is too
  restrictive. On each edge we define a local time step which is a
  power of two times the minimal time step. This local time step
  satisfies the CFL condition of the two neighboring cells. Then
  more time steps are performed on the small cells than on the big
  cells, with possible "rendez-vous" because the ratio of two different
  time steps is a power of two. The resulting
  scheme is stable, and the computation time is reduced by a significant factor.
\end{itemize}


In the fifth part we present numerical results obtained in the
case of an axisymmetrical gas generator. As we have said before,
the gas generator geometry is quite complex. Several kinds of
boundary conditions have to be considered. Because of the presence
of very small holes, the ratio between the biggest cell and the
smallest cell in the mesh is of the order of 10. All these facts
justify our previous approach. We are then able to present a
complete simulation of this industrial system. According to
preliminary measurements, our simulation gives, at least
qualitatively, good results. More details are given in \cite{rouy00}.\\

The sixth part is the conclusion of the paper.\\

We then end the paper with an appendix (seventh part) where some
computations are detailed:
\begin{itemize}
  \item First we perform classical computations concerning
  hyperbolicity and entropy. We also recall the mathematical
  equivalence of the conservative equations with the nonconservative
   form of Saurel-Abgrall. This fact would permit to
prove a Lax-Wendroff convergence result for the Saurel-Abgrall
  scheme and thus presents some interest.
  \item Second, we prove that the Riemann problem for a two-fluid
  flow governed by a stiffened gas law is globally well posed. In
  the case of strong rarefaction waves, it is necessary to
  introduce negative pressures and/or vacuum regions.
  The proof uses standard arguments
  but we have not found it in the literature. The notations that we
  have to set are also useful for a rigorous definition of the
  boundary conditions that are described in part 4.
\end{itemize}

%trsltx-end-ignoble

\section{A two-fluids model for air-water flows}


\subsection{Basic equations}

We are interested in the flow of a compressible continuous medium
characterized by its density \( \rho (t,x) \), its velocity \(
u(t,x) \), and its internal massic energy \( \varepsilon (t,x) \).
The time variable is denoted by \( t \), the space variable is \(
x \), and for simplicity we present the model in one space
dimension. The pressure $p(t,x)$ of the medium is expressed by a
stiffened gas Equation Of State (EOS)
\begin{equation} \label{presslaw}
p = (\gamma  - 1)\rho \varepsilon  - \gamma \pi .
\end{equation}
Because we are studying a flow of several fluids, the two
parameters $\gamma$ and $\pi$ of the pressure law also depend on
time and space
\begin{equation}
  \gamma=\gamma(t,x) \quad \text{and} \quad \pi=\pi(t,x).
\end{equation}
Conservation of mass, momentum and energy lead to the three Euler
equations\begin{eqnarray}
\rho _{t}+(\rho u)_{x} & = & 0,\nonumber \\
(\rho u)_{t}+(\rho u^{2}+p)_{x} & = & 0,\label{euler} \\
(\rho E)_{t}+((\rho E+p)u)_{x} & = & 0,\nonumber
\end{eqnarray}
where \( E \), the total massic energy, is defined by
\begin{equation}
E=\varepsilon +\frac{u^{2}}{2}.
\end{equation}
On the other hand, the pressure law parameters are convected with
the flow
\begin{equation}\label{convcoef}
\begin{gathered}
  \gamma _t  + u\gamma _x  = 0, \hfill \\
  \pi _t  + u\pi _x  = 0. \hfill \\
\end{gathered}
\end{equation}
If the gas is supposed to be perfect and polytropic (this will
always be the case in the sequel), we set
$\gamma=\gamma_{\text{gas}}$ and $\pi=0$. For the liquid, the
stiffened gas EOS is still valid. It reads \[ p=(\gamma
_{\,\text{liq}}-1)\rho \varepsilon -\gamma
_{\,\text{liq}}\pi_{\text{liq}} .\] The constants \( \gamma
_{\,\text{liq}} \) and \( \pi_{\text{liq}}  \) are chosen in order
to match physical measurements. Cocchi and Saurel in
\cite{saurel97} have proposed the following values for $\gamma
_{\,\text{liq}}$ and $\pi_{\text{liq}}$
\begin{equation}\label{valphi}
\begin{gathered}
  \gamma_{\,\text{liq}} = 5.5, \hfill \\
  \pi_{\text{liq}}  = 4900\text{ bar}\text{.} \hfill \\
\end{gathered}
\end{equation}
These values are based on sound speed and shock speed
measurements.

In this model, the interface between air and water can be located
by the discontinuity of $\gamma(t,x)$ or $\pi(t,x)$. It must be
pointed out that mathematically, the model is perfectly equivalent
to a level-set model (as the one of \cite{osher92}). In the
level-set model, equations (\ref{convcoef}) are replaced by the
convection of a level-set function
\begin{equation} \label{convlevset}
  \varphi_t+u\varphi_x=0,
\end{equation}
and the pressure law (\ref{presslaw}) by
\begin{equation} \label{preslevset}
  p=(\gamma(\varphi)-1)\rho\varepsilon -\gamma(\varphi) \pi(\varphi).
\end{equation}
But it appears that numerically, it is not equivalent to
discretize (\ref{euler}), (\ref{presslaw}), (\ref{convcoef}) or
(\ref{euler}), (\ref{preslevset}), (\ref{convlevset}).
\subsection{Properties of the model}

Our system (\ref{euler}), (\ref{presslaw}), (\ref{convcoef}) can
be expressed in the classical form of a system of conservation
laws (the equivalence between the non-conservative form and the
conservative form is rigorously proved in section \ref{appendix}:
see remark \ref{noncons2})
\begin{equation} \label{slc} W_t+F(W)_x=0,
\end{equation}
with\[ W=\left\{ \begin{array}{c}
\rho \\
\rho u\\
\rho E\\
\rho \gamma \\
\rho \pi
\end{array}\right\} ,\quad F(W)=\left( \begin{array}{c}
\rho u\\
\rho u^{2}+p\\
(\rho E+p)u\\
\rho \gamma u \\
\rho \pi u
\end{array}\right) ,\]
and the pressure law (\ref{presslaw}).

If $c$ is the sound speed associated with the pressure law
(\ref{presslaw}), it verifies
\begin{equation}\label{soundspeed}
  c^2=\gamma \frac{p+\pi}{\rho}.
\end{equation}
The hyperbolicity of the system (\ref{slc}) then implies that
\begin{equation}\label{condhyp}
  p+\pi \geq 0.
\end{equation}
Thus, the model admits negative pressure in the water. Is this
physically correct? Indeed, negative pressures can locally and
briefly appear in a liquid, they should then be called
\textsl{tensions}. But in the zone of negative pressures the
liquid is in a metastable state and is subject to vaporization.
This phenomenon is called \textsl{cavitation}. For a physical
description of the cavitation, we refer to the book of Franc $\&$
\textit{al} \cite{franc95}. We have proposed recently a simple
adaptation of the stiffened gas model to take into account
cavitation (see \cite{helluy02}) but before the phase transition,
or if the appearance of the tensions is very short, the stiffened
gas law is still a good physical model. It must be pointed out
that in our numerical simulations we will not use any special
treatment when negative pressures occur. Some authors (as
\cite{fedkiw99}) have proposed to correct the pressure by limiting
it to zero when it is negative. This amounts to forgoing energy
conservation and we think that it is worse from a physical point
of view than negative pressures. It also causes additional
numerical complications due to the kink in the limited gas law.
For example, it is necessary to envisage
a centered scheme on the cells where the pressure is limited.\\

There is another (mathematical) reason to keep this model. If one
considers the Riemann problem associated to the system (\ref{slc})
and (\ref{presslaw})
\begin{eqnarray}
W_t+F(W)_x=0,\\
W(0,x)=\left\{ \begin{array}{c}
W_{l}\textrm{ if }x<0,\\
W_{r}\textrm{ if }x>0.
\end{array} \right .
\end{eqnarray}
The self-similar solution is noted
\[ W(t,x)=R\left(\frac{x}{t},W_{l},W_{r}\right).\]
Then it can be shown that the solution is unique for any left and
right states $W_l$ and $W_r$ satisfying the positivity of density
and the hyperbolicity condition (\ref{condhyp}). The fact that the
global Riemann problem can be uniquely solved is well known in the
case of a monofluid flow. For example, it is solved in the book of
Godunov \cite{godunov79} for the case of a one-fluid flow
satisfying the stiffened gas law. In the case of strong
rarefaction waves, the solution can present a region of vacuum in
which
\begin{equation}\label{vacuum}
\begin{gathered}
  p =  - \pi_{\text{liq}} , \hfill \\
  \rho  = 0. \hfill \\
\end{gathered}
\end{equation}
The solvability result can be extended to our model. For the sake
of completeness, we prove it in the section \ref{appendix}. The
global solvability of the Riemann problem is fundamental when one
intends to use a Godunov type scheme, because it ensures the
robustness of the resulting scheme. Another important property of
the model is that it permits many equivalent formulations. Indeed,
any function $f(\gamma,\pi)$ of $\gamma$ and $\pi$ is also
convected with the flow. For example, the system (\ref{slc}),
(\ref{presslaw}) is equivalent to
\begin{equation} \label{abgralcons}
\begin{gathered}
 \rho _t  + (\rho u)_x  = 0, \hfill \\
 (\rho u)_t  + (\rho u^2  + p)_x  = 0, \hfill \\
 (\rho E)_t  + ((\rho E + p)u)_x  = 0, \hfill \\
 (\rho /(\gamma-1))_t  + (\rho u /(\gamma-1) )_x  = 0,\hfill \\
 (\rho \gamma \pi /(\gamma-1))_t  + (\rho u \gamma \pi /(\gamma-1) )_x  = 0,\hfill
\end{gathered}
\end{equation}
with the stiffened gas law (\ref{presslaw}).

It is also equivalent to the following nonconservative form
\begin{equation} \label{abgralnoncons}
\begin{gathered}
 \rho _t  + (\rho u)_x  = 0, \hfill \\
 (\rho u)_t  + (\rho u^2  + p)_x  = 0, \hfill \\
 (\rho E)_t  + ((\rho E + p)u)_x  = 0, \hfill \\
 ( 1/(\gamma-1))_t  + u (1/(\gamma-1) )_x  = 0,\hfill \\
 (\gamma \pi /(\gamma-1))_t  + u (\gamma \pi /(\gamma-1) )_x  = 0,\hfill
\end{gathered}
\end{equation}
with the stiffened gas law (\ref{presslaw}).

The nonconservative form (\ref{abgralnoncons}) plays a particular
role among the other forms on the numerical point of view as we
will see in the next section.


\section{Nonconservative finite volumes approximation}

This section is devoted to a short and simple presentation of the
pressure oscillations phenomenon in the conservative Godunov
schemes. It appears that for very simple one-dimensional test
cases, the classical first order conservative Godunov scheme gives
very bad results on every conservative form of the equations as
(\ref{slc}) or (\ref{abgralcons}). We first exhibit one of these
test
cases which is a simple Riemann problem.\\
Then, we present two fixes which permit to avoid the pressure
oscillations at the interface:
\begin{itemize}
  \item The first scheme is the Saurel-Abgrall scheme. The
  construction principle of this scheme is to require that it
  preserves the moving contact discontinuities. This
  condition leads to a nonconservative discretization of the
  transport of some pressure law coefficients.   Let us recall that
   the conservative 1D Godunov scheme also preserves moving contact discontinuities in the case
of a one-fluid flow. The nonconservative correction is only useful
for multifluid flows. The Saurel-Abgrall correction cannot
  be applied to other pressure laws than the stiffened gas law.
  \item The second scheme is a Lagrange plus remap scheme. This
  scheme works for any pressure laws but only for moderate shocks because it is not conservative
  for the mass fraction.
  It is based on the simple remark that during the Lagrangian step,
  the contact discontinuities are perfectly solved. In the remap
  step we thus project mass, momentum and energy as usual. We forget the mass
  fraction conservation and instead project the pressure. In this
  way, the pressure equilibrium of the two components is
  recovered.
\end{itemize}
These two schemes remove the pressure oscillation phenomenon and
can be extended without difficulty to higher dimensions. The
Saurel-Abgrall scheme is less diffusive than the
Lagrange-plus-remap scheme. The Lagrange-plus-remap scheme is not
convergent in its present form. It is possible to improve its
precision for strong shocks by employing a hybrid version of the
scheme: with the help of a level-set function, a conservative
scheme is used far from the interface and the Lagrange-plus-remap
approach near the contact. Hybrid schemes are described for
example in \cite{karni96} and \cite{seguin03}. Because we
concentrate on the stiffened gas law, only the Saurel-Abgrall
scheme is used in the sequel of the paper for
the numerical experiments in two dimensions. \\
In this section, we restrict ourself to a Riemann problem initial
condition. For the numerical experiments, we choose the following
values
\begin{equation}
\label{a1condini}\begin{array}{l}
 \rho _l  = 10\text{ kg/m}^3 \quad u_l  = 50\text{ m/s} \quad p_l  = 1.1\times10^5\text{ Pa}
 \quad \gamma _l  = 1.4 \quad \pi_l=0,\\
 \rho _r  = 1\text{ kg/m}^3\quad u_r  = 50\text{ m/s}\quad p_r  = 1\times10^5\text{ Pa}
 \quad \gamma _r  = 1.1 \quad \pi_r=0.\\
 \end{array}
\end{equation}
\subsection{Failure of the Godunov scheme} In this section, we
present numerical results obtained by a classical Godunov scheme.
The approximated system is (\ref{abgralcons}), but we would obtain
very similar results for any other conservative formulation.

Consider a space step \( h \) and a time step \( \tau \). The
discretization points are \( x_{i}=ih,\quad i\in Z \). The cells
\( C_{i} \) are centered on \( x_{i} \), \(
C_{i}=]x_{i-1/2},x_{i+1/2}[ \). We look for an approximation of \(
W \) in the cell \( C_{i} \) at time \( t_{n}=n\tau  \)\[
W_{i}^{n}\simeq W(t_{n},x),\quad x\in C_{i}.\] A general
conservative finite volumes scheme reads\[
W^{n+1}_{i}=W^{n}_{i}-\frac{\tau
}{h}(F^{n}_{i+1/2}-F^{n}_{i-1/2}).\] In the case of the Godunov
scheme, the numerical flux is given by the resolution of a Riemann
problem at each cell interface \( x_{i+1/2} \) and takes the
form\[ F^{n}_{i+1/2}=F(R(0^{+\textrm{ or
}-},W^{n}_{i},W^{n}_{i+1})).\]


%trsltx-begin-ignoble

The initial conditions are (\ref{a1condini}). We plot only the
pressure at time $t=1$ ms. The study interval is $]0,L[$ with
$L=1$ m. The number of cells is fixed at $N=400$ and the CFL
number is $0.7$. We observe pressure oscillations at the contact
discontinuity (which is also the material interface between the
two fluids). The results are in Figure \ref{a1godu_sys1}.
\begin{figure}[h]\label{a1godu_sys1}
\centering
%\resizebox*{8.cm}{6.cm}{\includegraphics{annexe1_eps/godu_rho.eps}}
%\resizebox*{8.cm}{6.cm}{\includegraphics{annexe1_eps/godu_vit.eps}}
%\resizebox*{8.cm}{6.cm}{\includegraphics{annexe1_eps/godu_pres.eps}}
%\resizebox*{8.cm}{6.cm}{\includegraphics{annexe1_eps/godu_gam.eps}}
\caption{\textit{Godunov scheme, pressure (line: exact; dots:
numeric) }}
\end{figure}

\subsection{Nonconservative transport of the pressure law
coefficients} The conservative scheme gives very bad results and
cannot be used for higher dimensional simulations. On the other
hand, numerical experiments indicate that the situation is not
better with another (approximate) Riemann solver.  A second order
MUSCL extension would slightly improve the results but it is not
sufficient.

In order to improve the precision of the Godunov scheme, it is
possible as proposed by Saurel and Abgrall in \cite{saurel99} to
give up the last two conservation laws of the system
(\ref{abgralcons}) and replace them by a nonconservative transport
equation to obtain (\ref{abgralnoncons}). We show now why the
special nonconservative form (\ref{abgralnoncons}) plays a
particular role for a finite volumes approximation. For this, let
us consider a general conservative (for the mass, momentum and
energy) upwind scheme. Suppose that we want to approximate a
general moving contact discontinuity of constant velocity $v$ and
pressure $p$. We suppose that $v\gg 1$ and that the constant flow
is supersonic. Then, because the speed $v>0$, the upwind scheme
reads
\begin{eqnarray}
 \rho _i^{n + 1}  &=& \rho _i^n  - \frac{\tau }{h}\left( {(\rho u)_i^n  - (\rho u)_{i - 1}^n } \right), \\
 (\rho u)_i^{n + 1}  &=& (\rho u)_i^n  - \frac{\tau }{h}\left( {(\rho u^2  + p)_i^n  - (\rho u^2  + p)_{i - 1}^n } \right), \\
 (\rho \varepsilon  &+& \rho \frac{{u^2 }}{2})_i^{n + 1}  = (\rho \varepsilon  + \rho \frac{{u^2 }}{2})_i^n  - \\
 & & \frac{\tau }{h}\left( {(\rho \varepsilon u + \rho u\frac{{u^2 }}{2} + pu)_i^n
  - (\rho \varepsilon u + \rho u\frac{{u^2 }}{2} + pu)_{i - 1}^n } \right). \nonumber
 \end{eqnarray}
We now impose that the scheme preserves the moving contact
discontinuities, i.e. that $u_i^{n+1}=u_i^{n}=v$ and
$p_i^{n+1}=p_i^{n}=p$. We obtain
\begin{eqnarray}
 \rho _i^{n + 1}  &=& \rho _i^n  -
  \frac{\tau }{h}\left( {(\rho v)_i^n  - (\rho v)_{i - 1}^n } \right), \\
 (\rho v)_i^{n + 1}  &=& (\rho v)_i^n  -
 \frac{\tau }{h}\left( {(\rho v^2  + p)_i^n  - (\rho v^2  + p)_{i - 1}^n } \right), \\
 (\rho \varepsilon  &+& \rho \frac{{v^2 }}{2})_i^{n + 1}  = (\rho \varepsilon  + \rho \frac{{v^2
 }}{2})_i^n\\
 & & -\frac{\tau }{h}\left( {(\rho \varepsilon v + \rho v\frac{{v^2 }}{2} + p v)_i^n
 - (\rho \varepsilon v + \rho v\frac{{v^2 }}{2} + p v)_{i - 1}^n }
 \right). \nonumber
 \end{eqnarray}
 The two first equations reduce then to
\begin{equation}
  \rho _i^{n + 1}  = \rho _i^n  - \frac{\tau }{h}v\left( {\rho_i^n  - \rho_{i - 1}^n }
  \right),
\end{equation}
while the last equation becomes
\begin{equation}
(\rho \varepsilon )_i^{n + 1}  = (\rho \varepsilon )_i^n  -
\frac{\tau }{h}v\left( {(\rho \varepsilon )_i^n  - (\rho
\varepsilon )_{i - 1}^n } \right).
\end{equation}
But because $\rho \varepsilon = (p+\gamma \pi)/(\gamma-1)$, the
only compatible approximations for $\gamma$ and $\pi$ are
\begin{equation} \label{a1noncons}
\begin{gathered}
 \left(\frac{1}{{\gamma   - 1}}\right)_i^{n + 1} =
\left(\frac{1}{{\gamma   - 1}}\right)_{i}^{n} - \frac{\tau
}{h}v\left( \left(\frac{1}{{\gamma   - 1}}\right)_{i}^{n} -
\left(\frac{1}{{\gamma   - 1}}\right)_{i-1}^{n} \right), \\
\left(\frac{\gamma \pi}{{\gamma   - 1}}\right)_i^{n + 1} =
\left(\frac{\gamma \pi}{{\gamma   - 1}}\right)_{i}^{n} -
\frac{\tau }{h}v\left( \left(\frac{\gamma \pi}{{\gamma   -
1}}\right)_{i}^{n} - \left(\frac{\gamma \pi}{{\gamma   -
1}}\right)_{i-1}^{n} \right).
\end{gathered}
\end{equation}
This is an upwind approximation of the transport equations
\begin{equation}
\begin{gathered}
  (1/(\gamma-1))_t+v(1/(\gamma-1))_x=0,\\
  (\gamma \pi/(\gamma-1))_t+v(\gamma \pi/(\gamma-1))_x=0.
  \end{gathered}
\end{equation}
\textsl{Any scheme that reduces to (\ref{a1noncons}) for constant
velocity and pressure will then preserve moving contact
discontinuities}.

We propose now such a scheme. First, we define the interface
values by the resolution of Riemann problems at the points \(
x_{i+1/2} \):\[ W^{n}_{i+1/2}=R(0,W^{n}_{i},W^{n}_{i+1}).\] For
density, momentum and energy, the classical conservative approach
is employed:\begin{eqnarray} \label{godupart}
\rho ^{n+1}_{i} & = & \rho ^{n}_{i}-\frac{\tau }{h}((\rho u)^{n}_{i+1/2}-(\rho u)^{n}_{i-1/2}),\nonumber \\
(\rho u)^{n+1}_{i} & = & (\rho u)^{n}_{i}-\frac{\tau }{h}((\rho u^{2}+p)^{n}_{i+1/2}-(\rho u^{2}+p)^{n}_{i-1/2}),\label{schemacons} \\
(\rho E)^{n+1}_{i} & = & (\rho E)^{n}_{i}-\frac{\tau
}{h}\left(((\rho E+p)u)^{n}_{i+1/2}-((\rho
E+p)u)^{n}_{i-1/2}\right).\nonumber
\end{eqnarray}
On the other hand, an upwind nonconservative scheme is used for
the last two equations of (\ref{abgralnoncons}). This
nonconservative scheme is based on the contact discontinuity
velocity of the Riemann problems solved at the points
$(x_{i+1/2})$. It reads
\begin{equation}
\label{schemanoncons} \alpha ^{n+1}_{i}=\alpha ^{n}_{i}-\frac{\tau
}{h}(\min (u_{i+1/2}^{n},0)(\alpha _{i+1}^{n}-\alpha
_{i}^{n})+\max (u_{i-1/2}^{n},0)(\alpha _{i}^{n}-\alpha
_{i-1}^{n})),
\end{equation}
where the quantity $\alpha$ is $1/(\gamma-1)$ or $\gamma \pi
/(\gamma-1)$. This choice is slightly different from the one of
Saurel and Abgrall in \cite{saurel99} which is based on the
approximate Riemann solver of Rusanov. It is easy to check that
the scheme
 (\ref{schemanoncons}) reduces to (\ref{a1noncons}) for constant
velocity and pressure states.
 \\

With the scheme (\ref{godupart}), (\ref{schemanoncons}), the
results on the same test case as above are given in Figure
\ref{a1grall_sys1}. There is an evident improvement.
\begin{figure}[h]\label{a1grall_sys1}
\centering
%\resizebox*{8.cm}{6.cm}{\includegraphics{annexe1_eps/grall_rho.eps}}
%\resizebox*{8.cm}{6.cm}{\includegraphics{annexe1_eps/grall_vit.eps}}
%\resizebox*{8.cm}{6.cm}{\includegraphics{annexe1_eps/grall_pres.eps}}
%\resizebox*{8.cm}{6.cm}{\includegraphics{annexe1_eps/grall_gam.eps}}
\caption{\textit{Saurel-Abgrall scheme, pressure (line: exact;
dots: numeric) }}
\end{figure}

Unfortunately, the previous reasoning cannot be extended to a
pressure law which is not linear with respect to $\rho
\varepsilon$. To be more general we thus present in the next
paragraph a general approach to deal with non-linear pressure
laws.
\subsection{A Lagrange plus
remap scheme for two-fluids flow} In this section we describe the
results obtained with a Lagrange plus remap scheme. In order to
have a clear description
of the scheme, we will recall the bases of the Lagrange scheme construction (see \cite{raviart96}).\\
\subsubsection*{Notations} We wish to approximate the system of
conservation laws.
\begin{equation}
\label{eq1} W_t + F(W)_x = 0.
\end{equation}
For this purpose, we consider an increasing sequence of instants
$(t^n)_{n \in {\rm N}} $ and a sequence of subdivisions of space
defined by the points $(x_i^n )_{i \in {\rm Z},n \in {\rm N}} $
which satisfy
\[
\forall i \in {\rm Z},\,\forall n \in {\rm N},\quad x_i^n < x_{i +
1}^n .
\]
The point $x_i^n $ will be the center of the cell $C_i^n $. In
order to define properly these cells, we thus introduce the
boundary points
\[
x_{i + 1 / 2}^n = \frac{x_i^n + x_{i + 1}^n }{2}.
\]
The cell $C_i^n $ is then
\[
C_i^n = \left] {x_{i - 1 / 2}^n ,x_{i + 1 / 2}^n } \right[.
\]
The time steps are
\[
\tau ^n = t^{n + 1} - t^n.
\]
The lengths of the cells are
\[
h_i^n = x_{i + 1 / 2}^n - x_{i - 1 / 2}^n .
\]
In a Lagrange scheme, the cell boundaries move between the time
step $t^n$ and $t^{n + 1}$ with the velocity $u_{i + 1 / 2}^n $.
Thus,
\[
x_{i + 1 / 2}^{n + 1} = x_{i + 1 / 2}^n + \tau ^nu_{i + 1 / 2}^n .
\]
A CFL condition has to be provided in order that the points $x_{i
+ 1 / 2}^{n + 1} $ stay ordered.

\subsubsection*{Scheme construction}

We suppose that at time $t^n$ we know an approximation $W^n$ of
the exact solution $W$. The approximation is supposed to be
constant in each cell
\[
W(t^n,x) \simeq W^n(x) = W_i^n ,\quad x \in C_i^n .
\]
We then compute exactly for a time $\tau ^n$, the entropic
solution of
\[
\begin{array}{l}
 V_t + F(V)_x = 0, \\
 V(0,x) = W^n(x),\quad x \in {\rm R}. \\
 \end{array}
\]
This exact resolution is possible under a CFL condition.


The new approximation of $W$ at time $t^{n + 1}$ is then defined
as the mean value of the exact solution in the new cells
\[
W_i^{n + 1} = \frac{1}{h_i^{n + 1} }\int_{C_i^{n + 1} } {V(\tau _n
,x)dx.}
\]
The Riemann problem reads
\[
\begin{array}{l}
 U_t + F(U)_x = 0, \\
 U(0,x) = \left\{ {{\begin{array}{*{20}c}
 {W_l \quad x < 0,} \hfill \\
 {W_r \quad x > 0.} \hfill \\
\end{array} }} \right. \\
 \end{array}
\]
The solution is self-similar; as before it is noted
\[
R(x/t,W_l ,W_r) = U(t,x).
\]
In order to have a simpler expression of the scheme, we express
the conservation law in the space-time trapezoid $Q$ whose
parallel sides are $C_i^n $ and $C_i^{n + 1} $.
\[
0 = \int_Q {\left( {W_t + F(W)_x } \right)} dx \wedge dt  =
\int_{\partial Q} {\left(F(W)dt - Wdx \right)}.
\]
The contour integral in the right hand side is the sum of four
contributions (bottom, top, right and left)
\[
\begin{array}{c}
 \int_{\partial Q} {\left(F(W)dt - Wdx \right)} = \int_{C_i^n } { - W_i^n dx} \\
 + \int_{C_i^{n + 1} } {W_i^{n + 1} dx} \\
 + \int_{t = 0}^{t = \tau ^n} {\left( {F(R(u_{i + 1 /2}^n,W_i^n ,W_{i + 1}^n  ))
  - R(u_{i + 1 / 2}^n,W_i^n ,W_{i + 1}^n )u_{i + 1 / 2}^n }
\right)dt} \\
 - \int_{t = 0}^{t = \tau ^n} {\left( {F(R(u_{i - 1 /2}^n,W_{i - 1}^n ,W_i^n  ))
  - R(u_{i - 1 / 2}^n,W_{i - 1}^n ,W_i^n  )u_{i - 1 / 2}^n }
\right)dt} . \\
 \end{array}
\]
This gives
\begin{equation} \label{a1lag}
\begin{array}{c}
 0 = h_i^{n + 1} W_i^{n + 1} - h_i^n W_i^n \\
 + \tau _n \left( {F(R(u_{i + 1 / 2}^n,W_i^n ,W_{i + 1}^n  ))
  - R(u_{i + 1 / 2}^n,W_i^n,W_{i + 1}^n  )u_{i + 1 / 2}^n } \right) \\
 - \tau _n \left( {F(R(u_{i - 1 / 2}^n,W_{i - 1}^n ,W_i^n  ))
  - R(u_{i - 1 / 2}^n,W_{i -1}^n ,W_i^n  )u_{i - 1 / 2}^n } \right). \\
 \end{array}
\end{equation}
 When the velocities at the cell boundaries $u_{i + 1 / 2}^n$
are zero, the scheme reduces to the classical Godunov scheme.
Another important case is when the velocity $u_{i + 1 / 2}^n$ is
equal to the contact discontinuity velocity of the Riemann problem
between the cells $C_i^n$ and $C_{i+1}$. With this choice, a
moving contact is perfectly resolved. The problem is now to come
back properly from the Lagrangian grid
$\left({C_{i}^{n+1}}\right)$ to the Eulerian grid
$\left({C_{i}^{n}}\right)$. This is the goal of the remap step.
\subsubsection*{Remap step} Let us first describe the remap step
of the classical Lagrange and projection scheme. Actually, the
formula (\ref{a1lag}) defines a value $W_{i}^{n+1/2}$ of the
conservative variables in the new cells $C_{i}^{n+1}$. This value
has now to be averaged on the old cells $C_{i}^{n}$. This is
usually done with the formula
\[
\displaylines{
  W_i^{n + 1}  = \frac{\tau }{h}\max (u_{i - 1/2}^n ,0)W_{i - 1}^{n + 1/2}  - \frac{\tau }{h}\min (u_{i + 1/2}^n ,0)W_{i + 1}^{n + 1/2}  +  \cr
  (1 - \frac{\tau }{h}\max (u_{i - 1/2}^n ,0) + \frac{\tau }{h}\min (u_{i + 1/2}^n ,0))W_i^{n + 1/2} . \cr}
\]
Our scheme is then a very simple correction of the
Lagrange-projection scheme. The projection is the same for
density, momentum and energy. But instead of projecting the last
conservative variable $\rho /(\gamma-1)$, we project the pressure
according to the formula
\[
\displaylines{
  p_i^{n + 1}  = \frac{\tau }{h}\max (u_{i - 1/2}^n ,0)p_{i - 1}^{n + 1/2}  - \frac{\tau }{h}\min (u_{i + 1/2}^n ,0)p_{i + 1}^{n + 1/2}  +  \cr
  (1 - \frac{\tau }{h}\max (u_{i - 1/2}^n ,0) + \frac{\tau }{h}\min (u_{i + 1/2}^n ,0))p_i^{n + 1/2} . \cr}
\]
It is then possible to compute $\gamma_i^{n + 1}$ thanks to the
pressure law (according to (\ref{a1condini}), the value of $\pi$
is indeed $0$). In test cases where $\pi\neq 0$ another quantity
has to be projected in order to recover $\gamma_i^{n + 1}$ and
$\pi_i^{n + 1}$. It could be for example the temperature.
\subsubsection*{Numerical results} With the Lagrange-projection scheme, we
obtain the results in Figure \ref{a1lag_sys1}. They are comparable
to the results of the Saurel-Abgrall approach but are a little bit
more diffusive. This extra diffusion is classical and is of course
due to the double projection: Godunov averaging and remap
averaging. This method can be applied though to more general
pressure laws.
\begin{figure}[h]\label{a1lag_sys1}
\centering
%\resizebox*{8.cm}{6.cm}{\includegraphics{annexe1_eps/lag_rho.eps}}
%\resizebox*{8.cm}{6.cm}{\includegraphics{annexe1_eps/lag_vit.eps}}
%\resizebox*{8.cm}{6.cm}{\includegraphics{annexe1_eps/lag_pres.eps}}
%\resizebox*{8.cm}{6.cm}{\includegraphics{annexe1_eps/lag_gam.eps}}
\caption{\textit{Lagrange and remap scheme, pressure (line: exact;
dots: numeric) }}
\end{figure}



\section{Axisymmetrical extension of the Godunov scheme}


\subsection{\label{num1}Scheme construction and properties}

In this paragraph, we present the extension of scheme (\ref{schemacons},
\ref{schemanoncons}) to an industrial axisymmetrical case. We will
first describe a 3D scheme without source term for which stability
under a CFL condition is well established. Then, using special meshes
with a rotational invariance we will deduce the 2D axisymmetrical
scheme keeping the same CFL stability condition. This 2D scheme presents
a pressure source term that is thus handled explicitly.

Consider a mesh of a 2D open set \( \Omega \subset R^{2} \), that
is to say a family of open sets \( (\Omega _{k})_{1\leq k\leq N} \)
satisfying\begin{eqnarray*}
\overline{\Omega }=\bigcup _{k}\overline{\Omega _{k}}, &  & \\
i\neq j\Rightarrow \Omega _{i}\cap \Omega _{j}=\oslash . &  &
\end{eqnarray*}
We suppose also that \( (x,z)\in \Omega \Rightarrow x>0 \). A 3D
mesh can be generated by a rotation of \( \Omega  \) around the axis
\( x=y=0 \) in a referential \( (x,y,z) \). Introducing cylindrical
coordinates \( (r,\phi ,z) \): \begin{eqnarray*}
x & = & r\cos \phi ,\\
y & = & r\sin \phi ,\\
z & = & z,
\end{eqnarray*}
we define the family \( (Q_{k,l}) \) ,\( \quad 1\leq k\leq N \),
\( \quad 0\leq l\leq P-1 \quad \) by \[ Q_{k,l}=\left\{ (x,y,z)\in
R^{3},(r,z)\in \Omega _{k}\textrm{ and }\phi \in
]\frac{2(l-1/2)\pi }{P},\frac{2(l+1/2)\pi }{P}[\right\} .\] In
order to simply define the 3D scheme, we will slightly change the
notations. The velocity \( \overrightarrow{u} \) is now a vector
\( \overrightarrow{u}=(u^{1},u^{2},u^{3})^{T} \). \( W \) will be
now the vector of conservative variables \( (\rho ,\rho
\overrightarrow{u},\rho E)^{T} \), with \( E=\varepsilon
+\frac{\overrightarrow{u}\cdot \overrightarrow{u}}{2} \). The
vector of nonconservative variables is denoted by \( Y=(\alpha
,\beta )^{T} \) with
\begin{equation}
  \alpha=\frac{1}{\gamma -1},\quad \beta=\frac{\gamma \pi}{\gamma
  -1}.
\end{equation}
 We define also a mixed vector as \(
V=(W,Y)^{T} \).

In 3D, the Euler equations read (\( I_{d} \) is the identity matrix
of size \( d\times d \))\begin{eqnarray*}
\rho _{t}+\nabla \cdot (\rho \overrightarrow{u}) & = & 0,\\
(\rho \overrightarrow{u})_{t}+\nabla \cdot (\rho \overrightarrow{u}\otimes \overrightarrow{u}+pI_{3}) & = & 0,\\
(\rho E)_{t}+\nabla \cdot ((\rho E+p)\overrightarrow{u}) & = & 0.
\end{eqnarray*}
Introducing the three fluxes:\begin{eqnarray*}
G^{1}(W)=(\rho u^{1},\rho u^{1}u^{1}+p,\rho u^{1}u^{2},\rho u^{1}u^{3},(\rho E+p)u^{1}), &  & \\
G^{2}(W)=(\rho u^{2},\rho u^{2}u^{1},\rho u^{2}u^{2}+p,\rho u^{2}u^{3},(\rho E+p)u^{2}), &  & \\
G^{3}(W)=(\rho u^{3},\rho u^{3}u^{1},\rho u^{3}u^{2},\rho u^{3}u^{3}+p,(\rho E+p)u^{3}), &  &
\end{eqnarray*}
and the vector flux \( G=(G^{1},G^{2},G^{3})^{T} \), the conservative
equations can also be written\[
W_{t}+\nabla \cdot G(W)=0.\]


Whereas for the nonconservative variables \( \alpha  \) and \(
\beta  \), the equations are\[ Y_{t}+\overrightarrow{u}\cdot
\nabla Y=0,\] and the pressure law is still the stiffened gas law
which becomes
\begin{equation}
\label{eosnoncons} p=\frac{1}{\alpha }\rho \varepsilon
-\frac{\beta }{\alpha }.
\end{equation}


Now, a 3D scheme reads\[ \int
_{Q_{k,l}}V^{n+1}_{k,l}-V^{n}_{k,l}+\tau \int _{\partial
Q_{k,l}}F(V^{n}_{k,l},V^{n}_{k',l'},\nu )=0,\] where \(
V^{n}_{k,l} \) is the approximation of \( V \) in \( \Omega _{k,l}
\) at time \( t_{n} \), \( \nu  \) is the outward normal vector to
\( Q_{k,l} \) on \( \partial Q_{k,l} \) and \( Q_{k',l'} \)
denotes the neighbors of \( Q_{k,l} \) along its boundary. The
quantity \( F(\cdot ,\cdot ,\nu ) \) is the numerical flux that we
will now precisely define. The definition of the numerical flux is
based on the rotational invariance of the Euler equations. This is
very classical (see \cite{raviart96}). The only originality is the
special treatment of the nonconservative variables.

First, a unit vector \( \nu =(\nu ^{1},\nu ^{2},\nu ^{3})^{T} \)
is given. \( \nu  \) can also be written as \( \nu =(\cos \phi \sin \theta ,\sin \phi \sin \theta ,\cos \theta ) \).
We define then the rotation matrix \[
M(\nu )=\left[ \begin{array}{ccc}
\cos (\phi )\sin (\theta ) & \sin (\phi )\sin (\theta ) & \cos (\theta )\\
-\sin (\phi ) & \cos (\phi ) & 0\\
-\cos (\phi )\cos (\theta ) & -\sin (\phi )\cos (\theta ) & \sin (\theta )
\end{array}\right] ,\]
which satisfies \( M(\nu )\nu =(1,0,0)^{T} \), and\[
N(\nu )=\left[ \begin{array}{ccc}
1 &  & 0\\
 & M(\nu ) & \\
0 &  & I_{3}
\end{array}\right] .\]


Consider now two states \( V_{a} \) and \( V_{b} \). In order to
compute \( F(V_{a},V_{b},\nu ) \), several steps are performed:

\begin{enumerate}
\item Two rotated states are defined by \( \widetilde{V}_{a}=M(\nu )V_{a} \)
and \( \widetilde{V}_{b}=M(\nu )V_{b} \).
\item The following augmented Riemann problem in the normal direction is
then solved \begin{eqnarray*}
\widetilde{W}_{t}+G^{1}(\widetilde{W})_{x}=0, &  & \\
\widetilde{Y}_{t}+\widetilde{u^{1}}\widetilde{Y}_{x}=0, &  & \\
\widetilde{V}(0,x)=\left\{ \begin{array}{c}
\widetilde{V}_{a}\textrm{ if }x<0,\\
\widetilde{V}_{b}\textrm{ if }x>0,
\end{array}\right.  &  &
\end{eqnarray*}
and the solution of this Riemann problem at \( x/t=0 \) is denoted
by \( \widetilde{V^{*}} \).
\item An interface state is recovered by the inverse rotation \( V^{*}=M(\nu )^{-1}\widetilde{V^{*}} \).
\item The numerical flux is then set to
\begin{equation} \label{flux3d}
F(V_{a},V_{b},\nu )=(G(W^{*})\cdot \nu ,\min (u^{*}\cdot \nu
,0)(Y_{b}^{*}-Y_{a}^{*}))^{T}.
\end{equation}
It must be noted that \textsl{this numerical flux is
nonconservative on the $Y$ variables}, as in the Saurel-Abgrall
scheme that we presented in 1D cases. It can be proved, as in the
1D case, that the resulting scheme preserves constant pressure and
velocity states.
\end{enumerate}
In our case the scheme will reduce to a 2D one thanks to several
simplifications.

First, thanks to the rotation matrix \( J\left(\phi \right)=N(\cos
\phi ,-\sin \phi ,0) \) the axisymmetry condition reads\[
V^{n}_{k,l'}=J(\frac{2(l'-l)\pi }{P})V^{n}_{k,l}\: .\] Thus, the
velocity vector can be written \(
\overrightarrow{u}_{k,l}=(u_{k}\cos (\frac{2l\pi }{P}),u_{k}\sin
(\frac{2l\pi }{P}),v_{k})^{T} \) and the other variables do not
depend on \( l \). In this way, the scheme has only to be written
on the cells \( Q_{k,0} \): \[ \int
_{Q_{k,0}}(V^{n+1}_{k,0}-V^{n}_{k,0})+\tau \int _{\partial
Q_{k,0}}F(V^{n}_{k,0},J\left(\frac{2l'\pi
}{P}\right)V^{n}_{k',0},\nu )=0.\] Denoting \( V_{k,0} \) by \(
V_{k} \), this scheme then becomes\[ \int _{\Omega
_{k}}(V^{n+1}_{k}-V^{n}_{k})rdr\, dz+\tau \int _{\partial \Omega
_{k}}F(V^{n}_{k},V^{n}_{k'},\nu )rd\sigma +\]
\[
\frac{\tau P}{2\pi }\left( \int _{\Omega
_{k}}F(V^{n}_{k,0},J\left(\frac{2\pi }{P}\right)V^{n}_{k,0},\nu
)rdr\, dz+\int _{\Omega _{k}}F(V^{n}_{k,0},J\left(\frac{-2\pi
}{P}\right)V^{n}_{k,0},\nu )rdr\right) =0.\] It is then natural to
let \( P \) tend to \( \infty  \). In the Riemann problems of the
two last integrals only symmetric rarefaction waves occur. Thus
those two terms reduce to a pressure term
\begin{equation} \label{axischeme}
 \int _{\Omega
_{k}}(V^{n+1}_{k}-V^{n}_{k})rdr+\tau \int _{\partial \Omega
_{k}}F(V^{n}_{k},V^{n}_{k'},\nu )rd\sigma -\tau \int _{\Omega
_{k}}(0,p^{n}_{k},0\cdots 0)^{T}rdr=0.
\end{equation}
 We recover, of course, an
approximation of the axisymmetrical equations,
namely\begin{eqnarray}
(\rho r)_{t}+(\rho ur)_{x}+(\rho vr)_{z} & = & 0,\nonumber \\
(\rho ur)_{t}+(\rho u^{2}r+pr)_{x}+(\rho uvr)_{z} & = & pr,\nonumber \\
(\rho ur)_{t}+(\rho v u r)_{x}+(\rho v^{2}r+pr)_{z} & = & 0,\nonumber \\
(\rho Er)_{t}+((\rho E+p)ur)_{x}+((\rho E+p)vr)_{z} & = & 0,\nonumber \\
\alpha _{t}+u\alpha _{x}+v\alpha _{z} & = & 0,\nonumber \\
\beta _{t}+u\beta _{x}+v\beta _{z} & = & 0.\nonumber
\end{eqnarray}


One advantage of this approach is that the pressure source term
can be hand\-led explicitly without modifying the 3D CFL
condition. We have also avoided axisymmetrical source terms which
are singular on the axis of rotation. Finally, the resulting
scheme preserves constant pressure and velocity states. The
construction of a 2D scheme that preserves contact discontinuities
with discontinuous tangential velocity is proposed in Nkonga
\cite{nkonga02}.


\subsection{\label{num2}Boundary conditions}

For a boundary cell \( \Omega _{k} \), an artificial value \( V^{n}_{k'} \)
has to be defined for the part of \( \partial \Omega _{k} \) that
meets the boundary. For simplicity, suppose that the normal vector
is \( \nu =(1,0) \). We index by \( (i) \) (as {}``inside'') the
components of \( V^{n}_{k} \) and by \( (o) \) (as {}``outside'')
the (unknown) components of \( V^{n}_{k'} \). Several boundary conditions
can then be used:

\begin{itemize}
\item {}``Supersonic'' inlet:\[
V_{o}=\textrm{given state}.\]

\item {}``Supersonic'' outlet:\[
V_{o}=V_{i}.\]

\item Pressure imposed ({}``subsonic'' outlet). The outside state with
pressure \( p_{o} \) is linked to the inside state by a one-wave
(shock or rarefaction). With the notations of §\ref{global}, we find
\begin{eqnarray*}
\rho _{o}=1/H_{i}(p_{o}), &  & \\
u_{o}=u_{i}-X_{i}(p_{o}), &  & \\
v_{o}=v_{i}, &  & \\
\alpha _{o}=\alpha _{i}, &  & \\
\beta _{o}=\beta _{i}. &  &
\end{eqnarray*}

\item Pressure and density imposed ({}``subsonic'' inlet). Pressure \( p_{o} \)
and density \( \rho _{o} \) of the outside state are given. The nature
of the outside state (\( \alpha _{o} \), \( \beta _{o} \)) and the
tangential velocity (\( v_{o} \)) are also supposed to be known.
Here, the outside state is linked to the inside state by a one-wave
(shock or rarefaction) and a contact discontinuity. This permits to
compute the unknown normal velocity\[
u_{o}=u_{i}-X_{i}(p_{o}).\]

\item Mirror state. This boundary condition is used at a solid boundary. All the components
of the state \( W_{o} \) are the same as those of state \( W_{i} \)
but the normal velocity\[
u_{o}=-u_{i}.\]

\end{itemize}
It is important here to point out that the terminology
{}``subsonic'' or {}``supersonic'' has nothing to do with the true
nature of the flow at the boundary. It is only linked to what is
expected when \( W_{o}\simeq W_{i} \). Indeed, we can imagine
imposing a {}``supersonic'' inlet boundary condition and
observing, at this boundary a supersonic outflow! The resolution
of a Riemann problem ensures that the redundant information will
be forgotten, if necessary. For more details about this technique,
we refer to Dubois and his theory of partial Riemann problems
\cite{dubois01}.

\subsection{Optimisation of the CFL number\label{cfl}}

An important constraint in any finite volumes scheme is the
Courant-Friedrichs-Lewy condition. It expresses that, for any
finite volume $\Omega_k$ the time step $\tau$ must verify
\begin{equation}\label{cflcond}
  \tau < \frac{\text{surf}(\Omega_k)}{\text{length}(\partial
  \Omega_k)V^*}\quad,
\end{equation}
where $V^*$ is the maximal wave speed in the solution of all the
Riemann problems at the cell interfaces. For a classical Godunov
scheme on an unstructured mesh, it is proved in \cite{vila91},
that the CFL condition (\ref{cflcond}) implies the positivity of
the scheme (the density remains >0) and the fact that it is
entropic for any Lax entropy. We assume that this is still true
for the Saurel-Abgrall scheme. In the case where large and small
cells are mixed, the CFL condition on the small cells imposes the
global CFL condition. The mixing of large and small cells can be
imposed by the geometry (an example is given below) and not
necessarily by a required precision of the computation. In order
to reduce the computational cost it is possible to use several
time steps, a small one for the small cells and a bigger one for
the big cells. Then, several time steps are performed on the small
cells and less time steps are performed on the big cells. The time
reduction of the computation can be significant.

The time marching algorithm can be formalized as follow.

First, in a initialization procedure, an ideal time step is
computed for each cell $\Omega_k$. Let $\delta$ be the desired CFL
number (for example $\delta=0.7$). The local time step is defined
by
\begin{equation}
\tau _k  = \delta \frac{{\text{surf}(\Omega _k )}}
{{\text{length}(\partial \Omega _k )V^* }}\quad .
\end{equation}
The maximal time step in the mesh is noted $\tau_{\text{max}}$. In
the same way, the minimal time step is $\tau_{\text{min}}$. Let
$n_0$ be the smallest integer such that
$2^{n_0+1}\tau_{\text{min}}>\tau_{\text{max}}$
 We will then say that the cell $\Omega_k$ has a
\textsl{CFL level} of $j$ if
\begin{equation}
2^{j - 1} \tau_{\text{min}}  \leq \tau _k  < 2^j \tau
_{\text{min}} .
\end{equation}
Thus a CFL level of $1$ corresponds to the smallest cells and a
CFL level of $n_0$ corresponds to the biggest cells. We define
also a CFL level for the edges. An edge $E_l$ has a CFL level
which is the smallest CFL level of its two neighboring cells. In
order to advance by a global time step $2^{n_0}
\tau_{\text{min}}$, the algorithm is:
\begin{itemize}
  \item for all integer $j=1 \dots n_0$ do
  \begin{itemize}
    \item for all edges of CFL level $\leq j$ do
    \begin{itemize}
        \item compute the flux
        \item distribute it to the two neighboring cells
    \end{itemize}
    \item enddo
    \item \textsl{update only the cells of level $\leq j$}
    \end{itemize}
  \item enddo
\end{itemize}
In this way, the scheme remains conservative and stable. The gain
in computation time is of order $2^{n_0}$ if the number of small
cells is small.

\section{An industrial application}

\subsection{Description}
The industrial system that we wish to simulate is a gas generator
whose geometry and working order are indicated in Figure
\ref{gen1}. The gas generator is made of a combustion chamber
(top), a cooling chamber (middle part) and an evacuation pipe
(bottom, not represented). Only the working order in the cooling
chamber will be numerically simulated.

The cooling chamber and the evacuation pipe are separated by a metallic
membrane. This membrane can withstand a pressure of \( 40\textrm{ bar} \).
The cooling chamber is around \( 1\textrm{ m} \) high, it is itself
split into several chambers: a central one and a se\-condary one
which communicate through an intermediate chamber and two series of
holes.

At time \( t=0\textrm{ ms} \), the cooling chamber is filled with
motionless air and water at a pressure of \( 5\textrm{ bar} \).
Gas at high pressure (\( \sim 100\textrm{ bar} \)) and high
temperature (\( \sim 2500\textrm{ K} \)) are then produced in the
combustion chamber. They rush into the cooling chamber and impact
on the water surface causing a shock wave to propagate in the
water. When it reaches the bottom membrane, it smashes it. A part
of the water is then drained in the evacuation pipe, the rest of
the water is transferred into the secondary chamber where it is
finally re-injected through very small holes in the draining pipe
and mixed with the gas still rushing from the combustion chamber.
The entire process, from the beginning of combustion to the
beginning of the liquid re-injection has an approximate duration
of \( 50\textrm{ ms} \). This justifies the fact that we neglect
vaporization. Of course, for longer simulations, vaporization
should be taken into account.

As one can note in Figure \ref{gen1}, the real geometry of the
cooling chamber is not axisymmetrical due to the presence of
re-injection holes near the draining pipe and connection holes
between the several chambers. For the simulation we thus replace
these holes by slits of equivalent area. The simplified
axisymmetrical geometry is represented in Figure \ref{geomaxi}.
Here we plot the density. Red corresponds to density values of the
liquid (\( \sim 1000\textrm{ kg}/\textrm{m}^{3} \)) whereas black
corresponds to density values of the gas (\( \sim 10\textrm{
kg}/\textrm{m}^{3} \)).

\subsection{Results}
A part of the mesh is represented in Figure \ref{mesh}. It appears
that very small cells are necessary in the injection slits. The
CFL stability condition is thus very restrictive on these cells.
In order to avoid a overly long computation we use the algorithm
described in §\ref{cfl}. With this technique it is possible to
perform a 50 ms simulation in 5 hours CPU on an 1.4 GHz computer.
It should be noted that the exact Riemann solver requires at most
5 Newton iterations for convergence to within $10^{-10}p_0$ where
$p_0$ is the atmospheric pressure. Most of the zones in the fluid
flow do not require as many iterations - 2 or 3 only. In
conclusion, the classical Godunov scheme is not as expensive as is
so often proclaimed in the literature.



We then run the scheme presented in §\ref{num1} and §\ref{num2}.
Gravity is neglected. The boundary conditions and initial
conditions are depicted in Figure \ref{geomaxi}. The boundary
conditions are imposed according to the technique described in
§\ref{num2}:

\begin{itemize}
\item At the top entrance of the cooling chamber, which corresponds to the
exit of the combustion chamber, we impose pressure \( p(t) \) and
density \( \rho (t) \). The time evolution of these quantities is
determined by experimental measurements. The pressure increases
from \( 10 \) to \( 120\textrm{ bar} \) in several milliseconds.
We observe that the speed increases from \( 10 \) to \(
700\textrm{ m}/\textrm{s} \). The flow thus remains subsonic.
\item At the bottom exit, the boundary condition is, at first, a solid wall
con\-dition. When the pressure reaches \( 40\textrm{ bar } \)(this
occurs around \( 6\textrm{ ms} \)), the boundary condition is
changed into an outflow condition. We then impose an outside
pressure of \( 5\textrm{ bar} \). The pressure then drops
progressively from \( 40 \) to \( 5\textrm{ bar} \), and the
velocity increases from \( 0 \) to \( 1500\textrm{ m}/\textrm{s}
\). The flow is thus supersonic at the end of the computation.
\end{itemize}
Then, in Figures \ref{cavit}, \ref{working1} and \ref{working2}
iso-densities for several instants are plotted. We can point out
the following:
\begin{itemize}
\item At \( 10\textrm{ ms} \), the gas begins to push the water. The free
surface is slightly deformed. We observe a smoothing of the
density profiles due to the numerical diffusion of the interface.
But despite the first order scheme (and thanks to a quite fine
mesh..)., the interface is easily recognized.
\item The instant \( 15\textrm{ ms} \) is after the bursting of the membrane.
We can see that the central part of the water in the cooling chamber
has been drained into the evacuation pipe.
\item At instants \( 20\textrm{ ms} \) and \( 25\textrm{ ms} \), water
has already entered the secondary chamber, forming a jet against its
boundary. The jet is numerically diffused but still visible.
\item At instant \( 50\textrm{ ms} \), the jet has impacted on the free
surface in the secondary chamber. We observe that the re-injection
has started (small jet at the bottom).
\item In Figure \ref{cavit}, we  plot pressure in order to
demonstrate the appearance of negative values of the pressure.
These negative values appear in the bottom nozzle where a strong
drop of pressure is probably triggering cavitation.
\end{itemize}
\vspace{0.3cm} Measurements on a real gas generator were performed
at the ``Direction des Constructions Navales'' (DCN) in Toulon
(France). Excellent agreement is observed in the central part of
the cooling chamber. For example, the bursting time of the
separating membrane is predicted with an error of a few percent.
More precise comparisons with experiments have now to be performed
in the secondary chambers.

\vspace{0.3cm}
%{
% \begin{figure}

% \caption{Gas generator\label{gen1}}

% {\centering \subfigure[gas-generator (start)]
% {\resizebox*{!}{8cm}{\includegraphics{images_gen/generateur1.eps}}}
% \subfigure[gas-generator
% (middle)]{\resizebox*{!}{8cm}{\includegraphics{images_gen/generateur2.ps}}}
% \subfigure[gas-generator
% (end)]{\resizebox*{!}{8cm}{\includegraphics{images_gen/generateur3.ps}}}
% \par}
% \end{figure}

% \begin{figure}

% \caption{Boundaries\label{geomaxi}}

% {\centering \subfigure[axisymmetrical geometry and boundaries]
% {\resizebox*{!}{10cm}{\includegraphics{images_gen/boundariessansgaz.eps}}}
% \par}
% \end{figure}

% \begin{figure}

% \caption{Mesh\label{mesh}}

% {\centering \subfigure[Mesh (partial view)]{\resizebox*{!}{6cm}{\includegraphics{mesh.eps}}} \par}
% \end{figure}

% \begin{figure}

% \caption{Cavitation\label{cavit}}

% {\centering \subfigure[density (kg/m$^3$) at 10
% ms]{\resizebox*{!}{10cm}{\includegraphics{rho10ms.eps}}}
% \subfigure[pressure (bar) at 10
% ms]{\resizebox*{!}{10cm}{\includegraphics{pres10ms.eps}}} \par}
% \end{figure}

% \begin{figure}

% \caption{Density plots (\protect\( kg/m^{3}\protect
% \))\label{working1}}

% {\centering \subfigure[density at 15
% ms]{\resizebox*{!}{10cm}{\includegraphics{rho15ms.eps}}}
% \subfigure[density at 20
% ms]{\resizebox*{!}{10cm}{\includegraphics{rho20ms.eps}}} \par}

% \end{figure}

% \begin{figure}

% \caption{Density plots (\protect\( kg/m^{3}\protect
% \))\label{working2}}

% {\centering \subfigure[density at 25
% ms]{\resizebox*{!}{10cm}{\includegraphics{rho25ms.eps}}}
% \subfigure[density at 50
% ms]{\resizebox*{!}{10cm}{\includegraphics{rho50ms.eps}}}
% \par}
% \end{figure}
% \par}
% \vspace{0.3cm}

\section{Conclusion}
In this paper we have first recalled basic facts on compressible
multifluid flows. We have also carefully described the spurious
pressure oscillations phenomenon that arises in any conservative
Godunov scheme applied to multifluid flows.

We focused on two remedies to suppress these oscillations. The
first fix has been proposed by Saurel and Abgrall in
\cite{saurel99}. It is based on a nonconservative transport of the
pressure law coefficients and works only for a stiffened gas
pressure law. The second fix is, to our knowledge, new and is
based on a Lagrange plus projection scheme. In the projection step
we project the pressure instead of the pressure law coefficients.
The resulting scheme is more diffusive than the Saurel-Abgrall
scheme but also more general (it works for any pressure law).

Because in our application the stiffened gas law is sufficient we
decided to exploit the Saurel-Abgrall scheme in a 2D
axisymmetrical and complex geometry. We had then to deal with some
practical problems: negative pressures, axisymmetry, unstructured
meshes, boundary conditions, multi time steps.

After having solved these problems, we were in a situation to
present a useful industrial numerical simulation.\\

Future studies could now follow several directions:
\begin{itemize}
  \item The scheme should be extended to second order. We have not
  done it because in axisymmetrical cases, the classical second
  order extensions (as the MUSCL method of Van Leer \cite{vanleer79}) are
  surprisingly not straightforward.
  \item The second important aspect is to be able to deal with
  true cavitation, i.e. the vaporization of the liquid in a
  metastable state. Some progress has been obtained in the case
  of a liquid-vapor flow in \cite{helluy02}. The case of the
  three-phase flow with air, liquid and vapor is being studied. Some
  preliminary results can be found in \cite{barberon02}.
\end{itemize}

\section{Appendix \label{appendix}}

\subsection{Entropy and hyperbolicity}

We study here the hyperbolicity of (\ref{slc}), (\ref{presslaw}).
For this purpose, it is classical to introduce the specific
entropy\[ s^{0}=\frac{p+\pi }{\rho ^{\gamma }}=s^{0}(W).\] A
simple computation shows that if \( W \) is a regular solution of
(\ref{presslaw}), (\ref{slc}), then \( s^{0} \) satisfies the
advection equation\[ s^{0}_t+us^{0}_x=0.\] Thus, for any function
\( g \) of \( s^{0} \), $\gamma$ and $\pi$ an additional
conservation law is satisfied by \( \rho g(s^{0},\gamma,\pi ) \)
\[ (\rho g(s^{0},\gamma,\pi
))_t+(\rho u g(s^{0},\gamma,\pi ))_x=0.\] If \( W\rightarrow
S(W)=\rho g(s^{0},\gamma,\pi ) \) is convex, we get in this way
all the Lax entropies of system (\ref{slc}), (\ref{presslaw}). For
a proof of this result, we refer to the review paper of
\cite{harten98}. According to Mock's theorem \cite{mock80} the
convexity of \( S \) would then imply hyperbolicity. Here, we
prefer to carry out a more direct calculation. Always for regular
solutions, we set\[ Y=\left(
\begin{array}{c}
\rho \\
u\\
s^{0}\\
\gamma \\
\pi
\end{array}\right) .\]
We then have\[ Y_t+B(Y)Y_x=0,\] with \[ B(Y)=\left(
\begin{array}{ccccc}
u & \rho  & 0 & 0 & 0\\
\gamma \frac{p+\pi }{\rho ^{2}} & u & \rho ^{\gamma -1} &
 \frac{(p+\pi )\ln (\rho )}{\rho}& -\frac{1}{\rho }\\
0 & 0 & u & 0 &0\\
0 & 0 & 0 & u &0\\
0 & 0 & 0 & 0 & u
\end{array}\right) .\]
The eigenvalues of \( B \) are \( (u-c,u,u,u,u+c) \) with \(
c^{2}=\gamma \frac{p+\pi }{\rho }. \) Thus, if \( \rho >0 \), the
system is hyperbolic if and only if \[ p+\pi \geq 0.\]


\begin{rem}
\label{detente}When \( p \) tends to \( -\pi  \) keeping the
specific entropy constant, which is the case in a rarefaction
wave, we get:\[ \rho =C\, \left( p+\pi \right) ^{1/\gamma
}\rightarrow \, 0.\] Thus, the limiting case \( p=-\pi  \)
corresponds to a zero density \( \rho =0 \). This means that, in a
liquid, vacuum corresponds to a negative value of the pressure.
\end{rem}

\subsection{\label{global}Global resolution of the Riemann problem}

As in the case of gas dynamics for one fluid, the fields \( 1 \)
and \( 3 \), which correspond to the eigenvalues \( u-c \) and \(
u+c \) are genuinely non-linear whereas the field \( 2 \)
corresponding to the multiple eigenvalue \( u \) is linearly
degenerate (contact discontinuity).

At present, we have written several forms (conservative or not)
for the convection equations. All these forms are formally
equivalent. It is important to verify that they are correct also
for discontinuous solutions.

Let us consider a discontinuity propagating with velocity \(
\sigma  \). Indexes (a) and (b) will be relative to the two sides
of the discontinuity. Rankine-Hugoniot relations read, in this
case
\[ \sigma (W_{a}-W_{b})=F(W_{a})-F(W_{b}).\] Introducing the
relative velocity to the discontinuity and the specific volume \[
v=u-\sigma ,\qquad \tau =\frac{1}{\rho },\] the jump relations
become \begin{eqnarray*}
M=\rho _{a}v_{a} & = & \rho _{b}v_{b},\\
\rho _{a}v^{2}_{a}+p_{a} & = & \rho _{b}v^{2}_{b}+p_{b},\\
(\rho _{a}(\varepsilon _{a}+\frac{v^{2}_{a}}{2}+p_{a})v_{a} & = & (\rho _{b}(\varepsilon _{b}+\frac{v^{2}_{b}}{2})+p_{b})v_{b},\\
M\gamma _{a} & = & M\gamma _{b}, \\
M\pi _{a} & = & M\pi _{b}.
\end{eqnarray*}


The last two relations implie that \( \gamma  \) and $\pi$ can
jump only at the contact discontinuity (when \( M=0). \) On the
other hand, a simple computation shows that \( \gamma  \) and
$\pi$ are Riemann invariants for the fields \( 1 \) and \( 3 \).

\begin{rem}
\label{noncons1}These two properties imply that in genuinely
non-linear fields the coefficients \( \gamma  \) and \( \pi  \)
are constant. Outside the contact discontinuity, the computations
are thus identical to the case of a single fluid. These classical
computations can be found for example in the book of Godlewski and
Raviart \cite{raviart96}. They are briefly sketched below.
\end{rem}
\begin{rem}
\label{noncons2}We have also given a sense to the nonconservative
products in the last two transport equations in
(\ref{abgralnoncons}) because \( u \) and the pressure law
coefficients cannot present a simultaneous jump.
\end{rem}
Solving the Riemann problem means finding the weak entropy
solution of\[
\begin{array}{c}
W_t+F(W)_x=0,\\
W(0,x)=\left\{ \begin{array}{c}
W_{l}\textrm{ if }x<0,\\
W_{r}\textrm{ if }x>0.
\end{array}\right.
\end{array}\]
This solution is supposed to be self-similar\[
W(t,x)=R\left(\frac{x}{t},W_{l},W_{r}\right).\] It is made up of
constant states separated by shock waves, rarefaction waves or a
contact discontinuity. It is thus of the form \[ R(\xi
,W_{l},W_{r})=\left\{ \begin{array}{lll}
W_{l}&\textrm{ if }&\xi <\lambda ^{-}_{1},\\
W_{I}&\textrm{ if }&\lambda ^{+}_{1}<\xi <\lambda _{2},\\
W_{II}&\textrm{ if }&\lambda _{2}<\xi <\lambda ^{-}_{3},\\
W_{r}&\textrm{ if }&\lambda ^{+}_{3}<\xi ,
\end{array}\right. \]
where the unknowns are \( W_{I} \), \( W_{II} \) and the
velocities \( \lambda _{2} \), \( \lambda ^{\pm }_{i} \), \( i=1,3
\) which satisfy \( \lambda ^{-}_{1}\leq \lambda ^{+}_{1}<\lambda
_{2}<\lambda ^{-}_{3}\leq \lambda ^{+}_{3} \).

Furthermore, if \( \lambda ^{-}_{i}<\lambda ^{+}_{i} \) (resp. if
\( \lambda ^{-}_{i}=\lambda ^{+}_{i} \)) then the \( i \)-wave is
a rarefaction wave (resp. a shock of velocity \( \sigma =\lambda
^{-}_{i}=\lambda ^{+}_{i} \)). When the \( i \)-wave is a
rarefaction, the computation of \( W=R(\xi ,W_{l},W_{r}) \) , for
\( \lambda ^{-}_{i}<\xi <\lambda ^{+}_{i} \) is classically
carried out by expressing that the three Riemann invariants are
constant in the \( i \)-rarefaction (see \cite{raviart96}).

On the other hand, we have \( p_{I}=p_{II}=p^{\star }. \) If no
vacuum occurs, we can also write \( u_{I}=u_{II}=u^{\star }. \)
Moreover, from remark \ref{noncons1}, we have that \( \gamma
_{I}=\gamma _{l} \), \( \gamma _{II}=\gamma _{r} \), \( \pi
_{I}=\pi _{l} \), \( \pi _{II}=\pi _{r} \). It is then classical
to compute the 1- and 3-waves from the pressure \( p^{\star } \)
common to the two intermediate states \( W_{I} \) et \( W_{II} \).
For this purpose, we introduce the functions\[ h_{a}(p^{\star
})=\tau _{a}\frac{(\gamma _{a}+1)(p_{a}+\pi _{a})+(\gamma
_{a}-1)(p^{\star }+\pi _{a})}{(\gamma _{a}+1)(p^{\star }+\pi
_{a})+(\gamma _{a}-1)(p_{a}+\pi _{a})},\quad a=l\textrm{ or }r,\]
\[
\Phi _{a}(p^{\star })=\sqrt{(p^{\star }-p_{a})(\tau
_{a}-h_{a}(p^{\star }))},\]
\[
g_{a}(p^{\star })=\tau _{a}\left( \frac{p_{a}+\pi _{a}}{p^{\star
}+\pi _{a}}\right) ^{1/\gamma _{a}},\]
\[
\Psi _{a}(p^{\star })=\frac{2}{\gamma _{a}-1}(\tau _{a}\gamma
_{a}(p_{a}+\pi _{a}))^{1/2}\left( (\frac{p^{\star }+\pi
_{a}}{p_{a}+\pi _{a}})^{\frac{\gamma _{a}-1}{2\gamma
_{a}}}-1\right) ,\]
\[
X_{a}(p^{\star })=\left\{ \begin{array}{ccc}
\Phi _{a}(p^{\star }) & \textrm{if} & p^{\star }>p_{a},\\
\Psi _{a}(p^{\star }) & \textrm{if} & p^{\star }<p_{a},
\end{array}\right. \]
\[
H_{a}(p^{\star })=\left\{ \begin{array}{ccc}
h_{a}(p^{\star }) & \textrm{if} & p^{\star }>p_{a},\\
g_{a}(p^{\star }) & \textrm{if} & p^{\star }<p_{a}.
\end{array}\right. \]
We thus get\[ u_{I}=u_{l}-X_{l}(p^{\star }),\]
\[
u_{II}=u_{r}+X_{r}(p^{\star }),\]
\[
\tau _{I}=H_{l}(p^{\star }),\]
\[
\tau _{II}=H_{r}(p^{\star }),\] and the Riemann problem is solved
when \( p^{\star } \) is known.

If no vacuum region appears, the following theorem holds.

\begin{thm}
Let \( p_{0}=\min (\pi _{l},\pi _{r}) \). If \begin{equation}
\label{classique} u_{r}-u_{l}\leq -\left(
X_{l}(-p_{0})+X_{r}(-p_{0})\right) ,
\end{equation}
then the Riemann problem has a unique solution. The pressure \(
p^{\star }\geq -p_{0} \) is the unique solution of\[
u_{l}-X_{l}(p^{\star })=u_{r}+X_{r}(p^{\star }).\]

\end{thm}

This result is quite similar to the case of the Riemann problem
for a single fluid. For the proof we refer (for example) to
\cite{godunov79}, \cite{raviart96}.

When inequality (\ref{classique}) is not true, a vacuum has to be
introduced. This vacuum region appears in the fluid whose
coefficient \( \pi  \) is the smallest.

\begin{thm}
If \[ u_{r}-u_{l}>-\left( X_{l}(-p_{0})+X_{r}(-p_{0})\right) ,\]
the Riemann problem has still an entropy solution. For example, if
\( p_{0}=\pi _{l} \), then we have \( p^{\star }=-p_{0} \), \(
\rho _{I}=0 \), \( u^{\star }=u_{II}=u_{r}+X_{r}(p^{\star }) \).
\( u_{I}=u_{l}-X_{l}(p^{\star }) \), and, in general \( u_{I}\neq
u_{II} \).
\end{thm}

\begin{proof}

Suppose that \( p_{0}=\min (\pi _{l},\pi _{r})=\pi _{l} \). In the
two open sets \( x<u^{\star }t \) and \( x>u^{\star }t \), the
computation of the 1- and 3-wave curves is identical to the
mono-fluid case. Thus, \( W(t,x) \) is indeed an entropy solution
of the Riemann problem in these two open sets. It is then
sufficient to verify that, at the contact discontinuity \(
x/t=u^{\star } \), Rankine-Hugoniot jump relations are satisfied,
together with the entropy condition. The discontinuity velocity is
\( \sigma =u^{\star }. \) We thus have \( v_{II}=u^{\star }-\sigma
=0 \). Mass conservation \( \rho _{I}v_{I}=0=\rho _{II}v_{II} \)
is then satisfied. In the same way, \( \rho _{I}v_{I}^{2}+p^{\star
}=p^{\star }=\rho _{II}v_{II}^{2}+p^{\star } \). The jump relation
for the conservation of \( \rho \varphi  \) is also satisfied: \(
\rho _{I}v_{I}\varphi _{I}=0=\rho _{II}v_{II}\varphi _{II} \). For
the energy jump relation, we use the fact that the 1-wave is
necessarily a rarefaction because \( p^{\star }=-p_{0}\leq p_{l}
\). However, in a rarefaction, when \( p\rightarrow -\pi  \), then
\( \rho \varepsilon +\pi \rightarrow 0 \) (see remark
\ref{detente}) and we have \( (\rho _{I}\varepsilon _{I}+p^{\star
})v_{I}+\rho _{I}\frac{v^{3}_{I}}{2}=0=(\rho _{II}\varepsilon
_{II}+p^{\star })v_{II}+\rho _{II}\frac{v^{3}_{II}}{2} \).
Finally, the entropy inequality (which degenerates to an equality)
is also satisfied: \( \rho _{I}v_{I}s_{I}=0=\rho _{II}v_{II}s_{II}
\).
\end{proof}
%trsltx-end-ignoble
\bibliographystyle{plain}
\bibliography{axigen}
\end{document}
