%% LyX 2.3.7 created this file.  For more info, see http://www.lyx.org/.
%% Do not edit unless you really know what you are doing.
\documentclass[french]{article}
\usepackage[T1]{fontenc}
\usepackage[utf8]{inputenc}
\usepackage{babel}
\makeatletter
\addto\extrasfrench{%
   \providecommand{\og}{\leavevmode\flqq~}%
   \providecommand{\fg}{\ifdim\lastskip>\z@\unskip\fi~\frqq}%
}

%\newcommand{\oper}[1]{\sqrt{#1}}
\makeatother
\begin{document}

\title{Exemple de petit fichier \LaTeX}
\author{Philippe Helluy}
\maketitle
\begin{abstract}
Ceci est un exemple minimal de fichier \LaTeX{} en français pour tester
\texttt{trsltx}.
\end{abstract}
\section{Objectifs}
L'objectif du projet trsltx est de réaliser un outil pour traduire
des documents \LaTeX{} d'une langue à une autre en conservant la syntaxe.
%trsltx-split

\begin{itemize}
\item Le document peut contenir des formules en ligne comme $x=\sqrt{3}$
ou
\item hors ligne comme $$x=\sqrt{3}$$ ou
\begin{equation}
\mathbf{A}=\left[\begin{array}{cc}
1 & 2\\
2 & 1
\end{array}\right].\label{eq_matrice} 
\end{equation}
\item Considérons également la fraction
\begin{equation}
\frac{x}{y} \label{eq_fraction}
\end{equation}
\end{itemize}
Il faut pouvoir aussi faire des références aux étiquettes existante
comme (\ref{eq_fraction}) ou \cite{tutu}  sans casser les dépendances.
\section{Conclusion}
%trsltx-begin-ignore

Cette phrase n'est
 pas traduite.
%trsltx-end-ignore

L'objectif est-il rempli ? Si ce document compile et semble écrit
en anglais, peut-être !
%trsltx-begin-ignore

\begin{thebibliography}{1}
\bibitem{tutu}Un auteur, \emph{Un titre}, une revue, 2021.
\end{thebibliography}
%trsltx-end-ignore


\end{document}
