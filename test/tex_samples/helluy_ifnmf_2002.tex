\documentclass{fldauth}
\usepackage[T1]{fontenc}
\usepackage{graphics}
\usepackage{ae}
\usepackage{hyperref}

\usepackage{latexsym}
\usepackage{graphicx}

\theoremstyle{plain}
\newtheorem{thm}{Theorem}
\theoremstyle{plain}
\newtheorem{lem}{Lemma}
\theoremstyle{plain}
\newtheorem{prop}{Proposition}
\theoremstyle{plain}
\newtheorem{defn}{Definition}
\theoremstyle{plain}
\newtheorem{rem}{Remark}
\theoremstyle{plain}
\newtheorem{proof}{Proof}

\begin{document}
\FLD{1}{6}{00}{28}{00}

\runningheads{Coquel, Helluy, Schneider} {Second order entropy
scheme}

\title{Second order entropy diminishing scheme for the Euler equations}


\author{F. Coquel \affil{1}, P. Helluy \affil{2}\comma\corrauth, J. Schneider
\affil{2}}

\address{\affilnum{1} Laboratoire d'Analyse Num\'erique, Universit\'e Paris VI,
France, \affilnum{2} Laboratoire ANAM/MNC, Université de Toulon,
France}

\corraddr{\texttt{helluy@univ-tln.fr}, Laboratoire ANAM/MNC,
ISITV, BP 56, 83162 La Valette CEDEX, France}
%\footnotetext[2]{Please ensure that you use the most up to date class file, available from the FLD Home Page at\\
%\texttt{http://www.interscience.wiley.com/jpages/0271-2091/}
%%\href{http://www.interscience.wiley.com/jpages/0271-2091/}{\texttt{http://www.interscience.wiley.com/jpages/0271-2091/}}
%}

%\cgsn{Publishing Arts Research Council}{98--1846389}

\received{3 December 2002} \revised{10 December 2004}
\noaccepted{}



\begin{abstract}
In several papers of Bouchut, Bourdarias, Perthame and Coquel, Le
Floch (\cite{bouchut_pertham93}, \cite{coquel_lefloch95}...), a
general methodology has been developed to construct second order
finite volume schemes for hyperbolic systems of conservation laws
satisfying the entire family of entropy inequalities. This
approach is mainly based on the construction of an \textbf{entropy
diminishing projection}. Unfortunately, the explicit computation
of this projection is not always easy. In the first part of this
paper, we carry out this computation in the important case of the
Euler equations of gas dynamics. In the second part, we present
several numerical applications of the projection in the context of
finite volume schemes.
\end{abstract}

\keywords{Godunov scheme, entropy, second order}

\section*{Introduction}

In this paper, a methodology to build second order generalized
Godunov schemes satisfying all the entropy inequalities is
described. Our goal is to adapt the Godunov's original idea, which
leads to first order schemes, in order to obtain a second order
scheme. The construction is done completely in the case of the
Euler equations.

Let us recall that in the classical first order scheme, each time-step of the
time-marching procedure is made up of two stages:

\begin{enumerate}
\item Exact (or high order) resolution of the system of conservation laws starting
from the cell averages of the previous time-step. This resolution is performed
during a short time \( \tau  \) so that the Riemann problems of the two sides
of each cell do not interact.
\item The result of the previous stage is no longer constant in the cells. A projection
is then realized in order to recover a piecewise constant approximation. For
a first order scheme, the only conservative projection is cell averaging.
\end{enumerate}
This description is of course theoretical. In practice, a numerical flux can
be associated to this procedure so that the two stages become transparent in
the implementation of the scheme.

The first scheme designed in this way is described by Godunov in
\cite{godunov59}. We shall call in the sequel this kind of scheme,
based on an exact or approximate Riemann solver, a \textbf{Godunov
scheme}. This terminology is developed in the review paper of
Harten, Lax and Van Leer \cite{harten_lax_vanleer_83}. Most of the
conservative schemes can be put in this framework as the HLL
scheme \cite{harten_lax_vanleer_83}, HLLC scheme of Toro, Spruce
and Speares \cite{toro94}, Roe scheme \cite{roe81}, Engquist-Osher
scheme \cite{osher81}, kinetic schemes \cite{deshpande88},
\cite{perthame90}, \cite{mazet89}, and many others...

The original Godunov scheme has the important property that it
respects on the discrete level the decrease of all the Lax
entropies of the approximated hyperbolic system. This is an
important property which is closely linked to the stability of the
scheme and with the possibility of non-physical waves if the
scheme is not entropic. For example, the HLL, HLLC and
Engquist-Osher schemes are entropic whereas the original Roe
scheme is not entropic. With a simple modification, it can be made
entropic (for example, see the fix of Harten and Hyman
\cite{harten83}).

Another important feature emerging from the construction of the
Godunov schemes is that the cell values are high order
approximations of the exact mean values. They are also second
order approximations of the exact values in the center of the
cells. Unfortunately, because they are only first order
approximations on the cell sides, an error of order one is
committed in the computation of the numerical flux.

These remarks have been used to improve the precision of Godunov schemes. The
most widespread improvement consists of reconstructing a more precise approximation
of the solution by using cell averages and Taylor expansions. This reconstruction
allows one to compute more accurately the fluxes, but has to be corrected by
a limitation procedure in order to avoid oscillations or non physical values.

The main criterion in these methods is a TVD (Total Variation
Diminishing) criterion. Many works deal with second order Godunov
schemes (we can cite for example the works of Van Leer
\cite{vanleer79} or Harten \cite{harten_osher82}) which give good
numerical results in various practical computations. Anyway, since
the work of Rauch \cite{rauch86} it is well known that the TVD
criterions are inadequate in the theoretical study of systems of
conservation laws in higher dimensions. There is no hope to prove
convergence results in a general framework with these
limitations\footnote{It should also be noticed that the TVD
criterion is limited to cartesian grids in the multi-d case, even
for scalar equations.}.

In this paper, another approach is followed, which is closer to
the Godunov's original idea. On the one hand, we shall consider
the resolution of a generalized Riemann problem with piecewise
linear (instead of constant) initial data. The solution will then
have to be projected back onto a set of piecewise linear
functions. On the other hand, in the projection step, we
substitute for the TVD criterion a mean entropy decreasing
criterion which seems more adequate for systems. An important
feature is that this \textbf{stability criterion will be verified
for the whole family of entropies of the system}. Such an approach
has been initiated for scalar conservation laws by Bouchut,
Bourdarias and Perthame in \cite{bouchut_pertham93} and has been
extended to 2x2 systems of conservation laws and to the 3x3 system
of Lagrangian gas dynamics by Coquel and Le Floch in
\cite{coquel_lefloch95}. Our paper deals with the case of the
Euler system.

The first part is devoted to a mathematical study of the
projection stage. The problem can be stated in the following way:
the three conservative variables, after the Riemann problem step
are, generally speaking, piecewise regular functions in each cell
of the finite volume method. The goal of the projection is to
recover linear variables in the cells in order to pursue the
resolution. Because the mean values of these variables are given
by the conservation property, it is sufficient to compute three
slopes. The projection can then be seen as the operator which
gives these slopes from the original piecewise regular
conservatives variables in the cells. It is known that this
operator can not be linear. Indeed, it can be proved that a
general linear projection (such as the classical $L^2$ projection)
will not always respect the positivity of the projected density
and pressure. It is of course linked to the fact that there is no
second order linear three point scheme which is also TVD for the
scalar conservation laws (see \cite{raviart}). Because we are
working on systems we will require that the projection operator is
entropic. By entropic, we mean that the mean value in each cell of
the entropy of the projection is smaller than the mean entropy of
the original conservative variables. It is very interesting that
\textbf{this entropy diminishing property gives also the
positivity of the projected variables}. This result is proved in
remark \ref{posit}.

Here, we are able to prove the existence of a non-linear
projection for the Euler equations and provide explicit formulas.
The construction is based on several ingredients:
\begin{itemize}
  \item First, we recall the theory of second order entropic
  projections of \cite{bouchut_pertham93}. This theory is based on the definition of an
  approximate derivative (Definition \ref{defN}) and a sufficient
  condition in order to have an entropic projection (Theorem
  \ref{theorem:perth}). A very nice feature of this theory is
  that, when applied to a scalar conservation law, the entropic
  projection is very similar to a classical minmod limiter.
  \item This sufficient condition is not exploitable as is for the Euler system, and it does
  not give easily an explicit formula for the slopes of the
  projection. Thus we propose to seek the projection as the
  composition of two non-linear operators. To build the first operator, we
  work with special variables (density, momentum and a particular
  entropy) for which the sufficient condition of Theorem \ref{theorem:perth}
  can be computed. We then reduce the
  first step of the projection to the solution of \textbf{a
  triangular set of inequalities} (\ref{inegener1}), (\ref{inegener2})
  and (\ref{inegentrop}) for the three slopes.
  \item The first operator is not conservative for the energy. The
  mean value of the projected energy has thus to be corrected. We
  prove that the correction is still entropy diminishing and thus does
  not impede the whole process. This fact can also be used by
 to built simple entropic schemes for the Euler
  equations based on the intermediate solution of the entropy
  conservation law.
  \item Finally, we provide formulas that explicitly
  solve the fundamental triangular set of inequalities for the slopes.
  These formulas are summed up in Theorem \ref{maintheo}.
\end{itemize}


The second part of the paper is then devoted to several numerical
experiments with the previously constructed projection. One
approach could have been to develop a generalized Riemann solver
as in \cite{coquel_lefloch95} for the gas dynamics equations in
Lagrangian coordinates. The theory of the generalized Riemann
problem can be found, for example, in the papers of Bourgeade, Le
Floch, Raviart \cite{bourgeade89}, Ben Artzi, Falcovitz
\cite{benartzi86}. It is also sketched in the book of Godlewski
and Raviart \cite{raviart}. The problem is that the implementation
of this solver is very complex. So we prefer to follow here much
simpler approaches. We present two kinds of numerical results:
\begin{itemize}
    \item The first are obtained with a second order kinetic scheme.
    We use the Boltzmann kinetic interpretation of the Euler equations in
    order to construct an approximate second order Riemann solver.
    For the computations, we choose a compactly
supported Maxwellian proposed by Perthame \cite{perthame90}. But
instead of taking a piecewise constant density function in the
microscopic scheme, we take a piecewise linear density function.
The free transport Boltzmann equation can then be solved exactly.
A second order approximate Riemann solver is then obtained by
taking the moments of the resulting microscopic solution. After
the Boltzmann step, the solution is piecewise polynomial in each
cell and can be computed explicitly. We are then in a position to
apply the results of the first part of the paper and provide some
numerical experiments which validate the whole procedure.
    \item The previous construction is quite complicated. So we
    propose also another approach. We do not try to construct a
    high order Riemann solver but reconstruct a high order
    approximation of the solution from its cell averages. The cell
    averages are computed with a standard Godunov flux. Without
    limitations, the scheme would present oscillations. We then
    apply the entropic projection to the reconstruction in order
    to evaluate the damping of the oscillations.
\end{itemize}

We conclude the paper with some comments about possible extensions
and applications of the entropic projection.

\section{Entropy solution of Euler equations}


\subsection{Euler equations}

In the present paper we focus our attention on the numerical approximation of
the discontinuous solutions of the Euler system for polytropic gases. With standard
notation, this system reads:
\begin{eqnarray}
\partial _{t}w+\partial _{x}f(w) & = & 0,\quad t>0,\quad x\in R,\label{slc} \\
w(0,x) & = & w_{0}(x),\nonumber
\end{eqnarray}
 where
\[
w=\left( \begin{array}{c}
\rho \\
\rho u\\
E
\end{array}\right) ,\quad f(w)=\left( \begin{array}{c}
\rho u\\
\rho u^{2}+p\\
(E+p)u
\end{array}\right) ,\]
and
\begin{eqnarray*}
E & = & \rho \varepsilon +\frac{(\rho u)^{2}}{2\rho },\\
p & = & (\gamma -1)\rho \varepsilon ,\quad \gamma >1.
\end{eqnarray*}
It is well known that this system is strictly hyperbolic over the phase space
\[
\Omega =\{w\in R^{3},\rho >0,\rho u\in R,E-\frac{(\rho u)^{2}}{2\rho }>0\}.\]



\subsection{Entropy condition}

Generally, the weak solution of (\ref{slc}) is not unique. Assuming that theoretical
results for scalar conservation laws extend to systems, an entropy condition
has to be added to the Euler system in order to recover uniqueness.

\begin{defn}
Let \( U(w) \) be a convex function of \( w \), and let \( F(w) \) be a function
such that the following additional conservation law holds whenever \( w \)
is a  strong solution
\[
\partial _{t}U(w)+\partial _{x}F(w)=0.\]
 \( (U,F) \) is then called a Lax entropy pair for the system (\ref{slc})
(\( U \) is the entropy and \( F \) the entropy flux).
\begin{defn}
A weak solution \( w(t,x) \) of (\ref{slc}) is an entropy solution if and
only if for every entropy pair \( (U,F) \), the following inequality holds
\[
\partial _{t}U(w)+\partial _{x}F(w)\leq 0,\quad t>0,\quad x\in R.\]

\end{defn}
The previous notions have been introduced by Lax in \cite{lax72} for general
systems of conservation laws. The mathematical existence and uniqueness of the
entropy solution is still an open problem. In this paper this well-posedness
result is supposed to hold.

The practical computation of the Lax entropies for the Euler system is given
for example (among many others) in the PhD thesis of Croisille \cite{croisille91}
or in the book of Raviart and Godlewski \cite{raviart96}. It appears that a
family of regular entropies can be constructed in the following way: let us
introduce the following quantity
\[
S=(\rho \varepsilon )^{1/\gamma }.\]
 It can be checked that \( (-S,-uS) \) is a Lax entropy pair of the Euler system.
We now consider the family \( (U,uU) \) defined by
\[
U=\rho G(\frac{S}{\rho }),\]
 where \( G \) is a \( C^{2} \) function on \( R^{+*} \) such that
\[
G'<0\textrm{ and }G''>0.\]
 It is proved in \cite{croisille91} that this construction gives all the \( C^{2} \)
entropies of the Euler system of the form \( (U,F) \) with \( F=uU \). Another
expression of the entropies is given by
\[
U=\rho H(\frac{\rho }{S}),\]
 where \( H \) is a \( C^{2} \) function on \( R^{+*} \) such that
\[
H'(x)+xH''(x)>0,\quad x\in R^{+*}.\]



\section{Generalized Godunov scheme}

\end{defn}
The framework of generalized second order Godunov schemes approximating (\ref{slc})
can now be stated.

A time-step \( \tau >0 \), and a space-step \( h>0 \) being given, we consider
the following discretization in time \( t_{n}=n\tau  \) \( (n\geq 0) \) and
space \( x_{i}=ih \) \( (i\in \mathbf{Z}) \). The cell \( i \) is defined
by \( C_{i}=\left] x_{i-1/2},x_{i+1/2}\right[  \).

We start with an approximation of \( w(t_{n},x) \) at time \( t_{n} \)
\[
w^{n}(x)\approx w(t_{n},x),\]
 which is supposed to be piecewise linear with
\[
w^{n}(x)=w_{i}^{n}+s_{i}^{n}(x-x_{i}),\quad \textrm{ }\forall x\in C_{i},\]
 (in the classical Godunov scheme \( s_{n}^{i}=0 \)).

\begin{rem}
\label{changevar}To be more general, we can also consider another set of variables
given by a regular transformation:
\[
\phi =\phi (w),\quad w\in \Omega ,\]
and suppose that \( \phi  \) is piecewise linear:
\[
w^{n}(x)=\phi ^{-1}(\phi _{i}^{n}+d_{i}^{n}(x-x_{i})),\quad \textrm{ }\forall x\in C_{i}.\]
This approach will be developed in the next part.
\end{rem}
In order to compute a new \( w^{n+1}(x) \), two steps are then performed:

\begin{enumerate}
\item First, the Euler equations and the entropy conditions are exactly solved during
a short period of time \( \tau : \)
\begin{eqnarray*}
 &  & \partial _{t}v+\partial _{x}f(v)=0\quad t>0,\quad x\in R,\\
 &  & \partial _{t}U(v)+\partial _{x}F(v)\leq 0,\\
 &  & v(0,x)=w^{n}(x),
\end{eqnarray*}
 and this is done for all Lax entropy pairs \( (U,F) \). The solution \( v(\tau ,x) \)
will be denoted by \( w^{n+1,-}(x) \) and is generally no longer linear in
each cell.
\item \( w^{n+1,-}(x) \) has to be approximated again by a piecewise linear function.
Let \( P^{1}_{h} \) be the space of piecewise (per cell) linear functions,
then we look for a (possibly non linear) operator \( \Pi :L^{\infty }(R)\rightarrow P^{1}_{h} \)
such that
\[
w^{n+1}(x)=\Pi \, w^{n+1,-}(x).\]
 \( \Pi  \) is also supposed to be a projection in the sense that \( \Pi w=w \)
whenever \( w\in P^{1}_{h} \).
\end{enumerate}
In the case of the classical first order Godunov scheme the projection consists
simply of cell-averaging
\[
\Pi w(x)=\overline{w}=\frac{1}{h}\int _{C_{i}}w,\quad x\in C_{i},\quad i\in Z.\]


If the projection is an approximation of order 2 (with respect to \( h \))
of \( w^{n+1,-} \), then the resulting scheme is of order 2 in space (and of
order 1 in time) in the sense of consistency. But good precision will be achieved
only under some stability conditions. We shall now describe a possible set of
stability conditions.


\section{entropy diminishing projection}

As mentioned in the introduction, we wish to generalize the Godunov idea to
second order schemes. The first step consists of solving exactly (or at least
with given precision) generalized Riemann problems between two cells starting
from a piecewise linear approximation of the solution. Actually, in order to
be more general, we shall consider in each cell the case of an initial condition
in a finite dimensional manifold \( M \) of regular functions defined on the
cell. It is clear that after the exact resolution step, the solution is, generally
speaking, no longer in \( M \). A regular approximation in the cell has to
be recovered, with the decrease of entropies.

A second order Riemann solver will be constructed later on, and in this part
we focus only on the projection step of the Godunov method.


\subsection{Second order entropy diminishing projections}

Because we are looking for a projection defined locally, i.e. on each cell \( C_{i} \),
we can suppose, for simplicity, that \( C_{i}=]0,1[ \). The projection problem
can then be presented in the following abstract setting (introduced in \cite{bouchut_pertham93},
\cite{coquel_lefloch95}):

\begin{defn}
Let \( M \) be a finite dimension manifold included in \( C^{\infty }([0,1],\Omega ) \).
Let \( w=(\rho ,\rho u,E)^{T} \) be an element of \( L^{\infty }\cap BV([0,1],\Omega ) \),
and let \( \Pi  \) be a (generally non-linear) projection from \( L^{\infty }\cap BV([0,1],\Omega ) \)
into \( M \) which satisfies the following property: for all Lax entropy pairs
\( (U,F) \),
\begin{equation}
\label{entronum}
\int ^{1}_{0}U(\Pi w(x))dx\leq \int ^{1}_{0}U(w(x))dx.
\end{equation}
 Such an operator will be denoted as an \textbf{entropy diminishing projection}
on \( M \).
\end{defn}
Let us note that condition (\ref{entronum}) implies a conservation property.
Indeed, when applied to the following degenerate entropies:
\[
U(w)=\pm \rho ,\quad U(w)=\pm \rho u,\quad U(w)=\pm E,\]
we get
\[
\overline{w}:=\int ^{1}_{0}\Pi w(x)dx=\int ^{1}_{0}w(x)dx.\]


In the classical Godunov method, \( M \) is the set of constant states which
are in \( \Omega  \) and the projection is given by
\[
\Pi w=\overline{w}.\]
Inequality (\ref{entronum}) then holds thanks to the Jensen inequality. The
problem here is that we are looking for an approximation of the function \( w \)
which is of order two when rescaled to an interval of size \( h \). Let us
define precisely this second order property.

\begin{defn}
Let \( \Pi  \) be an entropy diminishing projection from \( L^{\infty }\cap BV([0,1],\Omega ) \)
into \( M \). Let \( \widetilde{w} \) be an element of \( C^{2}([0,h],\Omega ) \)
and \( w(x)=\widetilde{w}(xh),\quad x\in [0,1]. \) After projection of \( w \)
we can define
\[
\widetilde{\Pi }\widetilde{w}(t):=\Pi w(\frac{t}{h}),\quad t\in [0,h],\]
then \( \Pi  \) is said to be a \textbf{second order entropy diminishing projection}
if
\begin{equation}
\label{ordre2}
\left| \widetilde{\Pi }\widetilde{w}(t)-\widetilde{w}(t)\right| \leq \lambda h^{2},
\end{equation}
where \( \lambda  \) is a constant which depends only on the \( C^{2} \) norm
of \( \widetilde{w} \).
\end{defn}
As we have seen before, in the case of first order Godunov scheme, the first
order entropy diminishing projection is necessarily unique. On the other hand,
there exist many second order entropy diminishing projections with various choices
of manifolds \( M \). In the sequel, we recall a general framework which allows
the construction of second order entropy diminishing projections. We then use
special features of the Euler equations to build a simple explicit projection
for this hyperbolic system.


\subsection{Pseudo-derivative}

We are going to derive a sufficient condition on \( \Pi  \) in order to satisfy
the very important inequality (\ref{entronum}). It is based on the following
notion of (what we decided to call) pseudo-derivative which was introduced first
by Bouchut, Bourdarias and Perthame in \cite{bouchut_pertham93} in the case
of a scalar equation. It was also studied by Coquel and Le Floch in \cite{coquel_lefloch95}
for systems.

\begin{defn}\label{defN}
Let \( w \) be an element of \( L^{\infty }\cap BV([0,1],\Omega ) \) . The
\textbf{pseudo-derivative} of \( w \) is the continuous function \( N(w)\in C(]0,1[,R^{3}) \)
defined by
\[
N(w)(x)=\frac{2}{1-x}\int _{x}^{1}w(t)dt-\frac{2}{x}\int _{0}^{x}w(t)dt.\]

\end{defn}
The linear operator \( N \) satisfies the following properties:

\begin{itemize}
\item For all constants \( C \), \( N(w+C)=N(w). \) \item If \( w
\) is regular enough, \( N(w)(x)=\int _{0}^{1}\varphi (x,t)w'(t)dt
\). Where the graph of \( t\rightarrow \varphi(x,t)  \) is given
below

{\par\centering \unitlength=1.5cm \begin{picture}(6,5)
\put(0,0){\vector(0,4){5}} \put(1,0){\circle*{.1}} \put(1,-.3){0}
\put(4,0){\circle*{.1}} \put(4,-.3){1} \put(1,0){\line(1,3){1}}
\put(4,0){\line(-2,3){2}} \put(2,3){\circle*{.1}}
\put(2,3){\line(-1,0){2}} \put(2,3){\line(0,-1){3}} \put(2,-.3){x}
\put(5,-.3){t} \put(.3,4){\(\varphi(x,t)\)} \put(-.3,3){\(2\)}
\put(0,0){\vector(1,0){5}}
\end{picture}\par}

\end{itemize}
\vspace{1cm}

\begin{itemize}
\item If \( w \) is regular, for all \( x\in ]0,1[ \) there is a \( \theta (x)\in ]0,1[ \)
such that
\begin{equation}
\label{deriv}
N(w)(x)=w^{\prime }(\theta (x)).
\end{equation}

\item If \( \overline{w}=\int _{0}^{1}w(t)dt=0 \) then
\[
-\frac{x(1-x)}{2}N(w)(x)=\int _{0}^{x}w(t)dt.\]

\item Finally, if \( \overline{w}=\int _{0}^{1}w(t)dt=0 \), the following integration
by parts formula holds
\[
\int _{0}^{1}v(t)w(t)dt=\int _{0}^{1}\frac{t(1-t)}{2}N(w)(t)v^{\prime }(t)dt.\]

\end{itemize}
A sufficient condition for the decrease of the mean entropy can then be stated.

\begin{thm}
\label{theorem:perth}Let \( \Pi  \) be a projection from \( L^{\infty }\cap BV([0,1],\Omega ) \)
into \( M \). If for all \( w\in L^{\infty }\cap BV([0,1],\Omega ) \), all
\( x\in ]0,1[ \), and all entropies \( U \),
\begin{equation}
\label{cs}
d_{x}\Pi w\cdot U''(\Pi w)(x)(N(w)(x)-d_{x}\Pi w)\geq 0,
\end{equation}
 then, \( \Pi  \) is an entropy diminishing projection.
\end{thm}
\begin{proof}
This is a consequence of the basic convex inequality
\[
\int _{0}^{1}U(w)-U(\Pi w)\geq \int _{0}^{1}U'(\Pi w)(w-\Pi w)\]

\[
=\int _{0}^{1}U'(\Pi w)(\overline{w}-\overline{\Pi w})\]
\begin{equation}
\label{ineg}
+\int _{0}^{1}\frac{t(1-t)}{2}d_{x}\Pi w\cdot U''(\Pi w)(t)(N(w)(t)-d_{x}\Pi w)dt.
\end{equation}

\end{proof}
At this stage, condition (\ref{cs}) would provide us with a practical definition
of \( \Pi w \) if we were studying a scalar equation instead of the Euler system
(see \cite{bouchut_pertham93}). Using the convexity of entropies \( U \),
the previous condition reduces to
\begin{equation}
\label{minmod:scal} d_{x}\Pi w \cdot (N(w)(x)-d_{x}\Pi w)\geq 0.
\end{equation}


Define the minmod function of a set of reals as
\[
\textrm{minmod }E=\left| \begin{array}{l}
\textrm{inf }E\textrm{ if }E\subset R^{+}\\
\textrm{sup }E\textrm{ if }E\subset R^{-}\\
0\textrm{ otherwise}
\end{array}\right. ,\]


then a possible choice for \( \Pi w \) is the linear function \( \Pi w(x)=\overline{w}+d_{x}\Pi w(x-1/2) \)
where the real number \( d_{x}\Pi w \) is defined by
\begin{equation}
\label{minmodscal}
d_{x}\Pi w=\textrm{minmod}\{N(w)(x),\quad x\in ]0,1[\}.
\end{equation}


This defines a second order entropy diminishing projection thanks to property
(\ref{deriv}).

In the case of a system of conservation laws, \( U''(\Pi w)(x) \) are positive
definite matrices which define a metric for all values of \( x\in ]0,1[ \).
If that metric was constant then we may define \( d_{x}\Pi w \) as the vector
in the convex envelope \( N^{\circ } \) of \( \{N(w)(x),\; x\in ]0,1[\} \)
that minimizes the norm associated to the metric i.e.:
\begin{equation}
d_{x}\Pi w^{T}\cdot U''\cdot d_{x}\Pi w=\min _{v\in N^{\circ }}v^{T}\cdot U''\cdot v\quad .
\end{equation}
 But since the metric is changing with \( x \), the practical use of condition
(\ref{cs}) is not at all straightforward. The inequality can however be used
to establish the existence of a solution (see \cite{coquel_lefloch95}) and
to derive a family of projections that partially (that is to say up to a certain
order \( n \)) fulfill it (see for an example \cite{coquel_lefloch95}).

In the next section, we adapt the previous method in order to build an exact
second order entropy diminishing projection in the case of the Euler equations.


\section{Explicit computation of the projection: Euler case\label{expl}}

The starting point of the explicit projection procedure is an element
\[
w_{0}=(\rho _{0},q_{0}=\rho _{0}u_{0},E_{0})^{T}\in L^{\infty }\cap BV([0,1],\Omega ).\]
In the sequel , we will denote by \( \Pi  \) a second order entropy diminishing
projection. The projection \( w_{2}=\Pi w_{0} \) thus belongs to \( C^{\infty }([0,1],\Omega ). \)
For practical reasons, \( w_{2} \) will be obtained from an intermediate state,
noted \( w_{1}=(\rho _{1},q_{1}=\rho _{1}u_{1},E_{1})^{T}\in C^{\infty }([0,1],\Omega ) \).
In other words, \( w_{1}=\Pi _{1}w_{0} \), and our explicit projection \( \Pi  \)
is the composition of two operators \( \Pi _{1} \) and \( \Pi _{2} \):
\[
\Pi =\Pi _{2}\circ \Pi _{1}.\]


Let us first describe operator \( \Pi _{1}. \) We have seen in the introduction
that all the \( C^{2} \) entropies of the Euler system have a rather simple
form when expressed as a function of the quantity \( S=(\rho \varepsilon )^{1/\gamma } \)
. We thus define
\[
S_{0}=(E_{0}-\frac{1}{2}\rho _{0}u_{0}^{2})^{1/\gamma }.\]
On the other hand, we wish \( w_{1} \) to be \( C^{\infty } \). A simple possibility
is to set
\begin{equation}
\label{defS1}
S_{1}=\overline{S_{0}}+dS_{1}(x-1/2)
\end{equation}
\begin{equation}
\label{defrho1}
\rho _{1}=\overline{\rho _{0}}+d\rho _{1}(x-1/2)
\end{equation}
\begin{equation}
\label{defq1}
q_{1}=\overline{q_{0}}+dq_{1}(x-1/2)
\end{equation}
where, as before, \( \overline{z} \) is a notation for the mean value of the
quantity \( z \) on the interval \( [0,1]. \) The real numbers \( dS_{1}, \)
\( d\rho _{1}, \) \( dq_{1} \) are to be guessed (and will be defined explicitly
in the sequel).With our choice, the quantities \( S_{1}, \) \( \rho _{1}, \)
\( q_{1} \) are thus linear in the cell \( [0,1] \) and have the same mean
values as \( S_{0}, \) \( \rho _{0}, \) \( q_{0} \).

Setting
\begin{equation}
\label{defw1}
w_{1}=(\rho _{1},q_{1},E_{1}=\frac{1}{2}\rho _{1}u^{2}_{1}+S_{1}^{\gamma }),
\end{equation}
 it is clear that \( w_{1} \) is regular but not linear in the cell. It is
also clear that, since \( \overline{S_{1}}=\overline{S_{0}} \), the first projection
operator \( \Pi _{1} \) is not conservative in the sense that, in general,
density and impulsion are conserved but not energy. Thus
\[
\overline{q_{1}}=\overline{q_{0}},\quad \overline{\rho _{1}}=\overline{\rho _{0}},\quad \overline{E_{1}}\neq \overline{E_{0}}.\]
This fact justifies the necessity of a correction \( w_{2}=\Pi _{2}w_{1} \),
if this approach is employed, in order to recover the conservation of energy.

The simplest way to obtain \( w_{2} \) is to correct the energy by taking
\[
\rho _{2}=\rho _{1},\]
\[
q_{2}=\rho _{2}u_{2}=q_{1},\]
\[
E_{2}=E_{1}-\overline{E_{1}}+\overline{E_{0}},\]


But with this choice, the operator \( w_{0}\rightarrow w_{2} \) is not a projection.
For this reason, we prefer to take
\begin{equation}
\label{correcE}
E_{2}=\frac{1}{2}\rho _{1}u^{2}_{1}+(K+S_{1})^{\gamma },
\end{equation}
where the constant \( K \) should be chosen in such a way that
\begin{equation}
\label{conservE}
\overline{E_{2}}=\overline{E_{0}}.
\end{equation}
We suggest then to define the approximation manifold \( M \) by:

\begin{defn}
\label{defM}\( M \) is the set of vector functions \( w=x\rightarrow (\rho (x),q(x),\frac{q(x)^{2}}{2\rho (x)}+S(x)^{\gamma }) \)
defined on \( [0,1] \) such that the functions \( \rho  \), \( q \), and
\( S \) are linear and such that \( \forall x\in [0,1],\quad w(x)\in \Omega  \)
(which is equivalent to \( \forall x\in [0,1],\quad \rho >0,\quad S>0 \)).
\end{defn}
In other words, according to the notations of Remark
\ref{changevar} (page \pageref{changevar}), we have simply set \(
\phi =(\rho ,\rho u,S) \).

The problem is now to verify that the correction (\ref{correcE}) can be done
with the decrease of any entropy \( U \):
\[
\int U(\rho _{2},q_{2},E_{2})\leq \int U(\rho _{1},q_{1},E_{1}).\]


But the convexity of \( U \) with respect to the conservative variables gives
\[
\int U(\rho _{1},q_{1},E_{1})\geq \int U(\rho _{2},q_{2},E_{2})+\int \frac{\partial U}{\partial E}(w_{2})(E_{1}-E_{2}).\]
On the other hand, we know that
\[
\frac{\partial U}{\partial E}(w)=\rho G'(\frac{S}{\rho })\frac{\partial S}{\partial E},\]
\[
\textrm{and }S=(E-\frac{q^{2}}{2\rho })^{1/\gamma }\Rightarrow \frac{\partial S}{\partial E}=\frac{1}{\gamma }S^{1-\gamma }>0,\]
 which implies that \( \frac{\partial U}{\partial E}<0 \) on the phase space
\( \Omega  \). It is thus natural to require that, \( E_{1}-E_{0}<0 \) on
the cell, or, in an equivalent way, that
\[
K\geq 0.\]
Actually, it will be more convenient (and it is possible) to ask a little bit
more. We know that the energy is the sum of two terms:
\[
E=\frac{q^{2}}{2\rho }+S^{\gamma },\]
 and we will require the decrease of these two terms separately:
\begin{equation}
\label{inegener1}
\int \frac{q_{1}^{2}}{2\rho _{1}}\leq \int \frac{q_{0}^{2}}{2\rho _{0}},
\end{equation}
and
\begin{equation}
\label{inegener2}
\int C(S_{1})\leq \int C(S_{0})\quad \forall C\in C^{2}(R,R),\textrm{ convex}.
\end{equation}
In order that the global process \( \Pi =\Pi _{2}\circ \Pi _{1} \)be entropy
decreasing, it is finally sufficient to require the last family of inequalities
\begin{equation}
\label{inegentrop}
\int \rho _{1}H(\frac{S_{1}}{\rho _{1}})dx\leq \int \rho _{0}H(\frac{S_{0}}{\rho _{0}}),
\end{equation}
where \( H \) is any \( C^{2} \) function on \( R^{+*} \) such that
\[
H'(x)+xH''(x)>0,\quad x\in R^{+*}.\]


\begin{rem}\label{posit}
Inequality (\ref{inegener2}) automatically enforces \( S_{1}>0 \) on \( ]0,1[ \).
Indeed, consider a \( C^{2}(R,R) \) convex function \( C \) satisfying
\begin{eqnarray*}
C(y)=0 & \textrm{when} & y\geq 0,\\
C(y)>0 & \textrm{when} & y<0,
\end{eqnarray*}
then, because \( S_{0}>0 \), we have \( 0\leq \int C(S_{1})\leq \int C(S_{0})=0 \)
and then \( S_{1}>0 \). In the same way, inequality (\ref{inegentrop}) implies
that \( \rho _{1}>0 \) if we suppose that \( S_{0}>0 \), \( S_{1}>0 \) and
\( \rho _{0}>0 \). These properties are very important for the numerical approximation
of the Euler equations. They ensure that the resulting second order entropy
diminishing scheme is also a positive scheme.
\end{rem}
We sum up the previous construction in the following proposition:

\begin{prop}
Let \( w_{0}=(\rho _{0},q_{0}=\rho _{0}u_{0},E_{0})^{T}\in
L^{\infty }\cap BV([0,1],\Omega ) \), and let \( w_{1}=(\rho
_{1},q_{1}=\rho _{1}u_{1},E_{1}=\frac{1}{2}\rho
_{1}u_{1}^{2}+S_{1}^{\gamma })^{T}\in C^{\infty }([0,1],\Omega )
\) where \( \rho _{1} \), \( q_{1} \), and \( S_{1} \) are linear
functions defined by (\ref{defS1}), (\ref{defrho1}),
(\ref{defq1}). Suppose that the slopes of \( \rho _{1} \), \(
q_{1} \), and \( S_{1} \) are computed in order to satisfy
(\ref{inegener1}), (\ref{inegener2}), (\ref{inegentrop}). Let
finally \( w_{2}=(\rho _{1},q_{1},E_{2})^{T}\in C^{\infty
}([0,1],\Omega ) \), where \( E_{2} \) is corrected according to
(\ref{correcE}) and (\ref{conservE}). Then, the non-linear
operator \( \Pi :w_{0}\rightarrow w_{2} \) is an entropy
diminishing projection on the manifold \( M \) of definition
\ref{defM} (on page \pageref{defM}).
\end{prop}
The construction of an entropy diminishing second order projection is now reduced
to the computation of three slopes \( dS_{1} \), \( d\rho _{1} \), and \( dq_{1} \)
satisfying inequalities (\ref{inegener1}), (\ref{inegener2}) and (\ref{inegentrop}).
The important fact is that we have now to solve a \textbf{triangular} set of
inequalities. In practice, we will have first to find a \( dS_{1} \) satisfying
(\ref{inegener2}). Then our choice of \( S_{1} \) will be inserted in (\ref{inegentrop})
in order to get \( d\rho _{1} \). Then, knowing \( S_{1} \) and \( \rho _{1} \),
we are in a position to solve (\ref{inegener1}) and compute \( q_{1} \).

The computation of \( S_{1} \) is quite simple if we apply the computations
leading to formula (\ref{minmodscal}). We thus propose
\begin{equation}
\label{minmodS}
dS_{1}=\textrm{minmod}\left\{ N(S_{0})(x),\quad x\in [0,1]\right\} .
\end{equation}
For the evaluation of \( d\rho _{1} \) a longer computation has to be performed.

\begin{lem}
Let \( \rho _{0} \) and \( S_{0} \) be two positive functions in \( L^{\infty }\cap BV([0,1],R) \).
Let \( \rho _{1} \) and \( S_{1} \) be two positive and linear functions defined
on \( [0,1] \):
\begin{equation}
\label{linear:S}
S_{1}=\overline{S_{0}}+dS_{1}(x-1/2),
\end{equation}
\begin{equation}
\label{linear:rho}
\rho _{1}=\overline{\rho _{0}}+d\rho _{1}(x-1/2).
\end{equation}
Then, a sufficient condition on the slope \( d\rho _{1} \) in order to have
(\ref{inegentrop}) is that the following inequality holds on \( [0,1] \):
\begin{equation}
\label{ineg:rho}
\alpha \cdot \left( \frac{S_{1}}{\overline{S_{0}}}(\overline{S_{0}}N(\rho _{0})-\overline{\rho _{0}}N(S_{0}))-\frac{\overline{S_{0}}+N(S_{0})(x-1/2)}{\overline{S_{0}}}\alpha \right) \geq 0,
\end{equation}
where \( \alpha  \) is defined by \( \alpha =\overline{S_{0}}d\rho _{1}-\overline{\rho _{0}}dS_{1} \).
\end{lem}
\begin{proof}
It is easy to check that \( (\rho ,S)\rightarrow \rho H(S/\rho )
\) is a convex function. Thus, using Theorem \ref{theorem:perth},
a sufficient condition in order to have (\ref{inegentrop}) is
\[
(d\rho _{1},dS_{1})\left( \begin{array}{cc}
1 & -\frac{\rho _{1}}{S_{1}}\\
-\frac{\rho _{1}}{S_{1}} & \left( \frac{\rho _{1}}{S_{1}}\right) ^{2}
\end{array}\right) \left( \begin{array}{c}
N(\rho _{0})-d\rho _{1}\\
N(S_{0})-dS_{1}
\end{array}\right) \geq 0,\]
or
\[
(S_{1}d\rho _{1}-\rho _{1}dS_{1})(S_{1}N(\rho _{0})-\rho _{1}N(S_{0})-(S_{1}d\rho _{1}-\rho _{1}dS_{1}))\geq 0.\]
using (\ref{linear:S}) and (\ref{linear:rho}), and thanks to basic computations,
we find (\ref{ineg:rho}).
\end{proof}
The slope \( dq_{1} \) is computed in the same spirit.

\begin{lem}
Let \( q_{0} \) and \( \rho _{0} \) be two functions in \( L^{\infty }\cap BV([0,1],R) \),
with \( \rho _{0}>0 \). Let \( q_{1} \) and \( \rho _{1} \) be two linear
functions defined on \( [0,1] \), with \( \rho _{1}>0 \).
\begin{equation}
\label{linear:q}
q_{1}=\overline{q_{0}}+dq_{1}(x-1/2),
\end{equation}
\[
\rho _{1}=\overline{\rho _{0}}+d\rho _{1}(x-1/2).\]
Then, a sufficient condition on the slope \( dq_{1} \) in order to have (\ref{inegener1})
is that the following inequality holds on \( [0,1] \)
\begin{equation}
\label{ineg:q}
\beta \cdot \left( \frac{\rho _{1}}{\overline{\rho _{0}}}(\overline{\rho _{0}}N(q_{0})-\overline{q_{0}}N(\rho _{0}))-\frac{\overline{\rho _{0}}+N(\rho _{0})(x-1/2)}{\overline{\rho _{0}}}\beta \right) \geq 0,
\end{equation}
where \( \beta  \) is defined by \( \beta =(\overline{\rho _{0}}dq_{1}-\overline{q_{0}}d\rho _{1}) \).
\end{lem}
The next theorem is devoted to the practical computation of the slopes \( dS_{1},d\rho _{1},dq_{1} \)
in order to solve inequalities (\ref{inegener1}), (\ref{inegener2}), (\ref{inegentrop})
and in order to maintain the second order property.

\begin{thm}\label{maintheo}
With the previous notations, consider:
\[
\alpha =\overline{S_{0}}d\rho _{1}-\overline{\rho _{0}}dS_{1},\quad \beta =(\overline{\rho _{0}}dq_{1}-\overline{q_{0}}d\rho _{1}),\]
\[
g_{\alpha }(x)=\frac{\overline{S_{0}}+N(S_{0})(x)(x-1/2)}{\overline{S_{0}}},\quad g_{\beta }(x)=\frac{\overline{\rho _{0}}+N(\rho _{0})(x)(x-1/2)}{\overline{\rho _{0}}},\]

\[
h_{\alpha }(x)=\frac{S_{1}}{\overline{S_{0}}}(\overline{S_{0}}N(\rho _{0})(x)-\overline{\rho _{0}}N(S_{0})(x))\quad \textrm{and}\quad h_{\beta }(x)=\frac{\rho _{1}}{\overline{\rho _{0}}}(\overline{\rho _{0}}N(q_{0})(x)-\overline{q_{0}}N(\rho _{0})(x)).\]
Then, if \( dS_{1} \) is defined by (\ref{minmodS}) and if \( \alpha  \)
and \( \beta  \) are defined by
\[
\frac{1}{\alpha }=\left\{ \begin{array}{c}
\max _{x\in [0,1]}\frac{g_{\alpha }(x)}{h_{\alpha }(x)}\textrm{ if }h_{\alpha }>0\\
\min _{x\in [0,1]}\frac{g_{\alpha }(x)}{h_{\alpha }(x)}\textrm{ if }h_{\alpha }<0\\
\infty \textrm{ otherwise}
\end{array}\right. ,\]
\[
\frac{1}{\beta }=\left\{ \begin{array}{c}
\max _{x\in [0,1]}\frac{g_{\beta }(x)}{h_{\beta }(x)}\textrm{ if }h_{\beta }>0\\
\min _{x\in [0,1]}\frac{g_{\beta }(x)}{h_{\beta }(x)}\textrm{ if }h_{\beta }<0\\
\infty \textrm{ otherwise}
\end{array}\right. ,\]
then, \( \rho _{1} \), \( S_{1} \), \( q_{1} \) are second order approximations
of respectively \( \rho _{0} \), \( S_{0} \), \( q_{0} \), and inequalities
(\ref{inegener2}), (\ref{ineg:rho}), (\ref{ineg:q}) are satisfied. In other
words, with this choice of slopes, \( \Pi  \) is a second order entropy diminishing
projection.
\end{thm}
\begin{proof}
The case of \( dS_{1} \) has already been treated before. The inequality to
solve for \( \alpha  \) is
\[
\alpha (h_{\alpha }(x)-\alpha g_{\alpha }(x))\geq 0.\]
\( \alpha =0 \) is a solution, but in order to achieve second order, \( \alpha  \)
has to be a first order approximation of \( h_{\alpha }(x) \). Thus, we can
assume that \( \alpha  \) and \( h_{\alpha } \) have the same signs. Then,
suppose that \( h_{\alpha }(x)>0 \) for all \( x\in [0,1] \). If \( \alpha >0 \),
we have to solve \( 1/\alpha \geq g_{\alpha }(x)/h_{\alpha }(x), \) \( x\in [0,1] \).
We thus take \( 1/\alpha =\textrm{max}_{[0,1]}\frac{g_{\alpha }(x)}{h_{\alpha }(x)}>0 \).
In the same way, if \( h_{\alpha }(x)<0 \) we choose \( 1/\alpha =\textrm{min}_{[0,1]}\frac{g_{\alpha }(x)}{h_{\alpha }(x)}<0 \).
Finally, if \( h_{\alpha } \) takes positive and negative values on \( [0,1] \),
we take \( \alpha =0 \). The computation of \( \beta  \) is completely similar.
\end{proof}

\section{First application: a second order Boltzmann scheme}


\subsection{An approximate Riemann solver based on a kinetic interpretation}

In this part, a possible computation of \( w^{n+1,-} \) is
presented. For practical reasons, we shall not use a classical
Riemann solver, but instead, a resolution step based on the
kinetic interpretation of the Euler equations. The kinetic
interpretation that we will describe below has been proposed by
Perthame in \cite{perthame90}. His model has the property of being
entropy diminishing for one particular entropy (see
\cite{perthame90}) and not necessarily for the other entropies.
For simplicity, it will be exposed in the case \( \gamma =3 \),
where the computations are easier, but can be extented to other
values of \( \gamma  \). Actually, it would be better to use the
more recent model of Bouchut as described into \cite{bouchut99},
which is entropy diminishing for all entropies.

Let us describe the original kinetic interpretation of Perthame.
For this purpose, we introduce the following function (called in
the sequel generalized Maxwellian).
\begin{equation}\label{maxwell} M_{w}(v)=\frac{\rho
}{2\sqrt{6\varepsilon }}Y(\frac{v-u}{\sqrt{6\varepsilon }}),
\end{equation}
 where:

\begin{itemize}
\item \( Y \) is the characteristic function of \( [-1,1] \):
\[
Y(t)=\left\{ \begin{array}{c}
1\textrm{ if }\left| t\right| <1\\
0\textrm{ if }\left| t\right| \geq 1
\end{array}\right. .\]

\item \( v \) is the microscopic speed.
\item \( w=\left( \begin{array}{c}
\rho \\
\rho u\\
E
\end{array}\right)  \) is the macroscopic state of the gas.
\end{itemize}
It is straightforward to check that
\[
\int _{v=-\infty }^{v=+\infty }M_{w}(v)\left( \begin{array}{c}
1\\
v\\
v^{2}/2
\end{array}\right) dv=w.\]
 Let us now set
\[
m(t,x,v)=M_{w(t,x)}(v).\]
 It is also easy to check that
\begin{equation}
\label{eulboltz} \int _{v=-\infty }^{v=+\infty }\partial
_{t}m+v\partial _{x}m=\partial _{t}w+\partial _{x}f(w).
\end{equation}
 This fact leads to a numerical resolution known as a Boltzmann scheme.
 Many papers have been published on this subject.
Without pretending to be exhaustive, we can cite for example the
works of Deshpande \cite{deshpande88}, Bourdel, Mazet, Delorme and
Croisille \cite{mazet89}. All these works are based on the
physical Maxwellian. Each time step of a Boltzmann scheme is made
of two substeps.

\begin{itemize}
\item Free transport step: \( w^{n}(x) \) being given, the following evolution problem
is solved during a time step:
\begin{eqnarray}
\partial _{t}g+v\partial _{x}g=0,\quad 0\leq t\leq \tau , &  & \nonumber \\
g(0,x,v)=M_{w^{n}(x)}(v).\label{freeboltz}
\end{eqnarray}

\item Collision step: from the solution at time \( t=\tau  \), a new macroscopic
state is recovered
\[
w^{n+1,-}(x)=\int _{v=-\infty }^{v=+\infty }g(\tau ,x,v)K(v)dv.\]
 where
\[
K(v)=\left( \begin{array}{c}
1\\
v\\
v^{2}/2
\end{array}\right) \]

\end{itemize}
It is important here to point out that the Boltzmann solver is only an approximate
Riemann solver. This is due to the fact that during the free transport procedure
(\ref{freeboltz}) the equality
\[
m(t,x,v)=g(t,x,v)\]
 does not hold (the microscopic state is not a generalized Maxwellian). The collision step
acts as a relaxation of the microscopic state to a Maxwellian state. Of course
this solver tends to an exact solver when \( \tau \rightarrow 0 \).

On the other hand, despite its simplicity, this approach leads to very tedious
computations for \( w^{n+1,-} \). Therefore, we propose the following simplification:
in the free transport step (\ref{freeboltz}), we replace the initial condition
by its linear interpolation on each cell
\[
f(0,x,v)=\frac{x-x_{i-1/2}}{h}M_{w_{i,r}}(v)+\frac{x_{i+1/2}-x}{h}M_{w_{i,l}}(v)\textrm{ for }x\in C_{i}\quad ,\]
 where
\[
w_{i,r}=w_{i}^{n}+\frac{h}{2}s_{i}^{n}\textrm{ and
}w_{i,l}=w_{i}^{n}-\frac{h}{2}s_{i}^{n}\quad .\] It must be
noticed that this approximation is conservative but unfortunately,
it is also necessarily entropy increasing because we replace a
Maxwellian state (which corresponds to a minimum of entropy) at
each point of the cell by a linear approximation.


\subsection{Computation of the approximate second-order Riemann solver}

Let \( x\in C_{i} \), we have:
\[
w^{n+1,-}(x)=\int _{v=-\infty }^{+\infty }f(0,x-v\tau ,v)K(v)dv.\]
 Consider then the speed \( v_{\min } \) (respectively \( v_{\max } \)) at
which a particle in \( x_{i+1/2} \) (respectively \( x_{i-1/2}) \) at time
\( t=0 \) reaches \( x \) at time \( t=\tau  \)
\begin{eqnarray*}
v_{\min } & = & \frac{x-x_{i+1/2}}{\tau }<0,\\
v_{\max } & = & \frac{x-x_{i-1/2}}{\tau }>0.
\end{eqnarray*}
 The time step \( \tau  \) is supposed to be smaller than \( hv^{*}, \) where
\( v^{*} \) is greater than the biggest support of all the
generalized Maxwellians plus the maximal speed of the flow. Thanks
to this CFL condition, the computation of \( w^{n+1,-}(x) \) can
then be split into three parts: a contribution from the left cell
\( C_{i-1} \), the right cell \( C_{i+1} \), and the middle cell
\( C_{i} \):
\[
w^{n+1,-}(x)=A_{l}+A_{m}+A_{r}\]
\begin{eqnarray*}
A_{l} & = & \int _{v=v_{\max }}^{+\infty }\left[ \frac{x-v\tau -x_{i-3/2}}{h}M_{w_{i-1,r}}(v)+\frac{x_{i-1/2}-x+v\tau }{h}M_{w_{i-1,l}}(v)\right] K(v)dv,\\
A_{r} & = & \int _{v=-\infty }^{v_{\min }}\left[ \frac{x-v\tau -x_{i+1/2}}{h}M_{w_{i+1,r}}(v)+\frac{x_{i+3/2}-x+v\tau }{h}M_{w_{i+1,l}}(v)\right] K(v)dv,\\
A_{m} & = & \int _{v=v_{\min }}^{v_{\max }}\left[ \frac{x-v\tau -x_{i-1/2}}{h}M_{w_{i,r}}(v)+\frac{x_{i+1/2}-x+v\tau }{h}M_{w_{i,l}}(v)\right] K(v)dv.
\end{eqnarray*}
 A more detailed expression of \( A_{l},A_{r},A_{m} \) is given in Appendix 1.
It can be checked that \( w^{n+1,-} \) is piecewise polynomial of degree \( \leq 4 \)
on each cell. The number of pieces is \( \leq 22. \)


\subsection{Numerical results}

For the numerical results that are presented in this section, we
decided to compute exactly \( w^{n+1,-} \) given by the kinetic
Riemann solver. This is done thanks to a C++ class of piecewise
polynomial functions. It appears then that \( S^{n+1,-} \) is not,
in general, piecewise polynomial. We thus had to construct a
piecewise polynomial approximation \( \tilde{S} \) of \( S^{n+1,-}
\). This is done with a Tchebychev interpolation with 3 points on
each piece of regularity of \( w^{n+1,-} \). Then \( N(\rho ) \),
\( N(q) \), and \( N(\tilde{S}) \) can be computed exactly. The
final limitation procedure has been performed numerically with a
sampling of the functions we wanted to maximize or minimize on
each cell.

We were able to verify that our projection operator acts at least
as a classical minmod limiter. Indeed, the numerical results
appear to be precise and present no oscillation on a 200 cells
mesh.

We tested the scheme on the classical case of a shock tube problem
with the following initial data:
\begin{center}
\begin{tabular}{|c|c|c|}
\hline Variable & Left state & Right state \\
\hline  Density & $\rho_L=2$ & $\rho_R=1$\\
\hline  Velocity & $u_L=0$ & $u_R=0$\\
\hline Pressure & $p_L=8$ & $p_R=2$\\
\hline
\end{tabular}
\end{center}

 In the Figures \ref{figrho}, \ref{figvit} and \ref{figpres} a comparison
is made between the first and the second order schemes with \( 100
\) or \( 200 \) cells. The results are given at time \( t=1 \). In
the case of the first order solution, the real (piecewise
constant) mathematical solution has been plotted. Density,
velocity and pressure are successively presented.
\begin{figure}[b]
\resizebox*{15cm}{!}{\rotatebox{0}{\includegraphics{rho100.eps}}}
\resizebox*{15cm}{!}{\rotatebox{0}{\includegraphics{rho200.eps}}}


\caption{Density. First or second order. 100 or 200 cells.\label{figrho}}
\end{figure}

\begin{figure}[b]
\resizebox*{15cm}{!}{\rotatebox{0}{\includegraphics{vit100.eps}}}
\resizebox*{15cm}{!}{\rotatebox{0}{\includegraphics{vit200.eps}}}


\caption{Velocity. First or Second order. 100 or 200
cells.\label{figvit}}
\end{figure}

\begin{figure}[b]
\resizebox*{15cm}{!}{\rotatebox{0}{\includegraphics{pres100.eps}}}
\resizebox*{15cm}{!}{\rotatebox{0}{\includegraphics{pres200.eps}}}


\caption{Pressure. First or second order. 100 or 200 cells.\label{figpres}}
\end{figure}

With a mesh refinement, oscillations start to appear. The
phenomenon can be observed with a classical MUSCL scheme. This is
due to the fact that the scheme is only first order in time.

The complexity of the scheme leaded us to abandon this approach.
The next section is thus devoted to a more tractable application
of the entropic projection.

\section{Second application: Mean value approach}

In this part, we envisage a simpler approach than the kinetic
approach. We first build a polynomial interpolation of the
solution, starting from cell averages, as in Harten's ENO schemes
\cite{harten89}. We use an interpolation without upwinding or
limitation in order to evaluate the effect of the entropic
projection.

More precisely, suppose that we know the mean value $w_i^n$ of the
solution  at time $n$ in the cell $C_i$. A second order extension
of the Godunov scheme reads
\begin{equation}\label{godu2}
w_i^{n + 1}  = w_i^n  - \frac{\tau } {h}(f_{i + 1/2}^{n + 1/2}  -
f_{i - 1/2}^{n + 1/2} ).
\end{equation}
The flux at interface $i+1/2$ and time ${n+1/2}$ is of the
form\begin{equation}\label{flux2} f_{i + 1/2}^{n + 1/2}  =
f(R(w_{i + 1/2, - }^{n + 1/2} ,w_{i + 1/2, + }^{n + 1/2} )).
\end{equation}
The quantity $R(w_L,w_R)$ denotes the solution of the first order
exact Riemann problem at the interface between $w_L$ and $w_R$.
The quantity $w_{i + 1/2, - }$ is the value of the reconstructed
solution in the cell $i$ at time $n+1/2$ and at the interface
$i+1/2$. The choice $w_{i + 1/2, - }=w_i^n$ corresponds to the
classical first order scheme. We focus now on the cell $i$. For
simplicity, we suppose that $C_i=]0,1[$.

The first step is to construct a high order approximation of $w$
from the cell averages. We thus suppose, with the notations of
Section \ref{expl}, that $\rho_0$, $q_0$ and $S_0$ are second
order polynomials in $]0,1[$. We also suppose that the
reconstruction is conservative \begin{equation} \int_j^{j + 1}
{w_0 (x)dx} = w_{i + j}^n ,\quad j =  - 1,0,1,
\end{equation}
with
\begin{equation}
w_0 (x) = (\rho _0 (x),q_0 (x),\frac{{q_0^2 (x)}} {{2\rho _0 (x)}}
+ S_0 (x)^\gamma  )^T .
\end{equation}
It is known that the resulting interpolation is not necessarily
positive for the density and the pressure, even if all the mean
values are positive. So we propose in Appendix 2 a simple
procedure to correct the interpolation, if necessary.

Because $\rho_0$, $q_0$ and $S_0$ are now second order
polynomials, the computations described in Theorem
\ref{theorem:perth} become almost explicit. It is then possible to
compute the limited variables $\rho_1$, $q_1$ and $S_1$, the
energy correction described in (\ref{correcE}) and then the fluxes
at cell interfaces. The second order in time is achieved with the
Hancock method. It uses the space slope estimate to compute a time
derivative estimate thanks to the conservation laws:
$w_t=-f(w)_x$. The time derivative estimate permits then to
compute the approximation of $w$ at time $n+1/2$.

We have tested the scheme on the Riemann problem whose data are
given in tables \ref{table1} and \ref{table2}. This case is chosen
in the book of Toro \cite{toro94}. It presents a sonic rarefaction
wave and a shock. We compare in Figures \ref{figrho2},
\ref{figvit2} and \ref{figpres2} the results of the
reconstruction-limitation scheme with a standard MUSCL-Hancock
scheme (described in \cite{toro94}). We observe a slight
improvement of the precision in the contact discontinuity, but
also small overshoots and undershoots in the right shock. Without
the correction described in Appendix 2 the computation would have
not ended. It is necessary only on one cell in the first time step
and only for the reconstruction of the density $\rho$ near the
contact discontinuity.
\begin{table}[h]
\begin{center}
\begin{tabular}{|c|c|c|}
\hline Variable & Left state & Right state \\
\hline  Density & $\rho_L=1$ & $\rho_R=0.125$\\
\hline  Velocity & $u_L=0.75$ & $u_R=0$\\
\hline Pressure & $p_L=1$ & $p_R=0.1$\\
\hline
\end{tabular}
\caption{\label{table1}\textit{Data of the Riemann problem.}}
\end{center}
\end{table}
\begin{table}[h]
\begin{center}
\begin{tabular}{|c|c|}
\hline Interval & $]-1/2,1/2[$  \\
\hline  Number of cells & 200 \\
\hline  CFL & $0.8$ \\
\hline Final time & $t=0.2$ \\
\hline $\gamma$ & $1.4$ \\
\hline
\end{tabular}
\caption{\label{table2}\textit{Computation characteristics.}}
\end{center}
\end{table}

\begin{figure}[b]
\resizebox*{15cm}{!}{\rotatebox{0}{\includegraphics{comprho.eps}}}
\caption{Comparison entropic scheme-MUSCL Hancock
scheme.\label{figrho2}}
\end{figure}
\begin{figure}[b]
\resizebox*{15cm}{!}{\rotatebox{0}{\includegraphics{compv.eps}}}
\caption{Comparison entropic scheme-MUSCL Hancock
scheme.\label{figvit2}}
\end{figure}
\begin{figure}[b]
\resizebox*{15cm}{!}{\rotatebox{0}{\includegraphics{comppr.eps}}}
\caption{Comparison entropic scheme-MUSCL Hancock
scheme.\label{figpres2}}
\end{figure}


In the next numerical experiment, we evaluate the rate of
convergence in the $L^1$ norm (for density $\rho$) of the
reconstruction-limitation scheme and compared it in
 Figure
\ref{rate} and Table \ref{ratet} with the standard MUSCL-Hancock
scheme. The rate is computed for a simple contact discontinuity
whose values are given in Table \ref{discon}. The characteristics
of this computation are summed up in Table \ref{resum}. We observe
that the projection scheme is more precise than the MUSCL scheme,
but the asymptotic rates of convergence seems to be approximately
the same. Recall that, for a simple contact discontinuity, the
standard MUSCL "second order" scheme has a convergence rate of
$2/3 \simeq 0.66666$.

\begin{figure}[b]
\resizebox*{15cm}{!}{\rotatebox{0}{\includegraphics{errvf.eps}}}
\caption{Comparison entropic scheme-MUSCL Hancock scheme. Rate of
convergence for a contact discontinuity.\label{rate}}
\end{figure}

\begin{table}[h]
\begin{center}
\begin{tabular}{|c|c|c|c|c|}
\hline $\ln(h)$ & $\ln($error$)$ MUSCL& $\ln($error$)$ proj. & rate MUSCL & rate proj.\\

\hline



-4.60517 &-4.10716 &-4.36288 & -&- \\ \hline

-5.29832 &-4.55118 &-4.84035 & 0.64058&0.68884 \\ \hline

-5.99146 &-4.99951 &-5.31210 & 0.64681&0.68060 \\ \hline

-6.68461 &-5.45112 &-5.77950 & 0.65153&0.67431 \\ \hline

-7.37776 &-5.90507 &-6.24700 & 0.65491&0.67446 \\ \hline
\end{tabular}
\caption{\label{ratet}\textit{Convergence test, contact
discontinuity.}}
\end{center}
\end{table}


\begin{table}[h]
\begin{center}
\begin{tabular}{|c|c|c|}
\hline Variable & Left state & Right state \\
\hline  Density & $\rho_L=2$ & $\rho_R=1$\\
\hline  Velocity & $u_L=1$ & $u_R=1$\\
\hline Pressure & $p_L=1$ & $p_R=1$\\
\hline
\end{tabular}
\caption{\label{discon}\textit{Contact discontinuity.}}
\end{center}
\end{table}

\begin{table}[h]
\begin{center}
\begin{tabular}{|c|c|}
\hline Interval & $]-1/2,1/2[$  \\
\hline  Number of cells & 200 to 1600 \\
\hline  CFL & $\simeq 0.45$ \\
\hline Final time & $t=0.2$ \\
\hline $\gamma$ & $1.4$ \\
\hline
\end{tabular}
\caption{\label{resum}\textit{Computation characteristics,
contact.}}
\end{center}
\end{table}

In the last numerical experiment, we evaluate the rate of
convergence in the $L^1$ norm (for density $\rho$) of the
reconstruction-limitation scheme and compared it in
 Figure
\ref{rate2} and Table \ref{ratet2} with the standard MUSCL-Hancock
scheme. The rate is computed for a simple shock whose values are
given in Table \ref{discon2}. The characteristics of this
computation are summed up in Table \ref{resum2}. We observe that
the projection scheme is more precise than the MUSCL scheme, but
the asymptotic rates of convergence seem to be approximately the
same. Recall that, for a simple shock, the standard MUSCL "second
order" scheme has a convergence rate of $1$.

\begin{figure}[b]
\resizebox*{15cm}{!}{\rotatebox{0}{\includegraphics{errvf2.eps}}}
\caption{Comparison entropic scheme-MUSCL Hancock scheme. Rate of
convergence for a shock.\label{rate2}}
\end{figure}

\begin{table}[h]
\begin{center}
\begin{tabular}{|c|c|c|c|c|}
\hline $\ln(h)$ & $\ln($error$)$ MUSCL& $\ln($error$)$ proj. & rate MUSCL & rate proj.\\

\hline







-4.60517 &-6.12021 &-6.45422 & -&- \\ \hline

-5.29832 &-6.79584 & -7.14075 & 0.97472&0.99045 \\ \hline

-5.99146 & -7.48411& -7.83202  &  0.99297 &0.99730 \\ \hline

-6.68461 & -8.17001 & -8.52117 & 0.98954&0.99423\\ \hline

-7.37776 &-8.84886 & -9.20612 &  0.97937&0.98817\\ \hline
\end{tabular}
\caption{\label{ratet2}\textit{Convergence test, shock.}}
\end{center}
\end{table}


\begin{table}[h]
\begin{center}
\begin{tabular}{|c|c|c|}
\hline Variable & Left state & Right state \\
\hline  Density & $\rho_L=1$ & $\rho_R=3/4$\\
\hline  Velocity & $u_L=0$ & $u_R=-1/3$\\
\hline Pressure & $p_L=1$ & $p_R=2/3$\\
\hline
\end{tabular}
\caption{\label{discon2}\textit{Shock wave of velocity 1.}}
\end{center}
\end{table}

\begin{table}[h]
\begin{center}
\begin{tabular}{|c|c|}
\hline Interval & $]-1/2,1/2[$  \\
\hline  Number of cells & 200 to 1600 \\
\hline  CFL & $\simeq 0.58$ \\
\hline Final time & $t=0.2$ \\
\hline $\gamma$ & $1.4$ \\
\hline
\end{tabular}
\caption{\label{resum2}\textit{Computation characteristics,
shock.}}
\end{center}
\end{table}

The program we used for the numerical results of this section can
be downloaded at
\begin{quote}
\url{http://helluy/entropy/index.html}
\end{quote}

\section{Conclusion}

In this paper, we have proposed a second order generalization of
the Godunov scheme for the Euler equations. In the first part, we
have built a second order entropic projection with explicit
formulas. In the second part, we have numerically tested the
projection. Because it is difficult to construct an exact second
order Riemann solver, we had to simplify the theoretical approach.
We proposed two applications: the slope limitation applied to an
approximate kinetic Riemann solver and the slope limitation
applied to a polynomial reconstruction with cell averages.

Our approach is rigorous and gives a clear justification to the
slope limitation procedure.

Many questions still remain. For example:
\begin{enumerate}
  \item \label{dim} Is it possible to extend the scheme to higher dimensions?
  \item \label{fast} Can we improve the efficiency of the computation?
  \item \label{neos} Is it possible to extend the method to more
  general Equation Of State (EOS) than the perfect gas EOS ?
\end{enumerate}

The answer to the first question is yes. It can be done with at
least two methods. The first method, which is the simplest could
be to use an alternate direction method. Then the scheme is
limited to cartesian grids. Another way would be to extend Theorem
\ref{theorem:perth} to triangles or quadrilaterals. We do not know
if the computation of the practical projection remains possible.

The answer to the second question is certainly yes. But we do not
know if it is possible to find a simpler computation without
abandoning an exact entropy decrease. By relaxing the exact
entropy decrease or approximating the formula of Theorem
\ref{maintheo}, some schemes have already been designed for the
Lagrangian equations in \cite{coquel_lefloch95}.

The answer to the third question is: maybe yes. The main
ingredient in the presented construction is the existence of an
entropy whose hessian is degenerated. For a given pressure EOS
\begin{equation}\label{eosg}
    p=p(\tau=1/\rho,\varepsilon),
\end{equation}
the Lax entropies of the associated Euler equations are
constructed, as described in \cite{harten98}, from the concave
solutions $s(\tau,\varepsilon)$ of
\begin{equation}\label{eqentr}
\frac{{\partial s}} {{\partial \tau }} - p\frac{{\partial s}}
{{\partial \varepsilon }} = 0.
\end{equation}
The Lax entropies are then $U=-\rho s$. The general concave
solutions of (\ref{eqentr}) are of the form $s=A(s_0)$ where $A$
is a function with monotony and convexity properties, and $s_0$ a
particular solution. If the hessian of $s_0$ is degenerated, maybe
that the previous construction can be generalized.

We would like to end this presentation by insisting on the fact
that the second order entropic projection can be used as a slope
limiter for other numerical methods. The Galerkin discontinous
method (see \cite{lesaint74}, \cite{cockburn89},
\cite{feistauer02}...) could be a candidate for such a limiter.
Indeed in this method a piecewise polynomial approximation at each
time step has to be limited in order to avoid spurious
oscillations.

\section*{Appendix 1: computations for the Boltzmann scheme}

We start with \( A_{m}: \)
\begin{eqnarray*}
A_{m} & = & \frac{x-x_{i-1/2}}{h}\int _{v=v_{\min }}^{v_{\max }}M_{w_{i,r}}(v)K(v)dv\\
 &  & -\frac{\tau }{h}\int _{v=v_{\min }}^{v_{\max }}M_{w_{i,r}}(v)vK(v)dv\\
 &  & +\frac{x_{i+1/2}-x}{h}\int _{v=v_{\min }}^{v_{\max }}M_{w_{i,l}}(v)K(v)dv\\
 &  & +\frac{\tau }{h}\int _{v=v_{\min }}^{v_{\max }}M_{w_{i,l}}(v)vK(v)dv
\end{eqnarray*}
\begin{eqnarray*}
A_{m} & = & \frac{x-x_{i-1/2}}{h}\frac{\rho _{i,r}}{2\sqrt{6\varepsilon _{i,r}}}\left[ \begin{array}{c}
v\\
v^{2}/2\\
v^{3}/6
\end{array}\right] _{\max (v_{\min },u_{i,r}-\sqrt{6\varepsilon _{i,r}})}^{\min (v_{\max },u_{i,r}+\sqrt{6\varepsilon _{i,r}})}\\
 &  & -\frac{\tau }{h}\frac{\rho _{i,r}}{2\sqrt{6\varepsilon _{i,r}}}\left[ \begin{array}{c}
v^{2}/2\\
v^{3}/3\\
v^{4}/8
\end{array}\right] _{\max (v_{\min },u_{i,r}-\sqrt{6\varepsilon _{i,r}})}^{\min (v_{\max },u_{i,r}+\sqrt{6\varepsilon _{i,r}})}\\
 &  & +\frac{x_{i+1/2}-x}{h}\frac{\rho _{i,l}}{2\sqrt{6\varepsilon _{i,l}}}\left[ \begin{array}{c}
v\\
v^{2}/2\\
v^{3}/6
\end{array}\right] _{\max (v_{\min },u_{i,l}-\sqrt{6\varepsilon _{i,l}})}^{\min (v_{\max },u_{i,l}+\sqrt{6\varepsilon _{i,l}})}\\
 &  & +\frac{\tau }{h}\frac{\rho _{i,l}}{2\sqrt{6\varepsilon _{i,l}}}\left[ \begin{array}{c}
v^{2}/2\\
v^{3}/3\\
v^{4}/8
\end{array}\right] _{\max (v_{\min },u_{i,l}-\sqrt{6\varepsilon _{i,l}})}^{\min (v_{\max },u_{i,l}+\sqrt{6\varepsilon _{i,l}})}
\end{eqnarray*}
 In the same way:
\begin{eqnarray*}
A_{l} & = & \frac{x-x_{i-3/2}}{h}\frac{\rho _{i-1,r}}{2\sqrt{6\varepsilon _{i-1,r}}}\left[ \begin{array}{c}
v\\
v^{2}/2\\
v^{3}/6
\end{array}\right] _{\max (v_{\max },u_{i-1,r}-\sqrt{6\varepsilon _{i-1,r}})}^{u_{i-1,r}+\sqrt{6\varepsilon _{i-1,r}}}\\
 &  & -\frac{\tau }{h}\frac{\rho _{i-1,r}}{2\sqrt{6\varepsilon _{i-1,r}}}\left[ \begin{array}{c}
v^{2}/2\\
v^{3}/3\\
v^{4}/8
\end{array}\right] _{\max (v_{\max },u_{i-1,r}-\sqrt{6\varepsilon _{i-1,r}})}^{u_{i-1,r}+\sqrt{6\varepsilon _{i-1,r}}}\\
 &  & +\frac{x_{i-1/2}-x}{h}\frac{\rho _{i-1,l}}{2\sqrt{6\varepsilon _{i-1,l}}}\left[ \begin{array}{c}
v\\
v^{2}/2\\
v^{3}/6
\end{array}\right] _{\max (v_{\max },u_{i-1,l}-\sqrt{6\varepsilon _{i-1,l}})}^{u_{i-1,l}+\sqrt{6\varepsilon _{i-1,l}}}\\
 &  & +\frac{\tau }{h}\frac{\rho _{i-1,l}}{2\sqrt{6\varepsilon _{i-1,l}}}\left[ \begin{array}{c}
v^{2}/2\\
v^{3}/3\\
v^{4}/8
\end{array}\right] _{\max (v_{\max },u_{i-1,l}-\sqrt{6\varepsilon _{i-1,l}})}^{u_{i-1,l}+\sqrt{6\varepsilon _{i-1,l}}}
\end{eqnarray*}
 and
\begin{eqnarray*}
A_{r} & = & \frac{x-x_{i+1/2}}{h}\frac{\rho _{i+1,r}}{2\sqrt{6\varepsilon _{i+1,r}}}\left[ \begin{array}{c}
v\\
v^{2}/2\\
v^{3}/6
\end{array}\right] _{u_{i+1,r}-\sqrt{6\varepsilon _{i+1,r}}}^{\min (v_{\min },u_{i+1,r}+\sqrt{6\varepsilon _{i+1,r}})}\\
 &  & -\frac{\tau }{h}\frac{\rho _{i+1,r}}{2\sqrt{6\varepsilon _{i+1,r}}}\left[ \begin{array}{c}
v^{2}/2\\
v^{3}/3\\
v^{4}/8
\end{array}\right] _{u_{i+1,r}-\sqrt{6\varepsilon _{i+1,r}}}^{\min (v_{\min },u_{i+1,r}+\sqrt{6\varepsilon _{i+1,r}})}\\
 &  & +\frac{x_{i+3/2}-x}{h}\frac{\rho _{i+1,l}}{2\sqrt{6\varepsilon _{i+1,l}}}\left[ \begin{array}{c}
v\\
v^{2}/2\\
v^{3}/6
\end{array}\right] _{u_{i+1,l}-\sqrt{6\varepsilon _{i+1,l}}}^{\min (v_{\min },u_{i+1,l}+\sqrt{6\varepsilon _{i+1,l}})}\\
 &  & +\frac{\tau }{h}\frac{\rho _{i+1,l}}{2\sqrt{6\varepsilon _{i+1,l}}}\left[ \begin{array}{c}
v^{2}/2\\
v^{3}/3\\
v^{4}/8
\end{array}\right] _{u_{i+1,l}-\sqrt{6\varepsilon _{i+1,l}}}^{\min (v_{\min },u_{i+1,l}+\sqrt{6\varepsilon _{i+1,l}})}
\end{eqnarray*}
 It is easy to check that \( w^{n+1,-} \) is piecewise polynomial and continuous
on each cell. For instance, the practical computation of a term like
\[
A=\left[ \begin{array}{c}
v\\
v^{2}/2\\
v^{3}/6
\end{array}\right] _{\max (v_{\min },v)}^{\min (v_{\max },V)}\]
 where \( v<V, \) gives
\begin{eqnarray*}
A & = & \left[ \begin{array}{c}
V-v\\
V^{2}/2-v^{2}/2\\
V^{3}/6-v^{3}/6
\end{array}\right] \textrm{ if }v_{\min }<v\textrm{ and }v_{\max }>V,\textrm{ i}.\textrm{e}.\textrm{ }x\in \left] V\tau +x_{i-1/2},v\tau +x_{i+1/2}\right[ \\
A & = & \left[ \begin{array}{c}
V-v_{\min }\\
V^{2}/2-v_{\min }^{2}/2\\
V^{3}/6-v_{\min }^{3}/6
\end{array}\right] \textrm{ if }v_{\min }>v\textrm{ and }v_{\max }>V,\textrm{ i}.\textrm{e}.\textrm{ }x\in \left] v\tau +x_{i+1/2},V\tau +x_{i+1/2}\right[ \\
A & = & \left[ \begin{array}{c}
v_{\max }-v\\
v_{\max }^{2}/2-v^{2}/2\\
v_{\max }^{3}/6-v^{3}/6
\end{array}\right] \textrm{ if }v_{\min }<v\textrm{ and }v_{\max }<V,\textrm{ i}.\textrm{e}.\textrm{ }x\in \left] v\tau +x_{i-1/2},V\tau +x_{i-1/2}\right[ \\
A & = & 0\textrm{ if }x>V\tau +x_{i+1/2}\textrm{ or }x<v\tau +x_{i-1/2}
\end{eqnarray*}


\textbf{Remark}: by the CFL condition the inequality \( V\tau +x_{i-1/2}<v\tau +x_{i+1/2} \)
necessarily holds.\bibliographystyle{plain}
\section*{Appendix 2: positive mean value interpolation}
This appendix is devoted to an algorithm in order to avoid
negative values in the interpolation process. In all the numerical
tests that we presented it was necessary to activate this
correction only on a few cells. This method can be interesting for
other purposes.

For this, we consider a scalar non-negative function $f$ defined
on the interval $[-1,2]$. We know the mean values of $f$ on the
sub-intervals $I_i=[i-1,i]$, $i=0,1,2$
\begin{equation}\label{subcell}
    f_i=
\int_{i - 1}^i {f(t)dt \ge 0.}
\end{equation}
A classical interpolation would be to find a second order
polynomial $P$ satisfying
\begin{equation}\label{meanpoly}
    f_i=
\int_{i - 1}^i {P(t)dt,}
\end{equation}
but it is known that this interpolation can be negative in some
point in the interval $[0,1]$, even if the three mean values $f_i$
are positive. Instead, we will consider a constrained optimization
problem. We consider a base $(P_i)$ of the second order
polynomials satisfying $ \int_{I_i } {P_j }  = \delta _{ij}$ where
$\delta _{ij}$ is the Kronecker symbol.
\begin{equation}\label{base}
\begin{gathered}
  P_1 (x) =  - x^2  + x + \frac{5}
{6}, \hfill \\
  P_0 (x) = \frac{{x^2 }}
{2} - x + \frac{1}
{3}, \hfill \\
  P_2 (x) = \frac{{x^2 }}
{2} - \frac{1}
{6}. \hfill \\
\end{gathered}
\end{equation}
The polynomial $P$ is searched under the form
\begin{equation}\label{poly}
P = g_0 P_0  + g_1 P_1  + g_2 P_2 .
\end{equation}
In this way, the mean values of $P$ on $[i-1,i]$ are $g_i$ for
$i=0,1,2$. The conservation property imposes
\begin{equation}\label{consf}
    g_1=f_1.
\end{equation}
We then consider the functional $ J(P) = \frac{1} {2}\left( {(g_0
- f_0 )^2  + (g_2  - f_2 )^2 } \right)$. We solve the optimization
problem: find $P\geq 0$ such that $g_1=f_1$ and $J(P)$ is minimal.
Note that if the interpolation polynomial defined in
(\ref{meanpoly}) is non-negative then it solves the minimization
problem and then $J(P)=0$.

Consider the Lagrangian $L(P,\mu)=J(P)-<\mu,P>$, where $\mu$ is in
the set $M^{1,+}$ of positive bounded measures on $[0,1]$. The
optimization problem is equivalent to
\begin{equation}\label{infsup}
\mathop {\inf } \limits_{\begin{subarray}{l}
  g_0 ,g_1 ,g_2  \\
  g_1  = f_1
\end{subarray}}  \mathop {\sup }\limits_{\mu  \in M^{1, + } } L(P,\mu ).
\end{equation}
The optimality condition classically reads\[
\begin{gathered}
  g_0  = f_0  +  < \mu ,P > , \hfill \\
  g_2  = f_2  +  < \mu ,P > , \hfill \\
  \forall x \in \left[ {0,1} \right],\quad \mu (x)P(x) = 0. \hfill \\
\end{gathered}
\]
But a second order polynomial has at most two roots. This means
that $\mu$ is a linear combination of at most two Dirac masses.
Due to the fact that $P$ has to be non-negative, we can then
distinguish the following cases:
\begin{enumerate}
    \item $P$ is positive, then $g_0=f_0$, $f_1=g_1$ and
    $g_2=f_2$.
    \item $P$ is positive in $]0,1]$ and $P(0)=0$ then
    $f_0=g_0+\mu_0 P_0(0)$, $f_1=g_1$ and $f_2=g_2$.
    \item $P$ is positive in $[0,1[$ and $P(1)=0$ then
    $f_2=g_2+\mu_2 P_2(1)$, $f_1=g_1$ and $f_0=g_0$.
    \item $P$ is positive in $]0,1[$ and $P(0)=P(1)=0$ then
    $f_0=g_0+\mu_0 P_0(0)$, $f_2=g_2+\mu_2 P_2(1)$ and $f_1=g_1$.
    \item $P$ is positive in $[0,1]-x_0$, with $x_0 \in ]0,1[$.
    Then, $f_0=g_0+\mu_0 P_0(x_0)$, $f_2=g_2+\mu_0 P_2(x_0)$,
    $f_1=g_1$,
    $P(x_0)=0$, $P'(x_0)=0$.
\end{enumerate}
Due to the positivity of $P$, case (4) never happens. The
algorithm to compute $P$ is then the following.

First, we have necessarily $f_1=g_1$. Then the following cases are
considered:
\begin{enumerate}
    \item If $g_1 \geq 1/2(g_0+g_2))$ take $f_0=g_0$, $f_1=g_1$, $f_2=g_2$. The
    resulting polynomial is concave and >0;
    \item if $g_1 \leq 1/2(g_0+g_2))$, try $f_0=g_0$, $f_1=g_1$,
    $f_2=g_2$. The resulting polynomial is convex. It is solution
    if $P(0)\geq 0$, $P'(0)\geq 0$ or $P(1)\geq 0$, $P'(1)\leq 0$.
    \item if $g_1 < 1/2(g_0+g_2))$ and $7/2g_1-1/2g_2\geq 0$ and
    $\mu_0=3/2g_2-15/2g_1-3g_0 \geq0$ then the solution is given
    by $f_0=g_0+\mu_0/3$, $f_2=g_2$;
    \item if $g_1 < 1/2(g_0+g_2))$ and $7/2g_1-1/2g_0\geq 0$ and
    $\mu_2=3/2g_0-15/2g_1-3g_2 \geq0$ then the solution is given
    by $f_0=g_0$, $f_2=g_2+\mu_2/3$;
    \item Finally, if $g_1 < 1/2(g_0+g_2))$ in all the other
    cases, solve $f_0=g_0+\mu_0 P_0(x_0)$, $f_2=g_2+\mu_0 P_2(x_0)$,
    $f_1=g_1$,
    $P(x_0)=0$, $P'(x_0)=0$ for $x_0$ and $\mu_0$.
\end{enumerate}
The algorithm, written in Maple and C++ languages, can be
downloaded and tested at
\begin{quote}\url{http://helluy/entropy/index.html}\end{quote}.
\bibliography{entropy}

\end{document}
