%% LyX 2.3.7 created this file.  For more info, see http://www.lyx.org/.
%% Do not edit unless you really know what you are doing.
\documentclass[oneside,english]{amsart}
\usepackage[T1]{fontenc}
\usepackage[utf8]{inputenc}
\usepackage{url}
\usepackage{amstext}
\usepackage{amsthm}
\usepackage{amssymb}
\usepackage{graphicx}

\makeatletter

%%%%%%%%%%%%%%%%%%%%%%%%%%%%%% LyX specific LaTeX commands.
%% Because html converters don't know tabularnewline
\providecommand{\tabularnewline}{\\}

%%%%%%%%%%%%%%%%%%%%%%%%%%%%%% Textclass specific LaTeX commands.
\numberwithin{equation}{section}
\numberwithin{figure}{section}
\newenvironment{lyxcode}
	{\par\begin{list}{}{
		\setlength{\rightmargin}{\leftmargin}
		\setlength{\listparindent}{0pt}% needed for AMS classes
		\raggedright
		\setlength{\itemsep}{0pt}
		\setlength{\parsep}{0pt}
		\normalfont\ttfamily}%
	 \item[]}
	{\end{list}}

\makeatother

\usepackage{babel}
\begin{document}
\title{Random sampling remap for compressible two-phase flows}
\author{M. Bachmann, P. Helluy, H. Mathis and S. Müller}
\address{RWTH Aachen and IRMA, Université de Strasbourg}
\email{helluy@math.unistra.fr}
\begin{abstract}
In this paper, we address the problem of solving accurately gas-liquid
compressible flows, without pressure oscillations at the gas-liquid
interface. We introduce a new Lagrange-projection scheme based on
a random sampling technique introduced by Chalons and Goatin in \cite{CG07}.
We compare it to a Ghost Fluid approach introduced in \cite{WLK06}
and \cite{MBKKH09}. Despite the non-conservative feature of the schemes,
we observe the numerical convergence towards the relevant weak solution,
for shock-contact interaction test cases. Finally, we apply the new
scheme to the computation of the oscillations of a spherical air bubble
inside water.
\end{abstract}

\keywords{Finite volume, Godunov scheme, Ghost fluid method, lagrange-projection,
Glimm scheme, gas bubble oscillations.}
\maketitle

\section*{Introduction}

The bad precision of conservative Godunov schemes applied to two-fluid
flows is a subject that has been studied now for more than twenty
years (see \cite{Abg88,Kar94,SA99a,BHR03,MBKKH09} and included references).
This bad precision implies perturbations on the pressure profiles,
that are often called the ``pressure oscillations`` phenomenon.

For the moment, it has not been possible to design a simple conservative
scheme that preserves the constant velocity-pressure states. This
property, which amounts to preserving the contact discontinuities
in one-dimensional flows, seems to be mandatory for obtaining reliable
schemes. Thus, many authors have proposed modified Godunov schemes
in order to achieve this property. Karni, in \cite{Kar94} proposes
to solve the pressure evolution equation instead of the mass fraction
evolution equation at the interface. Abgrall and Saurel \cite{Abg88,SA99a}
propose to solve the mass fraction equation in a non-conservative
way in order to recover the preservation of constant velocity-pressure
states (SA approach). Fedkiw and collaborators \cite{FAMO99} introduce
the Ghost Fluid (GF) method: at the interface, they propose to introduce
two virtual fluids in order to construct a scheme that only requires
a one-fluid Riemann solver. The GF method has been improved in many
works. We will concentrate here in one variant, the Real Ghost Fluid
Method (RGFM) \cite{WLK06,MBKKH09}. It is not possible to give a
comprehensive survey of this field, but many other attempts have been
proposed: changing the two-fluid model to a more general one \cite{SA99b,WK05,KL10},
using a Lagrangian approach at the interface \cite{HMM08}, \textit{etc.}

A common feature of all the above-mentioned approaches is that the
schemes are generally non-conservative. It is possible to construct
very exotic schemes that are conservative but they are then very complicated
and can be used only for academic test cases \cite{HMM08}. A natural
question arises, which is whether these schemes converge, or not,
towards the relevant solution of the initial two-fluid models. Indeed,
generally, non-conservative schemes converge towards wrong solutions.
It is a purely non-linear behavior \cite{HL94}, which is still not
yet well understood because a non-conservative Lax-Wendroff theory
does not exist (for a recent work on this aspect, see \cite{AK10}).
Here, the situation is rather subtle because the non-conservation
of the schemes is generally located at the contact discontinuity,
which is a linearly degenerated field. When all the discontinuous
waves (shocks and contacts) are well separated, it is thus not a full
paradox to observe convergence towards the good solution. However,
in case of complicated non-linear interactions, when all the waves
are mixed, it is difficult to understand why the non-conservative
approach leads to well converging schemes. For one-dimensional problems,
wave mixing occur in very simple situations: at the initial time of
a Riemann problem, for instance, or when a shock wave is sent over
a moving interface.

Our first objective in this paper is to provide a new non-conservative
scheme for solving two-fluid flows. Our approach is an adaptation
of previous works of Goatin, Chalons and Coquel \cite{CG07,CC08}
on Lagrange-projection schemes. The idea is to use a projection step
based on random sampling techniques, very similar to the Glimm scheme
method. The classical Glimm scheme \cite{Gl65} implies an exact Riemann
solver. Here, because the random sampling is only performed in the
projection step, it is possible to rely on approximate Riemann solvers
in the Lagrange step. We will see that in presence of strong shocks,
our approach has to be adapted, in order to avoid oscillations and
non-convergence: we simply propose to perform the random sampling
strategy only at the two fluids interface. As for other schemes, our
random sampling projection scheme is not conservative. Let us recall
that the Glimm scheme is not conservative too, but possesses statistically
conservation properties \cite{Gl65}. We hope that such properties
still hold for our sampling projection scheme. 

Our second objective is to perform a numerical convergence study for
several classical non-conservative schemes for two-fluid flows, and
compare them to our new scheme. We will observe, surprisingly, that
the numerical solutions seem to converge towards the good weak solutions.
According to our previous considerations, this behavior is absolutely
not obvious. Finally, we compare our new scheme to the RGFM in a more
complex configuration. We present a hard test case consisting in computing
the oscillations of a spherical gas bubble in a compressible liquid.
We present the results obtained with the RGFM and the random projection
scheme.

\section{The two-fluid model}

In this paper, we investigate the numerical resolution of the Euler
system for a compressible two-fluid mixture. The density of the mixture
is $\rho$, the velocity is $u$ and the internal energy is $e$.
We denote by $E$ the total energy defined by $E=e+u^{2}/2$. The
pressure is noted $p$. For simplicity, but without loss a generality,
we only consider one-dimensional flows. The unknowns depend on the
spatial position $x$ and of the time $t$. The PDE system is made
of mass, momentum and energy conservation laws 
\begin{eqnarray}
\partial_{t}\rho+\partial_{x}(\rho u) & = & 0,\label{eq:mass}\\
\partial_{t}(\rho u)+\partial_{x}(\rho u^{2}+p) & = & 0,\label{eq:momentum}\\
\partial_{t}(\rho E)+\partial_{x}((\rho E+p)u) & = & 0.\label{eq:energy}
\end{eqnarray}

In the case of a one-fluid flow, the pressure would be a function
of the density and the internal energy
\[
p=p(\rho,e).
\]
Because we consider two-fluid flows, our pressure is a function of
the density and the internal energy but also of a supplementary unknown
$\varphi$, the colour function
\begin{equation}
p=p(\rho,e,\varphi).\label{eq:pressure}
\end{equation}
The colour function is transported with the flow
\[
\partial_{t}\varphi+u\partial_{x}\varphi=0.
\]
Combining this transport equation with the mass conservation law (\ref{eq:mass})
gives a conservative form of the colour function equation
\begin{equation}
\partial_{t}(\rho\varphi)+\partial_{x}(\rho\varphi u)=0.\label{eq:colour}
\end{equation}
Finally, defining the conservative variables vector
\[
W=(\rho,\rho u,\rho E,\rho\varphi)^{T},
\]
and the flux vector 
\[
F(W)=(\rho u,\rho u^{2}+p,(\rho E+p)u,\rho\varphi u)^{T},
\]
the system (\ref{eq:mass})-(\ref{eq:energy}), (\ref{eq:colour}),
(\ref{eq:pressure}) can be written
\[
\partial_{t}W+\partial_{x}F(W)=0.
\]

For practical computations, we will use the pressure law of a mixture
a stiffened gas. We consider a gas and a liquid satisfying stiffened
gas laws
\[
p=(\gamma_{i}-1)\rho e-\gamma_{i}\pi_{i},
\]
with $i=1$ for the gas and $i=2$ for the liquid. The parameters
$\gamma_{i}>1$ and $\pi_{i}$ are obtained from physical measurements.
The mixture pressure is defined by
\begin{equation}
p(\rho,e,\varphi)=(\gamma(\varphi)-1)\rho e-\gamma(\varphi)\pi(\varphi).\label{eq:mix-stiff-gas}
\end{equation}
The mixture parameters are given by
\begin{eqnarray*}
\frac{1}{\gamma(\varphi)-1} & = & \varphi\frac{1}{\gamma_{2}-1}+(1-\varphi)\frac{1}{\gamma_{1}-1},\\
\frac{\gamma(\varphi)\pi(\varphi)}{\gamma(\varphi)-1} & = & \varphi\frac{\gamma_{2}\pi_{2}}{\gamma_{2}-1}+(1-\varphi)\frac{\gamma_{1}\pi_{1}}{\gamma_{1}-1},
\end{eqnarray*}
in such a way that $\varphi=1$ in the pure liquid phase and $\varphi=0$
in the pure gas phase. This system has nice mathematical properties:
it is hyperbolic and the Riemann problem has a unique solution, even
with large data \cite{BHR03}.

On the numerical side, the situation is more complicated. For instance
it is now well-known that standard conservative finite volume schemes
have a poor precision when applied to this kind of flow. Even worse,
in some configurations of liquid-gas flows, the explicit Godunov cannot
be used because it leads to negative densities.

\section{The Lagrange-projection approach}

For the finite volume approximation, we consider a sequence of times
$t_{n}$, $n\in\mathbb{N}$, such that $t_{0}=0$ and $\tau_{n}=t_{n+1}-t_{n}>0$.
We also consider mesh points $x_{i+1/2}^{n}$ at time $n$. The cell
$C_{i}^{n}$ is the interval $]x_{i-1/2}^{n},x_{i+1/2}^{n}[$. We
denote by $x_{i}^{n}$ the center of cell $C_{i}^{n}$
\[
x_{i}^{n}=\frac{x_{i-1/2}^{n}+x_{i+1/2}^{n}}{2}.
\]
The length of cell $C_{i}^{n}$ is noted $h_{i}^{n}=x_{i+1/2}^{n}-x_{i-1/2}^{n}$.
According to the notations, the mesh is moving but at some time step
we will go back to the initial mesh at $n=0$. We note
\[
x_{i}=x_{i}^{0},\quad C_{i}=C_{i}^{0},\quad h_{i}=h_{i}^{0},\quad etc.
\]

We are looking for an approximation of $W$ in the cell $C_{i}^{n}$
\[
W_{i}^{n}\simeq W(x,t),\quad x\in C_{i}^{n},\quad t\in]t_{n},t_{n+1}[.
\]

For the numerical resolution, we need an (exact or approximate) Riemann
solver. The (exact or approximate) solution of the Riemann problem
\begin{eqnarray*}
\partial_{t}V+\partial_{x}F(V) & = & 0,\\
V(x,0) & = & \left\{ \begin{array}{c}
W_{L}\text{ if }x<0,\\
W_{R}\text{ if }x>0,
\end{array}\right.
\end{eqnarray*}
is noted 
\[
R(\frac{x}{t},W_{L},W_{R})=V(x,t).
\]

Each time step of the Lagrange-projection scheme is made of two stages.
In the first stage, we approximate the solution with a Lagrange scheme
\[
h_{i}^{n+1/2}W_{i}^{n+1/2}-h_{i}^{n}W_{i}^{n}+\tau_{n}\left(F_{i+1/2}^{n}-F_{i-1/2}^{n}\right)=0.
\]
The Lagrange flux is defined by
\begin{eqnarray*}
F_{i+1/2}^{n} & = & F(W_{i+1/2}^{n})-u_{i+1/2}^{n}W_{i+1/2}^{n},\\
W_{i+1/2}^{n} & = & R(u_{i+1/2}^{n},W_{i}^{n},W_{i+1}^{n}),
\end{eqnarray*}
where the cell boundary $x_{i+1/2}^{n}$ moves at the velocity $u_{i+1/2}^{n}$
of the contact discontinuity in the resolution of the Riemann problem
between $W_{L}=W_{i}^{n}$ and $W_{R}=W_{i+1}^{n}$
\[
x_{i+1/2}^{n+1/2}=x_{i+1/2}^{n}+\tau_{n}u_{i+1/2}^{n}.
\]
In particular, this defines the new size of cell $C_{i}^{n}$
\[
h_{i}^{n+1/2}=x_{i+1/2}^{n+1/2}-x_{i-1/2}^{n+1/2}=h_{i}^{n}+\tau_{n}(u_{i+1/2}^{n}-u_{i-1/2}^{n}).
\]
This formula is important because it can be generalized to higher
dimensions. It permits to avoid the actual computation of the moved
mesh.

After the Lagrange stage, we have to go back to the initial Euler
mesh. This can be done with several methods

\subsection{The averaging projection}

In this approach, we average back on the Euler grid, with a simple
$L^{2}$ projection
\begin{eqnarray}
W_{i}^{n+1} & = & \frac{\tau_{n}}{h_{i}}\max(u_{i-1/2}^{n},0)W_{i-1}^{n+1/2}-\frac{\tau_{n}}{h_{i}}\min(u_{i+1/2}^{n},0)W_{i+1}^{n+1/2}\nonumber \\
 &  & +\left(1-\frac{\tau_{n}}{h_{i}}\max(u_{i-1/2}^{n},0)+\frac{\tau_{n}}{h_{i}}\min(u_{i+1/2}^{n},0)\right)W_{i}^{n+1/2}.\label{eq:cons-remap}
\end{eqnarray}
And we go back to the initial Euler grid
\[
C_{i}^{n+1}=C_{i},\quad h_{i}^{n+1}=h_{i}^{n}.
\]
It can also be written
\begin{gather}
W_{i}^{n+1}=W_{i}^{n+1/2}-\frac{\tau_{n}}{h_{i}}\left(\max(u_{i-1/2}^{n},0)(W_{i}^{n+1/2}-W_{i-1}^{n+1/2})+\right.\nonumber \\
\left.\min(u_{i+1/2}^{n+1/2},0)(W_{i+1}^{n+1/2}-W_{i}^{n+1/2})\right).\label{eq:trans-remap}
\end{gather}
In this way, it is clear that the projection step is an upwind approximation
of
\[
\partial_{t}W+u\partial_{x}W=0.
\]
This method is fully conservative and thus has a bad precision for
multifluid problems \cite{BHR03}. It is possible to improve the precision
by the Saurel-Abgrall approach. It consists in performing a non-conservative
projection on the colour function. Instead of projecting $\rho\varphi$
as in (\ref{eq:cons-remap}) we project directly $\varphi$, which
gives
\begin{gather*}
\varphi_{i}^{n+1}=\varphi_{i}^{n+1/2}-\frac{\tau_{n}}{h_{i}}\left(\max(u_{i-1/2}^{n},0)(\varphi_{i}^{n+1/2}-\varphi_{i-1}^{n+1/2})+\right.\\
\left.\min(u_{i+1/2}^{n+1/2},0)(\varphi_{i+1}^{n+1/2}-\varphi_{i}^{n+1/2})\right).
\end{gather*}
This approach results in a globally non-conservative scheme: it induces
a numerical mass transfer between the two phases. In the case of the
stiffened gas pressure law (\ref{eq:mix-stiff-gas}) it can be proved
that the resulting scheme preserves constant $(u,p)$ states.

\subsection{The Glimm projection}

In this approach, we construct a sequence of random or pseudo-random
numbers $\omega_{n}\in[0,1].$ According to this number we take
\begin{gather}
W_{i}^{n+1}=W_{i-1}^{n+1/2}\text{ if }\omega_{n}<\frac{\tau_{n}}{h_{i}}\max(u_{i-1/2}^{n},0),\nonumber \\
W_{i}^{n+1}=W_{i+1}^{n+1/2}\text{ if }\omega_{n}>1+\frac{\tau_{n}}{h_{i}}\min(u_{i+1/2}^{n},0),\nonumber \\
W_{i}^{n}=W_{i}^{n+1/2}\text{ if }\frac{\tau_{n}}{h_{i}}\max(u_{i-1/2}^{n},0)\leq\omega_{n}\leq1+\frac{\tau_{n}}{h_{i}}\min(u_{i+1/2}^{n},0).\label{eq:glimm-remap}
\end{gather}
And we go back to the initial Euler grid
\[
h_{i}^{n+1}=h_{i}^{n}.
\]
This method is only statistically conservative \cite{Gl65}. It preserves
exactly constant velocity-pressure states. The contacts are solved
in one point. The drawback is that the solution may be noisy. In particular,
for strong shocks the whole method does not converge towards the correct
entropy solution.

A good choice for the pseudo-random sequence $\omega_{n}$ is the
$(k_{1},k_{2})$ van der Corput sequence, computed by the following
C algorithm
\begin{lyxcode}
float~corput(int~n,int~k1,int~k2)\{

~~float~corput=0;

~~float~s=1;

~~while(n>0)\{

~~~~s/=k1;

~~~~corput+=(k2{*}n\%k1)\%k1{*}s;

~~~~n/=k1;

~~\}

~~return~corput;

\}~
\end{lyxcode}
In this algorithm, $k_{1}$ and $k_{2}$ are two relatively prime
numbers and $k_{1}>k_{2}>0$. For more details, we refer to \cite{Tor99}.
In practice, we consider the $(5,3)$ van der Corput sequence.

\subsection{Mixed projection\label{subsec:Mixed-remap}}

In order to improve the convergence of the Glimm approach, it is possible
to follow the following mixed projection step. If cell $C_{i}$ and
its two neighbours are in the same fluid, i.e. if
\[
(\varphi_{i-1}^{n}-\frac{1}{2})(\varphi_{i}^{n}-\frac{1}{2})>0\text{ and }(\varphi_{i}^{n}-\frac{1}{2})(\varphi_{i+1}^{n}-\frac{1}{2})>0,
\]
then, we follow the projection given by (\ref{eq:trans-remap}). In
all the other cases, we follow the Glimm projection (\ref{eq:glimm-remap}).
This approach allows a better precision at the interface because it
is resolved in only one point.

\section{The modified ghost fluid approach}

\begin{figure}[t]
\begin{centering}
\includegraphics[width=0.6\linewidth]{rgfmpic_color} \caption{Sketch of the computation of the real and ghost fluid states from
the interfacial states $u_{I}$, $p_{I}$ and $\rho_{IL}$, $\rho_{IR}$
determined by solving a two-phase Riemann problem for the states $u_{L}$
and $u_{R}$.}
\par\end{centering}
\centering{}\label{fig-gfm} 
\end{figure}

The real Ghost Fluid method, developed by Wang, Liu and Khoo in 2006
\cite{WLK06} is an adaptation of the original ghost fluid method
of Fedkiw.

In this method, the interface between the liquid and the gas is located
by a function $\psi$. In the liquid, we have $\psi>0$ and in the
gas, $\psi<0$. And thus, the interface corresponds to the level-set
$\psi=0$. As in the previous method, the level-set function $\psi$
is transported in the flow
\begin{equation}
\partial_{t}\psi+u\partial_{x}\psi=0.\label{eq:transpsi}
\end{equation}

And we switch from the pressure law of the liquid to the pressure
law of the gas according to the sign of $\psi$. It is clear that
the differences between the color function model and the level-set
model are only formal. However, the numerical implementations are
rather different.

The level-set function $\psi$ is approximated in cell $C_{i}$ at
time $t_{n}$ by $\psi_{i}^{n}$. The solution is approximated by
a Godunov scheme
\begin{equation}
W_{i}^{n+1}=W_{i}^{n}-\frac{\tau_{n}}{h_{i}}\left(F_{i+1/2}^{n,-}-F_{i-1/2}^{n,+}\right),\label{eq:fv-nc}
\end{equation}
with a possible non-conservative flux $F_{i+1/2}^{n,-}\neq F_{i+1/2}^{n,+}$.
If the two cells $C_{i}$ and $C_{i+1}$ are filled with the same
phase, which is true if
\[
\psi_{i}^{n}\cdot\psi_{i+1}^{n}>0.
\]
Then, we take the classical conservative Godunov flux
\[
F_{i+1/2}^{n,-}=F_{i+1/2}^{n,+}=F_{i+1/2}^{n}=F(R(0,W_{i}^{n},W_{i+1}^{n})).
\]

If the phase boundary is lying between cell $i$ and cell $i+1$ where
cell $i$ corresponds to fluid $A$ and cell $i+1$ to fluid $B$,
then it means that
\[
\psi_{i}^{n}\cdot\psi_{i+1}^{n}<0.
\]
then the left and right state are taken from cells $i-1$ and $i+2$,
respectively, to ensure access to states of the pure phases
\[
W_{L}=W_{i-1}^{n},\quad W_{R}=W_{i+2}^{n}.
\]
We solve the Riemann problem between $W_{L}$ and $W_{R}$. Let $u_{I}$
be the contact velocity in this exact solution. We can define interfacial
states to the left and to the right of the contact wave by
\[
W_{IL}=\lim_{\xi\to u_{I}^{-}}R(\xi,W_{L},W_{R}),\quad W_{IR}=\lim_{\xi\to u_{I}^{+}}R(\xi,W_{L},W_{R}).
\]
 We have thus access to interfacial states for densities $\rho_{IL}$,
$\rho_{IR}$, pressure $p_{I}$ and velocity $u_{I}$ left and right
to the phase boundary. For fluid $A$, the state $\left(\rho_{IL},\,u_{I},\,p_{I}\right)$
replaces the states of the cell i and defines the ghost states. The
fluxes are thus
\[
F_{i+1/2}^{-}=F(R(0,W_{IL},W_{IL})),\quad F_{i+1/2}^{+}=F(R(0,W_{IR},W_{IR})).
\]
The evolution equation is also slightly modified by
\[
W_{i}^{n+1}=W_{IL}-\frac{\tau_{n}}{h_{i}}\left(F_{i+1/2}^{n,-}-F_{i-1/2}^{n,+}\right),
\]
and
\[
W_{i+1}^{n+1}=W_{IR}-\frac{\tau_{n}}{h_{i+1}}\left(F_{i+3/2}^{n,-}-F_{i+1/2}^{n,+}\right),
\]
This procedure is sketched in Fig. \ref{fig-gfm}. As a consequence,
only single-phase Riemann problems are solved for each cell interface
of fluid $A$ to provide the numerical fluxes with the ghost cells
as boundary conditions at the phase boundary. Then the same procedure
is used for fluid $B$.

Thus near the phase boundary, two fluxes $\mathbf{F}_{i+\frac{1}{2}}^{\,n,\pm}$,
one for each fluid, are defined. Away from the phase boundary, where
only one numerical flux is computed at a cell interface, the spatial
order is improved by using a second-order reconstruction of the primitive
variables $\rho$, $u$, $p$. The solution is advanced to the next
time step by the finite volume scheme (\ref{eq:fv-nc}).

On the other hand, the level set function has also to be advanced.
This is done first by solving numerically (\ref{eq:transpsi}) with
a standard upwind non-conservative finite volume scheme
\[
\psi_{i}^{n+1,-}=\psi_{i}^{n}-\frac{\tau_{n}}{h_{i}}\left(\max(u_{i-1/2}^{n},0)(\psi_{i}^{n}-\psi_{i-1}^{n})+\min(u_{i+1/2}^{n},0)(\psi_{i+1}^{n}-\psi_{i}^{n})\right)
\]

Periodically, the level-set function approximation is reinitialized
in such a way that it remains a signed distance to the interface.
This is formally obtained through the numerical resolution of an Hamilton-Jacobi
equation
\begin{eqnarray*}
\partial_{\tau}\widetilde{\psi}(x,\tau)+a(\widetilde{\psi})\partial_{x}\widetilde{\psi} & = & S(\widetilde{\psi}),\\
a(\widetilde{\psi}) & = & S(\widetilde{\psi})\frac{\partial_{x}\widetilde{\psi}}{\left|\partial_{x}\widetilde{\psi}\right|},\\
S(\widetilde{\psi}) & = & \left\{ \begin{array}{ccc}
-1 & \text{ if } & \widetilde{\psi}<0,\\
0 & \text{ if } & \widetilde{\psi}=0,\\
1 & \text{ if } & \widetilde{\psi}>0,
\end{array}\right.\\
\widetilde{\psi}(x,\tau=0) & = & \psi_{i}^{n+1,-},\quad x\in C_{i}.
\end{eqnarray*}
The level-set function is replaced by the reinitialized level-set
function, \textit{i.e.} we take 
\[
\psi_{i}^{n+1}=\widetilde{\psi}(x,\tau=\infty),\quad x\in C_{i}.
\]
This procedure is described in more details in \cite{MBKKH09}.

Finally, during the update of the level set function, a cell may switch
from one fluid to the other. This situation corresponds to a change
of the sign between time step$n$ and time $n+1$, \textit{i.e.} when
$\psi_{i}^{n}\cdot\psi_{i}^{n+1}<0$. In this case, it is necessary
to also update $W_{i}^{n+1}$ on the corresponding cell. The fluid
variables are recalculated using the equation of state of the new
fluid. The cell being very close to the phase boundary, the velocities
and the pressure, which are constant for both fluids at the phase
boundary, are preserved. This modification was suggested by Barberon\cite{Barberon2002}.
In addition to this approach, we propose the modification of the density.
As no exact value for the density is known, the density is replaced
by the density of the corresponding ghost cell. More precisely, if
$\psi_{i}^{n}\cdot\psi_{i}^{n+1}<0$, and if $\psi_{i}^{n}\cdot\psi_{i+1}^{n}<0$,
then, before computing the next time-step, we substitute the density
by
\[
\rho_{i}^{n+1}\leftarrow\rho_{IR}
\]
and the energy $e_{i}^{n+1}$ is also modified in such a way that
\[
p_{i}^{n+1}=p(\rho_{i}^{n+1},e_{i}^{n+1},\psi_{i}^{n+1})
\]
is not changed. This construction implies that the whole resulting
scheme will preserve constant $(u,p)$ states. On the other hand,
it is also clear that the scheme is not conservative. For instance,
the last update implies a mass and an energy transfer between the
two fluids.

\section{Numerical results}

\subsection{Academic validation}

The first test consists in a two-fluid shock tube. The stiffened gas
parameter are
\begin{gather*}
\gamma_{2}=2,\quad\pi_{2}=1,\\
\gamma_{1}=1.4,\quad\pi_{1}=0.
\end{gather*}
We take for the left and right initial data
\begin{gather*}
(\rho_{L},u_{L},p_{L},\varphi_{L})=(2,1/2,2,1),\\
(\rho_{R},u_{R},p_{R},\varphi_{R})=(1,1/2,1,0).
\end{gather*}

For the Lagrange-projection approach, the non-conservative projection
and the Glimm projection are compared. We observe numerical convergence
in the $L^{1}$ norm for the two methods and that the Glimm projection
is more precise than the averaging projection. See Figure \ref{fig:Convergence-study:-Glimm}.
The convergence rate for the two methods is approximately $0.6$.
\begin{figure}
\begin{centering}
\includegraphics[width=10cm]{conv-academic}
\par\end{centering}
\caption{Convergence study: Glimm projection versus averaging projection, academic
validation.\label{fig:Convergence-study:-Glimm}}
\end{figure}


\subsection{1D academic shock-interface interaction}

An interface between two fluids is located a time $t=0$ at position
$x=1$. The two fluids are moving to the left at the velocity $v=-1$.
The fluid (2) is on the left, while the fluid (1) is on the right.
A shock is arriving from the left at velocity $\sigma=4$. The initial
position of the contact and the shock are chosen in such way that
they will meet together at the abscissa $x=0$ at time $t=1$. The
EOS parameters are the following
\begin{gather*}
\gamma_{1}=1.4\quad\pi_{1}=0,\\
\gamma_{2}=2\quad\pi_{2}=7.
\end{gather*}
The initial data are, if $x<-4$
\begin{gather*}
(\rho_{L},u_{L},p_{L},\varphi_{L})=(3.4884,1.1333,23.333,1),
\end{gather*}
if $x>1$
\begin{gather*}
(\rho_{R},u_{R},p_{R},\varphi_{R})=(1,-1,2,0),
\end{gather*}
and if $-4\leq x\leq1$
\begin{gather*}
(\rho_{M},u_{M},p_{M},\varphi_{M})=(2,-1,2,1).
\end{gather*}
After that the shock and the contact waves have met at time $t=1$,
the solution is simply given by the resolution of a two-fluid Riemann
problem between states $(L)$ and $(R)$. The solutions is sketched
in Figure \ref{SC-1}. The numerical data are recalled in Table \ref{SC-T1}.

\subsubsection{Lagrange plus projection schemes}

In this case, we observe that the Glimm approach does not converge.
This behavior depends on the strength of the shock wave. A typical
plot is given on Figure \ref{fig:Glimm-approach,-density}, where
we compare the exact and the approximated densities at time $t=1.5$.
\begin{figure}
\begin{centering}
\includegraphics[width=10cm]{moche}
\par\end{centering}
\caption{Glimm approach, density plot. BV explosion due to wall-heating effect
propagation.\label{fig:Glimm-approach,-density}}
\end{figure}

In this case, we thus compare the averaging projection approach with
the mixed projection approach. We obtained the results of Figure \ref{fig:Academic-shock-interface-interaction.}
The mixed projection has a better precision than the averaging projection.

\begin{figure}
\begin{centering}
\includegraphics[width=10cm]{conv-mixed}
\par\end{centering}
\caption{Academic shock-interface interaction. Convergence study. Mixed projection
and averaging projection.\label{fig:Academic-shock-interface-interaction.}}
\end{figure}
We also provide on Figure \ref{fig:Density.-Comparison} a comparison
of the mixed and averaging projection schemes for the densities for
a mesh of 500 cells of the interval $[-5;2]$. The CFL number is fixed
to $0.7$. It is interesting to observe that the interface position
is very well resolved (in only one mesh point) by the mixed projection
scheme and that this good resolution of the contact wave also implies
an improvement of the precision in the left rarefaction wave.

\begin{figure}
\begin{centering}
\includegraphics[width=10cm]{rho-mixed-average}
\par\end{centering}
\caption{Density. Comparison of the mixed and averaging projection schemes.\label{fig:Density.-Comparison}}
\end{figure}


\subsubsection{Modified Ghost fluid approach}

In order to compare the non-conservative methods of Saurel-Abgrall
and the RGFM, several numerical solutions are compared with the exact
solution and a convergence study is performed. The coarse discretization
consists of 100 cells on which the multiscale-based transformation
is applied \cite{MBKKH09}. The convergence study is performed for
grids having from 5 to 13 refinement levels $L$, i.e the uniform
grid on the finest level consists of $2^{L}*100$ cells. The threshold
value in the grid adaptation is chosen as $\varepsilon=10^{-5}$.
This small value is chosen in such a way that the regions containing
the finest grid cells are large enough to avoid additional error from
larger cells. The errors obtained with the multiscale grid adaptation
are thus comparable with those obtain with a uniform grid. Tests are
performed with a CFL number of $0.9$. Summary:$\Omega$={[}-5;2{]}m,
$t\in${[}0;1.5{]}s $N_{0}=100$, $5\le L\le13$, $\epsilon_{L}$=1.e-5,
CFL=0.9

\begin{figure}[h]
\centering{}\includegraphics[width=0.8\linewidth]{testcase1} \caption{Initialization of the air-air shock-contact interaction\label{SC-1} }
\end{figure}

\begin{table}
\centering{}%
\begin{tabular}{|l|l|l|l|l|l|}
\hline 
 & $U_{W}$  & $U_{WS}$  & $U_{W*}$  & $U_{A*}$  & $U_{A}$ \tabularnewline
\hline 
$\rho$ {[}kg/m$^{3}${]}  & 3.488  & 2  & 2.89415  & 3.2953  & 1 \tabularnewline
$v$ {[}m/s{]}  & 1.13  & -1  & 1.87672  &  & -1\tabularnewline
$p$ {[}Pa{]}  & 23.33  & 2  & 13.88  & 13.88  & 2\tabularnewline
\hline 
\end{tabular}\caption{Air-Air shock-contact interaction.\label{SC-T1}}
\end{table}

In figures \ref{SC-2-L7} and \ref{SC-2-L12} are shown the comparison
of the density for the two approaches with the exact solution, the
dashed line, at $t=1.5$ ms. The first figure corresponds to a grid
with seven refinement levels and the second one to a very fine grid
with twelve refinement levels. The global result is pictured in the
middle and a zoom of the shock position on the bottom left, a zoom
of the plateau between the rarefaction wave and the contact on the
top left and a zoom very close to the contact position on the right.
In the last one, cell centers are marked by diamonds for the numerical
results.

The Saurel-Abgrall approach generates oscillations at the bottom of
the rarefaction wave due to the interaction between the shock and
the contact. This is not the case for the RGFM that coincides quite
well with the exact solution. With five more grid refinement levels,
shown in picture \ref{SC-2-L12}, the amplitude of the oscillations
largely reduce.

At the contact, there is a smearing of the density for the Saurel-Abgrall
approach. This is easy to see in the zoom on the right in which the
density for this approach decreases slowly instead of presenting a
jump with the RGFM method. Due to the construction of the ghost fluid
method, we obtain the desired jump at the contact but this jump is
a little bit shifted compared with the exact solution. As the cell
centers are represented by diamonds on the pictures, for 7 refinement
levels, we can see a shift in the position of the contact of 2 cells
and in the computation with twelve levels of refinement, the shift
is about 4 cells. This is summarized in Table \ref{SC-T4}.

Concerning the position of the shock, which is only visible in the
zoom because of the big jump of density, we remark that its position
is well predicted with the RGFM method for both computations. That
is not really the case for the Saurel-Abgrall approach. Under grid
refinement, the shift reduces as much as the smearing region at the
contact.

Concerning the Saurel-Abgrall approach, the oscillations and the error
in the position of the shock is only due to the previous shock-contact
interaction. When only the Riemann problem is computed, i.e. when
the computation starts at $t=0.5$ ms, we obtain the results of Figure
\ref{SC-2}. In this simplified situation, the results are in good
agreement everywhere with the exact solution. A comparison (shock-contact
interaction and direct resolution of the Riemann problem) is shown
in figure \ref{SC-2b-L7}.

\begin{figure}
\centering{}\includegraphics[width=0.7\linewidth]{case2_100_L7} \caption{Results of the shock-contact interaction for 7 refinement levels at
t=1.5 ms\label{SC-2-L7}}
\end{figure}

% \centering\includegraphics[width=0.7\linewidth]{./figures/case2_100_L7}
% \vspace*{-1cm}
% \foilhead{Convergence study : Results with L=12 at t=1.5 ms }

\begin{figure}
\centering{}\includegraphics[width=0.7\linewidth]{case2_100_L12}
\caption{Results of the shock-contact for 12 refinement levels at t=1.5 ms\label{SC-2-L12}}
\end{figure}

\begin{figure}
\centering{}\includegraphics[width=0.7\linewidth]{case2rp_100_L7}
\caption{Results for 7 refinement levels at t=1.5 ms\label{SC-2b-L7}}
\end{figure}

The $L^{1}$ error of the density are given in Table \ref{SC-T3}.
The order of convergence for the Saurel-Abgrall approach is approximately
$0.5$. Concerning the RGFM, the error seems to tend to the same order
under grid refinement. 

\begin{table}
\centering{}%
\begin{tabular}{|l||l|l||l|l|}
\hline 
 & \multicolumn{2}{|c|}{Saurel-Abgrall} & \multicolumn{2}{|c|}{RGFM}\tabularnewline
\hline 
Levels  & $L^{1}$ error  & Order  & $L^{1}$ error  & Order \tabularnewline
\hline 
%$25*23$   & 43.75  & -          & 14.78    & - \\
%$250*23$  & 16.72  & 0.60   & 2.58     & 1.084\\
$5$  & 10.20  & -  & 1.8  & -\tabularnewline
$7$  & 4.81  & 0.54  & 0.51  & 0.91\tabularnewline
$8$  & 3.27  & 0.56  & 0.31  & 0.72\tabularnewline
$9$  & 2.24  & 0.54  & 0.17  & 0.87\tabularnewline
$10$  & 1.55  & 0.53  & 0.11  & 0.67\tabularnewline
$11$  & 1.09  & 0.51  & 6.90E-02  & 0.62\tabularnewline
$12$  & 0.77  & 0.51  & 4.54E-02  & 0.60\tabularnewline
$13$  & 0.54  & 0.5  & 3.05E-02  & 0.57\tabularnewline
\hline 
\end{tabular}\caption{Water-Air global $L^{1}$ error.\label{SC-T3}}
\end{table}

\begin{table}
\centering{}%
\begin{tabular}{|l|l||l|l||l|l|}
\hline 
 &  & \multicolumn{2}{|c|}{Saurel-Abgrall} & \multicolumn{2}{|c|}{RGFM}\tabularnewline
\hline 
Levels  & $h_{L}$ {[}m{]}  & Error  & Order  & Error  & Order \tabularnewline
\hline 
%$25*23$     & 3.0E-02  & 4.93E-02 & -     & -        & -\\
%$250*23$    & 6.0E-03  & 5.64E-03 & 1.35 & -        & -\\
5  & 1.87E-03  & 7.12E-03  & -  & 3.55E-03  & - \tabularnewline
7  & 4.69E-04  & 3.51E-03  & 0.51  & 1.03E-03  & 0.89\tabularnewline
8  & 2.34E-04  & 2.46E-03  & 0.51  & 5.25E-04  & 0.97\tabularnewline
9  & 1.17E-04  & 1.72E-03  & 0.52  & 2.93E-04  & 0.84 \tabularnewline
10  & 5.86E-05  & 1.20E-03  & 0.52  & 1.59E-04  & 0.88 \tabularnewline
11  & 2.93E-05  & 8.42E-04  & 0.51  & 9.21E-05  & 0.79\tabularnewline
12  & 1.46E-05  & 5.92E-04  & 0.51  & 5.43E-05  & 0.76 \tabularnewline
13  & 7.32E-06  & 4.22E-04  & 0.49  & 3.38E-05  & 0.68 \tabularnewline
\hline 
\end{tabular}\caption{Water-Air error in the interface position where $h_{L}$ is the grid
size for $L$ refinement levels.\label{SC-T4}}
\end{table}


\subsection{Air-water test case}

For the RGFM, we propose an additional test case consisting in a shock
wave of Mach number $0.67$ at the position $x=-3$ m running in the
liquid that interacts with the air at $t=0.5$ms at the position $x=0$
m. The water ahead of the shock and the air are moving towards the
shock at the velocity of $100$ $m/s$. The computational domain is
$[-4;2]$m. The test is sketched in Figure \ref{SC-2} and the different
states are given in Table (\ref{SC-T2}). The material parameters
for the fluids are listed in Table \ref{SC-TMP2}.

\begin{table}
\centering{}%
\begin{tabular}{|l|l|l|l|l|l|}
\hline 
 & $U_{W}$  & $U_{WS}$  & $U_{W*}$  & $U_{A*}$  & $U_{A}$ \tabularnewline
\hline 
$\rho$ {[}kg/m$^{3}${]}  & 1620.6  & 1000  & 900  & 5.57  & 1 \tabularnewline
$v$ {[}m/s{]}  & 1087.1  & -100  & 2361.4  & 2361.4  & -100\tabularnewline
$p$ {[}Pa{]}  & 3.6801E+09  & 1E+05  & 7.48506E+06  & 7.48506E+06  & 1E+05\tabularnewline
\hline 
\end{tabular}\caption{Water-Air shock-contact interaction.\label{SC-T2}}
\end{table}

\begin{table}
\centering{}%
\begin{tabular}{|l|l|l|}
\hline 
 & $\gamma$ {[}-{]}  & $\pi$ {[}Pa{]} \tabularnewline
\hline 
Water  & 3.0  & 7.499e+8 \tabularnewline
Air  & 1.4  & 0\tabularnewline
\hline 
\end{tabular}\caption{Material parameters for water and air.\label{SC-TMP2}}
\end{table}

\begin{figure}[ht]
\centering{}\includegraphics[width=0.8\linewidth]{testcase2} \caption{Initialization of the water-air shock-contact interaction\label{SC-2}}
\end{figure}


\section{Gas bubble oscillations}

In this section we apply the random projection scheme to a bubble
oscillations test case described in\cite{HMM08,MBKKH09,Ma10}. Despite
the quasi one-dimensional framework, the test case implies very long
computations and very fine meshes. We compare with the results obtained
with the RGFM on arbitrary refined grids. 

\section{Conclusion}

In this paper, we have proposed a new scheme for computing two-fluid
flows. The pressure oscillations at the interface are avoided thanks
to a Lagrange and projection approach. In the Lagrange step, the contact
waves are perfectly resolved and the interface is not smeared. In
the projection step, we employ a random sampling strategy. The resulting
scheme preserves the constant velocity-pressure states and the interface
is solved within one grid point.

The whole approach performs well for weak shocks. But in presence
of strong shocks, it appears to be oscillating. Therefore, we had
to adapt the projection step and only apply it at the two-fluid interface,
which is located thanks to the jumps of the colour function. We proposed
then numerical results that demonstrate the good convergence of the
scheme, despite that it is not conservative. We surprisingly observed
this convergence property for other non-conservative schemes for two-fluid
flows.

Finally, we apply our scheme to a more challenging problem, which
consists in the simulation of the oscillations of gas bubble in a
compressible liquid. Our simple scheme gives good results, even if
its precision is less than the more sophisticated RGFM coupled with
arbitrary mesh refinement.

Our prospects are in several directions:
\begin{itemize}
\item first we would like to improve the precision of the random projection
scheme. The first obvious way to do it is to couple it with a second
order MUSCL extension. This extension has to be deactivated at the
interface, in order to avoid oscillations. For the spherical bubble
computations, another way to improve the precision is to modify the
scheme in order that it becomes well-balanced. This can be done by
adapting the method described in \cite{HHM10}.
\item a challenging extension would consist in extending the random projection
scheme to two- or three-dimensional computations. This could be tried
for example by a simple directional splitting algorithm. This will
be the object of a forthcoming work.
\end{itemize}
\begin{thebibliography}{MBKKH09}
\bibitem[Abg88]{Abg88}R. Abgrall. Generalisation of the roe scheme
for the computation of mixture of perfect gases. Recherche Aérospatiale,
6:31--43, 1988.

\bibitem[AK10]{AK10}Rémi Abgrall, Smadar Karni. A comment on the
computation of non-conservative products. Journal of Computational
Physics 229 (2010) 2759--2763

\bibitem[BHR03]{BHR03}T. Barberon, P. Helluy, and S. Rouy. Practical
computation of axisymmetrical multifluid flows. International Journal
of Finite Volumes, 1(1) :1--34, 2003.

\bibitem[CC08]{CC08}Chalons, C.; Coquel, F. Capturing infinitely
sharp discrete shock profiles with the Godunov scheme. Hyperbolic
problems: theory, numerics, applications, 363--370, Springer, Berlin,
2008.

\bibitem[CG07]{CG07}Chalons, Christophe; Goatin, Paola Transport-equilibrium
schemes for computing contact discontinuities in traffic flow modeling.
Commun. Math. Sci. 5 (2007), no. 3, 533--551.

\bibitem[FAMO99]{FAMO99}Fedkiw, Ronald P.; Aslam, Tariq; Merriman,
Barry; Osher, Stanley A non-oscillatory Eulerian approach to interfaces
in multimaterial flows (the ghost fluid method). J. Comput. Phys.
152 (1999), no. 2, 457--492.

\bibitem[Gl65]{Gl65}Glimm, James Solutions in the large for nonlinear
hyperbolic systems of equations. Comm. Pure Appl. Math. 18 1965 697--715.

\bibitem[HHM10]{HHM10}P Helluy, J Herard, H Mathis. A well-balanced
approximate Riemann solver for variable cross-section compressible
flows. AIAA Proc. 22-25 Jun, 2009

\bibitem[HMM08]{HMM08}Helluy, Philippe; Mathis, Hélène; Müller, Siegfried
An ALE averaging approach for the computing of bubble oscillations.
Finite volumes for complex applications V, 487--494, ISTE, London,
2008.

\bibitem[HL94]{HL94}Hou, Thomas Y.; LeFloch, Philippe G. Why nonconservative
schemes converge to wrong solutions: error analysis. Math. Comp. 62
(1994), no. 206, 497--530.

\bibitem[Kar94]{Kar94}Karni, Smadar Multicomponent flow calculations
by a consistent primitive algorithm. J. Comput. Phys. 112 (1994),
no. 1, 31--43.

\bibitem[KL10]{KL10}Kokh, S.; Lagoutière, F. An anti-diffusive numerical
scheme for the simulation of interfaces between compressible fluids
by means of a five-equation model. J. Comput. Phys. 229 (2010), no.
8, 2773--2809.

\bibitem[Ma10]{Ma10}Mathis, H. Étude théorique et numérique des écoulements
avec transition de phase. PhD thesis, Université de Strasbourg, 2010.
\url{http://tel.archives-ouvertes.fr/IRMA/tel-00516683/fr/}

\bibitem[MBKKH09]{MBKKH09}S. Müller, M. Bachmann, D. Kröninger, T.
Kurz, P. Helluy. Comparison and validation of compressible flow simulations
of laser-induced cavitation bubbles. Computers \& Fluids. 38 (9):1850-1862,
2009.

\bibitem[SA99a]{SA99a}Saurel, Richard; Abgrall, Rémi A simple method
for compressible multifluid flows. SIAM J. Sci. Comput. 21 (1999),
no. 3, 1115--1145

\bibitem[SA99b]{SA99b}Saurel, Richard; Abgrall, Rémi A multiphase
Godunov method for compressible multifluid and multiphase flows. J.
Comput. Phys. 150 (1999), no. 2, 425--467.

\bibitem[Tor99]{Tor99}Toro, Eleuterio F. Riemann solvers and numerical
methods for fluid dynamics. A practical introduction. Second edition.
Springer-Verlag, Berlin, 1999.

\bibitem[WK05]{WK05}Wackers, Jeroen; Koren, Barry A fully conservative
model for compressible two-fluid flow. 8th ICFD Conference on Numerical
Methods for Fluid Dynamics. Part 2. Internat. J. Numer. Methods Fluids
47 (2005), no. 10-11, 1337--1343

\bibitem[WLK06]{WLK06}Wang, C. W. ; Liu, T. G. ; Khoo, B. C. A real
ghost fluid method for the simulation of multimedium compressible
flow. SIAM J. Sci. Comput. 28 (2006), no. 1, 278--302

\end{thebibliography}

\end{document}
