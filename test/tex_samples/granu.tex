%% LyX 1.1 created this file.  For more info, see http://www.lyx.org/.
%% Do not edit unless you really know what you are doing.
\documentclass[a4paper]{amsart}
%\usepackage[dvips]{geometry}
\usepackage{hyperref}
\usepackage{pdfsync}
%\documentclass{amsart}
\usepackage{amsmath}
\usepackage{rotating}
\usepackage{amssymb}
\usepackage[T1]{fontenc}
\usepackage[utf8]{inputenc}
\def \dpar #1#2{\displaystyle{\partial #1\over \partial #2}}
%\def \text #1{\hbox{#1}}
\usepackage{graphics}
\usepackage{float}
\usepackage{subfigure}
\usepackage{ae}
\usepackage{comment}

\excludecomment{out}

\makeatletter


\newcommand{\p}{\partial}
\newtheorem{remark}{Remark}
\newtheorem{example}{Example}
\newtheorem{theorem}{Theorem}
\newtheorem{proposition}{Proposition}


\makeatother

\begin{document}

\title{Hyperbolic relaxation models for granular flows}


\author{Thierry Gallouët, Philippe Helluy, Jean-Marc Hérard, Julien Nussbaum}\address{IRMA, 7 rue Descartes, 67084 Strasbourg Cedex}

\begin{abstract}
In this work we describe an efficient model for the simulation of a
two-phase flow made of a gas and a granular solid. The starting point is the two-velocity
two-pressure model of Baer-Nunziato \cite{baer86}. The model is supplemented by
a relaxation source term in order
to take into account the pressure equilibrium between the two phases and
the granular stress in the solid phase.  We show that the relaxation
process can be made thermodynamically coherent with an adequate  choice of the granular stress.
 We then propose a numerical scheme based on a splitting approach. Each step of the time marching
algorithm is made of two stages. In the first stage, the homogeneous convection equations are solved
 by a standard finite volume Rusanov scheme. In the second stage, the volume fraction
is updated in order to take into account the equilibrium source term.
The whole procedure is entropy dissipative.
For simplified pressure laws (stiffened gas laws) we are able to prove that the approximated volume
fraction stays within its natural bounds.
\end{abstract}


\subjclass{76M12, 65M12}
%
\keywords{two-phase flow, hyperbolicity, relaxation, finite volume,
entropy.}

\date{}


\maketitle

\bibliographystyle{plain}

%\tableofcontents

\section*{Introduction}
We are interested in the numerical modeling of a
two-phase (granular-gas) flow with two velocities and two pressures
$p_1$ and $p_2$. In one space dimension, the model is made up of seven non-homogeneous
 partial differential equations: two mass balance laws,
two momentum balance laws, two energy balance laws and one volume
fraction evolution equation. It is similar to the initial model proposed by
Baer-Nunziato \cite{baer86}. The main feature of this model is that
the left hand side of the equations is hyperbolic.
 This property is very important because it ensures the
mathematical stability of the model.

However, in many industrial applications it is not realistic to
admit two independent pressures. Generally, an algebraic relation
between the two pressures is assumed. An example (among many others) of such a modeling
in the framework of internal ballistics is given by Gough in \cite{gough79}.
For a general presentation of two-phase flow models, we refer to the book of Gidaspow \cite{gidaspow94}.
The relation between the two pressures is classically of the form $p_2=p_1+R$
where $R$ is the granular stress. In the general case, the granular stress depends
 on all the thermodynamic variables of the two phases.

Because of the pressure relation, the system is now overdetermined.
The volume fraction equation can be eliminated and a six-equation model is obtained.
 Unfortunately, the new model has a reduced hyperbolicity domain. The worst situation
corresponds to a vanishing granular stress $R=0$. In this case, the model
is almost never hyperbolic.

In the case of a vanishing granular stress, several authors have proposed to relax
the algebraic relation $p_2=p_1$ by adding an adequate source term to the volume fraction
evolution \cite{toro89}, \cite{saurel99b}, \cite{ghs04}, \cite{hurisse}, \emph{etc}.
An important parameter of the source term is the characteristic equilibrium time.
When the equilibration time tends to zero, the six equation model is recovered. When the equilibrium
 time does not vanish the stability of the model is expected.

In this paper, we extend the relaxation approach to non-vanishing granular
stress $R>0$.  With a positive granular stress, the hyperbolicity domain of the six-equation model is
 slightly extended. But it is generally not possible to remove all the elliptic regions. Therefore, we
apply a relaxation source term to the volume fraction evolution equation of the seven-equation model.
This source term takes into account the granular stress (see (\ref{formsource})).
When the relaxation parameter $\tau_p$ tends to $0$, we recover the equilibrium six-equation model. When the relaxation parameter $\tau_p >0$, the stability of the model is recovered.

An important aspect of the model is that the granular stress cannot be chosen
arbitrarily once the pressure law of the solid phase is fixed. Indeed, it has to satisfy some thermodynamical relations in order that an entropy dissipation equation can be established.
We illustrate this fact when the equation of state of the solid phase is a stiffened gas law.
We propose a very simple but useful expression (see (\ref{nice})) for the granular stress, which is mathematically and physically relevant.




Then we propose a  numerical method to solve the relaxed system. For that purpose, we use a splitting algorithm. Each time step of the algorithm consists in:
\begin{itemize}
  \item evolving the seven equation model without the source term;
  \item solving the relaxed pressure equilibrium with
  granular stress;
  \item solving the other source terms.
\end{itemize}
In the second stage the relaxed pressure equilibrium implies to solve an
update for the volume fraction, keeping the conserved
variables constant. Under some monotony hypothesis on the granular stress,
and when the pressure laws of the two phases are stiffened gas laws,
we are able to prove the existence and uniqueness of the new volume
fraction in the interval $]0,1[$.

The relaxed equilibrium becomes an exact equilibrium
when the relaxation parameter $\tau_p=0$. Thus our method can also be used to approximate the
equilibrium six-equation model on coarse meshes, which is important in industrial applications.

Finally, we propose some numerical experiments. For academic test cases,
we highlight some behavior of the relaxed approach in the case of
a non-stable (elliptic) case. We then compare the results of our
new approach with the standard Gough model \cite{gough79,nuss-hell06} in the case of a simplified
internal ballistics problem.



\section{Notations and model}
We consider a two-phase flow of a granular solid mixed with a compressible gas.
The solid is denoted by the index (2) and the gas by the index (1).
For more generality, the solid is supposed to be compressible. The
unknowns are, for each phase $k=1,2$, the partial density $\rho_k$,
the velocity $u_k$, the internal energy $e_k$. The volume fractions
$\alpha_k$ satisfy $\alpha_1+\alpha_2=1$. The gas volume fraction
$\alpha_1$ is also called the \emph{porosity} in the context of granular flows. The pressure of each
phase is given by an equation of state of the form
% MathType!MTEF!2!1!+-
% faaafaart1ev1aaat0uyJj1BTfMBaerbuLwBLnhiov2DGi1BTfMBae
% XatLxBI9gBaerbd9wDYLwzYbItLDharqqtubsr4rNCHbGeaGqiVu0J
% e9sqqrpepC0xbbL8F4rqqrFfpeea0xe9Lq-Jc9vqaqpepm0xbba9pw
% e9Q8fs0-yqaqpepae9pg0FirpepeKkFr0xfr-xfr-xb9adbaqaaeGa
% ciGaaiaabeqaamaabaabaaGcbaGaamiCamaaBaaaleaacaWGRbaabe
% aakiabg2da9iaadchadaWgaaWcbaGaam4AaaqabaGccaGGOaGaeqyW
% di3aaSbaaSqaaiaadUgaaeqaaOGaaiilaiaadwgadaWgaaWcbaGaam
% 4AaaqabaGccaGGPaaaaa!3D9F!
\begin{equation}\label{stifgas}
p_k  = \psi_k (\rho _k ,e_k ).
\end{equation}
We note $\alpha_k \rho_k = m_k$. The principles of mass, momentum and energy conservation imply that some source terms will cancel. Thus, we introduce the following notation for the sign function% MathType!MTEF!2!1!+-
% faaafaart1ev1aaat0uyJj1BTfMBaerbuLwBLnhiov2DGi1BTfMBae
% XatLxBI9gBaerbtfMDVLxzGWurubqefe0yHrwzTLhif52zYfMDLrgi
% mvevaebbnrfifHhDYfgasaacH8srps0lbbf9q8WrFfeuY-Hhbbf9v8
% qqaqFr0xc9pk0xbba9q8WqFfea0-yr0RYxir-Jbba9q8aq0-yq-He9
% q8qqQ8frFve9Fve9Ff0dmeaabaqaciGacaGaaeqabaWaaqaafaaake
% aacqaHdpWCdaWgaaWcbaGaam4AaaqabaGccaGG6aGaeyypa0Zaaiqa
% aqaabeqaaiabgUcaRiaaigdacaqGGaGaaeyAaiaabAgacaqGGaGaam
% 4Aaiabg2da9iaaigdacaGGSaaabaGaeyOeI0IaaGymaiaabccacaqG
% PbGaaeOzaiaabccacaWGRbGaeyypa0JaaGOmaiaac6caaaGaay5Eaa
% aaaa!4DEA!
\begin{equation}\label{}
\sigma _k : = \left\{ \begin{gathered}
   + 1{\text{ if }}k = 1, \hfill \\
   - 1{\text{ if }}k = 2. \hfill \\ 
\end{gathered}  \right.
\end{equation}

The balance of mass, momentum and energy read% MathType!MTEF!2!1!+-
% feaafiart1ev1aaatCvAUfeBSjuyZL2yd9gzLbvyNv2CaerbuLwBLn
% hiov2DGi1BTfMBaeXatLxBI9gBaerbd9wDYLwzYbItLDharqqtubsr
% 4rNCHbGeaGqiVu0Je9sqqrpepC0xbbL8F4rqqrFfpeea0xe9Lq-Jc9
% vqaqpepm0xbba9pwe9Q8fs0-yqaqpepae9pg0FirpepeKkFr0xfr-x
% fr-xb9adbaqaaeGaciGaaiaabeqaamaabaabaaGceaqabeaacaWGTb
% WaaSbaaSqaaiaadUgacaGGSaGaamiDaaqabaGccqGHRaWkcaGGOaGa
% amyBamaaBaaaleaacaWGRbaabeaakiaadwhadaWgaaWcbaGaam4Aaa
% qabaGccaGGPaWaaSbaaSqaaiaadIhaaeqaaOGaeyypa0JaeyySaeRa
% amytaiaacYcaaeaacaGGOaGaamyBamaaBaaaleaacaWGRbaabeaaki
% aadwhadaWgaaWcbaGaam4AaaqabaGccaGGPaWaaSbaaSqaaiaadsha
% aeqaaOGaey4kaSIaaiikaiaad2gadaWgaaWcbaGaam4AaaqabaGcca
% WG1bWaa0baaSqaaiaadUgaaeaacaaIYaaaaOGaey4kaSIaeqySde2a
% aSbaaSqaaiaadUgaaeqaaOGaamiCamaaBaaaleaacaWGRbaabeaaki
% aacMcadaWgaaWcbaGaamiEaaqabaGccqGHsislcaWGWbWaaSbaaSqa
% aiaadMeaaeqaaOGaeqySde2aaSbaaSqaaiaadUgacaGGSaGaamiEaa
% qabaGccqGH9aqpcqGHXcqScaWGrbGaaiilaaqaaiaacIcacaWGTbWa
% aSbaaSqaaiaadUgaaeqaaOGaamyramaaBaaaleaacaWGRbaabeaaki
% aacMcadaWgaaWcbaGaamiDaaqabaGccqGHRaWkcaGGOaGaaiikaiaa
% d2gadaWgaaWcbaGaam4AaaqabaGccaWGfbWaaSbaaSqaaiaadUgaae
% qaaOGaey4kaSIaeqySde2aaSbaaSqaaiaadUgaaeqaaOGaamiCamaa
% BaaaleaacaWGRbaabeaakiaacMcacaWG1bWaaSbaaSqaaiaadUgaae
% qaaOGaaiykamaaBaaaleaacaWG4baabeaakiabgUcaRiaadchadaWg
% aaWcbaGaamysaaqabaGccqaHXoqydaWgaaWcbaGaam4AaiaacYcaca
% WG0baabeaakiabg2da9iabgglaXkaadofacaGGSaaabaGaeqySde2a
% aSbaaSqaaiaadUgacaGGSaGaamiDaaqabaGccqGHRaWkcaWG2bWaaS
% baaSqaaiaadMeaaeqaaOGaeqySde2aaSbaaSqaaiaadUgacaGGSaGa
% amiEaaqabaGccqGH9aqpcqGHXcqScaWGqbGaaiilaaaaaa!99CA!
\begin{equation}\label{dipha}
\begin{gathered}
  m_{k,t}  + (m_k u_k )_x  =  \sigma_k M, \hfill \\
  (m_k u_k )_t  + (m_k u_k^2  + \alpha _k p_k )_x  - p_I \alpha _{k,x}  =  \sigma_k Q, \hfill \\
  (m_k E_k )_t  + ((m_k E_k  + \alpha _k p_k )u_k )_x  + p_I \alpha _{k,t}  =  \sigma_k S, \hfill \\
  \alpha _{k,t}  + v_I \alpha _{k,x}  =  \sigma_k P, \hfill \\
\end{gathered}
\end{equation}

where
\begin{equation}
E_k  = e_k  + \frac{{u_k^2 }} {2}.
\end{equation}


For simplicity, we restrict ourselves to one-dimensional equations, but it is possible to extend the results of this paper to higher space dimensions. The right hand side terms $M$, $Q$, $P$, $S$ are internal exchange source terms
that will be discussed later. For the moment, introducing the vector of the seven main unknowns of the system (\ref{dipha})% MathType!MTEF!2!1!+-
% faaafaart1ev1aaat0uyJj1BTfMBaerbuLwBLnhiov2DGi1BTfMBae
% XatLxBI9gBaerbd9wDYLwzYbItLDharqqtubsr4rNCHbGeaGqiVu0J
% e9sqqrpepC0xbbL8F4rqqrFfpeea0xe9Lq-Jc9vqaqpepm0xbba9pw
% e9Q8fs0-yqaqpepae9pg0FirpepeKkFr0xfr-xfr-xb9adbaqaaeGa
% ciGaaiaabeqaamaabaabaaGcbaGaam4vaiabg2da9maabmaabaGaam
% yBamaaBaaaleaacaaIXaaabeaakiaacYcacaWGTbWaaSbaaSqaaiaa
% igdaaeqaaOGaamyDamaaBaaaleaacaaIXaaabeaakiaacYcacaWGTb
% WaaSbaaSqaaiaaigdaaeqaaOGaamyramaaBaaaleaacaaIXaaabeaa
% kiaacYcacaWGTbWaaSbaaSqaaiaaikdaaeqaaOGaaiilaiaad2gada
% WgaaWcbaGaaGOmaaqabaGccaWG1bWaaSbaaSqaaiaaikdaaeqaaOGa
% aiilaiaad2gadaWgaaWcbaGaaGOmaaqabaGccaWGfbWaaSbaaSqaai
% aaikdaaeqaaOGaaiilaiabeg7aHnaaBaaaleaacaaIXaaabeaaaOGa
% ayjkaiaawMcaamaaCaaaleqabaGaamivaaaakiaacYcaaaa!4FE2!
\begin{equation}\label{}
W = \left( {m_1 ,m_1 u_1 ,m_1 E_1 ,m_2 ,m_2 u_2 ,m_2 E_2 ,\alpha _1 } \right)^T ,
\end{equation}
we only suppose that $M$, $Q$, $P$, $S$ are general functions of $W$. Here, we recall that the sign $\sigma_k = +1$ if $k=1$ and $\sigma_k=-1$ if
$k=2$. We also suppose that there are no external force
and energy source (this explains the $\sigma_k$ signs in the source
terms).
The quantities $p_I$ and $v_I$ are respectively the interface
pressure and the interface velocity. In this paper, we take the
special choice of Baer-Nunziato
\begin{equation}\label{eq-bn-choice}
\begin{gathered}
  p_I  = p_1 , \hfill \\
  v_I  = u_2 . \hfill \\
\end{gathered}
\end{equation}

This model for the interface pressure and velocity is physically relevant when the solid phase is dilute in the gas-solid mixture (see \cite{baer86}). Of course, due to the perfect symmetry of the PDE system with respect to the phase index $k$, a symmetric choice is also possible
\begin{equation}\label{pkchoice}
\begin{gathered}
  p_I  = p_2 , \hfill \\
  v_I  = u_1 , \hfill \\
\end{gathered}
\end{equation}
when the solid phase is packed. It is also possible to consider interface pressure and velocity that are convex linear combinations of the phase pressures and velocities \cite{saurel99b}, \cite{coquel02}, \cite{ghs04}. Quite surprisingly, the convex linear combination is not generally considered as the good physical modeling.

In this paper, we focus on the dilute case (\ref{eq-bn-choice}) which enjoys good properties\footnote{The packed model (\ref{pkchoice}) and the convex combination models given in \cite{ghs04} enjoy the same properties.}:
\begin{itemize}
\item the left hand side of the system is hyperbolic (the proof is recalled in Section \ref{app});
\item this choice ensures that the non-conservative products are
well defined (at least in the non-resonant case). This is due to the fact that the volume fraction only
jumps in the linearly degenerated field associated to the eigenvalue $v_I$. In a linearly degenerated
field, the jump relations are simply provided by the Riemann
invariants of this field. See \cite{coquel02}, \cite{ghs04};
\item in the applications,
the Baer-Nunziato model is particularly adapted to granular flows. See \cite{baer86} and included references.
\end{itemize}



Using the volume fraction equation, the time derivative $\alpha_{k,t}$ can be
replaced by space derivatives
\begin{equation}
\begin{gathered}
  m_{k,t}  + (m_k u_k )_x  =  \sigma_k M, \hfill \\
  (m_k u_k )_t  + (m_k u_k^2 )_x  + \alpha _k p_{k,x}  + (p_k  - p_I )\alpha _{k,x}  =  \sigma_k Q, \hfill \\
  (m_k E_k )_t  + (m_k E_k u_k )_x  + \alpha _k (p_k u_k )_x  + (p_k u_k  - p_I v_I )\alpha _{k,x}  =  \sigma_k S -\sigma_k p_I P, \hfill \\
  \alpha _{k,t}  + v_I \alpha _{k,x}  =  \sigma_k P. \hfill \\
\end{gathered}
\end{equation}



The equations can then be written under the form of a first
order non-conservative and non-homogeneous system % MathType!MTEF!2!1!+-
% feaafiart1ev1aaatCvAUfeBSjuyZL2yd9gzLbvyNv2CaerbuLwBLn
% hiov2DGi1BTfMBaeXatLxBI9gBaerbd9wDYLwzYbItLDharqqtubsr
% 4rNCHbGeaGqiVu0Je9sqqrpepC0xbbL8F4rqqrFfpeea0xe9Lq-Jc9
% vqaqpepm0xbba9pwe9Q8fs0-yqaqpepae9pg0FirpepeKkFr0xfr-x
% fr-xb9adbaqaaeGaciGaaiaabeqaamaabaabaaGcbaGaam4vamaaBa
% aaleaacaWG0baabeaakiabgUcaRiaadAeacaGGOaGaam4vaiaacMca
% daWgaaWcbaGaamiEaaqabaGccqGHRaWkcaWGbbGaaiikaiaadEfaca
% GGPaGaamitaiaacIcacaWGxbGaaiykamaaBaaaleaacaWG4baabeaa
% kiabg2da9iaadofacaGGOaGaam4vaiaacMcacaGGSaaaaa!49E6!
\begin{equation}
W_t  + F(W)_x  + A(W)L(W)_x  = \Sigma(W),
\end{equation}

with% MathType!MTEF!2!1!+-
% faaafaart1ev1aaat0uyJj1BTfMBaerbuLwBLnhiov2DGi1BTfMBae
% XatLxBI9gBaerbd9wDYLwzYbItLDharqqtubsr4rNCHbGeaGqiVu0J
% e9sqqrpepC0xbbL8F4rqqrFfpeea0xe9Lq-Jc9vqaqpepm0xbba9pw
% e9Q8fs0-yqaqpepae9pg0FirpepeKkFr0xfr-xfr-xb9adbaqaaeGa
% ciGaaiaabeqaamaabaabaaGceaqabeaacaWGxbGaeyypa0ZaaeWaae
% aacaWGTbWaaSbaaSqaaiaaigdaaeqaaOGaaiilaiaad2gadaWgaaWc
% baGaaGymaaqabaGccaWG1bWaaSbaaSqaaiaaigdaaeqaaOGaaiilai
% aad2gadaWgaaWcbaGaaGymaaqabaGccaWGfbWaaSbaaSqaaiaaigda
% aeqaaOGaaiilaiaad2gadaWgaaWcbaGaaGOmaaqabaGccaGGSaGaam
% yBamaaBaaaleaacaaIYaaabeaakiaadwhadaWgaaWcbaGaaGOmaaqa
% baGccaGGSaGaamyBamaaBaaaleaacaaIYaaabeaakiaadweadaWgaa
% WcbaGaaGOmaaqabaGccaGGSaGaeqySde2aaSbaaSqaaiaaigdaaeqa
% aaGccaGLOaGaayzkaaWaaWbaaSqabeaacaWGubaaaOGaaiilaaqaai
% aadAeacaGGOaGaam4vaiaacMcacqGH9aqpdaqadaqaaiaad2gadaWg
% aaWcbaGaaGymaaqabaGccaWG1bWaaSbaaSqaaiaaigdaaeqaaOGaai
% ilaiaad2gadaWgaaWcbaGaaGymaaqabaGccaWG1bWaa0baaSqaaiaa
% igdaaeaacaaIYaaaaOGaaiilaiaad2gadaWgaaWcbaGaaGymaaqaba
% GccaWGfbWaaSbaaSqaaiaaigdaaeqaaOGaamyDamaaBaaaleaacaaI
% XaaabeaakiaacYcacaWGTbWaaSbaaSqaaiaaikdaaeqaaOGaamyDam
% aaBaaaleaacaaIYaaabeaakiaacYcacaWGTbWaaSbaaSqaaiaaikda
% aeqaaOGaamyDamaaDaaaleaacaaIYaaabaGaaGOmaaaakiaacYcaca
% WGTbWaaSbaaSqaaiaaikdaaeqaaOGaamyramaaBaaaleaacaaIYaaa
% beaakiaadwhadaWgaaWcbaGaaGOmaaqabaGccaGGSaGaaGimaaGaay
% jkaiaawMcaamaaCaaaleqabaGaamivaaaakiaacYcaaeaacaWGmbGa
% aiikaiaadEfacaGGPaGaeyypa0ZaaeWaaeaacaWGWbWaaSbaaSqaai
% aaigdaaeqaaOGaaiilaiaadchadaWgaaWcbaGaaGymaaqabaGccaWG
% 1bWaaSbaaSqaaiaaigdaaeqaaOGaaiilaiaadchadaWgaaWcbaGaaG
% OmaaqabaGccaGGSaGaamiCamaaBaaaleaacaaIYaaabeaakiaadwha
% daWgaaWcbaGaaGOmaaqabaGccaGGSaGaeqySde2aaSbaaSqaaiaaig
% daaeqaaaGccaGLOaGaayzkaaWaaWbaaSqabeaacaWGubaaaOGaaiil
% aaqaaiaadgeacaGGOaGaam4vaiaacMcacaWGmbGaaiikaiaadEfaca
% GGPaWaaSbaaSqaaiaadIhaaeqaaOGaeyypa0JaaiikaiaaicdacaGG
% SaGaeqySde2aaSbaaSqaaiaaigdaaeqaaOGaamiCamaaBaaaleaaca
% aIXaGaaiilaiaadIhaaeqaaOGaaiilaiabeg7aHnaaBaaaleaacaaI
% XaaabeaakiaacIcacaWGWbWaaSbaaSqaaiaaigdaaeqaaOGaamyDam
% aaBaaaleaacaaIXaaabeaakiaacMcadaWgaaWcbaGaamiEaaqabaGc
% cqGHRaWkcaWGWbWaaSbaaSqaaiaaigdaaeqaaOGaaiikaiaadwhada
% WgaaWcbaGaaGymaaqabaGccqGHsislcaWG1bWaaSbaaSqaaiaaikda
% aeqaaOGaaiykaiabeg7aHnaaBaaaleaacaaIXaGaaiilaiaadIhaae
% qaaOGaaiilaaqaaiaaicdacaGGSaGaeqySde2aaSbaaSqaaiaaikda
% aeqaaOGaamiCamaaBaaaleaacaaIYaGaaiilaiaadIhaaeqaaOGaey
% 4kaSIaaiikaiaadchadaWgaaWcbaGaaGOmaaqabaGccqGHsislcaWG
% WbWaaSbaaSqaaiaaigdaaeqaaOGaaiykaiabeg7aHnaaBaaaleaaca
% aIYaGaaiilaiaadIhaaeqaaOGaaiilaiabeg7aHnaaBaaaleaacaaI
% YaaabeaakiaacIcacaWGWbWaaSbaaSqaaiaaikdaaeqaaOGaamyDam
% aaBaaaleaacaaIYaaabeaakiaacMcadaWgaaWcbaGaamiEaaqabaGc
% cqGHRaWkcaWG1bWaaSbaaSqaaiaaikdaaeqaaOGaaiikaiaadchada
% WgaaWcbaGaaGOmaaqabaGccqGHsislcaWGWbWaaSbaaSqaaiaaigda
% aeqaaOGaaiykaiabeg7aHnaaBaaaleaacaaIYaGaaiilaiaadIhaae
% qaaOGaaiilaiaadAhadaWgaaWcbaGaaGOmaaqabaGccqaHXoqydaWg
% aaWcbaGaaGymaiaacYcacaWG4baabeaakiaacMcadaahaaWcbeqaai
% aadsfaaaGccaGGSaaabaGaam4uaiaacIcacaWGxbGaaiykaiabg2da
% 9maabeaabaGaamytaiaacIcacaWGxbGaaiykaiaacYcacaWGrbGaai
% ikaiaadEfacaGGPaGaaiilaiaadofacaGGOaGaam4vaiaacMcacqGH
% sislcaWGWbWaaSbaaSqaaiaaigdaaeqaaOGaamiuaiaacIcacaWGxb
% GaaiykaiaacYcaaiaawIcaaaqaamaabiaabaGaeyOeI0Iaamytaiaa
% cIcacaWGxbGaaiykaiaacYcacqGHsislcaWGrbGaaiikaiaadEfaca
% GGPaGaaiilaiabgkHiTiaadofacaGGOaGaam4vaiaacMcacqGHRaWk
% caWGWbWaaSbaaSqaaiaaigdaaeqaaOGaamiuaiaacIcacaWGxbGaai
% ykaiaacYcacaWGqbGaaiikaiaadEfacaGGPaaacaGLPaaadaahaaWc
% beqaaiaadsfaaaGccaGGUaaaaaa!19E3!
\begin{equation}\label{}
\begin{gathered}
  W = \left( {m_1 ,m_1 u_1 ,m_1 E_1 ,m_2 ,m_2 u_2 ,m_2 E_2 ,\alpha _1 } \right)^T , \hfill \\
  F(W) = \left( {m_1 u_1 ,m_1 u_1^2 ,m_1 E_1 u_1 ,m_2 u_2 ,m_2 u_2^2 ,m_2 E_2 u_2 ,0} \right)^T , \hfill \\
  L(W) = \left( {p_1 ,p_1 u_1 ,p_2 ,p_2 u_2 ,\alpha _1 } \right)^T , \hfill \\
  A(W)L(W)_x  = (0,\alpha _1 p_{1,x} ,\alpha _1 (p_1 u_1 )_x  + p_1 (u_1  - u_2 )\alpha _{1,x} , \hfill \\
  0,\alpha _2 p_{2,x}  + (p_2  - p_1 )\alpha _{2,x} ,\alpha _2 (p_2 u_2 )_x  + u_2 (p_2  - p_1 )\alpha _{2,x} ,v_2 \alpha _{1,x} )^T , \hfill \\
  \Sigma(W) = \left( {M(W),Q(W),S(W) - p_1 P(W),} \right. \hfill \\
  \left. { - M(W), - Q(W), - S(W) + p_1 P(W),P(W)} \right)^T . \hfill \\ 
\end{gathered} 
\end{equation}

Let us note that this writing is not unique. We have chosen a formulation in which the non-conservative terms
vanish when the pressures and velocities are constant, \emph{i.e.}% MathType!MTEF!2!1!+-
% faaafaart1ev1aaat0uyJj1BTfMBaerbuLwBLnharmWu51MyVXgaru
% WqVvNCPvMCaebbnrfifHhDYfgasaacH8srps0lbbf9q8WrFfeuY-Hh
% bbf9v8qqaqFr0xc9pk0xbba9q8WqFfea0-yr0RYxir-Jbba9q8aq0-
% yq-He9q8qqQ8frFve9Fve9Ff0dmeaabaqaciGacaGaaeqabaWaaeaa
% eaaakeaacaWGWbWaaSbaaSqaaiaaigdaaeqaaOGaeyypa0JaamiCam
% aaBaaaleaacaaIYaaabeaakiabg2da9iaadchadaWgaaWcbaGaaGim
% aaqabaGccqGH9aqpcaqGdbGaae4CaiaabshacaqGGaGaaeyyaiaab6
% gacaqGKbGaaeiiaiaadwhadaWgaaWcbaGaaGymaaqabaGccqGH9aqp
% caWG1bWaaSbaaSqaaiaaikdaaeqaaOGaeyypa0JaamyDamaaBaaale
% aacaaIWaaabeaakiabg2da9iaaboeacaqGZbGaaeiDaiaab6caaaa!4863!
\begin{equation}
p_1  = p_2  = p_0  = {\text{Cst and }}u_1  = u_2  = u_0  = {\text{Cst}}{\text{.}}
\end{equation}
This particular representation is useful to design numerical schemes that perfectly preserve
states where the velocity and the pressure are constant (see the definitions (\ref{rusa}), (\ref{rusa2})).

\section{Entropy dissipation}In this section, we establish an entropy dissipation
equation. This equation is very important because it permits to
select the source terms that are compatible with the second
principle of thermodynamics.

For that purpose we first rewrite the system as follows
% MathType!MTEF!2!1!+-
% feaafiart1ev1aaatCvAUfeBSjuyZL2yd9gzLbvyNv2CaerbuLwBLn
% hiov2DGi1BTfMBaeXatLxBI9gBaerbd9wDYLwzYbItLDharqqtubsr
% 4rNCHbGeaGqiVu0Je9sqqrpepC0xbbL8F4rqqrFfpeea0xe9Lq-Jc9
% vqaqpepm0xbba9pwe9Q8fs0-yqaqpepae9pg0FirpepeKkFr0xfr-x
% fr-xb9adbaqaaeGaciGaaiaabeqaamaabaabaaGceaqabeaacaWGTb
% WaaSbaaSqaaiaadUgaaeqaaOWaaeWaaeaacaWG1bWaaSbaaSqaaiaa
% dUgacaGGSaGaamiDaaqabaGccqGHRaWkcaWG1bWaaSbaaSqaaiaadU
% gaaeqaaOGaamyDamaaBaaaleaacaWGRbGaaiilaiaadIhaaeqaaaGc
% caGLOaGaayzkaaGaey4kaSIaaiikaiabeg7aHnaaBaaaleaacaWGRb
% aabeaakiaadchadaWgaaWcbaGaam4AaaqabaGccaGGPaWaaSbaaSqa
% aiaadIhaaeqaaOGaeyOeI0IaamiCamaaBaaaleaacaaIXaaabeaaki
% abeg7aHnaaBaaaleaacaWGRbGaaiilaiaadIhaaeqaaOGaeyypa0Ja
% eyySaeRaamyuaiabloHiTjaadwhadaWgaaWcbaGaam4AaaqabaGcca
% WGnbGaaiilaaqaaiaad2gadaWgaaWcbaGaam4AaaqabaGccaGGOaGa
% amyzamaaBaaaleaacaWGRbGaaiilaiaadshaaeqaaOGaey4kaSIaam
% yDamaaBaaaleaacaWGRbaabeaakiaadwgadaWgaaWcbaGaam4Aaiaa
% cYcacaWG4baabeaakiaacMcacqGHRaWkdaWcaaqaaiaadwhadaWgaa
% WcbaGaam4AaaqabaaakeaacaaIYaaaamaadmaabaGaaiikaiaad2ga
% daWgaaWcbaGaam4AaaqabaGccaWG1bWaaSbaaSqaaiaadUgaaeqaaO
% GaaiykamaaBaaaleaacaWG0baabeaakiabgUcaRiaacIcacaWGTbWa
% aSbaaSqaaiaadUgaaeqaaOGaamyDamaaDaaaleaacaWGRbaabaGaaG
% OmaaaakiabgUcaRiabeg7aHnaaBaaaleaacaWGRbaabeaakiaadcha
% daWgaaWcbaGaam4AaaqabaGccaGGPaWaaSbaaSqaaiaadIhaaeqaaO
% Gaey4kaSIaaiikaiabeg7aHnaaBaaaleaacaWGRbaabeaakiaadcha
% daWgaaWcbaGaam4AaaqabaGccaGGPaWaaSbaaSqaaiaadIhaaeqaaa
% GccaGLBbGaayzxaaaabaGaey4kaSYaaSaaaeaacaaIXaaabaGaaGOm
% aaaacaWGTbWaaSbaaSqaaiaadUgaaeqaaOGaamyDamaaBaaaleaaca
% WGRbaabeaakmaabmaabaGaamyDamaaBaaaleaacaWGRbGaaiilaiaa
% dshaaeqaaOGaey4kaSIaamyDamaaBaaaleaacaWGRbaabeaakiaadw
% hadaWgaaWcbaGaam4AaiaacYcacaWG4baabeaaaOGaayjkaiaawMca
% aiabgUcaRiabeg7aHnaaBaaaleaacaWGRbaabeaakiaadchadaWgaa
% WcbaGaam4AaaqabaGccaWG1bWaaSbaaSqaaiaadUgacaGGSaGaamiE
% aaqabaGccqGHRaWkcaWGWbWaaSbaaSqaaiaaigdaaeqaaOGaeqySde
% 2aaSbaaSqaaiaadUgacaGGSaGaamiDaaqabaGccqGH9aqpcqGHXcqS
% caWGtbGaeS4eI0MaamyzamaaBaaaleaacaWGRbaabeaakiaad2eaca
% GGUaaaaaa!B956!
\begin{equation}
\begin{gathered}
  m_k \left( {u_{k,t}  + u_k u_{k,x} } \right) + (\alpha _k p_k )_x  - p_1 \alpha _{k,x}  =  \sigma_k Q -\sigma_k u_k M, \hfill \\
  m_k (e_{k,t}  + u_k e_{k,x} ) + \frac{{u_k }}
{2}\left[ {(m_k u_k )_t  + (m_k u_k^2  + \alpha _k p_k )_x  + (\alpha _k p_k )_x } \right] \hfill \\
   + \frac{1}
{2}m_k u_k \left( {u_{k,t}  + u_k u_{k,x} } \right) + \alpha _k p_k u_{k,x}  + p_1 \alpha _{k,t}  =  \sigma_k S -\sigma_k e_k M. \hfill \\
\end{gathered}
\end{equation}
The last equation also reads
% MathType!MTEF!2!1!+-
% feaafiart1ev1aaatCvAUfeBSjuyZL2yd9gzLbvyNv2CaerbuLwBLn
% hiov2DGi1BTfMBaeXatLxBI9gBaerbd9wDYLwzYbItLDharqqtubsr
% 4rNCHbGeaGqiVu0Je9sqqrpepC0xbbL8F4rqqrFfpeea0xe9Lq-Jc9
% vqaqpepm0xbba9pwe9Q8fs0-yqaqpepae9pg0FirpepeKkFr0xfr-x
% fr-xb9adbaqaaeGaciGaaiaabeqaamaabaabaaGceaqabeaacaWGTb
% WaaSbaaSqaaiaadUgaaeqaaOGaaiikaiaadwgadaWgaaWcbaGaam4A
% aiaacYcacaWG0baabeaakiabgUcaRiaadwhadaWgaaWcbaGaam4Aaa
% qabaGccaWGLbWaaSbaaSqaaiaadUgacaGGSaGaamiEaaqabaGccaGG
% PaGaey4kaSYaaSaaaeaacaWG1bWaaSbaaSqaaiaadUgaaeqaaaGcba
% GaaGOmaaaadaWadaqaaiaadchadaWgaaWcbaGaaGymaaqabaGccqaH
% XoqydaWgaaWcbaGaam4AaiaacYcacaWG4baabeaakiabgglaXkaadg
% facqGHRaWkcaGGOaGaeqySde2aaSbaaSqaaiaadUgaaeqaaOGaamiC
% amaaBaaaleaacaWGRbaabeaakiaacMcadaWgaaWcbaGaamiEaaqaba
% aakiaawUfacaGLDbaaaeaacqGHRaWkdaWcaaqaaiaaigdaaeaacaaI
% YaaaaiaadwhadaWgaaWcbaGaam4AaaqabaGcdaqadaqaaiabgkHiTi
% aacIcacqaHXoqydaWgaaWcbaGaam4AaaqabaGccaWGWbWaaSbaaSqa
% aiaadUgaaeqaaOGaaiykamaaBaaaleaacaWG4baabeaakiabgUcaRi
% aadchadaWgaaWcbaGaaGymaaqabaGccqaHXoqydaWgaaWcbaGaam4A
% aiaacYcacaWG4baabeaakiabgglaXkaadgfacqWItisBcaWG1bWaaS
% baaSqaaiaadUgaaeqaaOGaamytaaGaayjkaiaawMcaaiabgUcaRiab
% eg7aHnaaBaaaleaacaWGRbaabeaakiaadchadaWgaaWcbaGaam4Aaa
% qabaGccaWG1bWaaSbaaSqaaiaadUgacaGGSaGaamiEaaqabaGccqGH
% RaWkcaWGWbWaaSbaaSqaaiaaigdaaeqaaOGaeqySde2aaSbaaSqaai
% aadUgacaGGSaGaamiDaaqabaGccqGH9aqpcqGHXcqScaWGtbGaeS4e
% I0MaamyzamaaBaaaleaacaWGRbaabeaakiaad2eacaGGUaaaaaa!90ED!
\begin{equation}
\begin{gathered}
  m_k (e_{k,t}  + u_k e_{k,x} ) + \frac{{u_k }}
{2}\left[ {p_1 \alpha _{k,x}  \sigma_k Q + (\alpha _k p_k )_x } \right] \hfill \\
   + \frac{1}
{2}u_k \left( { - (\alpha _k p_k )_x  + p_1 \alpha _{k,x}  \sigma_k Q -\sigma_k u_k M} \right) + \alpha _k p_k u_{k,x}  + p_1 \alpha _{k,t}  =  \sigma_k S -\sigma_k e_k M. \hfill \\
\end{gathered}
\end{equation}
and
% MathType!MTEF!2!1!+-
% faaafaart1ev1aaat0uyJj1BTfMBaerbuLwBLnhiov2DGi1BTfMBae
% XatLxBI9gBaerbd9wDYLwzYbItLDharqqtubsr4rNCHbGeaGqiVu0J
% e9sqqrpepC0xbbL8F4rqqrFfpeea0xe9Lq-Jc9vqaqpepm0xbba9pw
% e9Q8fs0-yqaqpepae9pg0FirpepeKkFr0xfr-xfr-xb9adbaqaaeGa
% ciGaaiaabeqaamaabaabaaGceaqabeaacaWGTbWaaSbaaSqaaiaadU
% gaaeqaaOGaaiikaiaadwgadaWgaaWcbaGaam4AaiaacYcacaWG0baa
% beaakiabgUcaRiaadwhadaWgaaWcbaGaam4AaaqabaGccaWGLbWaaS
% baaSqaaiaadUgacaGGSaGaamiEaaqabaGccaGGPaGaey4kaSIaeqyS
% de2aaSbaaSqaaiaadUgaaeqaaOGaamiCamaaBaaaleaacaWGRbaabe
% aakiaadwhadaWgaaWcbaGaam4AaiaacYcacaWG4baabeaakiabgUca
% RiaadchadaWgaaWcbaGaaGymaaqabaGcdaqadaqaaiaadwhadaWgaa
% WcbaGaam4AaaqabaGccqGHsislcaWG1bWaaSbaaSqaaiaaikdaaeqa
% aaGccaGLOaGaayzkaaGaeqySde2aaSbaaSqaaiaadUgacaGGSaGaam
% iEaaqabaGccqGH9aqpaeaacqGHXcqScaWGtbGaeS4eI0Maamyzamaa
% BaaaleaacaWGRbaabeaakiaad2eacqWItisBcaWG1bWaaSbaaSqaai
% aadUgaaeqaaOGaamyuaiabgglaXoaalaaabaGaaGymaaqaaiaaikda
% aaGaamyDamaaDaaaleaacaWGRbaabaGaaGOmaaaakiaad2eacqWIti
% sBcaWGWbWaaSbaaSqaaiaaigdaaeqaaOGaamiuaaaaaa!6DBB!
\begin{equation}
\begin{gathered}
  m_k (e_{k,t}  + u_k e_{k,x} ) + \alpha _k p_k u_{k,x}  + p_1 \left( {u_k  - u_2 } \right)\alpha _{k,x}  =  \hfill \\
   \sigma_k S -\sigma_k e_k M -\sigma_k u_k Q \sigma_k \frac{1}
{2}u_k^2 M -\sigma_k p_1 P \hfill \\
\end{gathered}
\end{equation}
Finally, we obtain% MathType!MTEF!2!1!+-
% faaafaart1ev1aaat0uyJj1BTfMBaerbuLwBLnhiov2DGi1BTfMBae
% XatLxBI9gBaerbd9wDYLwzYbItLDharqqtubsr4rNCHbGeaGqiVu0J
% e9sqqrpepC0xbbL8F4rqqrFfpeea0xe9Lq-Jc9vqaqpepm0xbba9pw
% e9Q8fs0-yqaqpepae9pg0FirpepeKkFr0xfr-xfr-xb9adbaqaaeGa
% ciGaaiaabeqaamaabaabaaGceaqabeaacqaHXoqydaWgaaWcbaGaam
% 4AaiaacYcacaWG0baabeaakiabgUcaRiaadwhadaWgaaWcbaGaaGOm
% aaqabaGccqaHXoqydaWgaaWcbaGaam4AaiaacYcacaWG4baabeaaki
% abg2da9iabgglaXkaadcfacaGGSaaabaGaeqySde2aaSbaaSqaaiaa
% dUgaaeqaaOGaaiikaiabeg8aYnaaBaaaleaacaWGRbGaaiilaiaads
% haaeqaaOGaey4kaSIaamyDamaaBaaaleaacaWGRbaabeaakiabeg8a
% YnaaBaaaleaacaWGRbGaaiilaiaadIhaaeqaaOGaaiykaiabgUcaRi
% abeg8aYnaaBaaaleaacaWGRbaabeaakiaacIcacaWG1bWaaSbaaSqa
% aiaadUgaaeqaaOGaeyOeI0IaamyDamaaBaaaleaacaaIYaaabeaaki
% aacMcacqaHXoqydaWgaaWcbaGaam4AaiaacYcacaWG4baabeaakiab
% gUcaRiaad2gadaWgaaWcbaGaam4AaaqabaGccaWG1bWaaSbaaSqaai
% aadUgacaGGSaGaamiEaaqabaGccqGH9aqpcqGHXcqScaWGnbGaeS4e
% I0MaeqyWdi3aaSbaaSqaaiaadUgaaeqaaOGaamiuaiaacYcaaeaaca
% WGTbWaaSbaaSqaaiaadUgaaeqaaOWaaeWaaeaacaWG1bWaaSbaaSqa
% aiaadUgacaGGSaGaamiDaaqabaGccqGHRaWkcaWG1bWaaSbaaSqaai
% aadUgaaeqaaOGaamyDamaaBaaaleaacaWGRbGaaiilaiaadIhaaeqa
% aaGccaGLOaGaayzkaaGaey4kaSIaaiikaiabeg7aHnaaBaaaleaaca
% WGRbaabeaakiaadchadaWgaaWcbaGaam4AaaqabaGccaGGPaWaaSba
% aSqaaiaadIhaaeqaaOGaeyOeI0IaamiCamaaBaaaleaacaaIXaaabe
% aakiabeg7aHnaaBaaaleaacaWGRbGaaiilaiaadIhaaeqaaOGaeyyp
% a0JaeyySaeRaamyuaiabloHiTjaadwhadaWgaaWcbaGaam4Aaaqaba
% GccaWGnbGaaiilaaqaaiaad2gadaWgaaWcbaGaam4AaaqabaGccaGG
% OaGaamyzamaaBaaaleaacaWGRbGaaiilaiaadshaaeqaaOGaey4kaS
% IaamyDamaaBaaaleaacaWGRbaabeaakiaadwgadaWgaaWcbaGaam4A
% aiaacYcacaWG4baabeaakiaacMcacqGHRaWkcqaHXoqydaWgaaWcba
% Gaam4AaaqabaGccaWGWbWaaSbaaSqaaiaadUgaaeqaaOGaamyDamaa
% BaaaleaacaWGRbGaaiilaiaadIhaaeqaaOGaey4kaSIaamiCamaaBa
% aaleaacaaIXaaabeaakmaabmaabaGaamyDamaaBaaaleaacaWGRbaa
% beaakiabgkHiTiaadwhadaWgaaWcbaGaaGOmaaqabaaakiaawIcaca
% GLPaaacqaHXoqydaWgaaWcbaGaam4AaiaacYcacaWG4baabeaakiab
% g2da9aqaaiabgglaXkaadofacqWItisBcaWGLbWaaSbaaSqaaiaadU
% gaaeqaaOGaamytaiabloHiTjaadwhadaWgaaWcbaGaam4AaaqabaGc
% caWGrbGaeyySae7aaSaaaeaacaaIXaaabaGaaGOmaaaacaWG1bWaa0
% baaSqaaiaadUgaaeaacaaIYaaaaOGaamytaiabloHiTjaadchadaWg
% aaWcbaGaaGymaaqabaGccaWGqbaaaaa!D349!
\begin{equation}\label{prim}
\begin{gathered}
  \alpha _{k,t}  + u_2 \alpha _{k,x}  =  \sigma_k P, \hfill \\
  \alpha _k (\rho _{k,t}  + u_k \rho _{k,x} ) + \rho _k (u_k  - u_2 )\alpha _{k,x}  + m_k u_{k,x}  =  \sigma_k M -\sigma_k \rho _k P, \hfill \\
  m_k \left( {u_{k,t}  + u_k u_{k,x} } \right) + (\alpha _k p_k )_x  - p_1 \alpha _{k,x}  =  \sigma_k Q -\sigma_k u_k M, \hfill \\
  m_k (e_{k,t}  + u_k e_{k,x} ) + \alpha _k p_k u_{k,x}  + p_1 \left( {u_k  - u_2 } \right)\alpha _{k,x}  =  \hfill \\
   \sigma_k S -\sigma_k e_k M -\sigma_k u_k Q \sigma_k \frac{1}
{2}u_k^2 M -\sigma_k p_1 P \hfill \\
\end{gathered}
\end{equation}
Now, we introduce entropies $s_k$ for the two phases. The entropy
of the phase $k=1,2$ satisfies the following first order partial differential equation% MathType!MTEF!2!1!+-
% faaafaart1ev1aaat0uyJj1BTfMBaerbuLwBLnhiov2DGi1BTfMBae
% XatLxBI9gBaerbd9wDYLwzYbItLDharqqtubsr4rNCHbGeaGqiVu0J
% e9sqqrpepC0xbbL8F4rqqrFfpeea0xe9Lq-Jc9vqaqpepm0xbba9pw
% e9Q8fs0-yqaqpepae9pg0FirpepeKkFr0xfr-xfr-xb9adbaqaaeGa
% ciGaaiaabeqaamaabaabaaGceaqabeaacaWGubWaaSbaaSqaaiaadU
% gaaeqaaOGaamizaiaadohadaWgaaWcbaGaam4AaaqabaGccqGH9aqp
% caWGKbGaamyzamaaBaaaleaacaWGRbaabeaakiabgkHiTmaalaaaba
% GaamiCamaaBaaaleaacaWGRbaabeaaaOqaaiabeg8aYnaaDaaaleaa
% caWGRbaabaGaaGOmaaaaaaGccaWGKbGaeqyWdi3aaSbaaSqaaiaadU
% gaaeqaaOGaeyOeI0YaaSaaaeaacaWGsbWaaSbaaSqaaiaadUgaaeqa
% aaGcbaGaamyBamaaBaaaleaacaWGRbaabeaaaaGccaWGKbGaeqySde
% 2aaSbaaSqaaiaadUgaaeqaaaGcbaGaeyypa0JaamivamaaBaaaleaa
% caWGRbaabeaakmaabmaabaWaaSaaaeaacqGHciITcaWGZbWaaSbaaS
% qaaiaadUgaaeqaaaGcbaGaeyOaIyRaamyzamaaBaaaleaacaWGRbaa
% beaaaaaakiaawIcacaGLPaaacaWGKbGaamyzamaaBaaaleaacaWGRb
% aabeaakiabgUcaRiaadsfadaWgaaWcbaGaam4AaaqabaGcdaqadaqa
% amaalaaabaGaeyOaIyRaam4CamaaBaaaleaacaWGRbaabeaaaOqaai
% abgkGi2kabeg8aYnaaBaaaleaacaWGRbaabeaaaaaakiaawIcacaGL
% PaaacaWGKbGaeqyWdi3aaSbaaSqaaiaadUgaaeqaaOGaey4kaSIaam
% ivamaaBaaaleaacaWGRbaabeaakmaabmaabaWaaSaaaeaacqGHciIT
% caWGZbWaaSbaaSqaaiaadUgaaeqaaaGcbaGaeyOaIyRaeqySde2aaS
% baaSqaaiaadUgaaeqaaaaaaOGaayjkaiaawMcaaiaadsgacqaHXoqy
% daWgaaWcbaGaam4AaaqabaGccaGGUaaaaaa!7CED!
\begin{equation}\label{tds}
\begin{gathered}
  T_k ds_k  = de_k  - \frac{{p_k }}
{{\rho _k^2 }}d\rho _k  - \frac{{R_k }}
{{m_k }}d\alpha _k  \hfill \\
   = T_k \left( {\frac{{\partial s_k }}
{{\partial e_k }}} \right)de_k  + T_k \left( {\frac{{\partial s_k }}
{{\partial \rho _k }}} \right)d\rho _k  + T_k \left( {\frac{{\partial s_k }}
{{\partial \alpha _k }}} \right)d\alpha _k . \hfill \\
\end{gathered}
\end{equation}

The temperature of phase $k$ is denoted by $T_k$. The granular stress in the phase $k$ is denoted by $R_k$. Physically, the term $-R_k/m_k d\alpha _k$ represents the work of the granular stress due to a change of volume $d \alpha_k$. For sake of simplicity and in order to be more realistic, we assume that the "granular" stress $R_1$ vanishes in the
gas phase, thus $R_1=0$. Without ambiguity, we can also denote the granular stress in the solid phase by $R=R_2$. Generally, the granular stress $R$ depends on $(\rho_2,e_2,\alpha_2)$. A more precise formulation of the granular stress $R$ is discussed in Section \ref{num}.

The chemical potential of phase $k$ is noted $\mu_k$ and is defined by % MathType!MTEF!2!1!+-
% feaafiart1ev1aaatCvAUfeBSjuyZL2yd9gzLbvyNv2CaerbuLwBLn
% hiov2DGi1BTfMBaeXatLxBI9gBaerbd9wDYLwzYbItLDharqqtubsr
% 4rNCHbGeaGqiVu0Je9sqqrpepC0xbbL8F4rqqrFfpeea0xe9Lq-Jc9
% vqaqpepm0xbba9pwe9Q8fs0-yqaqpepae9pg0FirpepeKkFr0xfr-x
% fr-xb9adbaqaaeGaciGaaiaabeqaamaabaabaaGcbaGaeqiVd02aaS
% baaSqaaiaadUgaaeqaaOGaeyypa0JaamyzamaaBaaaleaacaWGRbaa
% beaakiabgUcaRmaalaaabaGaamiCamaaBaaaleaacaWGRbaabeaaaO
% qaaiabeg8aYnaaBaaaleaacaWGRbaabeaaaaGccqGHsislcaWGubWa
% aSbaaSqaaiaadUgaaeqaaOGaam4CamaaBaaaleaacaWGRbaabeaaaa
% a!46D2!
\begin{equation}
\mu _k  = e_k  + \frac{{p_k }} {{\rho _k }} - T_k s_k
\end{equation}
We multiply the last equation in (\ref{prim}) by $1/T_k$, the second
by
$-p_k/ \rho_k/T_k$, the first by $-R_k/T_k$ and take the sum% MathType!MTEF!2!1!+-
% faaafaart1ev1aaat0uyJj1BTfMBaerbuLwBLnhiov2DGi1BTfMBae
% XatLxBI9gBaerbd9wDYLwzYbItLDharqqtubsr4rNCHbGeaGqiVu0J
% e9sqqrpepC0xbbL8F4rqqrFfpeea0xe9Lq-Jc9vqaqpepm0xbba9pw
% e9Q8fs0-yqaqpepae9pg0FirpepeKkFr0xfr-xfr-xb9adbaqaaeGa
% ciGaaiaabeqaamaabaabaaGceaqabeaacaWGTbWaaSbaaSqaaiaadU
% gaaeqaaOGaaiikaiaadohadaWgaaWcbaGaam4AaiaacYcacaWG0baa
% beaakiabgUcaRiaadwhadaWgaaWcbaGaam4AaaqabaGccaWGZbWaaS
% baaSqaaiaadUgacaGGSaGaamiEaaqabaGccaGGPaGaey4kaSYaaSaa
% aeaacaWGWbWaaSbaaSqaaiaaigdaaeqaaOGaeyOeI0IaamiCamaaBa
% aaleaacaWGRbaabeaaaOqaaiaadsfadaWgaaWcbaGaam4Aaaqabaaa
% aOGaaiikaiaadwhadaWgaaWcbaGaam4AaaqabaGccqGHsislcaWG1b
% WaaSbaaSqaaiaaikdaaeqaaOGaaiykaiabeg7aHnaaBaaaleaacaWG
% RbGaaiilaiaadIhaaeqaaOGaeyypa0dabaWaaSaaaeaacaaIXaaaba
% GaamivamaaBaaaleaacaWGRbaabeaaaaGcdaqadaqaaiabgglaXkaa
% dcfacaGGOaGaeyOeI0IaamOuamaaBaaaleaacaWGRbaabeaakiabgU
% caRiaadchadaWgaaWcbaGaam4AaaqabaGccqGHsislcaWGWbWaaSba
% aSqaaiaaigdaaeqaaOGaaiykaiabgglaXkaad2eacaGGOaWaaSaaae
% aacaWG1bWaa0baaSqaaiaadUgaaeaacaaIYaaaaaGcbaGaaGOmaaaa
% cqGHsisldaWcaaqaaiaadchadaWgaaWcbaGaam4Aaaqabaaakeaacq
% aHbpGCdaWgaaWcbaGaam4AaaqabaaaaOGaeyOeI0IaamyzamaaBaaa
% leaacaWGRbaabeaakiaacMcacqGHXcqScaWGrbGaaiikaiabgkHiTi
% aadwhadaWgaaWcbaGaam4AaaqabaGccaGGPaGaeyySaeRaam4uaaGa
% ayjkaiaawMcaaiaacYcaaeaacaGGOaGaamyBamaaBaaaleaacaWGRb
% aabeaakiaadohadaWgaaWcbaGaam4AaaqabaGccaGGPaWaaSbaaSqa
% aiaadshaaeqaaOGaey4kaSIaaiikaiaad2gadaWgaaWcbaGaam4Aaa
% qabaGccaWG1bWaaSbaaSqaaiaadUgaaeqaaOGaam4CamaaBaaaleaa
% caWGRbaabeaakiaacMcadaWgaaWcbaGaamiEaaqabaGccqGH9aqpae
% aadaWcaaqaaiaaigdaaeaacaWGubWaaSbaaSqaaiaadUgaaeqaaaaa
% kmaabmaabaGaeyySaeRaamiuaiaacIcacaWGWbWaaSbaaSqaaiaadU
% gaaeqaaOGaeyOeI0IaamOuamaaBaaaleaacaWGRbaabeaakiabgkHi
% TiaadchadaWgaaWcbaGaaGymaaqabaGccaGGPaGaeyySaeRaamytai
% aacIcacaWGubWaaSbaaSqaaiaadUgaaeqaaOGaam4CamaaBaaaleaa
% caWGRbaabeaakiabgUcaRmaalaaabaGaamyDamaaDaaaleaacaWGRb
% aabaGaaGOmaaaaaOqaaiaaikdaaaGaeyOeI0YaaSaaaeaacaWGWbWa
% aSbaaSqaaiaadUgaaeqaaaGcbaGaeqyWdi3aaSbaaSqaaiaadUgaae
% qaaaaakiabgkHiTiaadwgadaWgaaWcbaGaam4AaaqabaGccaGGPaGa
% eyySaeRaamyuaiaacIcacqGHsislcaWG1bWaaSbaaSqaaiaadUgaae
% qaaOGaaiykaiabgglaXkaadofaaiaawIcacaGLPaaacaGGUaaaaaa!C1CE!
\begin{equation}
\begin{gathered}
  m_k (s_{k,t}  + u_k s_{k,x} ) + \frac{{p_1  - p_k }}
{{T_k }}(u_k  - u_2 )\alpha _{k,x}  =  \hfill \\
  \frac{1}
{{T_k }}\left( { \sigma_k P( - R_k  + p_k  - p_1 ) \sigma_k M(\frac{{u_k^2 }}
{2} - \frac{{p_k }}
{{\rho _k }} - e_k ) \sigma_k Q( - u_k ) \sigma_k S} \right), \hfill \\
  (m_k s_k )_t  + (m_k u_k s_k )_x  =  \hfill \\
  \frac{1}
{{T_k }}\left( { \sigma_k P(p_k  - R_k  - p_1 ) \sigma_k M(T_k s_k  + \frac{{u_k^2 }}
{2} - \frac{{p_k }}
{{\rho _k }} - e_k ) \sigma_k Q( - u_k ) \sigma_k S} \right). \hfill \\
\end{gathered}
\end{equation}

Adding now the two entropy equations leads to the entropy
dissipation partial differential equation that we sum up in the following proposition.
\begin{proposition}
Consider a smooth solution of the system (\ref{dipha}) and two
entropy functions $s_1$ and $s_2$ satisfying (\ref{tds}). Then, the
smooth solution satisfies the following entropy dissipation equation
% MathType!MTEF!2!1!+-
% faaafaart1ev1aaat0uyJj1BTfMBaerbuLwBLnhiov2DGi1BTfMBae
% XatLxBI9gBaerbd9wDYLwzYbItLDharqqtubsr4rNCHbGeaGqiVu0J
% e9sqqrpepC0xbbL8F4rqqrFfpeea0xe9Lq-Jc9vqaqpepm0xbba9pw
% e9Q8fs0-yqaqpepae9pg0FirpepeKkFr0xfr-xfr-xb9adbaqaaeGa
% ciGaaiaabeqaamaabaabaaGceaqabeaacaGGOaWaaabqaeaacaWGTb
% WaaSbaaSqaaiaadUgaaeqaaOGaam4CamaaBaaaleaacaWGRbaabeaa
% aeqabeqdcqGHris5aOGaaiykamaaBaaaleaacaWG0baabeaakiabgU
% caRiaacIcadaaeabqaaiaad2gadaWgaaWcbaGaam4AaaqabaGccaWG
% 1bWaaSbaaSqaaiaadUgaaeqaaOGaam4CamaaBaaaleaacaWGRbaabe
% aaaeqabeqdcqGHris5aOGaaiykamaaBaaaleaacaWG4baabeaakiab
% g2da9maalaaabaGaamiuaaqaaiaadsfadaWgaaWcbaGaaGOmaaqaba
% aaaOWaaeWaaeaacaWGWbWaaSbaaSqaaiaaigdaaeqaaOGaey4kaSIa
% amOuaiabgkHiTiaadchadaWgaaWcbaGaaGOmaaqabaaakiaawIcaca
% GLPaaacqGHRaWkaeaacaWGnbWaaeWaaeaadaWcaaqaaiaadwhadaqh
% aaWcbaGaaGymaaqaaiaaikdaaaaakeaacaaIYaGaamivamaaBaaale
% aacaaIXaaabeaaaaGccqGHsisldaWcaaqaaiaadwhadaqhaaWcbaGa
% aGOmaaqaaiaaikdaaaaakeaacaaIYaGaamivamaaBaaaleaacaaIYa
% aabeaaaaGccqGHsisldaWcaaqaaiabeY7aTnaaBaaaleaacaaIXaaa
% beaaaOqaaiaadsfadaWgaaWcbaGaaGymaaqabaaaaOGaey4kaSYaaS
% aaaeaacqaH8oqBdaWgaaWcbaGaaGOmaaqabaaakeaacaWGubWaaSba
% aSqaaiaaikdaaeqaaaaaaOGaayjkaiaawMcaaiabgUcaRiaadgfada
% qadaqaamaalaaabaGaamyDamaaBaaaleaacaaIYaaabeaaaOqaaiaa
% dsfadaWgaaWcbaGaaGOmaaqabaaaaOGaeyOeI0YaaSaaaeaacaWG1b
% WaaSbaaSqaaiaaigdaaeqaaaGcbaGaamivamaaBaaaleaacaaIXaaa
% beaaaaaakiaawIcacaGLPaaacqGHRaWkcaWGtbWaaeWaaeaadaWcaa
% qaaiaaigdaaeaacaWGubWaaSbaaSqaaiaaigdaaeqaaaaakiabgkHi
% TmaalaaabaGaaGymaaqaaiaadsfadaWgaaWcbaGaaGOmaaqabaaaaa
% GccaGLOaGaayzkaaGaaiOlaaaaaa!80F3!
\begin{equation}\label{entroprate}
\begin{gathered}
  (\sum {m_k s_k } )_t  + (\sum {m_k u_k s_k } )_x  = \frac{P}
{{T_2 }}\left( {p_1  + R - p_2 } \right) +  \hfill \\
  M\left( {\frac{{u_1^2 }}
{{2T_1 }} - \frac{{u_2^2 }}
{{2T_2 }} - \frac{{\mu _1 }}
{{T_1 }} + \frac{{\mu _2 }}
{{T_2 }}} \right) + Q\left( {\frac{{u_2 }}
{{T_2 }} - \frac{{u_1 }}
{{T_1 }}} \right) + S\left( {\frac{1}
{{T_1 }} - \frac{1}
{{T_2 }}} \right). \hfill \\
\end{gathered}
\end{equation}
\end{proposition}

\begin{remark}
According to the second principle of thermodynamics the right hand side of (\ref{entroprate}) has to
be non-negative. But each term in the formula (\ref{entroprate}) has not a
clear physical meaning and the Galilean invariance is not obvious. It is often more convenient to rewrite the
source term in a different way. For example, we can set% MathType!MTEF!2!1!+-
% feaafiart1ev1aaatCvAUfeBSjuyZL2yd9gzLbvyNv2CaerbuLwBLn
% hiov2DGi1BTfMBaeXatLxBI9gBaerbd9wDYLwzYbItLDharqqtubsr
% 4rNCHbGeaGqiVu0Je9sqqrpepC0xbbL8F4rqqrFfpeea0xe9Lq-Jc9
% vqaqpepm0xbba9pwe9Q8fs0-yqaqpepae9pg0FirpepeKkFr0xfr-x
% fr-xb9adbaqaaeGaciGaaiaabeqaamaabaabaaGceaqabeaacaWGrb
% Gaeyypa0JaamyuamaaBaaaleaacaaIWaaabeaakiabgUcaRiaadwha
% daWgaaWcbaGaaGymaaqabaGccaWGnbGaaiilaaqaaiaadofacqGH9a
% qpcaWGtbWaaSbaaSqaaiaaicdaaeqaaOGaey4kaSIaamyDamaaBaaa
% leaacaaIXaaabeaakiaadgfadaWgaaWcbaGaaGimaaqabaGccqGHRa
% WkdaWcaaqaaiaadwhadaqhaaWcbaGaaGymaaqaaiaaikdaaaaakeaa
% caaIYaaaaiaad2eacqGHRaWkcqaH8oqBdaWgaaWcbaGaaGymaaqaba
% GccaWGnbGaaiOlaaaaaa!5053!
\begin{equation}
\begin{gathered}
  Q = Q_0  + u_1 M, \hfill \\
  S = S_0  + u_1 Q_0  + \frac{{u_1^2 }}
{2}M + \mu _1 M. \hfill \\
\end{gathered}
\end{equation}
In this way, the dissipation rate becomes% MathType!MTEF!2!1!+-
% feaafiart1ev1aaatCvAUfeBSjuyZL2yd9gzLbvyNv2CaerbuLwBLn
% hiov2DGi1BTfMBaeXatLxBI9gBaerbd9wDYLwzYbItLDharqqtubsr
% 4rNCHbGeaGqiVu0Je9sqqrpepC0xbbL8F4rqqrFfpeea0xe9Lq-Jc9
% vqaqpepm0xbba9pwe9Q8fs0-yqaqpepae9pg0FirpepeKkFr0xfr-x
% fr-xb9adbaqaaeGaciGaaiaabeqaamaabaabaaGcbaWaaSaaaeaaca
% WGqbaabaGaamivamaaBaaaleaacaaIYaaabeaaaaGcdaqadaqaaiaa
% dchadaWgaaWcbaGaaGymaaqabaGccqGHRaWkcaWGsbGaeyOeI0Iaam
% iCamaaBaaaleaacaaIYaaabeaaaOGaayjkaiaawMcaaiabgUcaRmaa
% laaabaGaamytaaqaaiaadsfadaWgaaWcbaGaaGOmaaqabaaaaOGaai
% ikaiabeY7aTnaaBaaaleaacaaIYaaabeaakiabgkHiTiabeY7aTnaa
% BaaaleaacaaIXaaabeaakiabgkHiTmaalaaabaGaaiikaiaadwhada
% WgaaWcbaGaaGOmaaqabaGccqGHsislcaWG1bWaaSbaaSqaaiaaigda
% aeqaaOGaaiykamaaCaaaleqabaGaaGOmaaaaaOqaaiaaikdaaaGaai
% ykaiabgUcaRmaalaaabaGaamyuamaaBaaaleaacaaIWaaabeaaaOqa
% aiaadsfadaWgaaWcbaGaaGOmaaqabaaaaOGaaiikaiaadwhadaWgaa
% WcbaGaaGOmaaqabaGccqGHsislcaWG1bWaaSbaaSqaaiaaigdaaeqa
% aOGaaiykaiabgUcaRmaalaaabaGaam4uamaaBaaaleaacaaIWaaabe
% aaaOqaaiaadsfadaWgaaWcbaGaaGymaaqabaGccaWGubWaaSbaaSqa
% aiaaikdaaeqaaaaakiaacIcacaWGubWaaSbaaSqaaiaaikdaaeqaaO
% GaeyOeI0IaamivamaaBaaaleaacaaIXaaabeaakiaacMcacaGGUaaa
% aa!6C02!
\begin{equation}
\frac{P}
{{T_2 }}\left( {p_1  + R - p_2 } \right) + \frac{M}
{{T_2 }}(\mu _2  - \mu _1  - \frac{{(u_2  - u_1 )^2 }}
{2}) + \frac{{Q_0 }}
{{T_2 }}(u_2  - u_1 ) + \frac{{S_0 }}
{{T_1 T_2 }}(T_2  - T_1 ).
\end{equation}

It is $>0$ if each term in the sum is $>0$. The source $S_0$ can
then be interpreted as the heat flux (it is $>0$ when $T_2 > T_1$,
i.e. when the phase 2 heats the phase 1). The source $Q_0$ is the
drag force. Finally, $M$ is the mass transfer due to chemical
reactions. When $u_1=u_2$, we recover that the chemical reaction
tends to create the phase with the smallest chemical potential. 

\end{remark}


\begin{remark}
Generally, the equations (\ref{tds}) satisfied by the entropies
$s_k$ have not a unique solution, once the pressure laws are given.
 For example, if $s_k$ is a
solution, $-s_k$ is also a solution. A supplementary condition has
thus to be given in order to fix the sign of the entropy dissipation
rate. In the case of conservative systems the entropies are supposed
to satisfy some convexity property. For a non-conservative system,
it is not possible to apply the Godunov-Mock theorem and it is
difficult to extend naturally the convexity approach. We propose
here only to forbid the change $s \to -s$ by imposing that the
temperature remains $>0$. It implies
\begin{equation}\label{sourcep}
\frac{1} {{T_k }} = \frac{{\partial s_k }} {{\partial e_k }} > 0.
\end{equation}

\end{remark}

In this paper, we will concentrate on the pressure relaxation source term.
We will assume the following form, which ensures a positive entropy dissipation
\begin{equation}\label{formsource}
P = \frac{1}{{\tau _p }}\alpha _1 \alpha _2 \left( {p_1  + R - p_2 } \right),\quad \tau _p  > 0,
\end{equation}
where $\tau_p$ is the relaxation parameter.
An instantaneous relaxation corresponds to the limit $\tau_p = 0$.

\begin{remark}
Let $p_\text{ref}$ be a reference pressure. We can define a characteristic time for the pressure equilibrium by
\begin{equation}\label{tref}
t_{{\text{ref}}}  = \frac{{\tau _p }}
{{p_{{\text{ref}}} }}.
\end{equation}
The knowledge of this characteristic time is important for a proper modeling.


\end{remark}

\section{Application to stiffened gas laws: admissible granular stress\label{num}}
\subsection{Admissible granular stress}
In many works (as in \cite{kuo80}, \cite{goldstein96} and \cite{kapila01}), the granular stress is supposed to depend only on
the solid volume fraction $\alpha_2$. This hypothesis is reasonable
when the solid phase is incompressible. However, this choice is not compatible with the existence of an entropy
satisfying (\ref{tds}) in the case of a compressible phase. The choice of the granular
stress expression cannot be arbitrary. That is why in this section, we compute a simple expression
of it when the pressure law of the solid phase is a stiffened gas equation of state.

%We then
%justify the choice that we have made in the previous section to take
%a granular stress of the form% MathType!MTEF!2!1!+-
%% feaafiart1ev1aaatCvAUfeBSjuyZL2yd9gzLbvyNv2CaerbuLwBLn
%% hiov2DGi1BTfMBaeXatLxBI9gBaerbd9wDYLwzYbItLDharqqtubsr
%% 4rNCHbGeaGqiVu0Je9sqqrpepC0xbbL8F4rqqrFfpeea0xe9Lq-Jc9
%% vqaqpepm0xbba9pwe9Q8fs0-yqaqpepae9pg0FirpepeKkFr0xfr-x
%% fr-xb9adbaqaaeGaciGaaiaabeqaamaabaabaaGcbaGaamOuaiabg2
%% da9iaad2gadaWgaaWcbaGaaGOmaaqabaGccqaHbpGCdaqhaaWcbaGa
%% aGOmaaqaaiabeo7aNnaaBaaameaacaqGYaaabeaaliabgkHiTiaaig
%% daaaGccqaH4oqCcaGGOaGaeqySde2aaSbaaSqaaiaaikdaaeqaaOGa
%% aiykaiabg2da9iabeg8aYnaaDaaaleaacaaIYaaabaGaeq4SdC2aaS
%% baaWqaaiaaikdaaeqaaaaakiaadEgacaGGOaGaeqySde2aaSbaaSqa
%% aiaaikdaaeqaaOGaaiykaiaac6caaaa!520E!
%\begin{equation}
%R = m_2 \rho _2^{\gamma _{\text{2}}  - 1} \theta (\alpha _2 ) = \rho
%_2^{\gamma _2 } g(\alpha _2 ).
%\end{equation}


Let us note% MathType!MTEF!2!1!+-
% faaafaart1ev1aaat0uyJj1BTfMBaerbuLwBLnhiov2DGi1BTfMBae
% XatLxBI9gBaerbd9wDYLwzYbItLDharqqtubsr4rNCHbGeaGqiVu0J
% e9sqqrpepC0xbbL8F4rqqrFfpeea0xe9Lq-Jc9vqaqpepm0xbba9pw
% e9Q8fs0-yqaqpepae9pg0FirpepeKkFr0xfr-xfr-xb9adbaqaaeGa
% ciGaaiaabeqaamaabaabaaGcbaGaeuiMdeLaaiikaiaaigdacaGGVa
% GaeqyWdi3aaSbaaSqaaiaaikdaaeqaaOGaaiilaiaadwgacaGGSaGa
% eqySde2aaSbaaSqaaiaaikdaaeqaaOGaaiykaiabg2da9maalaaaba
% GaamOuamaaBaaaleaacaaIYaaabeaakiaacIcacaaIXaGaai4laiab
% eg8aYnaaBaaaleaacaaIYaaabeaakiaacYcacaWGLbGaaiilaiabeg
% 7aHnaaBaaaleaacaaIYaaabeaakiaacMcaaeaacqaHXoqydaWgaaWc
% baGaaGOmaaqabaGccqaHbpGCdaWgaaWcbaGaaGOmaaqabaaaaOGaai
% Olaaaa!5057!
\begin{equation}
\Theta (1/\rho _2 ,e,\alpha _2 ) = \frac{{R_2 (1/\rho _2 ,e,\alpha _2 )}}
{{\alpha _2 \rho _2 }}.
\end{equation}


We omit  now the subscript $k=2$ because we concentrate only on the solid phase. We have to find an entropy $s$, a temperature $T$ and a function $\Theta$ (containing the granular modeling) such that% MathType!MTEF!2!1!+-
% feaafiart1ev1aaatCvAUfeBSjuyZL2yd9gzLbvyNv2CaerbuLwBLn
% hiov2DGi1BTfMBaeXatLxBI9gBaerbd9wDYLwzYbItLDharqqtubsr
% 4rNCHbGeaGqiVu0Je9sqqrpepC0xbbL8F4rqqrFfpeea0xe9Lq-Jc9
% vqaqpepm0xbba9pwe9Q8fs0-yqaqpepae9pg0FirpepeKkFr0xfr-x
% fr-xb9adbaqaaeGaciGaaiaabeqaamaabaabaaGcbaGaamivaiaads
% gacaWGZbGaeyypa0JaamizaiaadwgacqGHsisldaWcaaqaaiaadcha
% aeaacqaHbpGCdaahaaWcbeqaaiaaikdaaaaaaOGaamizaiabeg8aYj
% abgkHiTiabfI5arjaadsgacqaHXoqyaaa!47BA!
\begin{equation}
Tds = de - \frac{p} {{\rho ^2 }}d\rho  - \Theta d\alpha.
\end{equation}


We note% MathType!MTEF!2!1!+-
% feaafiart1ev1aaatCvAUfeBSjuyZL2yd9gzLbvyNv2CaerbuLwBLn
% hiov2DGi1BTfMBaeXatLxBI9gBaerbd9wDYLwzYbItLDharqqtubsr
% 4rNCHbGeaGqiVu0Je9sqqrpepC0xbbL8F4rqqrFfpeea0xe9Lq-Jc9
% vqaqpepm0xbba9pwe9Q8fs0-yqaqpepae9pg0FirpepeKkFr0xfr-x
% fr-xb9adbaqaaeGaciGaaiaabeqaamaabaabaaGceaqabeaacqaHep
% aDcqGH9aqpcaaIXaGaai4laiabeg8aYjaacYcaaeaacaWGubGaeyyp
% a0JaaGymaiaac+cacqaHvpGAcaGGUaaaaaa!4268!
\begin{equation}
\begin{gathered}
  \tau  = 1/\rho , \hfill \\
  T = 1/\varphi . \hfill \\
\end{gathered}
\end{equation}
Then, $\varphi=\varphi(\tau,e,\alpha)$ is an integrating factor for the form % MathType!MTEF!2!1!+-
% feaafiart1ev1aaatCvAUfeBSjuyZL2yd9gzLbvyNv2CaerbuLwBLn
% hiov2DGi1BTfMBaeXatLxBI9gBaerbd9wDYLwzYbItLDharqqtubsr
% 4rNCHbGeaGqiVu0Je9sqqrpepC0xbbL8F4rqqrFfpeea0xe9Lq-Jc9
% vqaqpepm0xbba9pwe9Q8fs0-yqaqpepae9pg0FirpepeKkFr0xfr-x
% fr-xb9adbaqaaeGaciGaaiaabeqaamaabaabaaGcbaGaamizaiaadw
% gacqGHRaWkcaWGWbGaamizaiabes8a0jabgkHiTiabfI5arjaadsga
% cqaHXoqyaaa!4131!
\begin{equation}
de + pd\tau  - \Theta d\alpha,
\end{equation}
which reads% MathType!MTEF!2!1!+-
% feaafiart1ev1aaatCvAUfeBSjuyZL2yd9gzLbvyNv2CaerbuLwBLn
% hiov2DGi1BTfMBaeXatLxBI9gBaerbd9wDYLwzYbItLDharqqtubsr
% 4rNCHbGeaGqiVu0Je9sqqrpepC0xbbL8F4rqqrFfpeea0xe9Lq-Jc9
% vqaqpepm0xbba9pwe9Q8fs0-yqaqpepae9pg0FirpepeKkFr0xfr-x
% fr-xb9adbaqaaeGaciGaaiaabeqaamaabaabaaGcbaGaamizaiaado
% hacqGH9aqpcqaHvpGAcaWGKbGaamyzaiabgUcaRiabew9aQjaadcha
% caGGOaGaeqiXdqNaaiilaiaadwgacaGGPaGaamizaiabes8a0jabgk
% HiTiabew9aQjabfI5arjaadsgacqaHXoqycaGGUaaaaa!4EE6!
\begin{equation}
ds = \varphi de + \varphi p(\tau ,e)d\tau  - \varphi \Theta d\alpha
.
\end{equation}

%Thus, the form is closed iff% MathType!MTEF!2!1!+-
%% feaafiart1ev1aaatCvAUfeBSjuyZL2yd9gzLbvyNv2CaerbuLwBLn
%% hiov2DGi1BTfMBaeXatLxBI9gBaerbd9wDYLwzYbItLDharqqtubsr
%% 4rNCHbGeaGqiVu0Je9sqqrpepC0xbbL8F4rqqrFfpeea0xe9Lq-Jc9
%% vqaqpepm0xbba9pwe9Q8fs0-yqaqpepae9pg0FirpepeKkFr0xfr-x
%% fr-xb9adbaqaaeGaciGaaiaabeqaamaabaabaaGceaqabeaacqaHvp
%% GAdaWgaaWcbaGaeqySdegabeaakiabg2da9iabgkHiTiabfI5arjab
%% ew9aQnaaBaaaleaacaWGLbaabeaakiaacYcaaeaacaWGWbGaeqy1dO
%% 2aaSbaaSqaaiabeg7aHbqabaGccqGH9aqpcqGHsislcqqHyoqucqaH
%% vpGAdaWgaaWcbaGaeqiXdqhabeaakiaacYcaaeaacaWGWbGaeqy1dO
%% 2aaSbaaSqaaiaadwgaaeqaaOGaey4kaSIaamiCamaaBaaaleaacaWG
%% Lbaabeaakiabew9aQjabg2da9iabew9aQnaaBaaaleaacqaHepaDae
%% qaaOGaaiOlaaaaaa!5B36!
%\begin{equation}
%\begin{gathered}
%  \varphi _\alpha   =  - \Theta \varphi _e , \hfill \\
%  p\varphi _\alpha   =  - \Theta \varphi _\tau  , \hfill \\
%  p\varphi _e  + p_e \varphi  = \varphi _\tau  . \hfill \\
%\end{gathered}
%\end{equation}
%If we take $\varphi_\alpha \neq 0$, then it implies $\varphi=0$,
%which is impossible. Thus necessarily $\varphi _\alpha = 0.$ The
%only solution is% MathType!MTEF!2!1!+-
%% feaafiart1ev1aaatCvAUfeBSjuyZL2yd9gzLbvyNv2CaerbuLwBLn
%% hiov2DGi1BTfMBaeXatLxBI9gBaerbd9wDYLwzYbItLDharqqtubsr
%% 4rNCHbGeaGqiVu0Je9sqqrpepC0xbbL8F4rqqrFfpeea0xe9Lq-Jc9
%% vqaqpepm0xbba9pwe9Q8fs0-yqaqpepae9pg0FirpepeKkFr0xfr-x
%% fr-xb9adbaqaaeGaciGaaiaabeqaamaabaabaaGcbaGaeuiMdeLaey
%% ypa0JaaGimaiaac6caaaa!39D6!
%\begin{equation}
%\Theta  = 0.
%\end{equation}
In order to construct a practical and simple model, we suppose that the granular
 stress only depends on the density and the volume fraction of the solid phase.
This leads to the choice
$\Theta=\Theta(\tau,\alpha)$. The differential form is closed if
\begin{equation}
\begin{gathered}
  \varphi _\alpha   =  - \Theta \varphi _e , \hfill \\
  p\varphi _\alpha   =  - \Theta \varphi _\tau   - \varphi \Theta _\tau  , \hfill \\
  p\varphi _e  + p_e \varphi  = \varphi _\tau  . \hfill \\
\end{gathered}
\end{equation}
The general case corresponds to $\varphi _e \neq 0$, $\varphi
_\alpha \neq 0$ and $\varphi \neq 0$. We then have necessarily% MathType!MTEF!2!1!+-
% feaafiart1ev1aaatCvAUfeBSjuyZL2yd9gzLbvyNv2CaerbuLwBLn
% hiov2DGi1BTfMBaeXatLxBI9gBaerbd9wDYLwzYbItLDharqqtubsr
% 4rNCHbGeaGqiVu0Je9sqqrpepC0xbbL8F4rqqrFfpeea0xe9Lq-Jc9
% vqaqpepm0xbba9pwe9Q8fs0-yqaqpepae9pg0FirpepeKkFr0xfr-x
% fr-xb9adbaqaaeGaciGaaiaabeqaamaabaabaaGcbaWaaSaaaeaacq
% qHyoqudaWgaaWcbaGaeqiXdqhabeaaaOqaaiabfI5arbaacqGH9aqp
% cqGHsislcaWGWbWaaSbaaSqaaiaadwgaaeqaaOGaaiOlaaaa!3FA0!
\begin{equation}
\frac{{\Theta _\tau  }}
{\Theta } =  - p_e .
\end{equation}

\subsection{Practical example}
Now we propose some computations when the pressure law (\ref{stifgas}) is a stiffened gas EOS
\begin{equation}p=\psi(\rho ,e) = (\gamma  - 1)\rho e - \gamma \pi .\end{equation}
The parameter $\gamma$ must be $>1$. The parameter $\pi$ has the dimension of a pressure and can be arbitrary. But in practice, for a solid phase, it is positive and large compared to a characteristic
 pressure of the flow.
In the case of a stiffened gas equation, we thus find% MathType!MTEF!2!1!+-
% feaafiart1ev1aaatCvAUfeBSjuyZL2yd9gzLbvyNv2CaerbuLwBLn
% hiov2DGi1BTfMBaeXatLxBI9gBaerbd9wDYLwzYbItLDharqqtubsr
% 4rNCHbGeaGqiVu0Je9sqqrpepC0xbbL8F4rqqrFfpeea0xe9Lq-Jc9
% vqaqpepm0xbba9pwe9Q8fs0-yqaqpepae9pg0FirpepeKkFr0xfr-x
% fr-xb9adbaqaaeGaciGaaiaabeqaamaabaabaaGcbaWaaSaaaeaacq
% qHyoqudaWgaaWcbaGaeqiXdqhabeaaaOqaaiabfI5arbaacqGH9aqp
% cqGHsisldaWcaaqaaiabeo7aNjabgkHiTiaaigdaaeaacqaHepaDaa
% GaeyO0H4TaeuiMdeLaaiikaiaaigdacaGGVaGaeqyWdiNaaiilaiab
% eg7aHjaacMcacqGH9aqpcqaH4oqCcaGGOaGaeqySdeMaaiykaiabeg
% 8aYnaaCaaaleqabaGaeq4SdCMaeyOeI0IaaGymaaaakiaacYcaaaa!5851!
\begin{equation}
\frac{{\Theta _\tau  }}
{\Theta } =  - \frac{{\gamma  - 1}}
{\tau } \Rightarrow \Theta (1/\rho ,\alpha ) = \theta (\alpha )\rho ^{\gamma  - 1} ,
\end{equation}
which leads to
% MathType!MTEF!2!1!+-
% feaafiart1ev1aaatCvAUfeBSjuyZL2yd9gzLbvyNv2CaerbuLwBLn
% hiov2DGi1BTfMBaeXatLxBI9gBaerbd9wDYLwzYbItLDharqqtubsr
% 4rNCHbGeaGqiVu0Je9sqqrpepC0xbbL8F4rqqrFfpeea0xe9Lq-Jc9
% vqaqpepm0xbba9pwe9Q8fs0-yqaqpepae9pg0FirpepeKkFr0xfr-x
% fr-xb9adbaqaaeGaciGaaiaabeqaamaabaabaaGcbaGaamOuaiaacI
% cacqaHepaDcaGGSaGaeqySdeMaaiykaiabg2da9iabeg7aHjabeI7a
% XjaacIcacqaHXoqycaGGPaGaeqyWdi3aaWbaaSqabeaacqaHZoWzaa
% GccaGGUaaaaa!47D4!
\begin{equation}\label{genegranu}
R(\tau ,\alpha ) = \alpha \theta (\alpha )\rho ^\gamma  .
\end{equation}



In this paper, we will perform numerical experiments with a very simple particular choice% MathType!MTEF!2!1!+-
% feaafiart1ev1aaatCvAUfeBSjuyZL2yd9gzLbvyNv2CaerbuLwBLn
% hiov2DGi1BTfMBaeXatLxBI9gBaerbd9wDYLwzYbItLDharqqtubsr
% 4rNCHbGeaGqiVu0Je9sqqrpepC0xbbL8F4rqqrFfpeea0xe9Lq-Jc9
% vqaqpepm0xbba9pwe9Q8fs0-yqaqpepae9pg0FirpepeKkFr0xfr-x
% fr-xb9adbaqaaeGaciGaaiaabeqaamaabaabaaGcbaGaamOuaiaacI
% cacqaHepaDcaGGSaGaeqySdeMaaiykaiabg2da9iabeU7aSjabeg8a
% YnaaCaaaleqabaGaeq4SdCgaaOGaeqySde2aaWbaaSqabeaacqaHZo
% WzaaGccaGGUaaaaa!46B8!
\begin{equation}\label{nice}
R(\tau ,\alpha ) = \kappa \rho ^\gamma  \alpha ^\gamma  .
\end{equation}
The value of $\kappa$ can be adjusted to experiments. With this choice, the parameter $R_0=\kappa \rho_2^{\gamma_2}$ has
the dimension of a pressure. It represents the maximal stress
corresponding to the maximal compaction $\alpha=\alpha_2=1$.
This model is not so different from classical approaches
(as described for example in \cite{goldstein96}):
usually, the granular stress vanishes under some critical solid volume fraction $\alpha_c$ (dilute case) and increases with $\alpha$ when $\alpha > \alpha_c$ (packed case). In our approach, the parameter
$\gamma$ allows to ensure that the granular stress is small when
$\alpha$ is small. Actually, the higher $\gamma$ is, the faster
the granular stress tends to zero when $\alpha$ tends to zero.

In this model, unlike in \cite{goldstein96}, the solid volume fraction $\alpha$ can approach one. This is due to the fact that the compressibility of the solid phase is taken into account. Of course, if this solid phase is slightly compressible a volume fraction $\alpha_2 \simeq 1$ would imply very high pressures. Mathematically this situation is not a problem, but physically it can be questioned. It is also related with the choice (\ref{eq-bn-choice}) for the interface pressure and velocity. This choice is physically recommended  for dilute flows. When the flow is not dilute, the model, while still mathematically robust, should probably be improved.

Of course, it would be also possible to consider the most general
case where the granular stress also depends on the internal energy% MathType!MTEF!2!1!+-
% feaafiart1ev1aaatCvAUfeBSjuyZL2yd9gzLbvyNv2CaerbuLwBLn
% hiov2DGi1BTfMBaeXatLxBI9gBaerbd9wDYLwzYbItLDharqqtubsr
% 4rNCHbGeaGqiVu0Je9sqqrpepC0xbbL8F4rqqrFfpeea0xe9Lq-Jc9
% vqaqpepm0xbba9pwe9Q8fs0-yqaqpepae9pg0FirpepeKkFr0xfr-x
% fr-xb9adbaqaaeGaciGaaiaabeqaamaabaabaaGcbaGaeuiMdeLaey
% ypa0JaeuiMdeLaaiikaiabeg7aHjaacYcacqaHepaDcaGGSaGaamyz
% aiaacMcaaaa!40E8!
\begin{equation}
\Theta  = \Theta (\alpha ,\tau ,e).
\end{equation}
However, we will see that the choice (\ref{nice}) is very
interesting for the modeling and the numerics because it ensures that the volume fraction stays within its natural bounds during the pressure
equilibrium resolution.

Is also possible to compute the whole thermodynamic underlying model.
The expressions for the associated entropy and temperature  are given in
Section \ref{assentrop}.

\subsection{Summary of the full model}
Before presenting the numerical part of our work, we sum up the model that will be used for the numerics. From now on, we will suppose that the pressure laws of the two phases are stiffened gas equations of state. We will also suppose that the granular stress is given by formula (\ref{nice}). These hypothesis are made because they permit a good balance between simplicity and generality. Of course our approach can be generalized to other physical laws.
We solve% MathType!MTEF!2!1!+-
% feaafiart1ev1aaatCvAUfeBSjuyZL2yd9gzLbvyNv2CaerbuLwBLn
% hiov2DGi1BTfMBaeXatLxBI9gBaerbd9wDYLwzYbItLDharqqtubsr
% 4rNCHbGeaGqiVu0Je9sqqrpepC0xbbL8F4rqqrFfpeea0xe9Lq-Jc9
% vqaqpepm0xbba9pwe9Q8fs0-yqaqpepae9pg0FirpepeKkFr0xfr-x
% fr-xb9adbaqaaeGaciGaaiaabeqaamaabaabaaGcbaGaam4vamaaBa
% aaleaacaWG0baabeaakiabgUcaRiaadAeacaGGOaGaam4vaiaacMca
% daWgaaWcbaGaamiEaaqabaGccqGHRaWkcaWGbbGaaiikaiaadEfaca
% GGPaGaamitaiaacIcacaWGxbGaaiykamaaBaaaleaacaWG4baabeaa
% kiabg2da9iaadofacaGGOaGaam4vaiaacMcacaGGSaaaaa!49E6!
\begin{equation}\label{first-eq}
W_t  + F(W)_x  + A(W)L(W)_x  = \Sigma(W).
\end{equation}
The unknowns are (with $m_k=\alpha_k \rho_k$, $E_k=e_k+u_k^2/2$, $k=1,2$)
% MathType!MTEF!2!1!+-
% faaafaart1ev1aaat0uyJj1BTfMBaerbuLwBLnhiov2DGi1BTfMBae
% XatLxBI9gBaerbd9wDYLwzYbItLDharqqtubsr4rNCHbGeaGqiVu0J
% e9sqqrpepC0xbbL8F4rqqrFfpeea0xe9Lq-Jc9vqaqpepm0xbba9pw
% e9Q8fs0-yqaqpepae9pg0FirpepeKkFr0xfr-xfr-xb9adbaqaaeGa
% ciGaaiaabeqaamaabaabaaGcbaGaam4vaiabg2da9maabmaabaGaam
% yBamaaBaaaleaacaaIXaaabeaakiaacYcacaWGTbWaaSbaaSqaaiaa
% igdaaeqaaOGaamyDamaaBaaaleaacaaIXaaabeaakiaacYcacaWGTb
% WaaSbaaSqaaiaaigdaaeqaaOGaamyramaaBaaaleaacaaIXaaabeaa
% kiaacYcacaWGTbWaaSbaaSqaaiaaikdaaeqaaOGaaiilaiaad2gada
% WgaaWcbaGaaGOmaaqabaGccaWG1bWaaSbaaSqaaiaaikdaaeqaaOGa
% aiilaiaad2gadaWgaaWcbaGaaGOmaaqabaGccaWGfbWaaSbaaSqaai
% aaikdaaeqaaOGaaiilaiabeg7aHnaaBaaaleaacaaIXaaabeaaaOGa
% ayjkaiaawMcaamaaCaaaleqabaGaamivaaaaaaa!4F28!
\begin{equation}\label{}
W = \left( {m_1 ,m_1 u_1 ,m_1 E_1 ,m_2 ,m_2 u_2 ,m_2 E_2 ,\alpha _1 } \right)^T .
\end{equation}
The "conservative" flux is given by
% MathType!MTEF!2!1!+-
% faaafaart1ev1aaat0uyJj1BTfMBaerbuLwBLnhiov2DGi1BTfMBae
% XatLxBI9gBaerbd9wDYLwzYbItLDharqqtubsr4rNCHbGeaGqiVu0J
% e9sqqrpepC0xbbL8F4rqqrFfpeea0xe9Lq-Jc9vqaqpepm0xbba9pw
% e9Q8fs0-yqaqpepae9pg0FirpepeKkFr0xfr-xfr-xb9adbaqaaeGa
% ciGaaiaabeqaamaabaabaaGcbaGaamOraiaacIcacaWGxbGaaiykai
% abg2da9maabmaabaGaamyBamaaBaaaleaacaaIXaaabeaakiaadwha
% daWgaaWcbaGaaGymaaqabaGccaGGSaGaamyBamaaBaaaleaacaaIXa
% aabeaakiaadwhadaqhaaWcbaGaaGymaaqaaiaaikdaaaGccaGGSaGa
% amyBamaaBaaaleaacaaIXaaabeaakiaadweadaWgaaWcbaGaaGymaa
% qabaGccaWG1bWaaSbaaSqaaiaaigdaaeqaaOGaaiilaiaad2gadaWg
% aaWcbaGaaGOmaaqabaGccaWG1bWaaSbaaSqaaiaaikdaaeqaaOGaai
% ilaiaad2gadaWgaaWcbaGaaGOmaaqabaGccaWG1bWaa0baaSqaaiaa
% ikdaaeaacaaIYaaaaOGaaiilaiaad2gadaWgaaWcbaGaaGOmaaqaba
% GccaWGfbWaaSbaaSqaaiaaikdaaeqaaOGaamyDamaaBaaaleaacaaI
% YaaabeaakiaacYcacaaIWaaacaGLOaGaayzkaaWaaWbaaSqabeaaca
% WGubaaaOGaaiilaaaa!5958!
\begin{equation}\label{}
F(W) = \left( {m_1 u_1 ,m_1 u_1^2 ,m_1 E_1 u_1 ,m_2 u_2 ,m_2 u_2^2 ,m_2 E_2 u_2 ,0} \right)^T ,
\end{equation}
and the "non-conservative" terms are given by
% MathType!MTEF!2!1!+-
% faaafaart1ev1aaat0uyJj1BTfMBaerbuLwBLnhiov2DGi1BTfMBae
% XatLxBI9gBaerbd9wDYLwzYbItLDharqqtubsr4rNCHbGeaGqiVu0J
% e9sqqrpepC0xbbL8F4rqqrFfpeea0xe9Lq-Jc9vqaqpepm0xbba9pw
% e9Q8fs0-yqaqpepae9pg0FirpepeKkFr0xfr-xfr-xb9adbaqaaeGa
% ciGaaiaabeqaamaabaabaaGceaqabeaacaWGmbGaaiikaiaadEfaca
% GGPaGaeyypa0ZaaeWaaeaacaWGWbWaaSbaaSqaaiaaigdaaeqaaOGa
% aiilaiaadchadaWgaaWcbaGaaGymaaqabaGccaWG1bWaaSbaaSqaai
% aaigdaaeqaaOGaaiilaiaadchadaWgaaWcbaGaaGOmaaqabaGccaGG
% SaGaamiCamaaBaaaleaacaaIYaaabeaakiaadwhadaWgaaWcbaGaaG
% OmaaqabaGccaGGSaGaeqySde2aaSbaaSqaaiaaigdaaeqaaaGccaGL
% OaGaayzkaaWaaWbaaSqabeaacaWGubaaaOGaaiilaaqaaiaadgeaca
% GGOaGaam4vaiaacMcacaWGmbGaaiikaiaadEfacaGGPaWaaSbaaSqa
% aiaadIhaaeqaaOGaeyypa0JaaiikaiaaicdacaGGSaGaeqySde2aaS
% baaSqaaiaaigdaaeqaaOGaamiCamaaBaaaleaacaaIXaGaaiilaiaa
% dIhaaeqaaOGaaiilaiabeg7aHnaaBaaaleaacaaIXaaabeaakiaacI
% cacaWGWbWaaSbaaSqaaiaaigdaaeqaaOGaamyDamaaBaaaleaacaaI
% XaaabeaakiaacMcadaWgaaWcbaGaamiEaaqabaGccqGHRaWkcaWGWb
% WaaSbaaSqaaiaaigdaaeqaaOGaaiikaiaadwhadaWgaaWcbaGaaGym
% aaqabaGccqGHsislcaWG1bWaaSbaaSqaaiaaikdaaeqaaOGaaiykai
% abeg7aHnaaBaaaleaacaaIXaGaaiilaiaadIhaaeqaaOGaaiilaaqa
% aiaaicdacaGGSaGaeqySde2aaSbaaSqaaiaaikdaaeqaaOGaamiCam
% aaBaaaleaacaaIYaGaaiilaiaadIhaaeqaaOGaey4kaSIaaiikaiaa
% dchadaWgaaWcbaGaaGOmaaqabaGccqGHsislcaWGWbWaaSbaaSqaai
% aaigdaaeqaaOGaaiykaiabeg7aHnaaBaaaleaacaaIYaGaaiilaiaa
% dIhaaeqaaOGaaiilaiabeg7aHnaaBaaaleaacaaIYaaabeaakiaacI
% cacaWGWbWaaSbaaSqaaiaaikdaaeqaaOGaamyDamaaBaaaleaacaaI
% YaaabeaakiaacMcadaWgaaWcbaGaamiEaaqabaGccqGHRaWkcaWG1b
% WaaSbaaSqaaiaaikdaaeqaaOGaaiikaiaadchadaWgaaWcbaGaaGOm
% aaqabaGccqGHsislcaWGWbWaaSbaaSqaaiaaigdaaeqaaOGaaiykai
% abeg7aHnaaBaaaleaacaaIYaGaaiilaiaadIhaaeqaaOGaaiilaiaa
% dAhadaWgaaWcbaGaaGOmaaqabaGccqaHXoqydaWgaaWcbaGaaGymai
% aacYcacaWG4baabeaakiaacMcadaahaaWcbeqaaiaadsfaaaGccaGG
% Saaaaaa!A42C!
\begin{equation}\label{}
\begin{gathered}
  L(W) = \left( {p_1 ,p_1 u_1 ,p_2 ,p_2 u_2 ,\alpha _1 } \right)^T , \hfill \\
  A(W)L(W)_x  = (0,\alpha _1 p_{1,x} ,\alpha _1 (p_1 u_1 )_x  + p_1 (u_1  - u_2 )\alpha _{1,x} , \hfill \\
  0,\alpha _2 p_{2,x}  + (p_2  - p_1 )\alpha _{2,x} ,\alpha _2 (p_2 u_2 )_x  + u_2 (p_2  - p_1 )\alpha _{2,x} ,v_2 \alpha _{1,x} )^T .\hfill \\ 
\end{gathered} 
\end{equation}
The pressures obey stiffened gas equations of state
% MathType!MTEF!2!1!+-
% faaafaart1ev1aaat0uyJj1BTfMBaerbuLwBLnhiov2DGi1BTfMBae
% XatLxBI9gBaerbd9wDYLwzYbItLDharqqtubsr4rNCHbGeaGqiVu0J
% e9sqqrpepC0xbbL8F4rqqrFfpeea0xe9Lq-Jc9vqaqpepm0xbba9pw
% e9Q8fs0-yqaqpepae9pg0FirpepeKkFr0xfr-xfr-xb9adbaqaaeGa
% ciGaaiaabeqaamaabaabaaGcbaGaamiCamaaBaaaleaacaWGRbaabe
% aakiabg2da9iaacIcacqaHZoWzdaWgaaWcbaGaam4AaaqabaGccqGH
% sislcaaIXaGaaiykaiabeg8aYnaaBaaaleaacaWGRbaabeaakiaadw
% gadaWgaaWcbaGaam4AaaqabaGccqGHsislcqaHZoWzdaWgaaWcbaGa
% am4AaaqabaGccqaHapaCdaWgaaWcbaGaam4AaaqabaGccaGGSaGaaG
% zbVlabeo7aNnaaBaaaleaacaWGRbaabeaakiabg6da+iaaigdacaGG
% SaGaaGzbVlaadUgacqGH9aqpcaaIXaGaaiilaiaaikdacaGGUaaaaa!53C1!
\begin{equation}\label{}
p_k  = (\gamma _k  - 1)\rho _k e_k  - \gamma _k \pi _k ,\quad \gamma _k  > 1,\quad k = 1,2.
\end{equation}
The constants $\gamma_k$ and $\pi_k$ for $k=1,2$ are fixed and obtained from physical measurements. 
The source terms vector is% MathType!MTEF!2!1!+-
% faaafaart1ev1aaat0uyJj1BTfMBaerbuLwBLnhiov2DGi1BTfMBae
% XatLxBI9gBaerbd9wDYLwzYbItLDharqqtubsr4rNCHbGeaGqiVu0J
% e9sqqrpepC0xbbL8F4rqqrFfpeea0xe9Lq-Jc9vqaqpepm0xbba9pw
% e9Q8fs0-yqaqpepae9pg0FirpepeKkFr0xfr-xfr-xb9adbaqaaeGa
% ciGaaiaabeqaamaabaabaaGcbaGaeu4OdmLaeyypa0Zaaeqaaeaaca
% aIWaGaaiilaiaaicdacaGGSaGaeyOeI0IaamiCamaaBaaaleaacaaI
% XaaabeaakiaadcfacaGGSaaacaGLOaaadaqacaqaaiaaicdacaGGSa
% GaaGimaiaacYcacqGHRaWkcaWGWbWaaSbaaSqaaiaaigdaaeqaaOGa
% amiuaiaacYcacaWGqbaacaGLPaaadaahaaWcbeqaaiaadsfaaaGcca
% GGUaaaaa!4660!
\begin{equation}\label{}
\Sigma(W)  = \left( {0,0, - p_1 P(W),} \right.\left. {0,0,  p_1 P(W),P(W)} \right)^T .
\end{equation}

The pressure relaxation source term is
% MathType!MTEF!2!1!+-
% faaafaart1ev1aaat0uyJj1BTfMBaerbuLwBLnhiov2DGi1BTfMBae
% XatLxBI9gBaerbd9wDYLwzYbItLDharqqtubsr4rNCHbGeaGqiVu0J
% e9sqqrpepC0xbbL8F4rqqrFfpeea0xe9Lq-Jc9vqaqpepm0xbba9pw
% e9Q8fs0-yqaqpepae9pg0FirpepeKkFr0xfr-xfr-xb9adbaqaaeGa
% ciGaaiaabeqaamaabaabaaGceaqabeaacaWGqbGaaiikaiaadEfaca
% GGPaGaeyypa0ZaaSaaaeaacaaIXaaabaGaeqiXdq3aaSbaaSqaaiaa
% dcfaaeqaaaaakiabeg7aHnaaBaaaleaacaaIXaaabeaakiabeg7aHn
% aaBaaaleaacaaIYaaabeaakiaacIcacaWGWbWaaSbaaSqaaiaaigda
% aeqaaOGaey4kaSIaamOuaiabgkHiTiaadchadaWgaaWcbaGaaGOmaa
% qabaGccaGGPaGaaiilaiaaywW7cqaHepaDdaWgaaWcbaGaamiuaaqa
% baGccqGH+aGpcaaIWaGaaiilaaqaaiaadkfacqGH9aqpcqaH6oWAca
% WGTbWaa0baaSqaaiaaikdaaeaacqaHZoWzdaWgaaadbaGaaGOmaaqa
% baaaaOGaaiilaiaaywW7cqaH6oWAcqGH+aGpcaaIWaGaaiOlaaaaaa!5BEB!
\begin{equation}\label{last-eq}
\begin{gathered}
  P(W) = \frac{1}
{{\tau _P }}\alpha _1 \alpha _2 (p_1  + R(\rho_2,\alpha_2) - p_2 ),\quad \tau _P  > 0, \hfill \\
  R(\rho_2,\alpha_2) = \kappa (\alpha_2 \rho_2)^{\gamma _2 } ,\quad \kappa  > 0. \hfill \\ 
\end{gathered} 
\end{equation}
The constants $\kappa$ and $\tau_P$ are obtained by physical measurements.
 In Section \ref{acad}, we will also present a realistic gun simulation. In this case we will propose a different expression of the source terms $\Sigma(W)$.

\section{Finite volume approach}

For the numerical implementation, we consider a finite volume discretisation and a splitting approach. The convection step is
solved by a standard Rusanov scheme \cite{rusanov61} already described in many works
as \cite{nuss-hell06}.  In the second stage, the source terms are applied. We concentrate on the pressure relaxation source term in the next section.

For the finite volume scheme, we consider a space step $h$, a time step ${\Delta t}$. The cells are intervals $]x_{i-1/2},x_{i+1/2}[$. For simplicity, we consider a regular mesh $x_i=ih$ (but this
of course is not mandatory). The vector $W$ is approximated in each cell at time $t_n$ by% MathType!MTEF!2!1!+-% faaafaart1ev1aaat0uyJj1BTfMBaerbuLwBLnhiov2DGi1BTfMBae% XatLxBI9gBaerbd9wDYLwzYbItLDharqqtubsr4rNCHbGeaGqiVu0J% e9sqqrpepC0xbbL8F4rqqrFfpeea0xe9Lq-Jc9vqaqpepm0xbba9pw% e9Q8fs0-yqaqpepae9pg0FirpepeKkFr0xfr-xfr-xb9adbaqaaeGa% ciGaaiaabeqaamaabaabaaGcbaGaam4vamaaDaaaleaacaWGPbaaba% GaamOBaaaakiabloKi7iaadEfacaGGOaGaamiEamaaBaaaleaacaWG% PbaabeaakiaacYcacaWG0bWaaSbaaSqaaiaad6gaaeqaaOGaaiykai% aac6caaaa!3D63!
\begin{equation}W_i^n  \simeq W(x_i ,t_n ).\end{equation}

The numerical scheme for the convective terms is a standard Rusanov scheme
 for non-conservative systems, which reads% MathType!MTEF!2!1!+-
% faaafaart1ev1aaat0uyJj1BTfMBaerbuLwBLnhiov2DGi1BTfMBae
% XatLxBI9gBaerbd9wDYLwzYbItLDharqqtubsr4rNCHbGeaGqiVu0J
% e9sqqrpepC0xbbL8F4rqqrFfpeea0xe9Lq-Jc9vqaqpepm0xbba9pw
% e9Q8fs0-yqaqpepae9pg0FirpepeKkFr0xfr-xfr-xb9adbaqaaeGa
% ciGaaiaabeqaamaabaabaaGcbaWaaSaaaeaacaWGxbWaa0baaSqaai
% aadMgaaeaacaWGUbGaey4kaSIaaGymaiaac+cacaaIYaaaaOGaeyOe
% I0Iaam4vamaaDaaaleaacaWGPbaabaGaamOBaaaaaOqaaiabes8a0b
% aacqGHRaWkdaWcaaqaaiaadAeadaqhaaWcbaGaamyAaiabgUcaRiaa
% igdacaGGVaGaaGOmaaqaaiaad6gaaaGccqGHsislcaWGgbWaa0baaS
% qaaiaadMgacqGHsislcaaIXaGaai4laiaaikdaaeaacaWGUbaaaaGc
% baGaamiAaaaacqGHRaWkcaWGbbGaaiikaiaadEfadaqhaaWcbaGaam
% yAaaqaaiaad6gaaaGccaGGPaWaaSaaaeaacaWGmbWaa0baaSqaaiaa
% dMgacqGHRaWkcaaIXaaabaGaamOBaaaakiabgkHiTiaadYeadaqhaa
% WcbaGaamyAaiabgkHiTiaaigdaaeaacaWGUbaaaaGcbaGaaGOmaiaa
% dIgaaaGaeyypa0JaaGimaaaa!5F41!
\begin{equation}\label{rusa}
 h({W_i^{n + 1,-}  - W_i^n })
  + {\Delta t} ({F_{i + 1/2}^n  - F_{i - 1/2}^n })
  + {\Delta t} A(W_i^n )\frac{{L_{i + 1}^n  - L_{i - 1}^n }}
{{2}} = 0.
\end{equation}

The conservative numerical flux is given by
\begin{equation}\label{rusa2}
\begin{gathered}
  F(W_L ,W_R ) = \frac{{F(W_L ) + F(W_R )}}
{2} - \zeta \frac{{W_R  - W_L }}
{2}, \hfill \\
  \zeta = \max \left( {\rho (B(Y_L )),\rho (B(Y_R ))} \right). \hfill \\
\end{gathered}
\end{equation}

where $\rho(B)$ denotes the spectral radius of the matrix $B$. The convection matrix $B(Y)$ in the primitive variables $Y$ (see (\ref{defy})) is given in (\ref{by}) in Section \ref{app}. Our choice for the numerical viscosity parameter $\zeta$ is classical (see for example \cite{harten83}). It usually leads to an entropy dissipative scheme. In exceptional cases, it may be necessary to compute the numerical  viscosity from interface values instead of the cell values $Y_L$ and $Y_R$.

Our particular choice of the non-conservative terms $L$ ensures that constant velocity-pressure states will be maintained by the Rusanov scheme. Of course, in many interesting computations the velocity and pressure are not constant. However, it has been observed that one obtains better numerical results if the scheme is able to capture exactly the constant velocity and pressure solutions.

This convection step permits to obtain a value $W_i^{n + 1,-}$ in each cell $i$. It has now  to be updated in order to take into account the pressure equilibrium source and obtain $W_i^{n + 1}$.

\section{Relaxation algorithm}
In this section, we address now the numerical approximation of
the pressure relaxation source term of the system (\ref{first-eq})-(\ref{last-eq}). As usual, we use a fractional
step method in order to separate the convection step and the pressure
equilibrium step.

Thus, we concentrate only on the description of the pressure
equilibrium step, which can be formally written
% MathType!MTEF!2!1!+-
% feaafiart1ev1aaatCvAUfeBSjuyZL2yd9gzLbvyNv2CaerbuLwBLn
% hiov2DGi1BTfMBaeXatLxBI9gBaerbd9wDYLwzYbItLDharqqtubsr
% 4rNCHbGeaGqiVu0Je9sqqrpepC0xbbL8F4rqqrFfpeea0xe9Lq-Jc9
% vqaqpepm0xbba9pwe9Q8fs0-yqaqpepae9pg0FirpepeKkFr0xfr-x
% fr-xb9adbaqaaeGaciGaaiaabeqaamaabaabaaGceaqabeaacqaHXo
% qydaWgaaWcbaGaam4AaiaacYcacaWG0baabeaakiabg2da9iabggla
% XkaadcfacaGGSaaabaGaamyBamaaBaaaleaacaWGRbGaaiilaiaads
% haaeqaaOGaeyypa0JaamyDamaaBaaaleaacaWGRbGaaiilaiaadsha
% aeqaaOGaeyypa0JaaGimaiaacYcaaeaacaGGOaGaamyBamaaBaaale
% aacaWGRbaabeaakiaadwgadaWgaaWcbaGaam4AaaqabaGccaGGPaWa
% aSbaaSqaaiaadshaaeqaaOGaey4kaSIaamiCamaaBaaaleaacaaIXa
% aabeaakiabeg7aHnaaBaaaleaacaWGRbGaaiilaiaadshaaeqaaOGa
% eyypa0JaaGimaiaac6caaaaa!5A34!
\begin{equation}\label{formeq}
\begin{gathered}
  \alpha _{k,t}  =  \sigma_k P, \hfill \\
  m_{k,t}  = u_{k,t}  = 0, \hfill \\
  (m_k e_k )_t  + p_1 \alpha _{k,t}  = 0. \hfill \\
\end{gathered}
\end{equation}

%Because, the equilibrium is supposed to be instantaneous we have
%actually to update the volume fraction in such a way that we recover
%the relation $p_2=p_1+R$.

In order to simplify the notations, we denote now by a 0 superscript the physical values in a given cell $i$ at
the end of the advection step. These values are computed from the vector $W_i^{n+1,-}$, at the end of  the convection step, 
given by the Rusanov scheme (\ref{rusa}). The updated values at time $n+1$ are noted
without any superscript.

Owing to mass and momentum
conservation we have $m_k=m_k^0$ and $u_k=u_k^0$. We have now to
compute $(\alpha_1,p_1,p_2)$ in order to pursue the computation. The system is% MathType!MTEF!2!1!+-
% faaafaart1ev1aaat0uyJj1BTfMBaerbuLwBLnhiov2DGi1BTfMBae
% XatLxBI9gBaerbd9wDYLwzYbItLDharqqtubsr4rNCHbGeaGqiVu0J
% e9sqqrpepC0xbbL8F4rqqrFfpeea0xe9Lq-Jc9vqaqpepm0xbba9pw
% e9Q8fs0-yqaqpepae9pg0FirpepeKkFr0xfr-xfr-xb9adbaqaaeGa
% ciGaaiaabeqaamaabaabaaGceaqabeaacaWGWbWaaSbaaSqaaiaaik
% daaeqaaOGaeyOeI0IaamOuaiabgkHiTiaadchadaWgaaWcbaGaaGym
% aaqabaGccqGH9aqpcqaHepaDdaWgaaWcbaGaamiCaaqabaGccqaHXo
% qydaWgaaWcbaGaaGOmaiaacYcacaWG0baabeaakiaacYcaaeaacaWG
% TbWaaSbaaSqaaiaaigdaaeqaaOGaamyzamaaBaaaleaacaaIXaaabe
% aakiabgUcaRiaad2gadaWgaaWcbaGaaGOmaaqabaGccaWGLbWaaSba
% aSqaaiaaikdaaeqaaOGaeyypa0JaamyBamaaDaaaleaacaaIXaaaba
% GaaGimaaaakiaadwgadaqhaaWcbaGaaGymaaqaaiaaicdaaaGccqGH
% RaWkcaWGTbWaa0baaSqaaiaaikdaaeaacaaIWaaaaOGaamyzamaaDa
% aaleaacaaIYaaabaGaaGimaaaakiaacYcaaeaacaGGOaGaamyBamaa
% BaaaleaacaaIXaaabeaakiaadwgadaWgaaWcbaGaaGymaaqabaGccq
% GHsislcaWGTbWaa0baaSqaaiaaigdaaeaacaaIWaaaaOGaamyzamaa
% DaaaleaacaaIXaaabaGaaGimaaaakiaacMcacqGHRaWkcaWGWbWaaS
% baaSqaaiaaigdaaeqaaOGaaiikaiabeg7aHnaaBaaaleaacaaIXaaa
% beaakiabgkHiTiabeg7aHnaaDaaaleaacaaIXaaabaGaaGimaaaaki
% aacMcacqGH9aqpcaaIWaGaaiOlaaaaaa!6EBF!
\begin{equation}\label{relaxstep}
\begin{gathered}
  p_2  - R - p_1  = \tau _p \alpha _{2,t} , \hfill \\
  m_1 e_1  + m_2 e_2  = m_1^0 e_1^0  + m_2^0 e_2^0 , \hfill \\
  (m_1 e_1  - m_1^0 e_1^0 ) + p_1 (\alpha _1  - \alpha _1^0 ) = 0. \hfill \\
\end{gathered}
\end{equation}


We recall that the pressures of the two phases obey
stiffened gas equations of state% MathType!Translator!2!1!AMS LaTeX.tdl!TeX -- AMS-LaTeX!
% MathType!MTEF!2!1!+-
% feaafiart1ev1aaatCvAUfeBSjuyZL2yd9gzLbvyNv2CaerbuLwBLn
% hiov2DGi1BTfMBaeXatLxBI9gBaerbd9wDYLwzYbItLDharqqtubsr
% 4rNCHbGeaGqiVu0Je9sqqrpepC0xbbL8F4rqqrFfpeea0xe9Lq-Jc9
% vqaqpepm0xbba9pwe9Q8fs0-yqaqpepae9pg0FirpepeKkFr0xfr-x
% fr-xb9adbaqaaeGaciGaaiaabeqaamaabaabaaGcbaGaamiCamaaBa
% aaleaacaWGRbaabeaakiaacIcacqaHbpGCdaWgaaWcbaGaam4Aaaqa
% baGccaGGSaGaamyzamaaBaaaleaacaWGRbaabeaakiaacMcacqGH9a
% qpcaGGOaGaeq4SdC2aaSbaaSqaaiaadUgaaeqaaOGaeyOeI0IaaGym
% aiaacMcacqaHbpGCdaWgaaWcbaGaam4AaaqabaGccaWGLbWaaSbaaS
% qaaiaadUgaaeqaaOGaeyOeI0Iaeq4SdC2aaSbaaSqaaiaadUgaaeqa
% aOGaeqiWda3aaSbaaSqaaiaadUgaaeqaaOGaaiilaiaaywW7cqaHZo
% WzdaWgaaWcbaGaam4AaaqabaGccqGH+aGpcaaIXaGaaiilaiaaywW7
% caWGRbGaeyypa0JaaGymaiaacYcacaaIYaGaaiOlaaaa!5F49!
\begin{equation}
p_k (\rho _k ,e_k ) = (\gamma _k  - 1)\rho _k e_k  - \gamma _k \pi _k ,\quad \gamma _k  > 1,\quad k = 1,2.
\end{equation}
% MathType!End!2!1!

It is physically reasonable to suppose that
% MathType!MTEF!2!1!+-
% feaafiart1ev1aaatCvAUfeBSjuyZL2yd9gzLbvyNv2CaerbuLwBLn
% hiov2DGi1BTfMBaeXatLxBI9gBaerbd9wDYLwzYbItLDharqqtubsr
% 4rNCHbGeaGqiVu0Je9sqqrpepC0xbbL8F4rqqrFfpeea0xe9Lq-Jc9
% vqaqpepm0xbba9pwe9Q8fs0-yqaqpepae9pg0FirpepeKkFr0xfr-x
% fr-xb9adbaqaaeGaciGaaiaabeqaamaabaabaaGcbaGaeqiWda3aaS
% baaSqaaiaaikdaaeqaaOGaeyOpa4JaeqiWda3aaSbaaSqaaiaaigda
% aeqaaaaa!3C48!
\begin{equation}\label{liq}
\pi _2  > \pi _1,
\end{equation}
because phase (2) is the solid phase and because in the gas phase (1) $\pi_1 \simeq 0$. We also recall that the granular stress $R$ is given by (\ref{nice}). Because of the
stiffened gas law, we have% MathType!MTEF!2!1!+-
% feaafiart1ev1aaatCvAUfeBSjuyZL2yd9gzLbvyNv2CaerbuLwBLn
% hiov2DGi1BTfMBaeXatLxBI9gBaerbd9wDYLwzYbItLDharqqtubsr
% 4rNCHbGeaGqiVu0Je9sqqrpepC0xbbL8F4rqqrFfpeea0xe9Lq-Jc9
% vqaqpepm0xbba9pwe9Q8fs0-yqaqpepae9pg0FirpepeKkFr0xfr-x
% fr-xb9adbaqaaeGaciGaaiaabeqaamaabaabaaGcbaGaamyBamaaBa
% aaleaacaWGRbaabeaakiaadwgadaWgaaWcbaGaam4AaaqabaGccqGH
% 9aqpcqaHXoqydaWgaaWcbaGaam4AaaqabaGcdaWcaaqaaiaadchada
% WgaaWcbaGaam4AaaqabaGccqGHRaWkcqaHZoWzdaWgaaWcbaGaam4A
% aaqabaGccqaHapaCdaWgaaWcbaGaam4AaaqabaaakeaacqaHZoWzda
% WgaaWcbaGaam4AaaqabaGccqGHsislcaaIXaaaaaaa!4B12!
\begin{equation}
m_k e_k  = \alpha _k \frac{{p_k  + \gamma _k \pi _k }} {{\gamma _k
- 1}}.
\end{equation}
Thus
we have to solve for $(\alpha_1,p_1,p_2)$ the following three-equation system, at each time step and in each cell % MathType!MTEF!2!1!+-
% faaafaart1ev1aaat0uyJj1BTfMBaerbuLwBLnhiov2DGi1BTfMBae
% XatLxBI9gBaerbd9wDYLwzYbItLDharqqtubsr4rNCHbGeaGqiVu0J
% e9sqqrpepC0xbbL8F4rqqrFfpeea0xe9Lq-Jc9vqaqpepm0xbba9pw
% e9Q8fs0-yqaqpepae9pg0FirpepeKkFr0xfr-xfr-xb9adbaqaaeGa
% ciGaaiaabeqaamaabaabaaGceaqabeaacaWGWbWaaSbaaSqaaiaaik
% daaeqaaOGaeyOeI0IaeqOUdSMaeqySde2aa0baaSqaaiaaikdaaeaa
% cqaHZoWzdaWgaaadbaGaaGOmaaqabaaaaOGaeqyWdi3aa0baaSqaai
% aaikdaaeaacqaHZoWzdaWgaaadbaGaaGOmaaqabaaaaOGaeyOeI0Ia
% amiCamaaBaaaleaacaaIXaaabeaakiabg2da9maalaaabaGaeqiXdq
% 3aaSbaaSqaaiaadchaaeqaaaGcbaGaeqySde2aaSbaaSqaaiaaikda
% aeqaaOGaaiikaiaaigdacqGHsislcqaHXoqydaWgaaWcbaGaaGOmaa
% qabaGccaGGPaaaaiabeg7aHnaaBaaaleaacaaIYaGaaiilaiaadsha
% aeqaaOGaaiilaaqaaiabeg7aHnaaBaaaleaacaaIYaaabeaakmaala
% aabaGaamiCamaaBaaaleaacaaIYaaabeaakiabgUcaRiabec8aWnaa
% BaaaleaacaaIYaaabeaaaOqaaiabeo7aNnaaBaaaleaacaaIYaaabe
% aakiabgkHiTiaaigdaaaGaeyOeI0IaeqySde2aa0baaSqaaiaaikda
% aeaacaaIWaaaaOWaaSaaaeaacaWGWbWaa0baaSqaaiaaikdaaeaaca
% aIWaaaaOGaey4kaSIaeqiWda3aaSbaaSqaaiaaikdaaeqaaaGcbaGa
% eq4SdC2aaSbaaSqaaiaaikdaaeqaaOGaeyOeI0IaaGymaaaacqGHRa
% WkcaGGOaGaamiCamaaBaaaleaacaaIXaaabeaakiabgUcaRiabec8a
% WnaaBaaaleaacaaIYaaabeaakiaacMcacaGGOaGaeqySde2aaSbaaS
% qaaiaaikdaaeqaaOGaeyOeI0IaeqySde2aa0baaSqaaiaaikdaaeaa
% caaIWaaaaOGaaiykaiabg2da9iaaicdacaGGSaaabaGaeqySde2aaS
% baaSqaaiaaigdaaeqaaOWaaSaaaeaacaWGWbWaaSbaaSqaaiaaigda
% aeqaaOGaey4kaSIaeqiWda3aaSbaaSqaaiaaigdaaeqaaaGcbaGaeq
% 4SdC2aaSbaaSqaaiaaigdaaeqaaOGaeyOeI0IaaGymaaaacqGHsisl
% cqaHXoqydaqhaaWcbaGaaGymaaqaaiaaicdaaaGcdaWcaaqaaiaadc
% hadaqhaaWcbaGaaGymaaqaaiaaicdaaaGccqGHRaWkcqaHapaCdaWg
% aaWcbaGaaGymaaqabaaakeaacqaHZoWzdaWgaaWcbaGaaGymaaqaba
% GccqGHsislcaaIXaaaaiabgUcaRiaacIcacaWGWbWaaSbaaSqaaiaa
% igdaaeqaaOGaey4kaSIaeqiWda3aaSbaaSqaaiaaigdaaeqaaOGaai
% ykaiaacIcacqaHXoqydaWgaaWcbaGaaGymaaqabaGccqGHsislcqaH
% XoqydaqhaaWcbaGaaGymaaqaaiaaicdaaaGccaGGPaGaeyypa0JaaG
% imaiaac6caaaaa!AEEC!
\begin{equation}\label{51}
\begin{gathered}
  p_2  - \kappa \alpha _2^{\gamma _2 } \rho _2^{\gamma _2 }  - p_1  = \frac{{\tau _p }}
{{\alpha _2 (1 - \alpha _2 )}}\alpha _{2,t} , \hfill \\
  \alpha _2 \frac{{p_2  + \pi _2 }}
{{\gamma _2  - 1}} - \alpha _2^0 \frac{{p_2^0  + \pi _2 }}
{{\gamma _2  - 1}} + (p_1  + \pi _2 )(\alpha _2  - \alpha _2^0 ) = 0, \hfill \\
  \alpha _1 \frac{{p_1  + \pi _1 }}
{{\gamma _1  - 1}} - \alpha _1^0 \frac{{p_1^0  + \pi _1 }}
{{\gamma _1  - 1}} + (p_1  + \pi _1 )(\alpha _1  - \alpha _1^0 ) = 0. \hfill \\ 
\end{gathered} 
\end{equation}





We have
% MathType!MTEF!2!1!+-
% faaafaart1ev1aaat0uyJj1BTfMBaerbuLwBLnhiov2DGi1BTfMBae
% XatLxBI9gBaerbd9wDYLwzYbItLDharqqtubsr4rNCHbGeaGqiVu0J
% e9sqqrpepC0xbbL8F4rqqrFfpeea0xe9Lq-Jc9vqaqpepm0xbba9pw
% e9Q8fs0-yqaqpepae9pg0FirpepeKkFr0xfr-xfr-xb9adbaqaaeGa
% ciGaaiaabeqaamaabaabaaGceaqabeaacaWGWbWaaSbaaSqaaiaaik
% daaeqaaOGaeyOeI0IaeqOUdSMaeqySde2aa0baaSqaaiaaikdaaeaa
% cqaHZoWzdaWgaaadbaGaaGOmaaqabaaaaOGaeqyWdi3aa0baaSqaai
% aaikdaaeaacqaHZoWzdaWgaaadbaGaaGOmaaqabaaaaOGaeyOeI0Ia
% amiCamaaBaaaleaacaaIXaaabeaakiabg2da9maalaaabaGaeqiXdq
% 3aaSbaaSqaaiaadchaaeqaaaGcbaGaeqySde2aaSbaaSqaaiaaikda
% aeqaaOGaaiikaiaaigdacqGHsislcqaHXoqydaWgaaWcbaGaaGOmaa
% qabaGccaGGPaaaaiabeg7aHnaaBaaaleaacaaIYaGaaiilaiaadsha
% aeqaaOGaaiilaaqaaiaacIcacqaHXoqydaWgaaWcbaGaaGOmaaqaba
% GccqGHRaWkcaGGOaGaeq4SdC2aaSbaaSqaaiaaikdaaeqaaOGaeyOe
% I0IaaGymaiaacMcacaGGOaGaeqySde2aaSbaaSqaaiaaikdaaeqaaO
% GaeyOeI0IaeqySde2aa0baaSqaaiaaikdaaeaacaaIWaaaaOGaaiyk
% aiaacMcacaGGOaGaamiCamaaBaaaleaacaaIYaaabeaakiabgUcaRi
% abec8aWnaaBaaaleaacaaIYaaabeaakiaacMcacqGHsislcqaHXoqy
% daqhaaWcbaGaaGOmaaqaaiaaicdaaaGccaGGOaGaamiCamaaDaaale
% aacaaIYaaabaGaaGimaaaakiabgUcaRiabec8aWnaaBaaaleaacaaI
% YaaabeaakiaacMcacqGHsislaeaacaGGOaGaeq4SdC2aaSbaaSqaai
% aaikdaaeqaaOGaeyOeI0IaaGymaiaacMcacaGGOaGaeqySde2aaSba
% aSqaaiaaikdaaeqaaOGaeqyWdi3aa0baaSqaaiaaikdaaeaacqaHZo
% WzdaWgaaadbaGaaGOmaaqabaaaaOGaeqiUdeNaaiikaiabeg7aHnaa
% BaaaleaacaaIYaaabeaakiaacMcacqGHRaWkcqaHepaDcqaHXoqyda
% WgaaWcbaGaaGOmaiaacYcacaWG0baabeaakiaacMcacaGGOaGaeqyS
% de2aaSbaaSqaaiaaikdaaeqaaOGaeyOeI0IaeqySde2aa0baaSqaai
% aaikdaaeaacaaIWaaaaOGaaiykaiabg2da9iaaicdacaGGSaaabaGa
% aiikaiabeg7aHnaaBaaaleaacaaIXaaabeaakiabgUcaRiaacIcacq
% aHZoWzdaWgaaWcbaGaaGymaaqabaGccqGHsislcaaIXaGaaiykaiaa
% cIcacqaHXoqydaWgaaWcbaGaaGymaaqabaGccqGHsislcqaHXoqyda
% qhaaWcbaGaaGymaaqaaiaaicdaaaGccaGGPaGaaiykaiaacIcacaWG
% WbWaaSbaaSqaaiaaigdaaeqaaOGaey4kaSIaeqiWda3aaSbaaSqaai
% aaigdaaeqaaOGaaiykaiabgkHiTiabeg7aHnaaDaaaleaacaaIXaaa
% baGaaGimaaaakiaacIcacaWGWbWaa0baaSqaaiaaigdaaeaacaaIWa
% aaaOGaey4kaSIaeqiWda3aaSbaaSqaaiaaigdaaeqaaOGaaiykaiab
% g2da9iaaicdacaGGUaaaaaa!C7AD!
\begin{equation}\label{}
\begin{gathered}
  p_2  - \kappa \alpha _2^{\gamma _2 } \rho _2^{\gamma _2 }  - p_1  = \frac{{\tau _p }}
{{\alpha _2 (1 - \alpha _2 )}}\alpha _{2,t} , \hfill \\
  (\alpha _2  + (\gamma _2  - 1)(\alpha _2  - \alpha _2^0 ))(p_2  + \pi _2 ) - \alpha _2^0 (p_2^0  + \pi _2 ) -  \hfill \\
  (\gamma _2  - 1)(\alpha _2 \rho _2^{\gamma _2 } \theta (\alpha _2 ) + \tau \alpha _{2,t} )(\alpha _2  - \alpha _2^0 ) = 0, \hfill \\
  (\alpha _1  + (\gamma _1  - 1)(\alpha _1  - \alpha _1^0 ))(p_1  + \pi _1 ) - \alpha _1^0 (p_1^0  + \pi _1 ) = 0. \hfill \\ 
\end{gathered} 
\end{equation}

We then note
% MathType!MTEF!2!1!+-
% feaafiart1ev1aaatCvAUfeBSjuyZL2yd9gzLbvyNv2CaerbuLwBLn
% hiov2DGi1BTfMBaeXatLxBI9gBaerbd9wDYLwzYbItLDharqqtubsr
% 4rNCHbGeaGqiVu0Je9sqqrpepC0xbbL8F4rqqrFfpeea0xe9Lq-Jc9
% vqaqpepm0xbba9pwe9Q8fs0-yqaqpepae9pg0FirpepeKkFr0xfr-x
% fr-xb9adbaqaaeGaciGaaiaabeqaamaabaabaaGceaqabeaacaWGbb
% WaaSbaaSqaaiaaigdaaeqaaOGaeyypa0JaeqySde2aa0baaSqaaiaa
% igdaaeaacaaIWaaaaOGaaiikaiaadchadaqhaaWcbaGaaGymaaqaai
% aaicdaaaGccqGHRaWkcqaHapaCdaWgaaWcbaGaaGymaaqabaGccaGG
% PaGaaiilaaqaaiaadgeadaWgaaWcbaGaaGOmaaqabaGccqGH9aqpcq
% aHXoqydaqhaaWcbaGaaGOmaaqaaiaaicdaaaGccaGGOaGaamiCamaa
% DaaaleaacaaIYaaabaGaaGimaaaakiabgUcaRiabec8aWnaaBaaale
% aacaaIYaaabeaakiaacMcacaGGUaaaaaa!527E!
\begin{equation}
\begin{gathered}
  A_1  = \alpha _1^0 (p_1^0  + \pi _1 ), \hfill \\
  A_2  = \alpha _2^0 (p_2^0  + \pi _2 ). \hfill \\
\end{gathered}
\end{equation}
For a stiffened gas law, the sound speed $c$ is given by the formula% MathType!MTEF!2!1!+-
% feaafiart1ev1aaatCvAUfeBSjuyZL2yd9gzLbvyNv2CaerbuLwBLn
% hiov2DGi1BTfMBaeXatLxBI9gBaerbd9wDYLwzYbItLDharqqtubsr
% 4rNCHbGeaGqiVu0Je9sqqrpepC0xbbL8F4rqqrFfpeea0xe9Lq-Jc9
% vqaqpepm0xbba9pwe9Q8fs0-yqaqpepae9pg0FirpepeKkFr0xfr-x
% fr-xb9adbaqaaeGaciGaaiaabeqaamaabaabaaGcbaGaam4yaiabg2
% da9maakaaabaWaaSaaaeaacqaHZoWzcaGGOaGaamiCaiabgUcaRiab
% ec8aWjaacMcaaeaacqaHbpGCaaaaleqaaOGaaiOlaaaa!4116!
\begin{equation}
c = \sqrt {\frac{{\gamma (p + \pi )}} {\rho }} .
\end{equation}
It implies that the two quantities $A_1$ and $A_2$ are $>0$.
Of course, we suppose that the initial volume fraction $0< \alpha_2^0
< 1$.

After the elimination of $p_1$ and $p_2$, the system can be
rewritten

% MathType!MTEF!2!1!+-
% faaafaart1ev1aaat0uyJj1BTfMBaerbuLwBLnhiov2DGi1BTfMBae
% XatLxBI9gBaerbd9wDYLwzYbItLDharqqtubsr4rNCHbGeaGqiVu0J
% e9sqqrpepC0xbbL8F4rqqrFfpeea0xe9Lq-Jc9vqaqpepm0xbba9pw
% e9Q8fs0-yqaqpepae9pg0FirpepeKkFr0xfr-xfr-xb9adbaqaaeGa
% ciGaaiaabeqaamaabaabaaGceaqabeaacaWGhbGaaiikaiabeg7aHn
% aaBaaaleaacaaIYaaabeaakiaacMcacqGH9aqpcaGGOaGaeqiWda3a
% aSbaaSqaaiaaikdaaeqaaOGaeyOeI0IaeqiWda3aaSbaaSqaaiaaig
% daaeqaaOGaaiykaiaacIcacqaHXoqydaWgaaWcbaGaaGymaaqabaGc
% cqGHRaWkcaGGOaGaeq4SdC2aaSbaaSqaaiaaigdaaeqaaOGaeyOeI0
% IaaGymaiaacMcacaGGOaGaeqySde2aaSbaaSqaaiaaigdaaeqaaOGa
% eyOeI0IaeqySde2aa0baaSqaaiaaigdaaeaacaaIWaaaaOGaaiykai
% aacMcacaGGOaGaeqySde2aaSbaaSqaaiaaikdaaeqaaOGaey4kaSIa
% aiikaiabeo7aNnaaBaaaleaacaaIYaaabeaakiabgkHiTiaaigdaca
% GGPaGaaiikaiabeg7aHnaaBaaaleaacaaIYaaabeaakiabgkHiTiab
% eg7aHnaaDaaaleaacaaIYaaabaGaaGimaaaakiaacMcacaGGPaaaba
% Gaey4kaSIaaiikaiabeQ7aRjabeg7aHnaaBaaaleaacaaIYaaabeaa
% kiaad2gadaqhaaWcbaGaaGOmaaqaaiabeo7aNnaaBaaameaacaaIYa
% aabeaaaaGccqGHRaWkdaWcaaqaaiabes8a0naaBaaaleaacaWGWbaa
% beaaaOqaaiaacIcacaaIXaGaeyOeI0IaeqySde2aaSbaaSqaaiaaik
% daaeqaaOGaaiykaaaacqaHXoqydaWgaaWcbaGaaGOmaiaacYcacaWG
% 0baabeaakiabgkHiTiaadgeadaWgaaWcbaGaaGOmaaqabaGccaGGPa
% Gaaiikaiabeg7aHnaaBaaaleaacaaIXaaabeaakiabgUcaRiaacIca
% cqaHZoWzdaWgaaWcbaGaaGymaaqabaGccqGHsislcaaIXaGaaiykai
% aacIcacqaHXoqydaWgaaWcbaGaaGymaaqabaGccqGHsislcqaHXoqy
% daqhaaWcbaGaaGymaaqaaiaaicdaaaGccaGGPaGaaiykaaqaaiabgU
% caRiaadgeadaWgaaWcbaGaaGymaaqabaGccaGGOaGaeqySde2aaSba
% aSqaaiaaikdaaeqaaOGaey4kaSIaaiikaiabeo7aNnaaBaaaleaaca
% aIYaaabeaakiabgkHiTiaaigdacaGGPaGaaiikaiabeg7aHnaaBaaa
% leaacaaIYaaabeaakiabgkHiTiabeg7aHnaaDaaaleaacaaIYaaaba
% GaaGimaaaakiaacMcacaGGPaGaeyypa0JaaGimaaaaaa!A886!
\begin{equation}\label{defG}
\begin{gathered}
  G(\alpha _2 ) = (\pi _2  - \pi _1 )(\alpha _1  + (\gamma _1  - 1)(\alpha _1  - \alpha _1^0 ))(\alpha _2  + (\gamma _2  - 1)(\alpha _2  - \alpha _2^0 )) \hfill \\
   + (\kappa \alpha _2 m_2^{\gamma _2 }  + \frac{{\tau _p }}
{{(1 - \alpha _2 )}}\alpha _{2,t}  - A_2 )(\alpha _1  + (\gamma _1  - 1)(\alpha _1  - \alpha _1^0 )) \hfill \\
   + A_1 (\alpha _2  + (\gamma _2  - 1)(\alpha _2  - \alpha _2^0 )) = 0 \hfill \\ 
\end{gathered} 
\end{equation}
We use an implicit first-order time discretisation of $\alpha _{2,t}$
% MathType!MTEF!2!1!+-
% feaafiart1ev1aaatCvAUfeBSjuyZL2yd9gzLbvyNv2CaerbuLwBLn
% hiov2DGi1BTfMBaeXatLxBI9gBaerbd9wDYLwzYbItLDharqqtubsr
% 4rNCHbGeaGqiVu0Je9sqqrpepC0xbbL8F4rqqrFfpeea0xe9Lq-Jc9
% vqaqpepm0xbba9pwe9Q8fs0-yqaqpepae9pg0FirpepeKkFr0xfr-x
% fr-xb9adbaqaaeGaciGaaiaabeqaamaabaabaaGcbaGaeqySde2aaS
% baaSqaaiaaikdacaGGSaGaamiDaaqabaGccqGH9aqpdaWcaaqaaiab
% eg7aHnaaBaaaleaacaaIYaaabeaakiabgkHiTiabeg7aHnaaDaaale
% aacaaIYaaabaGaaGimaaaaaOqaaiabfs5aejaadshaaaGaaiOlaaaa
% !4518!
\begin{equation}\label{impl}
\alpha _{2,t}  = \frac{{\alpha _2  - \alpha _2^0 }} {{{\Delta t}}}.
\end{equation}
(recall that ${\Delta t}$ is the time step in the convection step (\ref{rusa})). The implicit approach is natural because $\tau_p$ may be arbitrary small and thus the source term (\ref{formsource}) may be stiff.






We first compute $G$ at the left point of the interval $[0,1]$
% MathType!MTEF!2!1!+-
% faaafaart1ev1aaat0uyJj1BTfMBaerbuLwBLnhiov2DGi1BTfMBae
% XatLxBI9gBaerbd9wDYLwzYbItLDharqqtubsr4rNCHbGeaGqiVu0J
% e9sqqrpepC0xbbL8F4rqqrFfpeea0xe9Lq-Jc9vqaqpepm0xbba9pw
% e9Q8fs0-yqaqpepae9pg0FirpepeKkFr0xfr-xfr-xb9adbaqaaeGa
% ciGaaiaabeqaamaabaabaaGceaqabeaacaWGhbGaaiikaiaaicdaca
% GGPaGaeyypa0JaeyOeI0Iaaiikaiabec8aWnaaBaaaleaacaaIYaaa
% beaakiabgkHiTiabec8aWnaaBaaaleaacaaIXaaabeaakiaacMcaca
% GGOaGaeq4SdC2aaSbaaSqaaiaaikdaaeqaaOGaeyOeI0IaaGymaiaa
% cMcacqaHXoqydaqhaaWcbaGaaGOmaaqaaiaaicdaaaGccaGGOaGaaG
% ymaiabgUcaRiaacIcacqaHZoWzdaWgaaWcbaGaaGymaaqabaGccqGH
% sislcaaIXaGaaiykaiaacIcacaaIXaGaeyOeI0IaeqySde2aa0baaS
% qaaiaaigdaaeaacaaIWaaaaOGaaiykaiaacMcaaeaacqGHsislcaGG
% OaGaamyqamaaBaaaleaacaaIYaaabeaakiabgUcaRmaalaaabaGaeq
% iXdq3aaSbaaSqaaiaadchaaeqaaOGaeqySde2aa0baaSqaaiaaikda
% aeaacaaIWaaaaaGcbaGaeqiXdqhaaiaacMcacaGGOaGaaGymaiabgU
% caRiaacIcacqaHZoWzdaWgaaWcbaGaaGymaaqabaGccqGHsislcaaI
% XaGaaiykaiaacIcacaaIXaGaeyOeI0IaeqySde2aa0baaSqaaiaaig
% daaeaacaaIWaaaaOGaaiykaiaacMcaaeaacqGHsislcaWGbbWaaSba
% aSqaaiaaigdaaeqaaOGaaiikaiabeo7aNnaaBaaaleaacaaIYaaabe
% aakiabgkHiTiaaigdacaGGPaGaeqySde2aa0baaSqaaiaaikdaaeaa
% caaIWaaaaOGaeyipaWJaaGimaiaac6caaaaa!7F68!
\begin{equation}
\begin{gathered}
  G(0) =  - (\pi _2  - \pi _1 )(\gamma _2  - 1)\alpha _2^0 (1 + (\gamma _1  - 1)(1 - \alpha _1^0 )) \hfill \\
   - (A_2  + \frac{{\tau _p \alpha _2^0 }}
{{\Delta t} })(1 + (\gamma _1  - 1)(1 - \alpha _1^0 )) \hfill \\
   - A_1 (\gamma _2  - 1)\alpha _2^0  < 0, \hfill \\
\end{gathered}
\end{equation}
because of the hypothesis (\ref{condt}) and $A_1, A_2 >0$.
For the computation at the right point, we  introduce % MathType!MTEF!2!1!+-
% feaafiart1ev1aaatCvAUfeBSjuyZL2yd9gzLbvyNv2CaerbuLwBLn
% hiov2DGi1BTfMBaeXatLxBI9gBaerbd9wDYLwzYbItLDharqqtubsr
% 4rNCHbGeaGqiVu0Je9sqqrpepC0xbbL8F4rqqrFfpeea0xe9Lq-Jc9
% vqaqpepm0xbba9pwe9Q8fs0-yqaqpepae9pg0FirpepeKkFr0xfr-x
% fr-xb9adbaqaaeGaciGaaiaabeqaamaabaabaaGcbaGaeqOSdi2aaS
% baaSqaaiaaigdaaeqaaOGaeyypa0ZaaSaaaeaacqaHZoWzdaWgaaWc
% baGaaGymaaqabaGccqGHsislcaaIXaaabaGaeq4SdC2aaSbaaSqaai
% aaigdaaeqaaaaakiabeg7aHnaaDaaaleaacaaIXaaabaGaaGimaaaa
% aaa!43AE!
\begin{equation}
\beta _1  = \frac{{\gamma _1  - 1}} {{\gamma _1 }}\alpha _1^0.
\end{equation}
We have% MathType!MTEF!2!1!+-
% feaafiart1ev1aaatCvAUfeBSjuyZL2yd9gzLbvyNv2CaerbuLwBLn
% hiov2DGi1BTfMBaeXatLxBI9gBaerbd9wDYLwzYbItLDharqqtubsr
% 4rNCHbGeaGqiVu0Je9sqqrpepC0xbbL8F4rqqrFfpeea0xe9Lq-Jc9
% vqaqpepm0xbba9pwe9Q8fs0-yqaqpepae9pg0FirpepeKkFr0xfr-x
% fr-xb9adbaqaaeGaciGaaiaabeqaamaabaabaaGcbaGaeqySde2aaS
% baaSqaaiaaigdaaeqaaOGaey4kaSIaaiikaiabeo7aNnaaBaaaleaa
% caaIXaaabeaakiabgkHiTiaaigdacaGGPaGaaiikaiabeg7aHnaaBa
% aaleaacaaIXaaabeaakiabgkHiTiabeg7aHnaaDaaaleaacaaIXaaa
% baGaaGimaaaakiaacMcacqGH+aGpcaaIWaGaeyi1HSTaeqySde2aaS
% baaSqaaiaaigdaaeqaaOGaeyOpa4JaeqOSdi2aaSbaaSqaaiaaigda
% aeqaaOGaeyi1HSTaeqySde2aaSbaaSqaaiaaikdaaeqaaOGaeyipaW
% JaaGymaiabgkHiTiabek7aInaaBaaaleaacaaIXaaabeaaaaa!5B82!
\begin{equation}
\alpha _1  + (\gamma _1  - 1)(\alpha _1  - \alpha _1^0 ) > 0
\Leftrightarrow \alpha _1  > \beta _1  \Leftrightarrow \alpha _2  <
1 - \beta _1.
\end{equation}


We compute% MathType!MTEF!2!1!+-
% feaafiart1ev1aaatCvAUfeBSjuyZL2yd9gzLbvyNv2CaerbuLwBLn
% hiov2DGi1BTfMBaeXatLxBI9gBaerbd9wDYLwzYbItLDharqqtubsr
% 4rNCHbGeaGqiVu0Je9sqqrpepC0xbbL8F4rqqrFfpeea0xe9Lq-Jc9
% vqaqpepm0xbba9pwe9Q8fs0-yqaqpepae9pg0FirpepeKkFr0xfr-x
% fr-xb9adbaqaaeGaciGaaiaabeqaamaabaabaaGceaqabeaacaWGhb
% GaaiikaiaaigdacqGHsislcqaHYoGydaWgaaWcbaGaaGymaaqabaGc
% caGGPaGaeyypa0JaamyqamaaBaaaleaacaaIXaaabeaakiaacIcaca
% aIXaGaeyOeI0IaeqOSdi2aaSbaaSqaaiaaigdaaeqaaOGaey4kaSIa
% aiikaiabeo7aNnaaBaaaleaacaaIYaaabeaakiabgkHiTiaaigdaca
% GGPaGaaiikaiabeg7aHnaaDaaaleaacaaIXaaabaGaaGimaaaakiab
% gkHiTiabek7aInaaBaaaleaacaaIXaaabeaakiaacMcacaGGPaaaba
% Gaeyypa0ZaaSaaaeaacaWGbbWaaSbaaSqaaiaaigdaaeqaaaGcbaGa
% eq4SdC2aaSbaaSqaaiaaigdaaeqaaOGaeyOeI0IaaGymaaaadaqada
% qaaiaacIcacqaHZoWzdaWgaaWcbaGaaGymaaqabaGccqGHsislcaaI
% XaGaaiykaiaacIcacaaIXaGaeyOeI0IaeqOSdi2aaSbaaSqaaiaaig
% daaeqaaOGaaiykaiabgUcaRiaacIcacqaHZoWzdaWgaaWcbaGaaGOm
% aaqabaGccqGHsislcaaIXaGaaiykaiabek7aInaaBaaaleaacaaIXa
% aabeaaaOGaayjkaiaawMcaaiabg6da+iaaicdacaGGUaaaaaa!7286!
\begin{equation}
\begin{gathered}
  G(1 - \beta _1 ) = A_1 (1 - \beta _1  + (\gamma _2  - 1)(\alpha _1^0  - \beta _1 )) \hfill \\
   = \frac{{A_1 }}
{{\gamma _1  - 1}}\left( {(\gamma _1  - 1)(1 - \beta _1 ) + (\gamma _2  - 1)\beta _1 } \right) > 0. \hfill \\
\end{gathered}
\end{equation}


Thus we have existence of a solution $\alpha_2$ to $G(\alpha_2)=0$
in the interval $ ]0,1 - \beta _1 [$.




We have to check that this solution leads to physically relevant pressures
$p_1$ and $p_2$ \emph{i.e.} that $p_k+\pi_k>0$. We have
% MathType!MTEF!2!1!+-
% feaafiart1ev1aaatCvAUfeBSjuyZL2yd9gzLbvyNv2CaerbuLwBLn
% hiov2DGi1BTfMBaeXatLxBI9gBaerbd9wDYLwzYbItLDharqqtubsr
% 4rNCHbGeaGqiVu0Je9sqqrpepC0xbbL8F4rqqrFfpeea0xe9Lq-Jc9
% vqaqpepm0xbba9pwe9Q8fs0-yqaqpepae9pg0FirpepeKkFr0xfr-x
% fr-xb9adbaqaaeGaciGaaiaabeqaamaabaabaaGcbaGaamiCamaaBa
% aaleaacaaIXaaabeaakiabgUcaRiabec8aWnaaBaaaleaacaaIXaaa
% beaakiabg2da9maalaaabaGaamyqamaaBaaaleaacaaIXaaabeaaaO
% qaaiabeg7aHnaaBaaaleaacaaIXaaabeaakiabgUcaRiaacIcacqaH
% ZoWzdaWgaaWcbaGaaGymaaqabaGccqGHsislcaaIXaGaaiykaiaacI
% cacqaHXoqydaWgaaWcbaGaaGymaaqabaGccqGHsislcqaHXoqydaqh
% aaWcbaGaaGymaaqaaiaaicdaaaGccaGGPaaaaaaa!4F5C!
\begin{equation}\label{rap}
p_1  + \pi _1  = \frac{{A_1 }} {{\alpha _1  + (\gamma _1  -
1)(\alpha _1  - \alpha _1^0 )}}.
\end{equation}

This quantity is $>0$ if the solution satisfies $\alpha_2 <1-\beta_1$.
Finally we also have % MathType!MTEF!2!1!+-
% faaafaart1ev1aaat0uyJj1BTfMBaerbuLwBLnhiov2DGi1BTfMBae
% XatLxBI9gBaerbd9wDYLwzYbItLDharqqtubsr4rNCHbGeaGqiVu0J
% e9sqqrpepC0xbbL8F4rqqrFfpeea0xe9Lq-Jc9vqaqpepm0xbba9pw
% e9Q8fs0-yqaqpepae9pg0FirpepeKkFr0xfr-xfr-xb9adbaqaaeGa
% ciGaaiaabeqaamaabaabaaGcbaGaamiCamaaBaaaleaacaaIYaaabe
% aakiabgUcaRiabec8aWnaaBaaaleaacaaIYaaabeaakiabg2da9iaa
% dchadaWgaaWcbaGaaGymaaqabaGccqGHRaWkcqaHapaCdaWgaaWcba
% GaaGymaaqabaGccqGHRaWkcqaH6oWAcaWGTbWaa0baaSqaaiaaikda
% aeaacqaHZoWzdaWgaaadbaGaaGOmaaqabaaaaOGaey4kaSIaeqiXdq
% 3aaSbaaSqaaiaadchaaeqaaOWaaSaaaeaacqaHXoqydaWgaaWcbaGa
% aGOmaaqabaGccqGHsislcqaHXoqydaqhaaWcbaGaaGOmaaqaaiaaic
% daaaaakeaacqaHepaDcqaHXoqydaWgaaWcbaGaaGOmaaqabaGccaGG
% OaGaaGymaiabgkHiTiabeg7aHnaaBaaaleaacaaIYaaabeaakiaacM
% caaaGaey4kaSIaeqiWda3aaSbaaSqaaiaaikdaaeqaaOGaeyOeI0Ia
% eqiWda3aaSbaaSqaaiaaigdaaeqaaOGaaiOlaaaa!60D3!
\begin{equation}\label{}
p_2  + \pi _2  = p_1  + \pi _1  + \kappa m_2^{\gamma _2 }  + \tau _p \frac{{\alpha _2  - \alpha _2^0 }}
{{\tau \alpha _2 (1 - \alpha _2 )}} + \pi _2  - \pi _1 .
\end{equation}

When $\alpha _2  - \alpha _2^0 \geq 0$ the previous quantity is obviously $>0$. In the case $\alpha _2  - \alpha _2^0 <0$, we use the second equation of (\ref{51}) and we find
% MathType!MTEF!2!1!+-
% feaafiart1ev1aaatCvAUfeBSjuyZL2yd9gzLbvyNv2CaerbuLwBLn
% hiov2DGi1BTfMBaeXatLxBI9gBaerbd9wDYLwzYbItLDharqqtubsr
% 4rNCHbGeaGqiVu0Je9sqqrpepC0xbbL8F4rqqrFfpeea0xe9Lq-Jc9
% vqaqpepm0xbba9pwe9Q8fs0-yqaqpepae9pg0FirpepeKkFr0xfr-x
% fr-xb9adbaqaaeGaciGaaiaabeqaamaabaabaaGcbaGaeqySde2aaS
% baaSqaaiaaikdaaeqaaOWaaSaaaeaacaWGWbWaaSbaaSqaaiaaikda
% aeqaaOGaey4kaSIaeqiWda3aaSbaaSqaaiaaikdaaeqaaaGcbaGaeq
% 4SdC2aaSbaaSqaaiaaikdaaeqaaOGaeyOeI0IaaGymaaaacqGH9aqp
% cqaHXoqydaqhaaWcbaGaaGOmaaqaaiaaicdaaaGcdaWcaaqaaiaadc
% hadaqhaaWcbaGaaGOmaaqaaiaaicdaaaGccqGHRaWkcqaHapaCdaWg
% aaWcbaGaaGOmaaqabaaakeaacqaHZoWzdaWgaaWcbaGaaGOmaaqaba
% GccqGHsislcaaIXaaaaiabgkHiTiaacIcacaWGWbWaaSbaaSqaaiaa
% igdaaeqaaOGaey4kaSIaeqiWda3aaSbaaSqaaiaaigdaaeqaaOGaey
% 4kaSIaeqiWda3aaSbaaSqaaiaaikdaaeqaaOGaeyOeI0IaeqiWda3a
% aSbaaSqaaiaaigdaaeqaaOGaaiykaiaacIcacqaHXoqydaWgaaWcba
% GaaGOmaaqabaGccqGHsislcqaHXoqydaqhaaWcbaGaaGOmaaqaaiaa
% icdaaaGccaGGPaGaeyOpa4JaaGimaiaac6caaaa!6A9C!
\begin{equation}
\alpha _2 \frac{{p_2  + \pi _2 }}
{{\gamma _2  - 1}} = \alpha _2^0 \frac{{p_2^0  + \pi _2 }}
{{\gamma _2  - 1}} - (p_1  + \pi _1  + \pi _2  - \pi _1 )(\alpha _2  - \alpha _2^0 ) > 0.
\end{equation}






And the algorithm can continue.
\begin{remark}
Formula (\ref{rap}) shows that we have to discard any solution that
is not in the interval $]0,1-\beta_1[$. Lets us notice that in many
cases, we can find another solution to $G(\alpha_2)=0$ in
$]1-\beta_1,1[$.
\end{remark}

\begin{remark}
The previous reasoning can be easily generalized to the case where the granular stress is of the form% MathType!MTEF!2!1!+-
% faaafaart1ev1aaat0uyJj1BTfMBaerbuLwBLnhiov2DGi1BTfMBae
% XatLxBI9gBaerbd9wDYLwzYbItLDharqqtubsr4rNCHbGeaGqiVu0J
% e9sqqrpepC0xbbL8F4rqqrFfpeea0xe9Lq-Jc9vqaqpepm0xbba9pw
% e9Q8fs0-yqaqpepae9pg0FirpepeKkFr0xfr-xfr-xb9adbaqaaeGa
% ciGaaiaabeqaamaabaabaaGcbaGaamOuaiaacIcacqaHbpGCdaWgaa
% WcbaGaaGOmaaqabaGccaGGSaGaeqySde2aaSbaaSqaaiaaikdaaeqa
% aOGaaiykaiabg2da9iabeg8aYnaaDaaaleaacaaIYaaabaGaeq4SdC
% 2aaSbaaWqaaiaaikdaaeqaaaaakiabeg7aHnaaBaaaleaacaaIYaaa
% beaakiabeI7aXjaacIcacqaHXoqydaWgaaWcbaGaaGOmaaqabaGcca
% GGPaGaaiilaaaa!48B1!
\begin{equation}\label{granugene}
R(\rho _2 ,\alpha _2 ) = \rho _2^{\gamma _2 } \alpha _2 \theta (\alpha _2 ),
\end{equation}
and the function $\theta$ satisfies% MathType!MTEF!2!1!+-
% faaafaart1ev1aaat0uyJj1BTfMBaerbuLwBLnhiov2DGi1BTfMBae
% XatLxBI9gBaerbd9wDYLwzYbItLDharqqtubsr4rNCHbGeaGqiVu0J
% e9sqqrpepC0xbbL8F4rqqrFfpeea0xe9Lq-Jc9vqaqpepm0xbba9pw
% e9Q8fs0-yqaqpepae9pg0FirpepeKkFr0xfr-xfr-xb9adbaqaaeGa
% ciGaaiaabeqaamaabaabaaGceaqabeaacqaH4oqCcaqGGaGaae4yai
% aab+gacaqGUbGaaeiDaiaabMgacaqGUbGaaeyDaiaab+gacaqG1bGa
% ae4CaiaabccacaqGVbGaaeOBaiaabccacaGGBbGaaGimaiaacYcaca
% aIXaGaaiyxaiaacYcaaeaacqaH4oqCcaGGOaGaeqySdeMaaiykaiab
% gwMiZkaaicdacaGGSaaabaGaeqiUdeNaaiikaiabeg7aHjaacMcacq
% GH9aqpcaWGVbGaaiikaiabeg7aHnaaCaaaleqabaGaeq4SdC2aaSba
% aWqaaiaaikdaaeqaaSGaeyOeI0IaaGOmaaaakiaacMcacaqGGaGaae
% 4DaiaabIgacaqGLbGaaeOBaiaabccacqaHXoqycqGHsgIRcaaIWaGa
% aiOlaaaaaa!64ED!
\begin{equation}\label{condt}
\begin{gathered}
  \theta {\text{ continuous on }}[0,1], \hfill \\
  \theta (\alpha_2 ) \geqslant 0, \hfill \\
  \theta (\alpha_2 ) = o(\alpha_2 ^{\gamma _2  - 2} ){\text{ when }}\alpha_2  \to 0. \hfill \\
\end{gathered}
\end{equation}
These conditions are fulfilled by the granular stress that we proposed in (\ref{nice}) where% MathType!MTEF!2!1!+-
% faaafaart1ev1aaat0uyJj1BTfMBaerbuLwBLnhiov2DGi1BTfMBae
% XatLxBI9gBaerbd9wDYLwzYbItLDharqqtubsr4rNCHbGeaGqiVu0J
% e9sqqrpepC0xbbL8F4rqqrFfpeea0xe9Lq-Jc9vqaqpepm0xbba9pw
% e9Q8fs0-yqaqpepae9pg0FirpepeKkFr0xfr-xfr-xb9adbaqaaeGa
% ciGaaiaabeqaamaabaabaaGcbaGaeqiUdeNaaiikaiabeg7aHjaacM
% cacqGH9aqpcqaH7oaBcqaHXoqydaahaaWcbeqaaiabeo7aNnaaBaaa
% meaacaaIYaaabeaaliabgkHiTiaaigdaaaGccaGGUaaaaa!3F97!
\begin{equation}
\theta (\alpha_2 ) = \kappa \alpha_2 ^{\gamma _2  - 1} .
\end{equation}
\end{remark}

Now, we will prove that the solution $\alpha_2$ is unique in the interval
$]0,1-\beta_1[$. For this, we set% MathType!MTEF!2!1!+-
% faaafaart1ev1aaat0uyJj1BTfMBaerbuLwBLnhiov2DGi1BTfMBae
% XatLxBI9gBaerbd9wDYLwzYbItLDharqqtubsr4rNCHbGeaGqiVu0J
% e9sqqrpepC0xbbL8F4rqqrFfpeea0xe9Lq-Jc9vqaqpepm0xbba9pw
% e9Q8fs0-yqaqpepae9pg0FirpepeKkFr0xfr-xfr-xb9adbaqaaeGa
% ciGaaiaabeqaamaabaabaaGcbaGaamOzaiaacIcacqaHXoqycaGGPa
% Gaeyypa0JaeqOUdSMaeqySdeMaey4kaSYaaSaaaeaacqaHepaDdaWg
% aaWcbaGaamiCaaqabaaakeaacqaHepaDcaWGTbWaa0baaSqaaiaaik
% daaeaacqaHZoWzdaWgaaadbaGaaGOmaaqabaaaaaaakmaalaaabaGa
% eqySdeMaeyOeI0IaeqySde2aa0baaSqaaiaaikdaaeaacaaIWaaaaa
% GcbaGaaGymaiabgkHiTiabeg7aHbaaaaa!4CE9!
\begin{equation}\label{}
f(\alpha ) = \kappa \alpha  + \frac{{\tau _p }}
{{\tau m_2^{\gamma _2 } }}\frac{{\alpha  - \alpha _2^0 }}
{{1 - \alpha }}
\end{equation}
it is easy to check that % MathType!MTEF!2!1!+-
% feaafiart1ev1aaatCvAUfeBSjuyZL2yd9gzLbvyNv2CaerbuLwBLn
% hiov2DGi1BTfMBaeXatLxBI9gBaerbd9wDYLwzYbItLDharqqtubsr
% 4rNCHbGeaGqiVu0Je9sqqrpepC0xbbL8F4rqqrFfpeea0xe9Lq-Jc9
% vqaqpepm0xbba9pwe9Q8fs0-yqaqpepae9pg0FirpepeKkFr0xfr-x
% fr-xb9adbaqaaeGaciGaaiaabeqaamaabaabaaGcbaGaamOzaiaabc
% cacaqGPbGaae4CaiaabccacaqGJbGaae4Baiaab6gacaqG2bGaaeyz
% aiaabIhacaqGUaaaaa!4056!
\begin{equation}
f{\text{ is convex}}{\text{.}}
\end{equation}



We then have% MathType!MTEF!2!1!+-
% feaafiart1ev1aaatCvAUfeBSjuyZL2yd9gzLbvyNv2CaerbuLwBLn
% hiov2DGi1BTfMBaeXatLxBI9gBaerbd9wDYLwzYbItLDharqqtubsr
% 4rNCHbGeaGqiVu0Je9sqqrpepC0xbbL8F4rqqrFfpeea0xe9Lq-Jc9
% vqaqpepm0xbba9pwe9Q8fs0-yqaqpepae9pg0FirpepeKkFr0xfr-x
% fr-xb9adbaqaaeGaciGaaiaabeqaamaabaabaaGcbaGaam4raiaacE
% cacaGGNaGaaiikaiabeg7aHnaaBaaaleaacaaIYaaabeaakiaacMca
% cqGH9aqpcqGHsislcaaIYaGaeq4SdC2aaSbaaSqaaiaaigdaaeqaaO
% Gaeq4SdC2aaSbaaSqaaiaaikdaaeqaaOGaaiikaiabec8aWnaaBaaa
% leaacaaIYaaabeaakiabgkHiTiabec8aWnaaBaaaleaacaaIXaaabe
% aakiaacMcacqGHsislcaaIYaGaeq4SdC2aaSbaaSqaaiaaigdaaeqa
% aOGaamyBamaaDaaaleaacaaIYaaabaGaeq4SdC2aaSbaaWqaaiaaik
% daaeqaaaaakiaadAgacaGGNaGaaiikaiabeg7aHnaaBaaaleaacaaI
% YaaabeaakiaacMcacqGHRaWkcaGGOaGaeqySde2aaSbaaSqaaiaaig
% daaeqaaOGaey4kaSIaaiikaiabeo7aNnaaBaaaleaacaaIXaaabeaa
% kiabgkHiTiaaigdacaGGPaGaaiikaiabeg7aHnaaBaaaleaacaaIXa
% aabeaakiabgkHiTiabeg7aHnaaDaaaleaacaaIXaaabaGaaGimaaaa
% kiaacMcacaGGPaGaamyBamaaDaaaleaacaaIYaaabaGaeq4SdC2aaS
% baaWqaaiaaikdaaeqaaaaakiaadAgacaGGNaGaai4jaiaacIcacqaH
% XoqydaWgaaWcbaGaaGOmaaqabaGccaGGPaaaaa!77B9!
\begin{equation}\label{gpp}
G''(\alpha _2 ) =  - 2\gamma _1 \gamma _2 (\pi _2  - \pi _1 ) -
2\gamma _1 m_2^{\gamma _2 } f'(\alpha _2 ) + (\alpha _1  + (\gamma
_1  - 1)(\alpha _1  - \alpha _1^0 ))m_2^{\gamma _2 } f''(\alpha _2 ).
\end{equation}


The function $f$ is convex and satisfies $f(\alpha) > f(0)$ for
$0<\alpha <1$. It implies
that $f$ is also increasing. Then the two first terms in (\ref{gpp})
are $<0$ and the last is $>0$. On the other hand, it is sufficient
that $G$
is concave to prove the uniqueness. But% MathType!MTEF!2!1!+-
% faaafaart1ev1aaat0uyJj1BTfMBaerbuLwBLnhiov2DGi1BTfMBae
% XatLxBI9gBaerbd9wDYLwzYbItLDharqqtubsr4rNCHbGeaGqiVu0J
% e9sqqrpepC0xbbL8F4rqqrFfpeea0xe9Lq-Jc9vqaqpepm0xbba9pw
% e9Q8fs0-yqaqpepae9pg0FirpepeKkFr0xfr-xfr-xb9adbaqaaeGa
% ciGaaiaabeqaamaabaabaaGceaqabeaacaWGhbGaai4jaiaacEcaca
% GGOaGaeqySde2aaSbaaSqaaiaaikdaaeqaaOGaaiykaiabgsMiJkab
% gkHiTiaaikdacqaHZoWzdaWgaaWcbaGaaGymaaqabaGccqaHZoWzda
% WgaaWcbaGaaGOmaaqabaGccaGGOaGaeqiWda3aaSbaaSqaaiaaikda
% aeqaaOGaeyOeI0IaeqiWda3aaSbaaSqaaiaaigdaaeqaaOGaaiykai
% abgkHiTaqaaiaaikdacqaHZoWzdaWgaaWcbaGaaGymaaqabaGccaWG
% TbWaa0baaSqaaiaaikdaaeaacqaHZoWzdaWgaaadbaGaaGOmaaqaba
% aaaOGaamOzaiaacEcacaGGOaGaeqySde2aaSbaaSqaaiaaikdaaeqa
% aOGaaiykaiabgUcaRiabeo7aNnaaBaaaleaacaaIXaaabeaakiaad2
% gadaqhaaWcbaGaaGOmaaqaaiabeo7aNnaaBaaameaacaaIYaaabeaa
% aaGccaGGOaGaaGymaiabgkHiTiabeg7aHnaaBaaaleaacaaIYaaabe
% aakiaacMcacaWGMbGaai4jaiaacEcacaGGOaGaeqySde2aaSbaaSqa
% aiaaikdaaeqaaOGaaiykaiabgsMiJcqaaiaaikdacqaHZoWzdaWgaa
% WcbaGaaGymaaqabaGccaWGTbWaa0baaSqaaiaaikdaaeaacqaHZoWz
% daWgaaadbaGaaGOmaaqabaaaaOWaamWaaeaadaWcaaqaaiaaigdaae
% aacaaIYaaaaiaacIcacaaIXaGaeyOeI0IaeqySde2aaSbaaSqaaiaa
% ikdaaeqaaOGaaiykaiaadAgacaGGNaGaai4jaiaacIcacqaHXoqyda
% WgaaWcbaGaaGOmaaqabaGccaGGPaGaeyOeI0IaamOzaiaacEcacaGG
% OaGaeqySde2aaSbaaSqaaiaaikdaaeqaaOGaaiykaiabgkHiTmaala
% aabaGaeq4SdC2aaSbaaSqaaiaaikdaaeqaaaGcbaGaamyBamaaDaaa
% leaacaaIYaaabaGaeq4SdC2aaSbaaWqaaiaaikdaaeqaaaaaaaGcca
% GGOaGaeqiWda3aaSbaaSqaaiaaikdaaeqaaOGaeyOeI0IaeqiWda3a
% aSbaaSqaaiaaigdaaeqaaOGaaiykaaGaay5waiaaw2faaaaaaa!985B!
\begin{equation}
\begin{gathered}
  G''(\alpha _2 ) \leqslant  - 2\gamma _1 \gamma _2 (\pi _2  - \pi _1 ) -  \hfill \\
  2\gamma _1 m_2^{\gamma _2 } f'(\alpha _2 ) + \gamma _1 m_2^{\gamma _2 } (1 - \alpha _2 )f''(\alpha _2 ) \leqslant  \hfill \\
  2\gamma _1 m_2^{\gamma _2 } \left[ {\frac{1}
{2}(1 - \alpha _2 )f''(\alpha _2 ) - f'(\alpha _2 ) - \frac{{\gamma _2 }}
{{m_2^{\gamma _2 } }}(\pi _2  - \pi _1 )} \right] \hfill \\
\end{gathered}
\end{equation}




A sufficient condition to obtain uniqueness is then% MathType!MTEF!2!1!+-
% feaafiart1ev1aaatCvAUfeBSjuyZL2yd9gzLbvyNv2CaerbuLwBLn
% hiov2DGi1BTfMBaeXatLxBI9gBaerbd9wDYLwzYbItLDharqqtubsr
% 4rNCHbGeaGqiVu0Je9sqqrpepC0xbbL8F4rqqrFfpeea0xe9Lq-Jc9
% vqaqpepm0xbba9pwe9Q8fs0-yqaqpepae9pg0FirpepeKkFr0xfr-x
% fr-xb9adbaqaaeGaciGaaiaabeqaamaabaabaaGcbaWaaSaaaeaaca
% aIXaaabaGaaGOmaaaacaGGOaGaaGymaiabgkHiTiabeg7aHnaaBaaa
% leaacaaIYaaabeaakiaacMcacaWGMbGaai4jaiaacEcacaGGOaGaeq
% ySde2aaSbaaSqaaiaaikdaaeqaaOGaaiykaiabgkHiTiaadAgacaGG
% NaGaaiikaiabeg7aHnaaBaaaleaacaaIYaaabeaakiaacMcacqGHsi
% sldaWcaaqaaiabeo7aNnaaBaaaleaacaaIYaaabeaaaOqaaiaad2ga
% daqhaaWcbaGaaGOmaaqaaiabeo7aNnaaBaaameaacaaIYaaabeaaaa
% aaaOGaaiikaiabec8aWnaaBaaaleaacaaIYaaabeaakiabgkHiTiab
% ec8aWnaaBaaaleaacaaIXaaabeaakiaacMcacqGHKjYOcaaIWaaaaa!5BBB!
\begin{equation}\label{fondineq}
\frac{1} {2}(1 - \alpha _2 )f''(\alpha _2 ) - f'(\alpha _2 ) -
\frac{{\gamma _2 }} {{m_2^{\gamma _2 } }}(\pi _2  - \pi _1 )
\leqslant 0.
\end{equation}

The above inequality (\ref{fondineq}) can also be written% MathType!MTEF!2!1!+-
% feaafiart1ev1aaatCvAUfeBSjuyZL2yd9gzLbvyNv2CaerbuLwBLn
% hiov2DGi1BTfMBaeXatLxBI9gBaerbd9wDYLwzYbItLDharqqtubsr
% 4rNCHbGeaGqiVu0Je9sqqrpepC0xbbL8F4rqqrFfpeea0xe9Lq-Jc9
% vqaqpepm0xbba9pwe9Q8fs0-yqaqpepae9pg0FirpepeKkFr0xfr-x
% fr-xb9adbaqaaeGaciGaaiaabeqaamaabaabaaGcbaGaeyOeI0Iaeq
% 4UdWMaeyOeI0YaaSaaaeaacqaHZoWzdaWgaaWcbaGaaGOmaaqabaaa
% keaacaWGTbWaaSbaaSqaaiaaikdaaeqaaOWaaWbaaSqabeaacqaHZo
% WzdaWgaaadbaGaaGOmaaqabaaaaaaakiaacIcacqaHapaCdaWgaaWc
% baGaaGOmaaqabaGccqGHsislcqaHapaCdaWgaaWcbaGaaGymaaqaba
% GccaGGPaGaeyizImQaaGimaaaa!4AE1!
\begin{equation}
 - \kappa  - \frac{{\gamma _2 }}
{{m_2 ^{\gamma _2 } }}(\pi _2  - \pi _1 ) \leqslant 0
\end{equation}
and it is obviously satisfied, independantly of $\tau_p$.

%\begin{remark}
%It is not necessary to suppose that the granular constraint is
%monotone. It can be used to model the damage of the solid grains
%under the granular stress.
%\end{remark}
We sum up the previous computations in the following proposition,
which is useful for the implementation of the algorithm.
\begin{proposition}
Let the granular stress be defined by (\ref{nice}). Let
\begin{equation}
\begin{gathered}
  0 < \alpha _1^0  < 1, \hfill \\
  p_k^0  + \pi _k  > 0,\quad k = 1,2, \hfill \\
\end{gathered}
\end{equation}
then, the algebraic system (\ref{relaxstep}), (\ref{impl}) admits a unique
solution $(\alpha_1,p_1,p_2)$ that complies with
\begin{equation}
\begin{gathered}
  0 < \alpha _1  < 1, \hfill \\
  p_k  + \pi _k  > 0,\quad k = 1,2.\hfill \\
\end{gathered}
\end{equation}
Moreover  we also have
\begin{equation}
\alpha _1 > \frac{{\gamma _1  - 1}} {{\gamma _1 }}\alpha
_1^0 .
\end{equation}
Finally, the solution can be computed by the Newton's method by
solving equation (\ref{defG}) for $\alpha_2$. A safe
choice for the initialisation of the Newton's method is $\alpha_2=0$

\end{proposition}
{\bf Proof}: the proof is a consequence of the previous computations. The proposed initialization of the Newton's method comes from the concavity of $G$. $\square$
\begin{remark}
This uniqueness result can be easily generalized to a granular stress defined by (\ref{granugene}). A sufficient uniqueness condition is still (\ref{fondineq}) and $f$ convex but the function $f$ is now defined by% MathType!MTEF!2!1!+-
% faaafaart1ev1aaat0uyJj1BTfMBaerbuLwBLnhiov2DGi1BTfMBae
% XatLxBI9gBaerbd9wDYLwzYbItLDharqqtubsr4rNCHbGeaGqiVu0J
% e9sqqrpepC0xbbL8F4rqqrFfpeea0xe9Lq-Jc9vqaqpepm0xbba9pw
% e9Q8fs0-yqaqpepae9pg0FirpepeKkFr0xfr-xfr-xb9adbaqaaeGa
% ciGaaiaabeqaamaabaabaaGcbaGaamOzaiaacIcacqaHXoqycaGGPa
% Gaeyypa0JaeqySde2aaWbaaSqabeaacaaIYaGaeyOeI0Iaeq4SdC2a
% aSbaaWqaaiaaikdaaeqaaaaakiabeI7aXjaacIcacqaHXoqycaGGPa
% Gaey4kaSYaaSaaaeaacqaHepaDdaWgaaWcbaGaamiCaaqabaaakeaa
% cqaHepaDcaWGTbWaa0baaSqaaiaaikdaaeaacqaHZoWzdaWgaaadba
% GaaGOmaaqabaaaaaaakmaalaaabaGaeqySdeMaeyOeI0IaeqySde2a
% a0baaSqaaiaaikdaaeaacaaIWaaaaaGcbaGaaGymaiabgkHiTiabeg
% 7aHbaaaaa!5455!
\begin{equation}\label{}
f(\alpha ) = \alpha ^{2 - \gamma _2 } \theta (\alpha ) + \frac{{\tau _p }}
{{\tau m_2^{\gamma _2 } }}\frac{{\alpha  - \alpha _2^0 }}
{{1 - \alpha }}
\end{equation}
\end{remark}


\section{Numerical results}
\subsection{Academical test cases \label{acad}}
First, we consider two one-dimensional test cases in order to evaluate the influence
of the granular stress on the system stability. We take $\tau_p=0$, which corresponds to instantaneous
pressure equilibrium. It is known that generally, the equilibrium system is not hyperbolic
(the computations are recalled in Section \ref{app}). The numerical parameters are taken from
\cite{elamine97} (and also studied in \cite{hurisse}). We consider a
simple Riemann problem in the interval $[-1/2,1/2]$. The two phases
are supposed to satisfy perfect gas equations of state, with $\gamma_1=1.0924$ and
$\gamma_2=1.0182$. The initial condition is made of two constant
states jumping at $x=0$. We plot the solution at time $t=0.0008$.
The CFL number is fixed to $0.9$. The
initial data are% MathType!MTEF!2!1!+-
% feaafiart1ev1aaatCvAUfeBSjuyZL2yd9gzLbvyNv2CaerbuLwBLn
% hiov2DGi1BTfMBaeXatLxBI9gBaerbd9wDYLwzYbItLDharqqtubsr
% 4rNCHbGeaGqiVu0Je9sqqrpepC0xbbL8F4rqqrFfpeea0xe9Lq-Jc9
% vqaqpepm0xbba9pwe9Q8fs0-yqaqpepae9pg0FirpepeKkFr0xfr-x
% fr-xb9adbaqaaeGaciGaaiaabeqaamaabaabaaGcbaqbaeqabGWaaa
% aaaeaaaeaacaGGOaGaamitaiaacMcaaeaacaGGOaGaamOuaiaacMca
% aeaacqaHbpGCdaWgaaWcbaGaaGymaaqabaaakeaacaqG3aGaaeOnai
% aab6cacaqG0aGaaeynaiaabsdacaqGZaGaaeimaiaabcdacaqG5aGa
% ae4maaqaaiaabwdacaqG3aGaaeOlaiaabodacaqG0aGaaeimaiaabE
% dacaqGYaGaaeynaiaabAdacaqG4aaabaGaamyDamaaBaaaleaacaaI
% XaaabeaaaOqaaiaaicdaaeaacaaIWaaabaGaamiCamaaBaaaleaaca
% aIXaaabeaaaOqaaiaaikdacaaIWaGaaGimaiabgEna0kaabgdacaqG
% WaWaaWbaaSqabeaacaqG1aaaaaGcbaGaaeymaiaabwdacaqGWaGaey
% 41aqRaaeymaiaabcdadaahaaWcbeqaaiaabwdaaaaakeaacqaHbpGC
% daWgaaWcbaGaaGOmaaqabaaakeaacaqG4aGaae4maiaabAdacaqGUa
% GaaeymaiaabkdacaqGZaGaaeyoaiaabEdacaqGXaGaaeioaaqaaiaa
% bodacaqG1aGaaeioaiaab6cacaqG4aGaaeyoaiaabIdacaqGYaGaae
% OmaiaabkdacaqG2aaabaGaamyDamaaBaaaleaacaaIYaaabeaaaOqa
% aiaaicdaaeaacaaIWaaabaGaamiCamaaBaaaleaacaaIYaaabeaaaO
% qaaiaaikdacaaIWaGaaGimaiabgEna0kaabgdacaqGWaWaaWbaaSqa
% beaacaqG1aaaaaGcbaGaaeymaiaabwdacaqGWaGaey41aqRaaeymai
% aabcdadaahaaWcbeqaaiaabwdaaaaakeaacqaHXoqydaWgaaWcbaGa
% aGymaaqabaaakeaacaaIWaGaaiOlaiaaikdacaaI1aaabaGaaGimai
% aac6cacaaIYaGaaGynaaaaaaa!8C8A!
\begin{equation}
\begin{array}{*{20}c}
   {} & {(L)} & {(R)}  \\
   {\rho _1 } & {{\text{76}}{\text{.45430093}}} & {{\text{57}}{\text{.34072568}}}  \\
   {u_1 } & 0 & 0  \\
   {p_1 } & {200 \times {\text{10}}^{\text{5}} } & {{\text{150}} \times {\text{10}}^{\text{5}} }  \\
   {\rho _2 } & {{\text{836}}{\text{.1239718}}} & {{\text{358}}{\text{.8982226}}}  \\
   {u_2 } & 0 & 0  \\
   {p_2 } & {200 \times {\text{10}}^{\text{5}} } & {{\text{150}} \times {\text{10}}^{\text{5}} }  \\
   {\alpha _1 } & {0.25} & {0.25}  \\

 \end{array}
\end{equation}
We perform our algorithm with a granular stress $R=0$. With a
mesh of $1,000$ cells we observe that the solution is rather smooth
but what seems to be a small oscillation starts to develop in the center of the computational domain. The
volume fraction $\alpha_1$, the velocities and pressures are plotted
on
%Figures \ref{a2-eq-50}, \ref{u1u2-eq-50} and \ref{p1p2-eq-50}.
Figures \ref{a2-eq-1000}, \ref{u1u2-eq-1000} and \ref{p1p2-eq-1000}.

%\begin{figure}[!h]
%\resizebox*{10cm}{!}{\rotatebox{0}{\includegraphics{a2-eq-50.png}}}

%\caption{Void fraction, 50 cells, no granular stress
%.\label{a2-eq-50}}
%\end{figure}
%\begin{figure}[!h]
%\resizebox*{10cm}{!}{\rotatebox{0}{\includegraphics{u1u2-eq-50.png}}}

%\caption{Velocities $u_1$ and $u_2$, 50 cells, no granular stress
%.\label{u1u2-eq-50}}
%\end{figure}
%\begin{figure}[!h]
%\resizebox*{10cm}{!}{\rotatebox{0}{\includegraphics{p1p2-eq-50.png}}}

%\caption{Pressures, 50 cells, no granular stress
%.\label{p1p2-eq-50}}
%\end{figure}

% We
%perform the same computation for 1000 cells on Figures
%\ref{a2-eq-1000}, \ref{u1u2-eq-1000} and \ref{p1p2-eq-1000}.

\begin{figure}[b]
\resizebox*{10cm}{!}{\rotatebox{0}{\includegraphics{a2-eq-1000.png}}}
\caption{Void fraction, $1,000$ cells, $\tau_p=0$, $R=0$.
\label{a2-eq-1000}}
\end{figure}


\begin{figure}[!h]
\resizebox*{10cm}{!}{\rotatebox{0}{\includegraphics{u1u2-eq-1000.png}}}
\caption{Velocities $u_1$ and $u_2$, 1,000 cells, $\tau_p=0$, $R=0$.
\label{u1u2-eq-1000}}
\end{figure}


\begin{figure}[!h]
\resizebox*{10cm}{!}{\rotatebox{0}{\includegraphics{p1p2-eq-1000.png}}}
\caption{Pressures, 1,000 cells, $\tau_p=0$, $R=0$.
\label{p1p2-eq-1000}}
\end{figure}


The same computation is made with $10,000$ cells. We observe on
Figures \ref{a2-eq-10000}, \ref{u1u2-eq-10000} and
\ref{p1p2-eq-10000} that instabilities arise, probably due to the
non-hyperbolic behavior of the model.

\begin{figure}[!h]
\resizebox*{10cm}{!}{\rotatebox{0}{\includegraphics{a2-eq-10000.png}}}
\caption{Void fraction, 10,000 cells, $\tau_p=0$, $R=0$.
\label{a2-eq-10000}}
\end{figure}

\begin{figure}[!h]
\resizebox*{10cm}{!}{\rotatebox{0}{\includegraphics{u1u2-eq-10000.png}}}
\caption{Velocities $u_1$ and $u_2$, 10,000 cells, $\tau_p=0$, $R=0$.
\label{u1u2-eq-10000}}
\end{figure}


\begin{figure}[!h]
\resizebox*{10cm}{!}{\rotatebox{0}{\includegraphics{p1p2-eq-10000.png}}}
\caption{Pressures, 10,000 cells, $\tau_p=0$, $R=0$.
\label{p1p2-eq-10000}}
\end{figure}

 We have also performed a
computation on a $100,000$ cells mesh. The oscillations clearly
increase as can be seen on Figure \ref{a2-eq-100000} for the volume
fraction. Recall that thanks to the CFL condition, the Rusanov scheme computes positive cell values for void fractions and the partial masses.


\begin{figure}[!h]
\resizebox*{10cm}{!}{\rotatebox{0}{\includegraphics{a2-eq-100000.png}}}
\caption{Void fraction, 100,000 cells, $\tau_p=0$, $R=0$ (the two colors only correspond to the two processors used in the MPI computation).
\label{a2-eq-100000}}
\end{figure}



We perform then another computation on the finer mesh with a
granular stress given by (\ref{nice}). For the numerics, we have
chosen $\kappa=500$. We observe a very slight damping of the
oscillations on Figures \ref{a2-ne-10000} and \ref{u1u2-ne-10000} (to be carefully compared with Figures \ref{a2-eq-10000} and \ref{u1u2-eq-10000}).
We have also plotted the two pressures on Figure  \ref{p1p2-ne-10000} in order to show the difference of pressures imposed by the granular stress.
%, \ref{u1u2-ne-10000} and \ref{p1p2-ne-10000}.


It clearly arises that the magnitude of the granular stress is not
sufficient to recover a hyperbolic regime. This is confirmed on Figure \ref{ima-1000} where we compare a $L^2$ norm of
the imaginary parts of the
eigenvalues with $R=0$ or $R>0$. The computation of
the convection matrix of the equilibrium system in the case $\tau_p \to 0$ is given in
Section \ref{app}. The eigenvalues are evaluated numerically. The  quantity that has been plotted is
% MathType!MTEF!2!1!+-
% feaafiart1ev1aaatCvAUfeBSjuyZL2yd9gzLbvyNv2CaerbuLwBLn
% hiov2DGi1BTfMBaeXatLxBI9gBaerbd9wDYLwzYbItLDharqqtubsr
% 4rNCHbGeaGqiVu0Je9sqqrpepC0xbbL8F4rqqrFfpeea0xe9Lq-Jc9
% vqaqpepm0xbba9pwe9Q8fs0-yqaqpepae9pg0FirpepeKkFr0xfr-x
% fr-xb9adbaqaaeGaciGaaiaabeqaamaabaabaaGcbaGaamysaiabg2
% da9maakaaabaWaaabCaeaaciGGjbGaaiyBamaabmaabaGaeq4UdW2a
% aSbaaSqaaiaadMgaaeqaaaGccaGLOaGaayzkaaaaleaacaWGPbGaey
% ypa0JaaGymaaqaaiaaiAdaa0GaeyyeIuoakmaaCaaaleqabaGaaGOm
% aaaaaeqaaOGaaiOlaaaa!4552!
\begin{equation}
I = \sqrt {\sum\limits_{i = 1}^6 {\operatorname{Im} \left( {\lambda
_i } \right)} ^2 } .
\end{equation}
We observe on Figure \ref{ima-1000} that the imaginary part slightly decreases, owing
 to the introduction
of the granular stress; however, it does not vanish.


\begin{figure}[!h]
\resizebox*{10cm}{!}{\rotatebox{0}{\includegraphics{a2-ne-10000.png}}}
\caption{Void fraction, 10,000 cells, $\tau_p=0$, $R=500 m_2^{\gamma_2}$.
\label{a2-ne-10000}}
\end{figure}

\begin{figure}[!h]
\resizebox*{10cm}{!}{\rotatebox{0}{\includegraphics{u1u2-ne-10000.png}}}

\caption{Velocities $u_1$ and $u_2$, 10,000 cells,  $\tau_p=0$,  $R=500 m_2^{\gamma_2}$ .\label{u1u2-ne-10000}}
\end{figure}

\begin{figure}[!h]
\resizebox*{10cm}{!}{\rotatebox{0}{\includegraphics{p1p2-ne-10000.png}}}
\caption{Pressures, 10,000 cells, $\tau_p=0$, $R=500 m_2^{\gamma_2}$.
\label{p1p2-ne-10000}}
\end{figure}


\begin{figure}[!h]
\resizebox*{10cm}{!}{\rotatebox{0}{\includegraphics{ima.png}}}
\caption{Imaginary parts, 1,000 cells, $\tau_p=0$, $R=500 m_2^{\gamma_2}$ or $R=0$.
\label{ima-1000}}
\end{figure}

We now evaluate the influence of the positive time scale parameter $\tau_p$,
and on its ability to stabilize
the model. While setting $p_{ref}=10^{7}$,  we compare
in Figure \ref{influ-relax-1000} the pressures
obtained with different relaxation coefficients (corresponding with
true time scales $\frac{\tau_p}{p_{ref}}=0, 10^{-7}, 10^{-6}, 10^{-5}$ and $
10^{-4}$). On this rather coarse mesh of 1,000 cells, we observe that the pressures
are very similar for small enough relaxation time scales. When focusing on a finer mesh of 50,000 cells, the
stability of the approximations increases when the relaxation time scales are larger
(Figure \ref{p150kc}).
%(Figure \ref{influ-relax-10000}).
We also observe on this test case that even with high values of
$\tau_p$, the difference between both pressures $p_1$ and $p_2$ is indeed rather small.
%The same kind of computations are performed on Figure \ref{p150kc}

\begin{figure}[!h]
\resizebox*{10cm}{!}{\rotatebox{0}{\includegraphics{prerelax.png}}}
\caption{Pressure $p_1$, 1,000 cells, with $0\leq \tau_p \leq 1000$.\label{influ-relax-1000}}
\end{figure}

%\begin{figure}[!h]
%\resizebox*{10cm}{!}{\rotatebox{-90}{\includegraphics{c10000-pres-taup.png}}}
%\caption{Pressure $p_1$, 10,000 cells, with $\tau_p =0$ or $\tau_p=1$.\label{influ-relax-10000}}
%\end{figure}


\begin{figure}[b]
\resizebox*{10cm}{!}{\rotatebox{0}{\includegraphics{p150kc.png}}}
\caption{Pressure $p_1$, $50,000$ cells, $\tau_p=0,2,10$, $R=0$.
\label{p150kc}}
\end{figure}

Finally, we want to compare the stabilization effect of the parameter $\tau_P$. For this purpose, we first set $\tau_P=0$, which corresponds to an instantaneous relaxation. We show on Figure \ref{pres-taup-0} the evolution of the pressure plots when the mesh is refined from $10,000$ cells ("10k cells") to $200,000$ cells ("200k cells"). We observe a non-smooth behavior of the pressure graphs. It is in agreement with the fact that we approximate a non-hyperbolic, two-velocity, one-pressure model. It has also been observed in \cite{guillemaud}, \cite{hurisse}. If the relaxation parameter is set to $\tau_P=1$, we observe on Figure \ref{pres-taup-1} a stabilization of the pressure curves, which is confirmed by a grid refinement. With this relaxation parameter and  with $p_{ref}=10^{7}$, the corresponding time scale given by (\ref{tref}) is $t_{ref}=10^{-7}$. In comparison, the time step $\Delta t$ for a mesh of $100,000$ cells is of the order of $10^{-9}$. It means that on the smaller meshes, the relaxation time scale is actually captured. It seems that the pressure is converging towards a smooth limit. It would be interesting to perform computations on even finer meshes but it would last a very long time (our longer computation lasted more than 60 hours on eight CPU of a 2.0 GHz parallel computer).

\begin{figure}[b]
\resizebox*{10cm}{!}{\rotatebox{0}{\includegraphics{pres-taup-0.png}}}
\caption{Pressure,  $10,000$ to $200,000$ cells, $\tau_p=0$, $R=0$.
\label{pres-taup-0}}
\end{figure}

\begin{figure}[b]
\resizebox*{10cm}{!}{\rotatebox{0}{\includegraphics{pres-taup-1.png}}}
\caption{Pressure $p_1$, $10,000$ to $100,000$ cells, $\tau_p=1$, $R=0$.
\label{pres-taup-1}}
\end{figure}


\subsection{Simplified combustion chamber \label{sgun}}
We consider now a more realistic case taken from
\cite{nussbaum07}. We are interested in the modeling of a simplified gun.
The gun is modeled by a one-dimensional tube filled with
a solid phase (the powder grains) and a gas phase (the combustion gases). The breech is on the left and
the shot base at the right boundary. The shot base moves according to Newton's
law because the bullet is accelerated by the increase of pressure due to the combustion
of the powder.
For this test case, we have adapted and simplified the physical parameters
given in \cite{nuss-hell06}. The mass transfer term is
defined by the simplified relations% MathType!MTEF!2!1!+-
% feaafiart1ev1aaatCvAUfeBSjuyZL2yd9gzLbvyNv2CaerbuLwBLn
% hiov2DGi1BTfMBaeXatLxBI9gBaerbd9wDYLwzYbItLDharqqtubsr
% 4rNCHbGeaGqiVu0Je9sqqrpepC0xbbL8F4rqqrFfpeea0xe9Lq-Jc9
% vqaqpepm0xbba9pwe9Q8fs0-yqaqpepae9pg0FirpepeKkFr0xfr-x
% fr-xb9adbaqaaeGaciGaaiaabeqaamaabaabaaGceaqabeaacaWGnb
% Gaeyypa0JaeqySde2aaSbaaSqaaiaaikdaaeqaaOGaeqyWdi3aaSba
% aSqaaiaaikdaaeqaaOWaaSaaaeaacaaIZaGabmOCayaacaaabaGaam
% OCaaaaaeaaceWGYbGbaiaacqGH9aqpcaaI1aGaey41aqRaaGymaiaa
% icdadaahaaWcbeqaaiabgkHiTiaaiodaaaGccaWGTbGaai4laiaado
% hacaqGGaGaaeikaiaabogacaqGVbGaaeyBaiaabkgacaqG1bGaae4C
% aiaabshacaqGPbGaae4Baiaab6gacaqGGaGaaeODaiaabwgacaqGSb
% Gaae4BaiaabogacaqGPbGaaeiDaiaabMhacaqGGaGaae4BaiaabAga
% caqGGaGaaeiDaiaabIgacaqGLbGaaeiiaiaabEgacaqGYbGaaeyyai
% aabMgacaqGUbGaae4CaiaabMcaaeaacaWGYbGaeyypa0JaaGymaiaa
% icdadaahaaWcbeqaaiabgkHiTiaaiodaaaGccaWGTbGaaeiiaiaabI
% cacaqGYbGaaeyyaiaabsgacaqGPbGaaeyDaiaabohacaqGGaGaae4B
% aiaabAgacaqGGaGaaeiDaiaabIgacaqGLbGaaeiiaiaabEgacaqGYb
% GaaeyyaiaabMgacaqGUbGaae4CaiaabMcaaaaa!8449!
\begin{equation}
\begin{gathered}
  M = \alpha _2 \rho _2 \frac{{3\dot r}}
{r} \hfill \\
  \dot r = 5 \times 10^{ - 3} m/s{\text{ (combustion velocity of the grains)}} \hfill \\
  r = 10^{ - 3} m{\text{ (radius of the grains)}} \hfill \\
\end{gathered}
\end{equation}

The momentum source term is given by% MathType!MTEF!2!1!+-
% feaafiart1ev1aaatCvAUfeBSjuyZL2yd9gzLbvyNv2CaerbuLwBLn
% hiov2DGi1BTfMBaeXatLxBI9gBaerbd9wDYLwzYbItLDharqqtubsr
% 4rNCHbGeaGqiVu0Je9sqqrpepC0xbbL8F4rqqrFfpeea0xe9Lq-Jc9
% vqaqpepm0xbba9pwe9Q8fs0-yqaqpepae9pg0FirpepeKkFr0xfr-x
% fr-xb9adbaqaaeGaciGaaiaabeqaamaabaabaaGceaqabeaacaWGrb
% Gaeyypa0JaamytaiaadwhadaWgaaWcbaGaaGOmaaqabaGccqGHsisl
% caWGebaabaGaamiraiabg2da9iaadoeacqaHXoqydaWgaaWcbaGaaG
% ymaaqabaGccqaHXoqydaWgaaWcbaGaaGOmaaqabaGccqaHbpGCdaWg
% aaWcbaGaaGOmaaqabaGccaGGOaGaamyDamaaBaaaleaacaaIXaaabe
% aakiabgkHiTiaadwhadaWgaaWcbaGaaGOmaaqabaGccaGGPaWaaqWa
% aeaacaWG1bWaaSbaaSqaaiaaigdaaeqaaOGaeyOeI0IaamyDamaaBa
% aaleaacaaIYaaabeaaaOGaay5bSlaawIa7aiaabccacaqGOaGaaeiz
% aiaabkhacaqGHbGaae4zaiaabccacaqGMbGaae4BaiaabkhacaqGJb
% GaaeyzaiaabMcaaeaacaWGdbGaeyypa0ZaaSaaaeaacaaIZaaabaGa
% aGinaiaadkhaaaGaaeiiaiaabIcacaqGZbGaaeyAaiaab2gacaqGWb
% GaaeiBaiaabMgacaqGMbGaaeyAaiaabwgacaqGKbGaaeiiaiaaboha
% caqGObGaaeyyaiaabchacaqGLbGaaeiiaiaabAgacaqGHbGaae4yai
% aabshacaqGVbGaaeOCaiaabMcaaaaa!7A9C!
\begin{equation}
\begin{gathered}
  Q = Mu_2  - D \hfill \\
  D = C\alpha _1 \alpha _2 \rho _2 (u_1  - u_2 )\left| {u_1  - u_2 } \right|{\text{ (drag force)}} \hfill \\
  C = \frac{3}
{{4r}}{\text{ (simplified shape factor)}} \hfill \\
\end{gathered}
\end{equation}

The energy source terms are % MathType!MTEF!2!1!+-
% feaafiart1ev1aaatCvAUfeBSjuyZL2yd9gzLbvyNv2CaerbuLwBLn
% hiov2DGi1BTfMBaeXatLxBI9gBaerbd9wDYLwzYbItLDharqqtubsr
% 4rNCHbGeaGqiVu0Je9sqqrpepC0xbbL8F4rqqrFfpeea0xe9Lq-Jc9
% vqaqpepm0xbba9pwe9Q8fs0-yqaqpepae9pg0FirpepeKkFr0xfr-x
% fr-xb9adbaqaaeGaciGaaiaabeqaamaabaabaaGceaqabeaacaWGtb
% WaaSbaaSqaaiaaigdaaeqaaOGaeyypa0JaeyOeI0IaamyDamaaBaaa
% leaacaaIYaaabeaakiaadseacqGHRaWkcaWGnbGaamyuamaaBaaale
% aacaWGLbGaamiEaaqabaaakeaacaWGtbWaaSbaaSqaaiaaikdaaeqa
% aOGaeyypa0JaamyDamaaBaaaleaacaaIYaaabeaakiaadseaaeaaca
% WGrbWaaSbaaSqaaiaadwgacaWG4baabeaakiabg2da9iaaiodacaaI
% 3aGaaiOlaiaaiodacaaI4aGaaG4maiaaiMacqGHxdaTcaaIXaGaaG
% imamaaCaaaleqabaGaaGOnaaaakiaabccacaqGkbGaae4laiaabUga
% caqGNbGaaeiiaiaabIcacaqGJbGaaeiAaiaabwgacaqGTbGaaeyAai
% aabogacaqGHbGaaeiBaiaabccacaqGJbGaae4Baiaab2gacaqGIbGa
% aeyDaiaabohacaqG0bGaaeyAaiaab+gacaqGUbGaaeiiaiaabwgaca
% qGUbGaaeyzaiaabkhacaqGNbGaaeyEaiaabMcaaaaa!71C2!
\begin{equation}
\begin{gathered}
  S_1  =  - u_2 D + MQ_{ex}  \hfill \\
  S_2  = u_2 D \hfill \\
  Q_{ex}  = 37.3839 \times 10^6 {\text{ J/kg (chemical combustion energy)}} \hfill \\
\end{gathered}
\end{equation}
The source term vector is now% MathType!MTEF!2!1!+-
% faaafaart1ev1aaat0uyJj1BTfMBaerbuLwBLnhiov2DGi1BTfMBae
% XatLxBI9gBaerbd9wDYLwzYbItLDharqqtubsr4rNCHbGeaGqiVu0J
% e9sqqrpepC0xbbL8F4rqqrFfpeea0xe9Lq-Jc9vqaqpepm0xbba9pw
% e9Q8fs0-yqaqpepae9pg0FirpepeKkFr0xfr-xfr-xb9adbaqaaeGa
% ciGaaiaabeqaamaabaabaaGcbaGaam4uaiaacIcacaWGxbGaaiykai
% abg2da9maabeaabaGaamytaiaacYcacaWGrbGaaiilaiaadofadaWg
% aaWcbaGaaGymaaqabaGccqGHsislcaWGWbWaaSbaaSqaaiaaigdaae
% qaaOGaamiuaiaacYcaaiaawIcaamaabiaabaGaeyOeI0Iaamytaiaa
% cYcacqGHsislcaWGrbGaaiilaiabgkHiTiaadofadaWgaaWcbaGaaG
% OmaaqabaGccqGHRaWkcaWGWbWaaSbaaSqaaiaaigdaaeqaaOGaamiu
% aiaacYcacaWGqbaacaGLPaaadaahaaWcbeqaaiaadsfaaaGccaGGUa
% aaaa!4EAB!
\begin{equation}\label{}
S(W) = \left( {M,Q,S_1  - p_1 P,} \right.\left. { - M, - Q, - S_2  + p_1 P,P} \right)^T .
\end{equation}
Let us remark that the energy sources do not cancel when summed up.
This is only apparently a violation of the total energy conservation.
Actually, we can rewrite the model in
order to have opposite source terms. The rewriting is based on a translation of the
internal energy in the pressure laws.

\begin{remark}\label{transener}
If we set% MathType!MTEF!2!1!+-
% feaafiart1ev1aaatCvAUfeBSjuyZL2yd9gzLbvyNv2CaerbuLwBLn
% hiov2DGi1BTfMBaeXatLxBI9gBaerbd9wDYLwzYbItLDharqqtubsr
% 4rNCHbGeaGqiVu0Je9sqqrpepC0xbbL8F4rqqrFfpeea0xe9Lq-Jc9
% vqaqpepm0xbba9pwe9Q8fs0-yqaqpepae9pg0FirpepeKkFr0xfr-x
% fr-xb9adbaqaaeGaciGaaiaabeqaamaabaabaaGcbaGaamyzamaaBa
% aaleaacaWGRbaabeaakiabg2da9iaadwgacaGGNaWaaSbaaSqaaiaa
% dUgaaeqaaOGaeyOeI0IaamyzamaaDaaaleaacaWGRbaabaGaaGimaa
% aakiaac6caaaa!4028!
\begin{equation}
e_k  = e'_k  - e_k^0 ,
\end{equation}
where $e'_k$ is the translated internal energy of phase $k$, and
$e_k^0$ is a reference energy for phase $k$, and if we define the
translated total energies% MathType!MTEF!2!1!+-
% feaafiart1ev1aaatCvAUfeBSjuyZL2yd9gzLbvyNv2CaerbuLwBLn
% hiov2DGi1BTfMBaeXatLxBI9gBaerbd9wDYLwzYbItLDharqqtubsr
% 4rNCHbGeaGqiVu0Je9sqqrpepC0xbbL8F4rqqrFfpeea0xe9Lq-Jc9
% vqaqpepm0xbba9pwe9Q8fs0-yqaqpepae9pg0FirpepeKkFr0xfr-x
% fr-xb9adbaqaaeGaciGaaiaabeqaamaabaabaaGcbaGaamyraiaacE
% cadaWgaaWcbaGaam4AaaqabaGccqGH9aqpcaWGLbGaai4jamaaBaaa
% leaacaWGRbaabeaakiabgUcaRiaadwgadaqhaaWcbaGaam4Aaaqaai
% aaicdaaaGccqGHRaWkdaWcaaqaaiaadwhadaqhaaWcbaGaam4Aaaqa
% aiaaikdaaaaakeaacaaIYaaaaiaac6caaaa!4533!
\begin{equation}
E'_k  = e'_k  + e_k^0  + \frac{{u_k^2 }}
{2}.
\end{equation}

the energy balance equations can be rewritten % MathType!MTEF!2!1!+-
% feaafiart1ev1aaatCvAUfeBSjuyZL2yd9gzLbvyNv2CaerbuLwBLn
% hiov2DGi1BTfMBaeXatLxBI9gBaerbd9wDYLwzYbItLDharqqtubsr
% 4rNCHbGeaGqiVu0Je9sqqrpepC0xbbL8F4rqqrFfpeea0xe9Lq-Jc9
% vqaqpepm0xbba9pwe9Q8fs0-yqaqpepae9pg0FirpepeKkFr0xfr-x
% fr-xb9adbaqaaeGaciGaaiaabeqaamaabaabaaGcbaGaaiikaiaad2
% gadaWgaaWcbaGaam4AaaqabaGccaWGfbGaai4jamaaBaaaleaacaWG
% RbaabeaakiaacMcadaWgaaWcbaGaamiDaaqabaGccqGHRaWkcaGGOa
% Gaaiikaiaad2gadaWgaaWcbaGaam4AaaqabaGccaWGfbGaai4jamaa
% BaaaleaacaWGRbaabeaakiabgUcaRiabeg7aHnaaBaaaleaacaWGRb
% aabeaakiaadchadaWgaaWcbaGaam4AaaqabaGccaGGPaGaamyDamaa
% BaaaleaacaWGRbaabeaakiaacMcadaWgaaWcbaGaamiEaaqabaGccq
% GHRaWkcaWGWbWaaSbaaSqaaiaadMeaaeqaaOGaeqySde2aaSbaaSqa
% aiaadUgacaGGSaGaamiDaaqabaGccqGH9aqpcqGHXcqScaWGtbGaeS
% 4eI0MaamytaiaadwgadaqhaaWcbaGaam4Aaaqaaiaaicdaaaaaaa!5E5F!
\begin{equation}
(m_k E'_k )_t  + ((m_k E'_k  + \alpha _k p_k )u_k )_x  + p_I \alpha
_{k,t}  =  \sigma_k S -\sigma_k Me_k^0.
\end{equation}
Now the total translated energy $m_1 E'_1 + m_2 E'_2$ is no longer
conserved since the $\sigma_k$ terms do not cancel. The term $M(e_2^0 -
e_1^0 )$ can be identified to the chemical reaction heat.
\end{remark}



The other parameters of the computations are% MathType!MTEF!2!1!+-
% feaafiart1ev1aaatCvAUfeBSjuyZL2yd9gzLbvyNv2CaerbuLwBLn
% hiov2DGi1BTfMBaeXatLxBI9gBaerbd9wDYLwzYbItLDharqqtubsr
% 4rNCHbGeaGqiVu0Je9sqqrpepC0xbbL8F4rqqrFfpeea0xe9Lq-Jc9
% vqaqpepm0xbba9pwe9Q8fs0-yqaqpepae9pg0FirpepeKkFr0xfr-x
% fr-xb9adbaqaaeGaciGaaiaabeqaamaabaabaaGceaqabeaacqaHZo
% WzdaWgaaWcbaGaaGymaaqabaGccqGH9aqpcaaIXaGaaiOlaiaaisda
% aeaacqaHZoWzdaWgaaWcbaGaaGOmaaqabaGccqGH9aqpcaaIZaaaba
% GaeqiWda3aaSbaaSqaaiaaikdaaeqaaOGaeyypa0JaaGOmaiaac6ca
% caaIXaGaaG4maiaaiodacaaIZaGaey41aqRaaGymaiaaicdadaahaa
% WcbeqaaiaaiMdaaaGccaWGqbGaamyyaaqaaiabeg8aYnaaBaaaleaa
% caaIYaaabeaakiabg2da9iaaigdacaaI2aGaaGimaiaaicdacaWGRb
% Gaam4zaiaac+cacaWGTbWaaWbaaSqabeaacaaIZaaaaOGaaeiiaiaa
% bIcacaqGPbGaaeOBaiaabMgacaqG0bGaaeyAaiaabggacaqGSbGaae
% iiaiaabohacaqGVbGaaeiBaiaabMgacaqGKbGaaeiiaiaabsgacaqG
% LbGaaeOBaiaabohacaqGPbGaaeiDaiaabMhacaqGPaaabaGaamyBam
% aaBaaaleaacaWGWbaabeaakiabg2da9iaaiodacaaIWaGaaeiiaiaa
% bUgacaqGNbGaaeiiaiaabIcacaqGWbGaaeOCaiaab+gacaqGQbGaae
% yzaiaabogacaqG0bGaaeyAaiaabYgacaqGLbGaaeiiaiaab2gacaqG
% HbGaae4CaiaabohacaqGPaaabaGaamiCamaaBaaaleaacaWGYbaabe
% aakiabg2da9iaaigdacaaIWaWaaWbaaSqabeaacaaI4aaaaOGaaeii
% aiaabcfacaqGHbGaaeiiaiaabIcacaqGYbGaaeyzaiaabohacaqGPb
% Gaae4CaiaabshacaqGPbGaaeODaiaabwgacaqGGaGaaeiCaiaabkha
% caqGLbGaae4CaiaabohacaqG1bGaaeOCaiaabwgacaqGPaaabaGaam
% iCamaaBaaaleaacaaIWaaabeaakiabg2da9iaaigdacaaIWaWaaWba
% aSqabeaacaaI1aaaaOGaaeiiaiaabcfacaqGHbGaaeiiaiaabIcaca
% qGPbGaaeOBaiaabMgacaqG0bGaaeyAaiaabggacaqGSbGaaeiiaiaa
% bchacaqGYbGaaeyzaiaabohacaqGZbGaaeyDaiaabkhacaqGLbGaae
% ykaaqaaiaadsfadaWgaaWcbaGaaGimaaqabaGccqGH9aqpcaaIYaGa
% aGyoaiaaisdacaqGGaGaae4saiaabccacaqGOaGaaeyAaiaab6gaca
% qGPbGaaeiDaiaabMgacaqGHbGaaeiBaiaabccacaqG0bGaaeyzaiaa
% b2gacaqGWbGaaeyzaiaabkhacaqGHbGaaeiDaiaabwhacaqGYbGaae
% yzaiaabMcaaeaacqaHbpGCdaWgaaWcbaGaaGimaaqabaGccqGH9aqp
% caaIWaGaaiOlaiaaiIdacaaI3aGaaGymaiaaiodacaqGGaGaae4Aai
% aabEgacaqGVaGaaeyBamaaCaaaleqabaGaae4maaaakiaabccacaqG
% OaGaaeyAaiaab6gacaqGPbGaaeiDaiaabMgacaqGHbGaaeiBaiaabc
% cacaqGNbGaaeyyaiaabohacaqGGaGaaeizaiaabwgacaqGUbGaae4C
% aiaabMgacaqG0bGaaeyEaiaabMcaaeaacqaHXoqydaWgaaWcbaGaaG
% OmaiaacYcacaaIWaaabeaakiabg2da9iaaicdacaGGUaGaaGynaiaa
% iEdacaaIWaGaaGyoaiaabccacaqGOaGaaeyAaiaab6gacaqGPbGaae
% iDaiaabMgacaqGHbGaaeiBaiaabccacaqGWbGaae4BaiaabkhacaqG
% VbGaae4CaiaabMgacaqG0bGaaeyEaiaabMcaaeaacaWGKbGaamyAai
% aadggacaWGTbGaeyypa0JaaGymaiaaiodacaaIYaGaamyBaiaad2ga
% caqGGaGaaeikaiaabsgacaqGPbGaaeyyaiaab2gacaqGLbGaaeiDai
% aabwgacaqGYbGaaeiiaiaab+gacaqGMbGaaeiiaiaabshacaqGObGa
% aeyzaiaabccacaqGNbGaaeyDaiaab6gacaqGPaaabaGaamiBaiaadw
% gacaWGUbGaam4zaiaadshacaWGObGaeyypa0JaaG4naiaaiAdacaaI
% YaGaamyBaiaad2gacaqGGaGaaeikaiaabYgacaqGLbGaaeOBaiaabE
% gacaqG0bGaaeiAaiaabccacaqGVbGaaeOzaiaabccacaqG0bGaaeiA
% aiaabwgacaqGGaGaaeiDaiaabwhacaqGIbGaaeyzaiaabMcaaeaaca
% WGTbGaamiCaiaad+gacaWG3bGaeyypa0JaaGyoaiaac6cacaaI1aGa
% aGOmaiaaiwdacaaI1aGaam4AaiaadEgacaqGGaGaaeikaiaabchaca
% qGVbGaae4DaiaabsgacaqGLbGaaeOCaiaabccacaqGTbGaaeyyaiaa
% bohacaqGZbGaaeykaaqaaiabeU7aSjabg2da9iaaicdacaGGUaGaaG
% imaiaaiodacaqGGaGaaeikaiaabEgacaqGYbGaaeyyaiaab6gacaqG
% 1bGaaeiBaiaabggacaqGYbGaaeiiaiaabchacaqGHbGaaeOCaiaabg
% gacaqGTbGaaeyzaiaabshacaqGLbGaaeOCaiaabMcaaeaacaWGTbGa
% amyBaiaad+gacaWGSbGaeyypa0JaaGOmaiaaigdacaGGUaGaaG4mai
% aadEgacaGGVaGaamyBaiaad+gacaWGSbGaaeiiaiaabIcacaqGTbGa
% ae4BaiaabYgacaqGHbGaaeOCaiaabccacaqGTbGaaeyyaiaabohaca
% qGZbGaaeiiaiaab+gacaqGMbGaaeiiaiaabshacaqGObGaaeyzaiaa
% bccacaqGNbGaaeyyaiaabohacaqGPaaaaaa!9478!
\begin{equation}
\begin{gathered}
  \gamma _1  = 1.4 \hfill \\
  \gamma _2  = 3 \hfill \\
  \pi _2  = 2.1333 \times 10^9 Pa \hfill \\
  \rho _2  = 1600kg/m^3 {\text{ (initial solid density)}} \hfill \\
  m_p  = 30{\text{ kg (projectile mass)}} \hfill \\
  p_r  = 10^8 {\text{ Pa (resistive pressure)}} \hfill \\
  p_0  = 10^5 {\text{ Pa (initial pressure)}} \hfill \\
  T_0  = 294{\text{ K (initial temperature)}} \hfill \\
  \rho _0  = 0.8713{\text{ kg/m}}^{\text{3}} {\text{ (initial gas density)}} \hfill \\
  \alpha _{2,0}  = 0.5709{\text{ (initial solid volume fraction)}} \hfill \\
  diam = 132mm{\text{ (diameter of the gun)}} \hfill \\
  length = 762mm{\text{ (length of the tube)}} \hfill \\
  mpow = 9.5255kg{\text{ (powder mass)}} \hfill \\
  \kappa  = 0.03{\text{ (granular parameter)}} \hfill \\
  mmol = 21.3g/mol{\text{ (molar mass of the gas)}} \hfill \\
\end{gathered}
\end{equation}
 The fractional step method now involves a third step following the convection step and the pressure relaxation step. It is required for integrating the remaining source terms. Because these remaining source terms are not stiff, they are integrated by a simple explicit first-order Euler method.

In addition to this third step, we also have to take into account the motion of the shot base. When the gas pressure $p_1$ at the
shot base is greater than the resistive pressure $p_r$, the acceleration of the
projectile is given by
% MathType!Translator!2!1!AMS LaTeX.tdl!TeX -- AMS-LaTeX!
% MathType!MTEF!2!1!+-
% feaafiart1ev1aaatCvAUfeBSjuyZL2yd9gzLbvyNv2CaerbuLwBLn
% hiov2DGi1BTfMBaeXatLxBI9gBaerbd9wDYLwzYbItLDharqqtubsr
% 4rNCHbGeaGqiVu0Je9sqqrpepC0xbbL8F4rqqrFfpeea0xe9Lq-Jc9
% vqaqpepm0xbba9pwe9Q8fs0-yqaqpepae9pg0FirpepeKkFr0xfr-x
% fr-xb9adbaqaaeGaciGaaiaabeqaamaabaabaaGcbaGaamyBamaaBa
% aaleaacaWGWbaabeaakmaalaaabaGaamizaiaadAhaaeaacaWGKbGa
% amiDaaaacqGH9aqpcaGGOaGaamiCamaaBaaaleaacaaIXaaabeaaki
% abgkHiTiaadchadaWgaaWcbaGaamOCaaqabaGccaGGPaWaaSaaaeaa
% cqaHapaCcaaMi8UaaeizaiaabMgacaqGHbGaaeyBamaaCaaaleqaba
% GaaGOmaaaaaOqaaiaaisdaaaGaaiOlaaaa!4C9C!
\[
m_p \frac{{dv}}
{{dt}} = (p_1  - p_r )\frac{{\pi {\kern 1pt} {\text{diam}}^2 }}
{4}.
\]
% MathType!End!2!1!

The algorithm to move the right boundary is based on an
Arbitrary Lagrangian Eulerian (ALE) approach
described in \cite{nussbaum07}. As the domain enlarges, the number of
computational cells increases.

We compare our new compressible model with the classical Gough model
described for example in \cite{nuss-hell06}.

We obtain the following results for the projectile velocity
at the exit time% MathType!MTEF!2!1!+-
% feaafiart1ev1aaatCvAUfeBSjuyZL2yd9gzLbvyNv2CaerbuLwBLn
% hiov2DGi1BTfMBaeXatLxBI9gBaerbd9wDYLwzYbItLDharqqtubsr
% 4rNCHbGeaGqiVu0Je9sqqrpepC0xbbL8F4rqqrFfpeea0xe9Lq-Jc9
% vqaqpepm0xbba9pwe9Q8fs0-yqaqpepae9pg0FirpepeKkFr0xfr-x
% fr-xb9adbaqaaeGaciGaaiaabeqaamaabaabaaGcbaqbaeqabqabaa
% aaaeaaaeaacaqGhbGaae4BaiaabwhacaqGNbGaaeiAaiaabccacaqG
% TbGaae4BaiaabsgacaqGLbGaaeiBaaqaaiaabkfacaqGLbGaaeiBai
% aabggacaqG4bGaaeOlaiaabccacaqGUbGaae4BaiaabccacaqGNbGa
% aeOCaiaabggacaqGUbGaaeyDaiaabYgacaqGHbGaaeOCaiaabccaca
% qGZbGaaeiDaiaabkhacaqGLbGaae4CaiaabohaaeaacaqGsbGaaeyz
% aiaabYgacaqGHbGaaeiEaiaab6cacaqGGaGaae4DaiaabMgacaqG0b
% GaaeiAaiaabccacaqGNbGaaeOCaiaabggacaqGUbGaaeyDaiaabYga
% caqGHbGaaeOCaiaabccacaqGZbGaaeiDaiaabkhacaqGLbGaae4Cai
% aabohaaeaacaqG2bGaaeyzaiaabYgacaqGVbGaae4yaiaabMgacaqG
% 0bGaaeyEaiaabccacaqGOaGaaeyBaiaab+cacaqGZbGaaeykaaqaai
% aabsdacaqGYaGaaeynaaqaaiaabsdacaqGXaGaaeinaaqaaiaabsda
% caqGXaGaaeinaaqaaiaabwgacaqG4bGaaeyAaiaabshacaqGGaGaae
% iDaiaabMgacaqGTbGaaeyzaiaabccacaqGOaGaaeyBaiaabohacaqG
% PaaabaGaaeOmaiaab6cacaqG5aaabaGaae4maiaab6cacaqGWaGaae
% 4naaqaaiaabodacaqGUaGaaeimaiaabEdaaeaaaeaaaeaaaeaaaaaa
% aa!9471!
\begin{equation*}
\begin{array}{*{20}c}
   {} & {{\text{Gough model}}} & {{\text{Relax}}{\text{. no granular stress}}} & {{\text{Relax}}{\text{. with granular stress}}}  \\
   {{\text{velocity (m/s)}}} & {{\text{425}}} & {{\text{414}}} & {{\text{414}}}  \\
   {{\text{exit time (ms)}}} & {{\text{2}}{\text{.9}}} & {{\text{3}}{\text{.07}}} & {{\text{3}}{\text{.07}}}  \\
   {} & {} & {} & {}  \\

 \end{array}
\end{equation*}


On Figure \ref{pcpc}, we compare the pressure evolution at the
breech and the shot base of the projectile. We observe a good
qualitative agreement between the Gough model and the relaxation
model.

\begin{figure}[!h]
\resizebox*{10cm}{!}{\rotatebox{0}{\includegraphics{pculcul.png}}}
\caption{Pressure evolution at the breech and the shot base vs.
time. Comparison between the Gough and the relaxation
model.\label{pcpc}}
\end{figure}

Finally, we plot some quantities in the tube at the final time. The
porosity $\alpha_1$, the velocities and the pressures are given in Figures \ref{porofin},
\ref{velofin}, \ref{pressfin}. We also plot on Figure \ref{rho2fin}
the density $\rho_2$ of the solid phase at the final time in order
to check that the variations of the powder density are small when compared with the
initial density $\rho_2=1600$~kg/m$^3$.

\begin{figure}[!h]
\resizebox*{10cm}{!}{\rotatebox{0}{\includegraphics{porofin.png}}}
\caption{Porosity $\alpha_1$ at the final time. Relaxation model with granular
stress.\label{porofin}}
\end{figure}
\begin{figure}[!h]
\resizebox*{10cm}{!}{\rotatebox{0}{\includegraphics{velofin.png}}}
\caption{Velocities at the final time. Relaxation model with
granular stress.\label{velofin}}
\end{figure}
\begin{figure}[!h]
\resizebox*{10cm}{!}{\rotatebox{0}{\includegraphics{pressfin.png}}}
\caption{Pressures at the final time. Relaxation model with granular
stress.\label{pressfin}}
\end{figure}
\begin{figure}[!h]
\resizebox*{10cm}{!}{\rotatebox{0}{\includegraphics{rho2fin.png}}}
\caption{Density of the solid phase at the final time. Relaxation
model with granular stress.\label{rho2fin}}
\end{figure}




\section*{Conclusion}

In this paper, we have adapted the pressure relaxation method
described in \cite{saurel99b} and \cite{hurisse} to the case of a
non-vanishing granular stress.

Starting from the two-velocity, two-pressure multiphase model of
Baer-Nunziato, we have proposed a relaxation source term in the governing equation of the void
fraction that is compatible with the second principle of
thermodynamics. In this study, we have shown that
\begin{itemize}
\item the source term increases the entropy of the phase mixture;
\item the granular stress cannot have an arbitrary form. It is
related to the fact that the differential form satisfied by the
entropy is closed.
\end{itemize}

When the relaxation time tends to zero, we have then proposed a
numerical method based on the underlying two-pressure model to
approximate the single-pressure model. In the pressure relaxation step, the void
fraction is updated in order to equilibrate the jump of pressures
with the granular stress. We have proved existence and uniqueness of
the equilibrium void fraction under some hypothesis on the granular
stress. Those hypothesis are satisfied by physically reasonable
models. In particular, it is possible to simulate in a realistic way almost incompressible solid materials. 
It is also possible to compute realistic granular stress with our simple granular law (\ref{nice}).

Eventually, we have proposed some numerical experiments in order to
validate  our approach. In an ideal test case, we have checked
that when the mesh is refined, the instability of the one-pressure
model is (fortunately) not suppressed. We also checked that the
introduction of the granular stress slightly improves the whole
stability. We finally performed more realistic simulations where we were
able to reproduce correct quantitative features of a simplified gun.
%We checked that the compressibility of the solid phase remains weak.


The whole approach is thus very promising and must now be extended
to more sophisticated granular pressure laws, equations of state and geometries.





\section{Appendix I: hyperbolicity \label{app}}
\subsection{Relaxed system}
For the sake of completness, we recall the proof of hyperbolicity of
the convection part of the equations (\ref{dipha}). It is
convenient to study it in the variables% MathType!MTEF!2!1!+-
% feaafiart1ev1aaatCvAUfeBSjuyZL2yd9gzLbvyNv2CaerbuLwBLn
% hiov2DGi1BTfMBaeXatLxBI9gBaerbd9wDYLwzYbItLDharqqtubsr
% 4rNCHbGeaGqiVu0Je9sqqrpepC0xbbL8F4rqqrFfpeea0xe9Lq-Jc9
% vqaqpepm0xbba9pwe9Q8fs0-yqaqpepae9pg0FirpepeKkFr0xfr-x
% fr-xb9adbaqaaeGaciGaaiaabeqaamaabaabaaGcbaGaamywaiabg2
% da9iaacIcacqaHXoqydaWgaaWcbaGaaGymaaqabaGccaGGSaGaeqyW
% di3aaSbaaSqaaiaaigdaaeqaaOGaaiilaiaadwhadaWgaaWcbaGaaG
% ymaaqabaGccaGGSaGaam4CamaaBaaaleaacaaIXaaabeaakiaacYca
% cqaHbpGCdaWgaaWcbaGaaGOmaaqabaGccaGGSaGaamyDamaaBaaale
% aacaaIYaaabeaakiaacYcacaWGZbWaaSbaaSqaaiaaikdaaeqaaOGa
% aiykamaaCaaaleqabaGaamivaaaakiaac6caaaa!4EA9!
\begin{equation}\label{defy}
Y = (\alpha _1 ,\rho _1 ,u_1 ,s_1 ,\rho _2 ,u_2 ,s_2 )^T .
\end{equation}
In this set of variables the system becomes% MathType!MTEF!2!1!+-
% feaafiart1ev1aaatCvAUfeBSjuyZL2yd9gzLbvyNv2CaerbuLwBLn
% hiov2DGi1BTfMBaeXatLxBI9gBaerbd9wDYLwzYbItLDharqqtubsr
% 4rNCHbGeaGqiVu0Je9sqqrpepC0xbbL8F4rqqrFfpeea0xe9Lq-Jc9
% vqaqpepm0xbba9pwe9Q8fs0-yqaqpepae9pg0FirpepeKkFr0xfr-x
% fr-xb9adbaqaaeGaciGaaiaabeqaamaabaabaaGcbaGaamywamaaBa
% aaleaacaWG0baabeaakiabgUcaRiaadkeacaGGOaGaamywaiaacMca
% caWGzbWaaSbaaSqaaiaadIhaaeqaaOGaeyypa0JaaGimaiaacYcaaa
% a!405B!
\begin{equation}
Y_t  + B(Y)Y_x  = 0,
\end{equation}
with% MathType!MTEF!2!1!+-
% feaafiart1ev1aaatCvAUfeBSjuyZL2yd9gzLbvyNv2CaerbuLwBLn
% hiov2DGi1BTfMBaeXatLxBI9gBaerbd9wDYLwzYbItLDharqqtubsr
% 4rNCHbGeaGqiVu0Je9sqqrpepC0xbbL8F4rqqrFfpeea0xe9Lq-Jc9
% vqaqpepm0xbba9pwe9Q8fs0-yqaqpepae9pg0FirpepeKkFr0xfr-x
% fr-xb9adbaqaaeGaciGaaiaabeqaamaabaabaaGceaqabeaacaWGJb
% WaaSbaaSqaaiaadUgaaeqaaOGaeyypa0ZaaSaaaeaacqGHciITcaWG
% WbGaaiikaiabeg8aYnaaBaaaleaacaWGRbaabeaakiaacYcacaWGZb
% WaaSbaaSqaaiaadUgaaeqaaOGaaiykaaqaaiabgkGi2kabeg8aYnaa
% BaaaleaacaWGRbaabeaaaaGccaGGSaGaaGzbVlaadUgacqGH9aqpca
% aIXaGaaiilaiaaikdaaeaacaWGcbGaaiikaiaadMfacaGGPaGaeyyp
% a0ZaamWaaeaafaqabeWbhaaaaaqaaiaadwhadaWgaaWcbaGaaGOmaa
% qabaaakeaaaeaaaeaaaeaaaeaaaeaaaeaadaWcaaqaaiabeg8aYnaa
% BaaaleaacaaIXaaabeaakiaacIcacaWG1bWaaSbaaSqaaiaaigdaae
% qaaOGaeyOeI0IaamyDamaaBaaaleaacaaIYaaabeaakiaacMcaaeaa
% cqaHXoqydaWgaaWcbaGaaGymaaqabaaaaaGcbaGaamyDamaaBaaale
% aacaaIXaaabeaaaOqaaiabeg8aYnaaBaaaleaacaaIXaaabeaaaOqa
% aaqaaaqaaaqaaaqaaaqaamaalaaabaGaam4yamaaDaaaleaacaaIXa
% aabaGaaGOmaaaaaOqaaiabeg8aYnaaBaaaleaacaaIXaaabeaaaaaa
% keaacaWG1bWaaSbaaSqaaiaaigdaaeqaaaGcbaWaaSaaaeaacaWGWb
% WaaSbaaSqaaiaaigdacaGGSaGaam4CamaaBaaameaacaaIXaaabeaa
% aSqabaaakeaacqaHbpGCdaWgaaWcbaGaaGymaaqabaaaaaGcbaaaba
% aabaaabaaabaaabaaabaGaamyDamaaBaaaleaacaaIXaaabeaaaOqa
% aaqaaaqaaaqaaaqaaaqaaaqaaaqaaiaadwhadaWgaaWcbaGaaGOmaa
% qabaaakeaacqaHbpGCdaWgaaWcbaGaaGOmaaqabaaakeaaaeaadaWc
% aaqaaiaadchadaWgaaWcbaGaaGymaaqabaGccqGHsislcaWGWbWaaS
% baaSqaaiaaikdaaeqaaaGcbaGaamyBamaaBaaaleaacaaIYaaabeaa
% aaaakeaaaeaaaeaaaeaadaWcaaqaaiaadogadaqhaaWcbaGaaGOmaa
% qaaiaaikdaaaaakeaacqaHbpGCdaWgaaWcbaGaaGOmaaqabaaaaaGc
% baGaamyDamaaBaaaleaacaaIYaaabeaaaOqaamaalaaabaGaamiCam
% aaBaaaleaacaaIYaGaaiilaiaadohadaWgaaadbaGaaGOmaaqabaaa
% leqaaaGcbaGaeqyWdi3aaSbaaSqaaiaaikdaaeqaaaaaaOqaaaqaaa
% qaaaqaaaqaaaqaaaqaaiaadwhadaWgaaWcbaGaaGOmaaqabaaaaaGc
% caGLBbGaayzxaaaaaaa!918B!
\begin{equation}\label{by}
\begin{gathered}
  c_k  = \frac{{\partial p_k(\rho _k ,s_k )}}
{{\partial \rho _k }},\quad k = 1,2 \hfill \\
  B(Y) = \left[ {\begin{array}{*{20}c}
   {u_2 } & {} & {} & {} & {} & {} & {}  \\
   {\frac{{\rho _1 (u_1  - u_2 )}}
{{\alpha _1 }}} & {u_1 } & {\rho _1 } & {} & {} & {} & {}  \\
   {} & {\frac{{c_1^2 }}
{{\rho _1 }}} & {u_1 } & {\frac{{p_{1,s_1 } }}
{{\rho _1 }}} & {} & {} & {}  \\
   {} & {} & {} & {u_1 } & {} & {} & {}  \\
   {} & {} & {} & {} & {u_2 } & {\rho _2 } & {}  \\
   {\frac{{p_1  - p_2 }}
{{m_2 }}} & {} & {} & {} & {\frac{{c_2^2 }} {{\rho _2 }}} & {u_2 } &
{\frac{{p_{2,s_2 } }}
{{\rho _2 }}}  \\
   {} & {} & {} & {} & {} & {} & {u_2 }  \\

 \end{array} } \right] \hfill \\
\end{gathered}
\end{equation}

The characteristic polynomial is% MathType!MTEF!2!1!+-
% feaafiart1ev1aaatCvAUfeBSjuyZL2yd9gzLbvyNv2CaerbuLwBLn
% hiov2DGi1BTfMBaeXatLxBI9gBaerbd9wDYLwzYbItLDharqqtubsr
% 4rNCHbGeaGqiVu0Je9sqqrpepC0xbbL8F4rqqrFfpeea0xe9Lq-Jc9
% vqaqpepm0xbba9pwe9Q8fs0-yqaqpepae9pg0FirpepeKkFr0xfr-x
% fr-xb9adbaqaaeGaciGaaiaabeqaamaabaabaaGcbaGaamiuaiaacI
% cacqaH7oaBcaGGPaGaeyypa0JaaiikaiaadwhadaWgaaWcbaGaaGOm
% aaqabaGccqGHsislcqaH7oaBcaGGPaWaaWbaaSqabeaacaaIYaaaaO
% GaaiikaiaadwhadaWgaaWcbaGaaGymaaqabaGccqGHsislcqaH7oaB
% caGGPaGaaiikaiaadwhadaWgaaWcbaGaaGymaaqabaGccqGHsislca
% WGJbWaaSbaaSqaaiaaigdaaeqaaOGaeyOeI0Iaeq4UdWMaaiykaiaa
% cIcacaWG1bWaaSbaaSqaaiaaigdaaeqaaOGaey4kaSIaam4yamaaBa
% aaleaacaaIXaaabeaakiabgkHiTiabeU7aSjaacMcacaGGOaGaamyD
% amaaBaaaleaacaaIYaaabeaakiabgkHiTiaadogadaWgaaWcbaGaaG
% OmaaqabaGccqGHsislcqaH7oaBcaGGPaGaaiikaiaadwhadaWgaaWc
% baGaaGOmaaqabaGccqGHRaWkcaWGJbWaaSbaaSqaaiaaikdaaeqaaO
% GaeyOeI0Iaeq4UdWMaaiykaaaa!6A2D!
\begin{equation}\label{polB}
P(\lambda ) = (u_2  - \lambda )^2 (u_1  - \lambda )(u_1  - c_1  -
\lambda )(u_1  + c_1  - \lambda )(u_2  - c_2  - \lambda )(u_2  + c_2
- \lambda )
\end{equation}
We can then state the following proposition
\begin{proposition}
If $\left| {u_1  - u_2 } \right| \ne c_k$, $k=1,2$ then, the system
(\ref{dipha}) is hyperbolic (its eigenvalues are real and it has a full set of eigenvectors). If $\left| {u_1  - u_2 } \right| =c_k$
for $k=1$ or $2$ then the system is resonant (its eigenvalues are real but $B(Y)$ is not diagonalizable).
\end{proposition}

\subsection{Equilibrium system} We also study the
hyperbolicity behavior of the equilibrium system, which correspond to the limit $\tau_p=0$. When the granular stress vanishes, the results are given in several papers. We thus only detail the case $R>0$ with a granular stress satisfying (\ref{genegranu}). The computations given below have been used
to draw Figure \ref{ima-1000}.
We note, for any quantity $z$, % MathType!MTEF!2!1!+-
% feaafiart1ev1aaatCvAUfeBSjuyZL2yd9gzLbvyNv2CaerbuLwBLn
% hiov2DGi1BTfMBaeXatLxBI9gBaerbd9wDYLwzYbItLDharqqtubsr
% 4rNCHbGeaGqiVu0Je9sqqrpepC0xbbL8F4rqqrFfpeea0xe9Lq-Jc9
% vqaqpepm0xbba9pwe9Q8fs0-yqaqpepae9pg0FirpepeKkFr0xfr-x
% fr-xb9adbaqaaeGaciGaaiaabeqaamaabaabaaGcbaGaamiramaaBa
% aaleaacaWGRbaabeaakiaadQhacqGH9aqpcaWG6bWaaSbaaSqaaiaa
% dshaaeqaaOGaey4kaSIaamyDamaaBaaaleaacaWGRbaabeaakiaadQ
% hadaWgaaWcbaGaamiEaaqabaGccaGGUaaaaa!41F5!
\begin{equation}
D_k z = z_t  + u_k z_x .
\end{equation}
At equilibrium, we can remove the transport equations in $\alpha_k$
and replace them by the pressure relation% MathType!MTEF!2!1!+-
% feaafiart1ev1aaatCvAUfeBSjuyZL2yd9gzLbvyNv2CaerbuLwBLn
% hiov2DGi1BTfMBaeXatLxBI9gBaerbd9wDYLwzYbItLDharqqtubsr
% 4rNCHbGeaGqiVu0Je9sqqrpepC0xbbL8F4rqqrFfpeea0xe9Lq-Jc9
% vqaqpepm0xbba9pwe9Q8fs0-yqaqpepae9pg0FirpepeKkFr0xfr-x
% fr-xb9adbaqaaeGaciGaaiaabeqaamaabaabaaGcbaGaamiCamaaBa
% aaleaacaaIYaaabeaakiabg2da9iaadchadaWgaaWcbaGaaGymaaqa
% baGccqGHRaWkcqaHXoqydaWgaaWcbaGaaGOmaaqabaGccqaHbpGCda
% qhaaWcbaGaaGOmaaqaaiabeo7aNnaaBaaameaacaaIYaaabeaaaaGc
% cqaH4oqCcaGGOaGaeqySde2aaSbaaSqaaiaaikdaaeqaaOGaaiykai
% abg2da9iaadchadaWgaaWcbaGaaGymaaqabaGccqGHRaWkcqaHbpGC
% daqhaaWcbaGaaGOmaaqaaiabeo7aNnaaBaaameaacaaIYaaabeaaaa
% GccaWGNbGaaiikaiabeg7aHnaaBaaaleaacaaIYaaabeaakiaacMca
% caGGUaaaaa!57AF!
\begin{equation}
p_2  = p_1  + \alpha _2 \rho _2^{\gamma _2 } \theta (\alpha _2 ) =
p_1  + \rho _2^{\gamma _2 } g(\alpha _2 ).
\end{equation}
We note $h$ the inverse function of $g$,% MathType!MTEF!2!1!+-
% faaafaart1ev1aaat0uyJj1BTfMBaerbuLwBLnharmWu51MyVXgaru
% WqVvNCPvMCaebbnrfifHhDYfgasaacH8srps0lbbf9q8WrFfeuY-Hh
% bbf9v8qqaqFr0xc9pk0xbba9q8WqFfea0-yr0RYxir-Jbba9q8aq0-
% yq-He9q8qqQ8frFve9Fve9Ff0dmeaabaqaciGacaGaaeqabaWaaeaa
% eaaakeaacqGHaiIicqaHXoqycqGHiiIZcaGGDbGaaGimaiaacYcaca
% aIXaGaai4waiaaywW7caWGObGaaiikaiaadEgacaGGOaGaeqySdeMa
% aiykaiaacMcacqGH9aqpcqaHXoqycaGGUaaaaa!3F9A!
\begin{equation}
\forall \alpha  \in ]0,1[\quad h(g(\alpha )) = \alpha .
\end{equation}
Of course, when the granular stress vanishes, the inverse function of $g$ is not defined and the computations must be carried out in another way.
At equilibrium, we have% MathType!MTEF!2!1!+-
% feaafiart1ev1aaatCvAUfeBSjuyZL2yd9gzLbvyNv2CaerbuLwBLn
% hiov2DGi1BTfMBaeXatLxBI9gBaerbd9wDYLwzYbItLDharqqtubsr
% 4rNCHbGeaGqiVu0Je9sqqrpepC0xbbL8F4rqqrFfpeea0xe9Lq-Jc9
% vqaqpepm0xbba9pwe9Q8fs0-yqaqpepae9pg0FirpepeKkFr0xfr-x
% fr-xb9adbaqaaeGaciGaaiaabeqaamaabaabaaGceaqabeaacqaHXo
% qydaWgaaWcbaGaaGOmaaqabaGccqGH9aqpcaWGObWaaeWaaeaadaWc
% aaqaaiaadchadaWgaaWcbaGaaGOmaaqabaGccqGHsislcaWGWbWaaS
% baaSqaaiaaigdaaeqaaaGcbaGaeqyWdi3aa0baaSqaaiaaikdaaeaa
% cqaHZoWzdaWgaaadbaGaaGOmaaqabaaaaaaaaOGaayjkaiaawMcaaa
% qaaiabgkDiElaadsgacqaHXoqydaWgaaWcbaGaaGOmaaqabaGccqGH
% 9aqpcqaH0oazdaqadaqaamaabmaabaGaam4yamaaDaaaleaacaaIYa
% aabaGaaGOmaaaakiabgkHiTiabeo7aNnaaBaaaleaacaaIYaaabeaa
% kmaalaaabaGaamiCamaaBaaaleaacaaIYaaabeaakiabgkHiTiaadc
% hadaWgaaWcbaGaaGymaaqabaaakeaacqaHbpGCdaWgaaWcbaGaaGOm
% aaqabaaaaaGccaGLOaGaayzkaaGaamizaiabeg8aYnaaBaaaleaaca
% aIYaaabeaakiabgUcaRiaadchadaWgaaWcbaGaaGOmaiaacYcacaWG
% ZbWaaSbaaWqaaiaaikdaaeqaaaWcbeaakiaadsgacaWGZbWaaSbaaS
% qaaiaaikdaaeqaaOGaeyOeI0Iaam4yamaaDaaaleaacaaIXaaabaGa
% aGOmaaaakiaadsgacqaHbpGCdaWgaaWcbaGaaGymaaqabaGccqGHsi
% slcaWGWbWaaSbaaSqaaiaaigdacaGGSaGaam4CamaaBaaameaacaaI
% XaaabeaaaSqabaGccaWGKbGaam4CamaaBaaaleaacaaIXaaabeaaaO
% GaayjkaiaawMcaaaqaaiaabEhacaqGPbGaaeiDaiaabIgacaqGGaGa
% eqiTdqMaaeypamaalaaabaGaamiAaiaacEcadaqadaqaamaalaaaba
% GaamiCamaaBaaaleaacaaIYaaabeaakiabgkHiTiaadchadaWgaaWc
% baGaaGymaaqabaaakeaacqaHbpGCdaqhaaWcbaGaaGOmaaqaaiabeo
% 7aNnaaBaaameaacaaIYaaabeaaaaaaaaGccaGLOaGaayzkaaaabaGa
% eqyWdi3aa0baaSqaaiaaikdaaeaacqaHZoWzdaWgaaadbaGaaGOmaa
% qabaaaaaaakiabg6da+iaaicdacaGGUaaaaaa!9631!
\begin{equation}
\begin{gathered}
  \alpha _2  = h\left( {\frac{{p_2  - p_1 }}
{{\rho _2^{\gamma _2 } }}} \right) \hfill \\
   \Rightarrow d\alpha _2  = \delta \left( {\left( {c_2^2  - \gamma _2 \frac{{p_2  - p_1 }}
{{\rho _2 }}} \right)d\rho _2  + p_{2,s_2 } ds_2  - c_1^2 d\rho _1  - p_{1,s_1 } ds_1 } \right) \hfill \\
  {\text{with }}\delta {\text{ = }}\frac{{h'\left( {\frac{{p_2  - p_1 }}
{{\rho _2^{\gamma _2 } }}} \right)}}
{{\rho _2^{\gamma _2 } }} > 0. \hfill \\
\end{gathered}
\end{equation}
\begin{example}\label{ex1}
We can consider% MathType!MTEF!2!1!+-
% feaafiart1ev1aaatCvAUfeBSjuyZL2yd9gzLbvyNv2CaerbuLwBLn
% hiov2DGi1BTfMBaeXatLxBI9gBaerbd9wDYLwzYbItLDharqqtubsr
% 4rNCHbGeaGqiVu0Je9sqqrpepC0xbbL8F4rqqrFfpeea0xe9Lq-Jc9
% vqaqpepm0xbba9pwe9Q8fs0-yqaqpepae9pg0FirpepeKkFr0xfr-x
% fr-xb9adbaqaaeGaciGaaiaabeqaamaabaabaaGcbaGaeqiUdeNaai
% ikaiabeg7aHjaacMcacqGH9aqpcqaH7oaBcqaHXoqydaahaaWcbeqa
% aiabeo7aNnaaBaaameaacaaIYaaabeaaliabgkHiTiaaigdaaaGcca
% GGUaaaaa!4420!
\begin{equation}\label{pratic}
\theta (\alpha ) = \kappa \alpha ^{\gamma _2  - 1} .
\end{equation}
We then have % MathType!MTEF!2!1!+-
% feaafiart1ev1aaatCvAUfeBSjuyZL2yd9gzLbvyNv2CaerbuLwBLn
% hiov2DGi1BTfMBaeXatLxBI9gBaerbd9wDYLwzYbItLDharqqtubsr
% 4rNCHbGeaGqiVu0Je9sqqrpepC0xbbL8F4rqqrFfpeea0xe9Lq-Jc9
% vqaqpepm0xbba9pwe9Q8fs0-yqaqpepae9pg0FirpepeKkFr0xfr-x
% fr-xb9adbaqaaeGaciGaaiaabeqaamaabaabaaGcbaGaeqiTdqMaey
% ypa0ZaaSaaaeaacqaHXoqydaqhaaWcbaGaaGOmaaqaaiaaigdacqGH
% sislcaaIXaGaai4laiabeo7aNnaaBaaameaacaaIYaaabeaaaaaake
% aacqaH7oaBcqaHZoWzdaWgaaWcbaGaaGOmaaqabaGccqaHbpGCdaqh
% aaWcbaGaaGOmaaqaaiabeo7aNnaaBaaameaacaaIYaaabeaaaaaaaO
% GaaiOlaaaa!4B22!
\begin{equation}\label{phy}
\delta  = \frac{{\alpha _2^{1 - 1/\gamma _2 } }} {{\kappa \gamma _2
\rho _2^{\gamma _2 } }}.
\end{equation}










\end{example}
It is natural to introduce% MathType!MTEF!2!1!+-
% feaafiart1ev1aaatCvAUfeBSjuyZL2yd9gzLbvyNv2CaerbuLwBLn
% hiov2DGi1BTfMBaeXatLxBI9gBaerbd9wDYLwzYbItLDharqqtubsr
% 4rNCHbGeaGqiVu0Je9sqqrpepC0xbbL8F4rqqrFfpeea0xe9Lq-Jc9
% vqaqpepm0xbba9pwe9Q8fs0-yqaqpepae9pg0FirpepeKkFr0xfr-x
% fr-xb9adbaqaaeGaciGaaiaabeqaamaabaabaaGcbaGaamyyamaaDa
% aaleaacaaIYaaabaGaaGOmaaaakiabg2da9maalaaabaGaeq4SdC2a
% aSbaaSqaaiaaikdaaeqaaOGaeqyWdi3aaSbaaSqaaiaaigdaaeqaaa
% GcbaGaeq4SdC2aaSbaaSqaaiaaigdaaeqaaOGaeqyWdi3aaSbaaSqa
% aiaaikdaaeqaaaaakiaadogadaqhaaWcbaGaaGymaaqaaiaaikdaaa
% GccqGHRaWkcqaHZoWzdaWgaaWcbaGaaGOmaaqabaGcdaWcaaqaaiab
% ec8aWnaaBaaaleaacaaIYaaabeaakiabgkHiTiabec8aWnaaBaaale
% aacaaIXaaabeaaaOqaaiabeg8aYnaaBaaaleaacaaIYaaabeaaaaGc
% cqGH+aGpcaaIWaGaaiilaaaa!55BB!
\begin{equation}
a_2^2  = \frac{{\gamma _2 \rho _1 }} {{\gamma _1 \rho _2 }}c_1^2  +
\gamma _2 \frac{{\pi _2  - \pi _1 }} {{\rho _2 }} > 0,
\end{equation}

In such a way that we have also% MathType!MTEF!2!1!+-
% feaafiart1ev1aaatCvAUfeBSjuyZL2yd9gzLbvyNv2CaerbuLwBLn
% hiov2DGi1BTfMBaeXatLxBI9gBaerbd9wDYLwzYbItLDharqqtubsr
% 4rNCHbGeaGqiVu0Je9sqqrpepC0xbbL8F4rqqrFfpeea0xe9Lq-Jc9
% vqaqpepm0xbba9pwe9Q8fs0-yqaqpepae9pg0FirpepeKkFr0xfr-x
% fr-xb9adbaqaaeGaciGaaiaabeqaamaabaabaaGcbaGaamizaiabeg
% 7aHnaaBaaaleaacaaIYaaabeaakiabg2da9iabes7aKnaabmaabaGa
% amyyamaaDaaaleaacaaIYaaabaGaaGOmaaaakiaadsgacqaHbpGCda
% WgaaWcbaGaaGOmaaqabaGccqGHRaWkcaWGWbWaaSbaaSqaaiaaikda
% caGGSaGaam4CamaaBaaameaacaaIYaaabeaaaSqabaGccaWGKbGaam
% 4CamaaBaaaleaacaaIYaaabeaakiabgkHiTiaadogadaqhaaWcbaGa
% aGymaaqaaiaaikdaaaGccaWGKbGaeqyWdi3aaSbaaSqaaiaaigdaae
% qaaOGaeyOeI0IaamiCamaaBaaaleaacaaIXaGaaiilaiaadohadaWg
% aaadbaGaaGymaaqabaaaleqaaOGaamizaiaadohadaWgaaWcbaGaaG
% ymaaqabaaakiaawIcacaGLPaaaaaa!5B60!
\begin{equation}
d\alpha _2  = \delta \left( {a_2^2 d\rho _2  + p_{2,s_2 } ds_2  -
c_1^2 d\rho _1  - p_{1,s_1 } ds_1 } \right)
\end{equation}
It gives another expression of the source term $P$ at equilibrium% MathType!MTEF!2!1!+-
% feaafiart1ev1aaatCvAUfeBSjuyZL2yd9gzLbvyNv2CaerbuLwBLn
% hiov2DGi1BTfMBaeXatLxBI9gBaerbd9wDYLwzYbItLDharqqtubsr
% 4rNCHbGeaGqiVu0Je9sqqrpepC0xbbL8F4rqqrFfpeea0xe9Lq-Jc9
% vqaqpepm0xbba9pwe9Q8fs0-yqaqpepae9pg0FirpepeKkFr0xfr-x
% fr-xb9adbaqaaeGaciGaaiaabeqaamaabaabaaGcbaGaamiuaiabg2
% da9iabgkHiTiabes7aKnaabmaabaGaamyyamaaDaaaleaacaaIYaaa
% baGaaGOmaaaakiaadseadaWgaaWcbaGaaGOmaaqabaGccqaHbpGCda
% WgaaWcbaGaaGOmaaqabaGccqGHRaWkcaWGWbWaaSbaaSqaaiaaikda
% caGGSaGaam4CamaaBaaameaacaaIYaaabeaaaSqabaGccaWGebWaaS
% baaSqaaiaaikdaaeqaaOGaam4CamaaBaaaleaacaaIYaaabeaakiab
% gkHiTiaadogadaqhaaWcbaGaaGymaaqaaiaaikdaaaGccaWGebWaaS
% baaSqaaiaaikdaaeqaaOGaeqyWdi3aaSbaaSqaaiaaigdaaeqaaOGa
% eyOeI0IaamiCamaaBaaaleaacaaIXaGaaiilaiaadohadaWgaaadba
% GaaGymaaqabaaaleqaaOGaamiramaaBaaaleaacaaIYaaabeaakiaa
% dohadaWgaaWcbaGaaGymaaqabaaakiaawIcacaGLPaaaaaa!5CF0!
\begin{equation}
P =  - \delta \left( {a_2^2 D_2 \rho _2  + p_{2,s_2 } D_2 s_2  -
c_1^2 D_2 \rho _1  - p_{1,s_1 } D_2 s_1 } \right)
\end{equation}
We then rewrite the equilibrium system in the variables% MathType!MTEF!2!1!+-
% feaafiart1ev1aaatCvAUfeBSjuyZL2yd9gzLbvyNv2CaerbuLwBLn
% hiov2DGi1BTfMBaeXatLxBI9gBaerbd9wDYLwzYbItLDharqqtubsr
% 4rNCHbGeaGqiVu0Je9sqqrpepC0xbbL8F4rqqrFfpeea0xe9Lq-Jc9
% vqaqpepm0xbba9pwe9Q8fs0-yqaqpepae9pg0FirpepeKkFr0xfr-x
% fr-xb9adbaqaaeGaciGaaiaabeqaamaabaabaaGcbaGaamOwaiabg2
% da9iaacIcacqaHbpGCdaWgaaWcbaGaaGymaaqabaGccaGGSaGaamyD
% amaaBaaaleaacaaIXaaabeaakiaacYcacaWGZbWaaSbaaSqaaiaaig
% daaeqaaOGaaiilaiabeg8aYnaaBaaaleaacaaIYaaabeaakiaacYca
% caWG1bWaaSbaaSqaaiaaikdaaeqaaOGaaiilaiaadohadaWgaaWcba
% GaaGOmaaqabaGccaGGPaWaaWbaaSqabeaacaWGubaaaOGaaiOlaaaa
% !4B6A!
\begin{equation}
Z = (\rho _1 ,u_1 ,s_1 ,\rho _2 ,u_2 ,s_2 )^T ,
\end{equation}
In these variables, the system is% MathType!MTEF!2!1!+-
% feaafiart1ev1aaatCvAUfeBSjuyZL2yd9gzLbvyNv2CaerbuLwBLn
% hiov2DGi1BTfMBaeXatLxBI9gBaerbd9wDYLwzYbItLDharqqtubsr
% 4rNCHbGeaGqiVu0Je9sqqrpepC0xbbL8F4rqqrFfpeea0xe9Lq-Jc9
% vqaqpepm0xbba9pwe9Q8fs0-yqaqpepae9pg0FirpepeKkFr0xfr-x
% fr-xb9adbaqaaeGaciGaaiaabeqaamaabaabaaGcbaGaamOwamaaBa
% aaleaacaWG0baabeaakiabgUcaRiaadoeacaGGOaGaamOwaiaacMca
% caWGAbWaaSbaaSqaaiaadIhaaeqaaOGaeyypa0JaaGimaiaac6caaa
% a!4061!
\begin{equation}
Z_t  + C(Z)Z_x  = 0.
\end{equation}
For the sake of completeness, we give some details of the
computations
% MathType!MTEF!2!1!+-
% feaafiart1ev1aaatCvAUfeBSjuyZL2yd9gzLbvyNv2CaerbuLwBLn
% hiov2DGi1BTfMBaeXatLxBI9gBaerbd9wDYLwzYbItLDharqqtubsr
% 4rNCHbGeaGqiVu0Je9sqqrpepC0xbbL8F4rqqrFfpeea0xe9Lq-Jc9
% vqaqpepm0xbba9pwe9Q8fs0-yqaqpepae9pg0FirpepeKkFr0xfr-x
% fr-xb9adbaqaaeGaciGaaiaabeqaamaabaabaaGceaqabeaacqaHbp
% GCdaWgaaWcbaGaaGymaiaacYcacaWG0baabeaakiabgUcaRiaadwha
% daWgaaWcbaGaaGymaaqabaGccqaHbpGCdaWgaaWcbaGaaGymaiaacY
% cacaWG4baabeaakiabgkHiTmaalaaabaGaeqyWdi3aaSbaaSqaaiaa
% igdaaeqaaaGcbaGaeqySde2aaSbaaSqaaiaaigdaaeqaaaaakiaacI
% cacaWG1bWaaSbaaSqaaiaaigdaaeqaaOGaeyOeI0IaamyDamaaBaaa
% leaacaaIYaaabeaakiaacMcacqaH0oazdaqadaqaaiaadggadaqhaa
% WcbaGaaGOmaaqaaiaaikdaaaGccqaHbpGCdaWgaaWcbaGaaGOmaiaa
% cYcacaWG4baabeaakiabgUcaRiaadchadaWgaaWcbaGaaGOmaiaacY
% cacaWGZbWaaSbaaWqaaiaaikdaaeqaaaWcbeaakiaadohadaWgaaWc
% baGaaGOmaiaacYcacaWG4baabeaakiabgkHiTiaadogadaqhaaWcba
% GaaGymaaqaaiaaikdaaaGccqaHbpGCdaWgaaWcbaGaaGymaiaacYca
% caWG4baabeaakiabgkHiTiaadchadaWgaaWcbaGaaGymaiaacYcaca
% WGZbWaaSbaaWqaaiaaigdaaeqaaaWcbeaakiaadohadaWgaaWcbaGa
% aGymaiaacYcacaWG4baabeaaaOGaayjkaiaawMcaaiabgUcaRiabeg
% 8aYnaaBaaaleaacaaIXaaabeaakiaadwhadaWgaaWcbaGaaGymaiaa
% cYcacaWG4baabeaaaOqaaiabgkHiTmaalaaabaGaeqyWdi3aaSbaaS
% qaaiaaigdaaeqaaaGcbaGaeqySde2aaSbaaSqaaiaaigdaaeqaaaaa
% kiabes7aKnaabmaabaGaamyyamaaDaaaleaacaaIYaaabaGaaGOmaa
% aakiaadseadaWgaaWcbaGaaGOmaaqabaGccqaHbpGCdaWgaaWcbaGa
% aGOmaaqabaGccqGHRaWkcaWGWbWaaSbaaSqaaiaaikdacaGGSaGaam
% 4CamaaBaaameaacaaIYaaabeaaaSqabaGccaWGebWaaSbaaSqaaiaa
% ikdaaeqaaOGaam4CamaaBaaaleaacaaIYaaabeaakiabgkHiTiaado
% gadaqhaaWcbaGaaGymaaqaaiaaikdaaaGccaWGebWaaSbaaSqaaiaa
% ikdaaeqaaOGaeqyWdi3aaSbaaSqaaiaaigdaaeqaaOGaeyOeI0Iaam
% iCamaaBaaaleaacaaIXaGaaiilaiaadohadaWgaaadbaGaaGymaaqa
% baaaleqaaOGaamiramaaBaaaleaacaaIYaaabeaakiaadohadaWgaa
% WcbaGaaGymaaqabaaakiaawIcacaGLPaaacqGH9aqpcaaIWaGaaiil
% aaqaaiabeg8aYnaaBaaaleaacaaIYaGaaiilaiaadshaaeqaaOGaey
% 4kaSIaamyDamaaBaaaleaacaaIYaaabeaakiabeg8aYnaaBaaaleaa
% caaIYaGaaiilaiaadIhaaeqaaOGaey4kaSYaaSaaaeaacqaHbpGCda
% WgaaWcbaGaaGOmaaqabaaakeaacqaHXoqydaWgaaWcbaGaaGOmaaqa
% baaaaOGaeqiTdq2aaeWaaeaacaWGHbWaa0baaSqaaiaaikdaaeaaca
% aIYaaaaOGaamiramaaBaaaleaacaaIYaaabeaakiabeg8aYnaaBaaa
% leaacaaIYaaabeaakiabgUcaRiaadchadaWgaaWcbaGaaGOmaiaacY
% cacaWGZbWaaSbaaWqaaiaaikdaaeqaaaWcbeaakiaadseadaWgaaWc
% baGaaGOmaaqabaGccaWGZbWaaSbaaSqaaiaaikdaaeqaaOGaeyOeI0
% Iaam4yamaaDaaaleaacaaIXaaabaGaaGOmaaaakiaadseadaWgaaWc
% baGaaGOmaaqabaGccqaHbpGCdaWgaaWcbaGaaGymaaqabaGccqGHsi
% slcaWGWbWaaSbaaSqaaiaaigdacaGGSaGaam4CamaaBaaameaacaaI
% XaaabeaaaSqabaGccaWGebWaaSbaaSqaaiaaikdaaeqaaOGaam4Cam
% aaBaaaleaacaaIXaaabeaaaOGaayjkaiaawMcaaiabgUcaRiabeg8a
% YnaaBaaaleaacaaIYaaabeaakiaadwhadaWgaaWcbaGaaGOmaiaacY
% cacaWG4baabeaakiabg2da9iaaicdaaaaa!E4D1!
\begin{equation}
\begin{gathered}
  \rho _{1,t}  + u_1 \rho _{1,x}  - \frac{{\rho _1 }}
{{\alpha _1 }}(u_1  - u_2 )\delta \left( {a_2^2 \rho _{2,x}  + p_{2,s_2 } s_{2,x}  - c_1^2 \rho _{1,x}  - p_{1,s_1 } s_{1,x} } \right) + \rho _1 u_{1,x}  \hfill \\
   - \frac{{\rho _1 }}
{{\alpha _1 }}\delta \left( {a_2^2 D_2 \rho _2  + p_{2,s_2 } D_2 s_2  - c_1^2 D_2 \rho _1  - p_{1,s_1 } D_2 s_1 } \right) = 0, \hfill \\
  \rho _{2,t}  + u_2 \rho _{2,x}  + \frac{{\rho _2 }}
{{\alpha _2 }}\delta \left( {a_2^2 D_2 \rho _2  + p_{2,s_2 } D_2 s_2  - c_1^2 D_2 \rho _1  - p_{1,s_1 } D_2 s_1 } \right) + \rho _2 u_{2,x}  = 0 \hfill \\
\end{gathered}
\end{equation}
% MathType!MTEF!2!1!+-
% feaafiart1ev1aaatCvAUfeBSjuyZL2yd9gzLbvyNv2CaerbuLwBLn
% hiov2DGi1BTfMBaeXatLxBI9gBaerbd9wDYLwzYbItLDharqqtubsr
% 4rNCHbGeaGqiVu0Je9sqqrpepC0xbbL8F4rqqrFfpeea0xe9Lq-Jc9
% vqaqpepm0xbba9pwe9Q8fs0-yqaqpepae9pg0FirpepeKkFr0xfr-x
% fr-xb9adbaqaaeGaciGaaiaabeqaamaabaabaaGceaqabeaacqaHbp
% GCdaWgaaWcbaGaaGymaiaacYcacaWG0baabeaakiabgUcaRiaadwha
% daWgaaWcbaGaaGymaaqabaGccqaHbpGCdaWgaaWcbaGaaGymaiaacY
% cacaWG4baabeaakiabgkHiTmaalaaabaGaeqyWdi3aaSbaaSqaaiaa
% igdaaeqaaaGcbaGaeqySde2aaSbaaSqaaiaaigdaaeqaaaaakiaadw
% hadaWgaaWcbaGaaGymaaqabaGccqaH0oazdaqadaqaaiaadggadaqh
% aaWcbaGaaGOmaaqaaiaaikdaaaGccqaHbpGCdaWgaaWcbaGaaGOmai
% aacYcacaWG4baabeaakiabgUcaRiaadchadaWgaaWcbaGaaGOmaiaa
% cYcacaWGZbWaaSbaaWqaaiaaikdaaeqaaaWcbeaakiaadohadaWgaa
% WcbaGaaGOmaiaacYcacaWG4baabeaakiabgkHiTiaadogadaqhaaWc
% baGaaGymaaqaaiaaikdaaaGccqaHbpGCdaWgaaWcbaGaaGymaiaacY
% cacaWG4baabeaakiabgkHiTiaadchadaWgaaWcbaGaaGymaiaacYca
% caWGZbWaaSbaaWqaaiaaigdaaeqaaaWcbeaakiaadohadaWgaaWcba
% GaaGymaiaacYcacaWG4baabeaaaOGaayjkaiaawMcaaiabgUcaRiab
% eg8aYnaaBaaaleaacaaIXaaabeaakiaadwhadaWgaaWcbaGaaGymai
% aacYcacaWG4baabeaaaOqaaiabgkHiTmaalaaabaGaeqyWdi3aaSba
% aSqaaiaaigdaaeqaaaGcbaGaeqySde2aaSbaaSqaaiaaigdaaeqaaa
% aakiabes7aKnaabmaabaGaamyyamaaDaaaleaacaaIYaaabaGaaGOm
% aaaakiabeg8aYnaaBaaaleaacaaIYaGaaiilaiaadshaaeqaaOGaey
% 4kaSIaamiCamaaBaaaleaacaaIYaGaaiilaiaadohadaWgaaadbaGa
% aGOmaaqabaaaleqaaOGaam4CamaaBaaaleaacaaIYaGaaiilaiaads
% haaeqaaOGaeyOeI0Iaam4yamaaDaaaleaacaaIXaaabaGaaGOmaaaa
% kiabeg8aYnaaBaaaleaacaaIXaGaaiilaiaadshaaeqaaOGaeyOeI0
% IaamiCamaaBaaaleaacaaIXaGaaiilaiaadohadaWgaaadbaGaaGym
% aaqabaaaleqaaOGaam4CamaaBaaaleaacaaIXaGaaiilaiaadshaae
% qaaaGccaGLOaGaayzkaaGaeyypa0JaaGimaiaacYcaaaaa!A171!
\begin{equation}
\begin{gathered}
  \rho _{1,t}  + u_1 \rho _{1,x}  - \frac{{\rho _1 }}
{{\alpha _1 }}u_1 \delta \left( {a_2^2 \rho _{2,x}  + p_{2,s_2 } s_{2,x}  - c_1^2 \rho _{1,x}  - p_{1,s_1 } s_{1,x} } \right) + \rho _1 u_{1,x}  \hfill \\
   - \frac{{\rho _1 }}
{{\alpha _1 }}\delta \left( {a_2^2 \rho _{2,t}  + p_{2,s_2 } s_{2,t}  - c_1^2 \rho _{1,t}  - p_{1,s_1 } s_{1,t} } \right) = 0, \hfill \\
\end{gathered}
\end{equation}
% MathType!MTEF!2!1!+-
% feaafiart1ev1aaatCvAUfeBSjuyZL2yd9gzLbvyNv2CaerbuLwBLn
% hiov2DGi1BTfMBaeXatLxBI9gBaerbd9wDYLwzYbItLDharqqtubsr
% 4rNCHbGeaGqiVu0Je9sqqrpepC0xbbL8F4rqqrFfpeea0xe9Lq-Jc9
% vqaqpepm0xbba9pwe9Q8fs0-yqaqpepae9pg0FirpepeKkFr0xfr-x
% fr-xb9adbaqaaeGaciGaaiaabeqaamaabaabaaGceaqabeaacaWG1b
% WaaSbaaSqaaiaaigdacaGGSaGaamiDaaqabaGccqGHRaWkcaWG1bWa
% aSbaaSqaaiaaigdaaeqaaOGaamyDamaaBaaaleaacaaIXaGaaiilai
% aadIhaaeqaaOGaey4kaSYaaSaaaeaacaaIXaaabaGaeqyWdi3aaSba
% aSqaaiaaigdaaeqaaaaakiaadchadaWgaaWcbaGaaGymaiaacYcaca
% WG4baabeaakiabg2da9iaaicdacaGGSaaabaGaamyDamaaBaaaleaa
% caaIYaGaaiilaiaadshaaeqaaOGaey4kaSIaamyDamaaBaaaleaaca
% aIYaaabeaakiaadwhadaWgaaWcbaGaaGOmaiaacYcacaWG4baabeaa
% kiabgUcaRmaalaaabaGaaGymaaqaaiabeg8aYnaaBaaaleaacaaIYa
% aabeaaaaGccaWGWbWaaSbaaSqaaiaaikdacaGGSaGaamiEaaqabaGc
% cqGHRaWkdaWcaaqaaiaadchadaWgaaWcbaGaaGOmaaqabaGccqGHsi
% slcaWGWbWaaSbaaSqaaiaaigdaaeqaaaGcbaGaamyBamaaBaaaleaa
% caaIYaaabeaaaaGccqaH0oazdaqadaqaaiaadggadaqhaaWcbaGaaG
% OmaaqaaiaaikdaaaGccqaHbpGCdaWgaaWcbaGaaGOmaiaacYcacaWG
% 4baabeaakiabgUcaRiaadchadaWgaaWcbaGaaGOmaiaacYcacaWGZb
% WaaSbaaWqaaiaaikdaaeqaaaWcbeaakiaadohadaWgaaWcbaGaaGOm
% aiaacYcacaWG4baabeaakiabgkHiTiaadogadaqhaaWcbaGaaGymaa
% qaaiaaikdaaaGccqaHbpGCdaWgaaWcbaGaaGymaiaacYcacaWG4baa
% beaakiabgkHiTiaadchadaWgaaWcbaGaaGymaiaacYcacaWGZbWaaS
% baaWqaaiaaigdaaeqaaaWcbeaakiaadohadaWgaaWcbaGaaGymaiaa
% cYcacaWG4baabeaaaOGaayjkaiaawMcaaiabg2da9iaaicdaaaaa!8990!
\begin{equation}
\begin{gathered}
  u_{1,t}  + u_1 u_{1,x}  + \frac{1}
{{\rho _1 }}p_{1,x}  = 0, \hfill \\
  u_{2,t}  + u_2 u_{2,x}  + \frac{1}
{{\rho _2 }}p_{2,x}  + \frac{{p_2  - p_1 }}
{{m_2 }}\delta \left( {a_2^2 \rho _{2,x}  + p_{2,s_2 } s_{2,x}  - c_1^2 \rho _{1,x}  - p_{1,s_1 } s_{1,x} } \right) = 0 \hfill \\
\end{gathered}
\end{equation}
% MathType!MTEF!2!1!+-
% feaafiart1ev1aaatCvAUfeBSjuyZL2yd9gzLbvyNv2CaerbuLwBLn
% hiov2DGi1BTfMBaeXatLxBI9gBaerbd9wDYLwzYbItLDharqqtubsr
% 4rNCHbGeaGqiVu0Je9sqqrpepC0xbbL8F4rqqrFfpeea0xe9Lq-Jc9
% vqaqpepm0xbba9pwe9Q8fs0-yqaqpepae9pg0FirpepeKkFr0xfr-x
% fr-xb9adbaqaaeGaciGaaiaabeqaamaabaabaaGcbaGaam4CamaaBa
% aaleaacaWGRbGaaiilaiaadshaaeqaaOGaey4kaSIaamyDamaaBaaa
% leaacaWGRbaabeaakiaadohadaWgaaWcbaGaam4AaiaacYcacaWG4b
% aabeaakiabg2da9iaaicdaaaa!4241!
\begin{equation}
s_{k,t}  + u_k s_{k,x}  = 0
\end{equation}

% MathType!MTEF!2!1!+-
% feaafiart1ev1aaatCvAUfeBSjuyZL2yd9gzLbvyNv2CaerbuLwBLn
% hiov2DGi1BTfMBaeXatLxBI9gBaerbd9wDYLwzYbItLDharqqtubsr
% 4rNCHbGeaGqiVu0Je9sqqrpepC0xbbL8F4rqqrFfpeea0xe9Lq-Jc9
% vqaqpepm0xbba9pwe9Q8fs0-yqaqpepae9pg0FirpepeKkFr0xfr-x
% fr-xb9adbaqaaeGaciGaaiaabeqaamaabaabaaGceaqabeaacaGGOa
% GaaGymaiabgUcaRmaalaaabaGaeqyWdi3aaSbaaSqaaiaaigdaaeqa
% aOGaam4yamaaDaaaleaacaaIXaaabaGaaGOmaaaakiabes7aKbqaai
% abeg7aHnaaBaaaleaacaaIXaaabeaaaaGccaGGPaGaeqyWdi3aaSba
% aSqaaiaaigdacaGGSaGaamiDaaqabaGccqGHsisldaWcaaqaaiabeg
% 8aYnaaBaaaleaacaaIXaaabeaakiaadggadaqhaaWcbaGaaGOmaaqa
% aiaaikdaaaGccqaH0oazaeaacqaHXoqydaWgaaWcbaGaaGymaaqaba
% aaaOGaeqyWdi3aaSbaaSqaaiaaikdacaGGSaGaamiDaaqabaGccqGH
% RaWkcaWG1bWaaSbaaSqaaiaaigdaaeqaaOGaeqyWdi3aaSbaaSqaai
% aaigdacaGGSaGaamiEaaqabaaakeaacqGHRaWkdaWcaaqaaiabeg8a
% YnaaBaaaleaacaaIXaaabeaaaOqaaiabeg7aHnaaBaaaleaacaaIXa
% aabeaaaaGccqaH0oazdaqadaqaaiabgkHiTiaadggadaqhaaWcbaGa
% aGOmaaqaaiaaikdaaaGccaWG1bWaaSbaaSqaaiaaigdaaeqaaOGaeq
% yWdi3aaSbaaSqaaiaaikdacaGGSaGaamiEaaqabaGccqGHRaWkcaWG
% JbWaa0baaSqaaiaaigdaaeaacaaIYaaaaOGaamyDamaaBaaaleaaca
% aIXaaabeaakiabeg8aYnaaBaaaleaacaaIXaGaaiilaiaadIhaaeqa
% aOGaey4kaSIaaiikaiaadwhadaWgaaWcbaGaaGOmaaqabaGccqGHsi
% slcaWG1bWaaSbaaSqaaiaaigdaaeqaaOGaaiykaiaadchadaWgaaWc
% baGaaGOmaiaacYcacaWGZbWaaSbaaWqaaiaaikdaaeqaaaWcbeaaki
% aadohadaWgaaWcbaGaaGOmaiaacYcacaWG4baabeaaaOGaayjkaiaa
% wMcaaiabgUcaRiabeg8aYnaaBaaaleaacaaIXaaabeaakiaadwhada
% WgaaWcbaGaaGymaiaacYcacaWG4baabeaakiabg2da9iaaicdaaeaa
% caGGOaGaaGymaiabgUcaRmaalaaabaGaeqyWdi3aaSbaaSqaaiaaik
% daaeqaaOGaamyyamaaDaaaleaacaaIYaaabaGaaGOmaaaakiabes7a
% Kbqaaiabeg7aHnaaBaaaleaacaaIYaaabeaaaaGccaGGPaGaeqyWdi
% 3aaSbaaSqaaiaaikdacaGGSaGaamiDaaqabaGccqGHsisldaWcaaqa
% aiabeg8aYnaaBaaaleaacaaIYaaabeaakiaadogadaqhaaWcbaGaaG
% ymaaqaaiaaikdaaaGccqaH0oazaeaacqaHXoqydaWgaaWcbaGaaGOm
% aaqabaaaaOGaeqyWdi3aaSbaaSqaaiaaigdacaGGSaGaamiDaaqaba
% GccqGHRaWkcaWG1bWaaSbaaSqaaiaaikdaaeqaaOGaeqyWdi3aaSba
% aSqaaiaaikdacaGGSaGaamiEaaqabaaakeaacqGHRaWkdaWcaaqaai
% abeg8aYnaaBaaaleaacaaIYaaabeaaaOqaaiabeg7aHnaaBaaaleaa
% caaIYaaabeaaaaGccqaH0oazdaqadaqaaiaadggadaqhaaWcbaGaaG
% OmaaqaaiaaikdaaaGccaWG1bWaaSbaaSqaaiaaikdaaeqaaOGaeqyW
% di3aaSbaaSqaaiaaikdacaGGSaGaamiEaaqabaGccqGHsislcaWGJb
% Waa0baaSqaaiaaigdaaeaacaaIYaaaaOGaamyDamaaBaaaleaacaaI
% Yaaabeaakiabeg8aYnaaBaaaleaacaaIXaGaaiilaiaadIhaaeqaaO
% GaeyOeI0IaamiCamaaBaaaleaacaaIXaGaaiilaiaadohadaWgaaad
% baGaaGymaaqabaaaleqaaOGaaiikaiaadwhadaWgaaWcbaGaaGOmaa
% qabaGccqGHsislcaWG1bWaaSbaaSqaaiaaigdaaeqaaOGaaiykaiaa
% dohadaWgaaWcbaGaaGymaiaacYcacaWG4baabeaaaOGaayjkaiaawM
% caaiabgUcaRiabeg8aYnaaBaaaleaacaaIYaaabeaakiaadwhadaWg
% aaWcbaGaaGOmaiaacYcacaWG4baabeaakiabg2da9iaaicdaaaaa!ECC0!
\begin{equation}
\begin{gathered}
  (1 + \frac{{\rho _1 c_1^2 \delta }}
{{\alpha _1 }})\rho _{1,t}  - \frac{{\rho _1 a_2^2 \delta }}
{{\alpha _1 }}\rho _{2,t}  + u_1 \rho _{1,x}  \hfill \\
   + \frac{{\rho _1 }}
{{\alpha _1 }}\delta \left( { - a_2^2 u_1 \rho _{2,x}  + c_1^2 u_1 \rho _{1,x}  + (u_2  - u_1 )p_{2,s_2 } s_{2,x} } \right) + \rho _1 u_{1,x}  = 0 \hfill \\
  (1 + \frac{{\rho _2 a_2^2 \delta }}
{{\alpha _2 }})\rho _{2,t}  - \frac{{\rho _2 c_1^2 \delta }}
{{\alpha _2 }}\rho _{1,t}  + u_2 \rho _{2,x}  \hfill \\
   + \frac{{\rho _2 }}
{{\alpha _2 }}\delta \left( {a_2^2 u_2 \rho _{2,x}  - c_1^2 u_2 \rho _{1,x}  - p_{1,s_1 } (u_2  - u_1 )s_{1,x} } \right) + \rho _2 u_{2,x}  = 0 \hfill \\
\end{gathered}
\end{equation}
% MathType!MTEF!2!1!+-
% feaafiart1ev1aaatCvAUfeBSjuyZL2yd9gzLbvyNv2CaerbuLwBLn
% hiov2DGi1BTfMBaeXatLxBI9gBaerbd9wDYLwzYbItLDharqqtubsr
% 4rNCHbGeaGqiVu0Je9sqqrpepC0xbbL8F4rqqrFfpeea0xe9Lq-Jc9
% vqaqpepm0xbba9pwe9Q8fs0-yqaqpepae9pg0FirpepeKkFr0xfr-x
% fr-xb9adbaqaaeGaciGaaiaabeqaamaabaabaaGceaqabeaacaGGOa
% GaaGymaiabgUcaRmaalaaabaGaeqyWdi3aaSbaaSqaaiaaigdaaeqa
% aOGaam4yamaaDaaaleaacaaIXaaabaGaaGOmaaaakiabes7aKbqaai
% abeg7aHnaaBaaaleaacaaIXaaabeaaaaGccaGGPaGaeqyWdi3aaSba
% aSqaaiaaigdacaGGSaGaamiDaaqabaGccqGHsisldaWcaaqaaiabeg
% 8aYnaaBaaaleaacaaIXaaabeaakiaadggadaqhaaWcbaGaaGOmaaqa
% aiaaikdaaaGccqaH0oazaeaacqaHXoqydaWgaaWcbaGaaGymaaqaba
% aaaOGaeqyWdi3aaSbaaSqaaiaaikdacaGGSaGaamiDaaqabaGccqGH
% RaWkcaGGOaGaaGymaiabgUcaRmaalaaabaGaeqyWdi3aaSbaaSqaai
% aaigdaaeqaaOGaam4yamaaDaaaleaacaaIXaaabaGaaGOmaaaakiab
% es7aKbqaaiabeg7aHnaaBaaaleaacaaIXaaabeaaaaGccaGGPaGaam
% yDamaaBaaaleaacaaIXaaabeaakiabeg8aYnaaBaaaleaacaaIXaGa
% aiilaiaadIhaaeqaaaGcbaGaey4kaSYaaSaaaeaacqaHbpGCdaWgaa
% WcbaGaaGymaaqabaaakeaacqaHXoqydaWgaaWcbaGaaGymaaqabaaa
% aOGaeqiTdq2aaeWaaeaacqGHsislcaWGHbWaa0baaSqaaiaaikdaae
% aacaaIYaaaaOGaamyDamaaBaaaleaacaaIXaaabeaakiabeg8aYnaa
% BaaaleaacaaIYaGaaiilaiaadIhaaeqaaOGaey4kaSIaamiCamaaBa
% aaleaacaaIYaGaaiilaiaadohadaWgaaadbaGaaGOmaaqabaaaleqa
% aOGaaiikaiaadwhadaWgaaWcbaGaaGOmaaqabaGccqGHsislcaWG1b
% WaaSbaaSqaaiaaigdaaeqaaOGaaiykaiaadohadaWgaaWcbaGaaGOm
% aiaacYcacaWG4baabeaaaOGaayjkaiaawMcaaiabgUcaRiabeg8aYn
% aaBaaaleaacaaIXaaabeaakiaadwhadaWgaaWcbaGaaGymaiaacYca
% caWG4baabeaakiabg2da9iaaicdaaeaacaGGOaGaaGymaiabgUcaRm
% aalaaabaGaeqyWdi3aaSbaaSqaaiaaikdaaeqaaOGaamyyamaaDaaa
% leaacaaIYaaabaGaaGOmaaaakiabes7aKbqaaiabeg7aHnaaBaaale
% aacaaIYaaabeaaaaGccaGGPaGaeqyWdi3aaSbaaSqaaiaaikdacaGG
% SaGaamiDaaqabaGccqGHsisldaWcaaqaaiabeg8aYnaaBaaaleaaca
% aIYaaabeaakiaadogadaqhaaWcbaGaaGymaaqaaiaaikdaaaGccqaH
% 0oazaeaacqaHXoqydaWgaaWcbaGaaGOmaaqabaaaaOGaeqyWdi3aaS
% baaSqaaiaaigdacaGGSaGaamiDaaqabaGccqGHRaWkcaGGOaGaaGym
% aiabgUcaRmaalaaabaGaeqyWdi3aaSbaaSqaaiaaikdaaeqaaOGaam
% yyamaaDaaaleaacaaIYaaabaGaaGOmaaaakiabes7aKbqaaiabeg7a
% HnaaBaaaleaacaaIYaaabeaaaaGccaGGPaGaamyDamaaBaaaleaaca
% aIYaaabeaakiabeg8aYnaaBaaaleaacaaIYaGaaiilaiaadIhaaeqa
% aaGcbaGaey4kaSYaaSaaaeaacqaHbpGCdaWgaaWcbaGaaGOmaaqaba
% aakeaacqaHXoqydaWgaaWcbaGaaGOmaaqabaaaaOGaeqiTdq2aaeWa
% aeaacqGHsislcaWGJbWaa0baaSqaaiaaigdaaeaacaaIYaaaaOGaam
% yDamaaBaaaleaacaaIYaaabeaakiabeg8aYnaaBaaaleaacaaIXaGa
% aiilaiaadIhaaeqaaOGaeyOeI0IaamiCamaaBaaaleaacaaIXaGaai
% ilaiaadohadaWgaaadbaGaaGymaaqabaaaleqaaOGaaiikaiaadwha
% daWgaaWcbaGaaGOmaaqabaGccqGHsislcaWG1bWaaSbaaSqaaiaaig
% daaeqaaOGaaiykaiaadohadaWgaaWcbaGaaGymaiaacYcacaWG4baa
% beaaaOGaayjkaiaawMcaaiabgUcaRiabeg8aYnaaBaaaleaacaaIYa
% aabeaakiaadwhadaWgaaWcbaGaaGOmaiaacYcacaWG4baabeaakiab
% g2da9iaaicdaaaaa!F324!
\begin{equation}
\begin{gathered}
  (1 + \frac{{\rho _1 c_1^2 \delta }}
{{\alpha _1 }})\rho _{1,t}  - \frac{{\rho _1 a_2^2 \delta }}
{{\alpha _1 }}\rho _{2,t}  + (1 + \frac{{\rho _1 c_1^2 \delta }}
{{\alpha _1 }})u_1 \rho _{1,x}  \hfill \\
   + \frac{{\rho _1 }}
{{\alpha _1 }}\delta \left( { - a_2^2 u_1 \rho _{2,x}  + p_{2,s_2 } (u_2  - u_1 )s_{2,x} } \right) + \rho _1 u_{1,x}  = 0 \hfill \\
  (1 + \frac{{\rho _2 a_2^2 \delta }}
{{\alpha _2 }})\rho _{2,t}  - \frac{{\rho _2 c_1^2 \delta }}
{{\alpha _2 }}\rho _{1,t}  + (1 + \frac{{\rho _2 a_2^2 \delta }}
{{\alpha _2 }})u_2 \rho _{2,x}  \hfill \\
   + \frac{{\rho _2 }}
{{\alpha _2 }}\delta \left( { - c_1^2 u_2 \rho _{1,x}  - p_{1,s_1 } (u_2  - u_1 )s_{1,x} } \right) + \rho _2 u_{2,x}  = 0 \hfill \\
\end{gathered}
\end{equation}

% MathType!MTEF!2!1!+-
% feaafiart1ev1aaatCvAUfeBSjuyZL2yd9gzLbvyNv2CaerbuLwBLn
% hiov2DGi1BTfMBaeXatLxBI9gBaerbd9wDYLwzYbItLDharqqtubsr
% 4rNCHbGeaGqiVu0Je9sqqrpepC0xbbL8F4rqqrFfpeea0xe9Lq-Jc9
% vqaqpepm0xbba9pwe9Q8fs0-yqaqpepae9pg0FirpepeKkFr0xfr-x
% fr-xb9adbaqaaeGaciGaaiaabeqaamaabaabaaGceaqabeaacaWG1b
% WaaSbaaSqaaiaaigdacaGGSaGaamiDaaqabaGccqGHRaWkcaWG1bWa
% aSbaaSqaaiaaigdaaeqaaOGaamyDamaaBaaaleaacaaIXaGaaiilai
% aadIhaaeqaaOGaey4kaSYaaSaaaeaacaWGJbWaa0baaSqaaiaaigda
% aeaacaaIYaaaaaGcbaGaeqyWdi3aaSbaaSqaaiaaigdaaeqaaaaaki
% abeg8aYnaaBaaaleaacaaIXaGaaiilaiaadIhaaeqaaOGaey4kaSYa
% aSaaaeaacaWGWbWaaSbaaSqaaiaaigdacaGGSaGaam4CamaaBaaame
% aacaaIXaaabeaaaSqabaaakeaacqaHbpGCdaWgaaWcbaGaaGymaaqa
% baaaaOGaam4CamaaBaaaleaacaaIXaGaaiilaiaadIhaaeqaaOGaey
% ypa0JaaGimaiaacYcaaeaacaWG1bWaaSbaaSqaaiaaikdacaGGSaGa
% amiDaaqabaGccqGHRaWkcaWG1bWaaSbaaSqaaiaaikdaaeqaaOGaam
% yDamaaBaaaleaacaaIYaGaaiilaiaadIhaaeqaaOGaey4kaSYaaSaa
% aeaacaaIXaaabaGaeqyWdi3aaSbaaSqaaiaaikdaaeqaaaaakiaacI
% cacaWGJbWaa0baaSqaaiaaikdaaeaacaaIYaaaaOGaey4kaSYaaSaa
% aeaacaWGWbWaaSbaaSqaaiaaikdaaeqaaOGaeyOeI0IaamiCamaaBa
% aaleaacaaIXaaabeaaaOqaaiabeg7aHnaaBaaaleaacaaIYaaabeaa
% aaGccqaH0oazcaWGHbWaa0baaSqaaiaaikdaaeaacaaIYaaaaOGaai
% ykaiabeg8aYnaaBaaaleaacaaIYaGaaiilaiaadIhaaeqaaOGaey4k
% aSYaaSaaaeaacaWGWbWaaSbaaSqaaiaaikdacaGGSaGaam4CamaaBa
% aameaacaaIYaaabeaaaSqabaaakeaacqaHbpGCdaWgaaWcbaGaaGOm
% aaqabaaaaOGaaiikaiaaigdacqGHRaWkdaWcaaqaaiaadchadaWgaa
% WcbaGaaGOmaaqabaGccqGHsislcaWGWbWaaSbaaSqaaiaaigdaaeqa
% aaGcbaGaeqySde2aaSbaaSqaaiaaikdaaeqaaaaakiabes7aKjaacM
% cacaWGZbWaaSbaaSqaaiaaikdacaGGSaGaamiEaaqabaaakeaacqGH
% RaWkdaWcaaqaaiaadchadaWgaaWcbaGaaGOmaaqabaGccqGHsislca
% WGWbWaaSbaaSqaaiaaigdaaeqaaaGcbaGaamyBamaaBaaaleaacaaI
% YaaabeaaaaGccqaH0oazdaqadaqaaiabgkHiTiaadogadaqhaaWcba
% GaaGymaaqaaiaaikdaaaGccqaHbpGCdaWgaaWcbaGaaGymaiaacYca
% caWG4baabeaakiabgkHiTiaadchadaWgaaWcbaGaaGymaiaacYcaca
% WGZbWaaSbaaWqaaiaaigdaaeqaaaWcbeaakiaadohadaWgaaWcbaGa
% aGymaiaacYcacaWG4baabeaaaOGaayjkaiaawMcaaiabg2da9iaaic
% daaaaa!B0E7!
\begin{equation}
\begin{gathered}
  u_{1,t}  + u_1 u_{1,x}  + \frac{{c_1^2 }}
{{\rho _1 }}\rho _{1,x}  + \frac{{p_{1,s_1 } }}
{{\rho _1 }}s_{1,x}  = 0, \hfill \\
  u_{2,t}  + u_2 u_{2,x}  + \frac{1}
{{\rho _2 }}(c_2^2  + \frac{{p_2  - p_1 }} {{\alpha _2 }}\delta
a_2^2 )\rho _{2,x}  + \frac{{p_{2,s_2 } }} {{\rho _2 }}(1 +
\frac{{p_2  - p_1 }}
{{\alpha _2 }}\delta )s_{2,x}  \hfill \\
   + \frac{{p_2  - p_1 }}
{{m_2 }}\delta \left( { - c_1^2 \rho _{1,x}  - p_{1,s_1 } s_{1,x} } \right) = 0 \hfill \\
\end{gathered}
\end{equation}
Finally, setting % MathType!MTEF!2!1!+-
% feaafiart1ev1aaatCvAUfeBSjuyZL2yd9gzLbvyNv2CaerbuLwBLn
% hiov2DGi1BTfMBaeXatLxBI9gBaerbd9wDYLwzYbItLDharqqtubsr
% 4rNCHbGeaGqiVu0Je9sqqrpepC0xbbL8F4rqqrFfpeea0xe9Lq-Jc9
% vqaqpepm0xbba9pwe9Q8fs0-yqaqpepae9pg0FirpepeKkFr0xfr-x
% fr-xb9adbaqaaeGaciGaaiaabeqaamaabaabaaGcbaGaeuiLdqKaey
% ypa0JaeqySde2aaSbaaSqaaiaaigdaaeqaaOGaeqySde2aaSbaaSqa
% aiaaikdaaeqaaOGaey4kaSIaeqiTdqMaaiikaiabeg7aHnaaBaaale
% aacaaIXaaabeaakiabeg8aYnaaBaaaleaacaaIYaaabeaakiaadgga
% daqhaaWcbaGaaGOmaaqaaiaaikdaaaGccqGHRaWkcqaHXoqydaWgaa
% WcbaGaaGOmaaqabaGccqaHbpGCdaWgaaWcbaGaaGymaaqabaGccaWG
% JbWaa0baaSqaaiaaigdaaeaacaaIYaaaaOGaaiykaiaacYcaaaa!529B!
\begin{equation}
\Delta  = \alpha _1 \alpha _2  + \delta (\alpha _1 \rho _2 a_2^2  +
\alpha _2 \rho _1 c_1^2 ),
\end{equation}
we find

\begin{center}
\begin{sideways}
\hbox{% MathType!MTEF!2!1!+-
% feaafiart1ev1aaatCvAUfeBSjuyZL2yd9gzLbvyNv2CaerbuLwBLn
% hiov2DGi1BTfMBaeXatLxBI9gBaerbd9wDYLwzYbItLDharqqtubsr
% 4rNCHbGeaGqiVu0Je9sqqrpepC0xbbL8F4rqqrFfpeea0xe9Lq-Jc9
% vqaqpepm0xbba9pwe9Q8fs0-yqaqpepae9pg0FirpepeKkFr0xfr-x
% fr-xb9adbaqaaeGaciGaaiaabeqaamaabaabaaGcbaGaam4qaiaacI
% cacaWGAbGaaiykaiabg2da9maadmaabaqbaeqabyGbaaaaaeaacaWG
% 1bWaaSbaaSqaaiaaigdaaeqaaOGaey4kaSYaaSaaaeaacqaHbpGCda
% WgaaWcbaGaaGymaaqabaGccqaHbpGCdaWgaaWcbaGaaGOmaaqabaGc
% caWGJbWaa0baaSqaaiaaigdaaeaacaaIYaaaaOGaamyyamaaDaaale
% aacaaIYaaabaGaaGOmaaaakiabes7aKnaaCaaaleqabaGaaGOmaaaa
% kiaacIcacaWG1bWaaSbaaSqaaiaaigdaaeqaaOGaeyOeI0IaamyDam
% aaBaaaleaacaaIYaaabeaakiaacMcaaeaacqqHuoaraaaabaWaaSaa
% aeaacqaHXoqydaWgaaWcbaGaaGymaaqabaGccqaHbpGCdaWgaaWcba
% GaaGymaaqabaGccaGGOaGaeqySde2aaSbaaSqaaiaaikdaaeqaaOGa
% ey4kaSIaeqyWdi3aaSbaaSqaaiaaikdaaeqaaOGaamyyamaaDaaale
% aacaaIYaaabaGaaGOmaaaakiabes7aKjaacMcaaeaacqqHuoaraaaa
% baWaaSaaaeaacqaHbpGCdaWgaaWcbaGaaGymaaqabaGccqaHbpGCda
% WgaaWcbaGaaGOmaaqabaGccaWGHbWaa0baaSqaaiaaikdaaeaacaaI
% YaaaaOGaeqiTdq2aaWbaaSqabeaacaaIYaaaaOGaaiikaiaadwhada
% WgaaWcbaGaaGymaaqabaGccqGHsislcaWG1bWaaSbaaSqaaiaaikda
% aeqaaOGaaiykaiaadchadaWgaaWcbaGaaGymaiaacYcacaWGZbWaaS
% baaWqaaiaaigdaaeqaaaWcbeaaaOqaaiabfs5aebaaaeaadaWcaaqa
% aiabeg8aYnaaBaaaleaacaaIXaaabeaakiaadggadaqhaaWcbaGaaG
% OmaaqaaiaaikdaaaGccqaH0oazcaGGOaGaeqySde2aaSbaaSqaaiaa
% ikdaaeqaaOGaey4kaSIaeqyWdi3aaSbaaSqaaiaaikdaaeqaaOGaam
% yyamaaDaaaleaacaaIYaaabaGaaGOmaaaakiabes7aKjaacMcacaGG
% OaGaamyDamaaBaaaleaacaaIYaaabeaakiabgkHiTiaadwhadaWgaa
% WcbaGaaGymaaqabaGccaGGPaaabaGaeuiLdqeaaaqaamaalaaabaGa
% eqySde2aaSbaaSqaaiaaikdaaeqaaOGaeqyWdi3aaSbaaSqaaiaaig
% daaeqaaOGaeqyWdi3aaSbaaSqaaiaaikdaaeqaaOGaamyyamaaDaaa
% leaacaaIYaaabaGaaGOmaaaakiabes7aKbqaaiabfs5aebaaaeaada
% Wcaaqaaiabeg8aYnaaBaaaleaacaaIXaaabeaakiabes7aKjaacIca
% cqaHXoqydaWgaaWcbaGaaGOmaaqabaGccqGHRaWkcqaHbpGCdaWgaa
% WcbaGaaGOmaaqabaGccaWGHbWaa0baaSqaaiaaikdaaeaacaaIYaaa
% aOGaeqiTdqMaaiykaiaacIcacaWG1bWaaSbaaSqaaiaaikdaaeqaaO
% GaeyOeI0IaamyDamaaBaaaleaacaaIXaaabeaakiaacMcacaWGWbWa
% aSbaaSqaaiaaikdacaGGSaGaam4CamaaBaaameaacaaIYaaabeaaaS
% qabaaakeaacqqHuoaraaaabaWaaSaaaeaacaWGJbWaa0baaSqaaiaa
% igdaaeaacaaIYaaaaaGcbaGaeqyWdi3aaSbaaSqaaiaaigdaaeqaaa
% aaaOqaaiaadwhadaWgaaWcbaGaaGymaaqabaaakeaadaWcaaqaaiaa
% dchadaWgaaWcbaGaaGymaiaacYcacaWGZbWaaSbaaWqaaiaaigdaae
% qaaaWcbeaaaOqaaiabeg8aYnaaBaaaleaacaaIXaaabeaaaaaakeaa
% caaIWaaabaGaaGimaaqaaiaaicdaaeaacaaIWaaabaGaaGimaaqaai
% aadwhadaWgaaWcbaGaaGymaaqabaaakeaacaaIWaaabaGaaGimaaqa
% aiaaicdaaeaadaWcaaqaaiabeg8aYnaaBaaaleaacaaIYaaabeaaki
% aadogadaqhaaWcbaGaaGymaaqaaiaaikdaaaGccqaH0oazcaGGOaGa
% eqySde2aaSbaaSqaaiaaigdaaeqaaOGaey4kaSIaeqyWdi3aaSbaaS
% qaaiaaigdaaeqaaOGaam4yamaaDaaaleaacaaIXaaabaGaaGOmaaaa
% kiabes7aKjaacMcacaGGOaGaamyDamaaBaaaleaacaaIXaaabeaaki
% abgkHiTiaadwhadaWgaaWcbaGaaGOmaaqabaGccaGGPaaabaGaeuiL
% dqeaaaqaamaalaaabaGaeqySde2aaSbaaSqaaiaaigdaaeqaaOGaeq
% yWdi3aaSbaaSqaaiaaigdaaeqaaOGaeqyWdi3aaSbaaSqaaiaaikda
% aeqaaOGaam4yamaaDaaaleaacaaIXaaabaGaaGOmaaaakiabes7aKb
% qaaiabfs5aebaaaeaadaWcaaqaaiabeg8aYnaaBaaaleaacaaIYaaa
% beaakiabes7aKjaacIcacqaHXoqydaWgaaWcbaGaaGymaaqabaGccq
% GHRaWkcqaHbpGCdaWgaaWcbaGaaGymaaqabaGccaWGJbWaa0baaSqa
% aiaaigdaaeaacaaIYaaaaOGaeqiTdqMaaiykaiaacIcacaWG1bWaaS
% baaSqaaiaaigdaaeqaaOGaeyOeI0IaamyDamaaBaaaleaacaaIYaaa
% beaakiaacMcacaWGWbWaaSbaaSqaaiaaigdacaGGSaGaam4CamaaBa
% aameaacaaIXaaabeaaaSqabaaakeaacqqHuoaraaaabaGaamyDamaa
% BaaaleaacaaIYaaabeaakiabgUcaRmaalaaabaGaeqyWdi3aaSbaaS
% qaaiaaigdaaeqaaOGaeqyWdi3aaSbaaSqaaiaaikdaaeqaaOGaam4y
% amaaDaaaleaacaaIXaaabaGaaGOmaaaakiaadggadaqhaaWcbaGaaG
% OmaaqaaiaaikdaaaGccqaH0oazdaahaaWcbeqaaiaaikdaaaGccaGG
% OaGaamyDamaaBaaaleaacaaIYaaabeaakiabgkHiTiaadwhadaWgaa
% WcbaGaaGymaaqabaGccaGGPaaabaGaeuiLdqeaaaqaamaalaaabaGa
% eqySde2aaSbaaSqaaiaaikdaaeqaaOGaeqyWdi3aaSbaaSqaaiaaik
% daaeqaaOGaaiikaiabeg7aHnaaBaaaleaacaaIXaaabeaakiabgUca
% Riabeg8aYnaaBaaaleaacaaIXaaabeaakiaadogadaqhaaWcbaGaaG
% ymaaqaaiaaikdaaaGccqaH0oazcaGGPaaabaGaeuiLdqeaaaqaamaa
% laaabaGaeqyWdi3aaSbaaSqaaiaaigdaaeqaaOGaeqyWdi3aaSbaaS
% qaaiaaikdaaeqaaOGaam4yamaaDaaaleaacaaIXaaabaGaaGOmaaaa
% kiabes7aKnaaCaaaleqabaGaaGOmaaaakiaacIcacaWG1bWaaSbaaS
% qaaiaaikdaaeqaaOGaeyOeI0IaamyDamaaBaaaleaacaaIXaaabeaa
% kiaacMcacaWGWbWaaSbaaSqaaiaaikdacaGGSaGaam4CamaaBaaame
% aacaaIYaaabeaaaSqabaaakeaacqqHuoaraaaabaWaaSaaaeaacaGG
% OaGaamiCamaaBaaaleaacaaIXaaabeaakiabgkHiTiaadchadaWgaa
% WcbaGaaGOmaaqabaGccaGGPaGaeqiTdqMaam4yamaaDaaaleaacaaI
% XaaabaGaaGOmaaaaaOqaaiabeg7aHnaaBaaaleaacaaIYaaabeaaki
% abeg8aYnaaBaaaleaacaaIYaaabeaaaaaakeaacaaIWaaabaWaaSaa
% aeaacaGGOaGaamiCamaaBaaaleaacaaIXaaabeaakiabgkHiTiaadc
% hadaWgaaWcbaGaaGOmaaqabaGccaGGPaGaeqiTdqMaamiCamaaBaaa
% leaacaaIXaGaaiilaiaadohadaWgaaadbaGaaGymaaqabaaaleqaaa
% GcbaGaeqySde2aaSbaaSqaaiaaikdaaeqaaOGaeqyWdi3aaSbaaSqa
% aiaaikdaaeqaaaaaaOqaamaalaaabaGaeqiTdqMaaiikaiaadchada
% WgaaWcbaGaaGOmaaqabaGccqGHsislcaWGWbWaaSbaaSqaaiaaigda
% aeqaaOGaaiykaiabgUcaRiabeg7aHnaaBaaaleaacaaIYaaabeaaki
% aadogadaqhaaWcbaGaaGOmaaqaaiaaikdaaaaakeaacqaHXoqydaWg
% aaWcbaGaaGOmaaqabaGccqaHbpGCdaWgaaWcbaGaaGOmaaqabaaaaa
% GcbaGaamyDamaaBaaaleaacaaIYaaabeaaaOqaamaalaaabaGaaiik
% aiabeg7aHnaaBaaaleaacaaIYaaabeaakiabgUcaRiabes7aKjaacI
% cacaWGWbWaaSbaaSqaaiaaikdaaeqaaOGaeyOeI0IaamiCamaaBaaa
% leaacaaIXaaabeaakiaacMcacaGGPaGaamiCamaaBaaaleaacaaIYa
% GaaiilaiaadohadaWgaaadbaGaaGOmaaqabaaaleqaaaGcbaGaeqyS
% de2aaSbaaSqaaiaaikdaaeqaaOGaeqyWdi3aaSbaaSqaaiaaikdaae
% qaaaaaaOqaaiaaicdaaeaacaaIWaaabaGaaGimaaqaaiaaicdaaeaa
% caaIWaaabaGaamyDamaaBaaaleaacaaIYaaabeaaaaaakiaawUfaca
% GLDbaaaaa!AE5E!
$ C(Z) = \left[ {\begin{array}{*{20}c}
   {u_1  + \frac{{\rho _1 \rho _2 c_1^2 a_2^2 \delta ^2 (u_1  - u_2 )}}
{\Delta }} & {\frac{{\alpha _1 \rho _1 (\alpha _2  + \rho _2 a_2^2
\delta )}} {\Delta }} & {\frac{{\rho _1 \rho _2 a_2^2 \delta ^2 (u_1
- u_2 )p_{1,s_1 } }} {\Delta }} & {\frac{{\rho _1 a_2^2 \delta
(\alpha _2  + \rho _2 a_2^2 \delta )(u_2  - u_1 )}} {\Delta }} &
{\frac{{\alpha _2 \rho _1 \rho _2 a_2^2 \delta }} {\Delta }} &
{\frac{{\rho _1 \delta (\alpha _2  + \rho _2 a_2^2 \delta )(u_2  -
u_1 )p_{2,s_2 } }}
{\Delta }}  \\
   {\frac{{c_1^2 }}
{{\rho _1 }}} & {u_1 } & {\frac{{p_{1,s_1 } }}
{{\rho _1 }}} & 0 & 0 & 0  \\
   0 & 0 & {u_1 } & 0 & 0 & 0  \\
   {\frac{{\rho _2 c_1^2 \delta (\alpha _1  + \rho _1 c_1^2 \delta )(u_1  - u_2 )}}
{\Delta }} & {\frac{{\alpha _1 \rho _1 \rho _2 c_1^2 \delta }}
{\Delta }} & {\frac{{\rho _2 \delta (\alpha _1  + \rho _1 c_1^2
\delta )(u_1  - u_2 )p_{1,s_1 } }} {\Delta }} & {u_2  + \frac{{\rho
_1 \rho _2 c_1^2 a_2^2 \delta ^2 (u_2  - u_1 )}} {\Delta }} &
{\frac{{\alpha _2 \rho _2 (\alpha _1  + \rho _1 c_1^2 \delta )}}
{\Delta }} & {\frac{{\rho _1 \rho _2 c_1^2 \delta ^2 (u_2  - u_1
)p_{2,s_2 } }}
{\Delta }}  \\
   {\frac{{(p_1  - p_2 )\delta c_1^2 }}
{{\alpha _2 \rho _2 }}} & 0 & {\frac{{(p_1  - p_2 )\delta p_{1,s_1 }
}} {{\alpha _2 \rho _2 }}} & {\frac{{\delta (p_2  - p_1 ) + \alpha
_2 c_2^2 }} {{\alpha _2 \rho _2 }}} & {u_2 } & {\frac{{(\alpha _2  +
\delta (p_2  - p_1 ))p_{2,s_2 } }}
{{\alpha _2 \rho _2 }}}  \\
   0 & 0 & 0 & 0 & 0 & {u_2 }  \\

 \end{array} } \right]
$

}
\end{sideways}
\end{center}
It is not easy to compute the eigenvalues analytically. It is also
difficult to give a practical sufficient condition on all the
parameters in order to prove that the eigenvalues are all real. In
the case $\delta=0$, corresponding to an infinite granular
stress, the characteristic polynomial is% MathType!MTEF!2!1!+-
% feaafiart1ev1aaatCvAUfeBSjuyZL2yd9gzLbvyNv2CaerbuLwBLn
% hiov2DGi1BTfMBaeXatLxBI9gBaerbd9wDYLwzYbItLDharqqtubsr
% 4rNCHbGeaGqiVu0Je9sqqrpepC0xbbL8F4rqqrFfpeea0xe9Lq-Jc9
% vqaqpepm0xbba9pwe9Q8fs0-yqaqpepae9pg0FirpepeKkFr0xfr-x
% fr-xb9adbaqaaeGaciGaaiaabeqaamaabaabaaGcbaGaamiuaiaacI
% cacqaH7oaBcaGGPaGaeyypa0JaaiikaiaadwhadaWgaaWcbaGaaGOm
% aaqabaGccqGHsislcqaH7oaBcaGGPaGaaiikaiaadwhadaWgaaWcba
% GaaGymaaqabaGccqGHsislcqaH7oaBcaGGPaGaaiikaiaadwhadaWg
% aaWcbaGaaGymaaqabaGccqGHsislcaWGJbWaaSbaaSqaaiaaigdaae
% qaaOGaeyOeI0Iaeq4UdWMaaiykaiaacIcacaWG1bWaaSbaaSqaaiaa
% igdaaeqaaOGaey4kaSIaam4yamaaBaaaleaacaaIXaaabeaakiabgk
% HiTiabeU7aSjaacMcacaGGOaGaamyDamaaBaaaleaacaaIYaaabeaa
% kiabgkHiTiaadogadaWgaaWcbaGaaGOmaaqabaGccqGHsislcqaH7o
% aBcaGGPaGaaiikaiaadwhadaWgaaWcbaGaaGOmaaqabaGccqGHRaWk
% caWGJbWaaSbaaSqaaiaaikdaaeqaaOGaeyOeI0Iaeq4UdWMaaiykaa
% aa!693A!
\begin{equation}
P(\lambda ) = (u_2  - \lambda )(u_1  - \lambda )(u_1  - c_1  -
\lambda )(u_1  + c_1  - \lambda )(u_2  - c_2  - \lambda )(u_2  + c_2
- \lambda )
\end{equation}
We recover the same eigenvalues as in (\ref{polB}).

With a small $\kappa$, which corresponds to a big $\delta$ as can
be seen by formula (\ref{phy}), we observe numerically that the
system is elliptic when $u_1 \neq u_2$. When $\kappa$ increases,
$\delta$ decreases and we recover a hyperbolic behavior.

{\bf Numerical application}: we take% MathType!MTEF!2!1!+-
% feaafiart1ev1aaatCvAUfeBSjuyZL2yd9gzLbvyNv2CaerbuLwBLn
% hiov2DGi1BTfMBaeXatLxBI9gBaerbd9wDYLwzYbItLDharqqtubsr
% 4rNCHbGeaGqiVu0Je9sqqrpepC0xbbL8F4rqqrFfpeea0xe9Lq-Jc9
% vqaqpepm0xbba9pwe9Q8fs0-yqaqpepae9pg0FirpepeKkFr0xfr-x
% fr-xb9adbaqaaeGaciGaaiaabeqaamaabaabaaGceaqabeaacqaHZo
% WzdaWgaaWcbaGaaGymaaqabaGccqGH9aqpcaqGXaGaaeOlaiaabcda
% caqG5aGaaeOmaiaabsdaaeaacqaHZoWzdaWgaaWcbaGaaeOmaaqaba
% GccqGH9aqpcaqGXaGaaeOlaiaabcdacaqGXaGaaeioaiaabkdaaeaa
% cqaHapaCdaWgaaWcbaGaaeymaaqabaGccqGH9aqpcqaHapaCdaWgaa
% WcbaGaaGOmaaqabaGccqGH9aqpcaaIWaaabaGaeqySde2aaSbaaSqa
% aiaabgdaaeqaaOGaeyypa0JaaGimaiaac6cacaaIYaGaaGynaaqaai
% aadchadaWgaaWcbaGaaGymaaqabaGccqGH9aqpcaaIWaGaaiOlaiaa
% ikdacqGHxdaTcaaIXaGaaGimamaaCaaaleqabaGaaGioaaaaaOqaai
% abeU7aSjabg2da9iaaicdacaGGUaGaaGimaiaaigdacqGHshI3caWG
% WbWaaSbaaSqaaiaaikdaaeqaaOGaeyypa0JaaGimaiaac6cacaqGYa
% GaaeimaiaabcdacaqGWaGaaeimaiaabcdacaqGWaGaae4naiaabcda
% caqG1aGaaeimaiaabIdacaqG4aGaaeymaiabgEna0kaaigdacaaIWa
% WaaWbaaSqabeaacaaI4aaaaaGcbaGaamyDamaaBaaaleaacaaIYaaa
% beaakiabg2da9iabgkHiTiaadwhadaWgaaWcbaGaaGymaaqabaGccq
% GH9aqpcaaI1aGaaGimaaqaaiabeg8aYnaaBaaaleaacaaIXaaabeaa
% kiabg2da9iaabEdacaqG2aGaaeOlaiaabsdacaqG1aGaaeinaiaabo
% dacaqGWaGaaeimaiaabMdacaqGZaaabaGaeqyWdi3aaSbaaSqaaiaa
% bkdaaeqaaOGaeyypa0JaaeioaiaabodacaqG2aGaaeOlaiaabgdaca
% qGYaGaae4maiaabMdacaqG3aGaaeymaiaabIdaaaaa!97FF!
\begin{equation}
\begin{gathered}
  \gamma _1  = {\text{1}}{\text{.0924}} \hfill \\
  \gamma _{\text{2}}  = {\text{1}}{\text{.0182}} \hfill \\
  \pi _{\text{1}}  = \pi _2  = 0 \hfill \\
  \alpha _{\text{1}}  = 0.25 \hfill \\
  p_1  = 0.2 \times 10^8  \hfill \\
  \kappa  = 0.01 \Rightarrow p_2  = 0.{\text{20000007050881}} \times 10^8  \hfill \\
  u_2  =  - u_1  = 50 \hfill \\
  \rho _1  = {\text{76}}{\text{.45430093}} \hfill \\
  \rho _{\text{2}}  = {\text{836}}{\text{.1239718}} \hfill \\
\end{gathered}
\end{equation}
The eigenvalues are % MathType!MTEF!2!1!+-
% feaafiart1ev1aaatCvAUfeBSjuyZL2yd9gzLbvyNv2CaerbuLwBLn
% hiov2DGi1BTfMBaeXatLxBI9gBaerbd9wDYLwzYbItLDharqqtubsr
% 4rNCHbGeaGqiVu0Je9sqqrpepC0xbbL8F4rqqrFfpeea0xe9Lq-Jc9
% vqaqpepm0xbba9pwe9Q8fs0-yqaqpepae9pg0FirpepeKkFr0xfr-x
% fr-xb9adbaqaaeGaciGaaiaabeqaamaabaabaaGceaqabeaacaGGTa
% GaaG4maiaaigdacaaIWaGaaiOlaiaaiEdacaaI5aaabaGaaiylaiaa
% iwdacaaIWaGaaiOlaiaaicdacaaIWaaabaGaaGinaiaaiIdacaGGUa
% GaaGyoaiaaiAdacaGGTaGaaGyoaiaac6cacaaIZaGaaGOnaiaadMga
% aeaacaaI0aGaaGioaiaac6cacaaI5aGaaGOnaiabgUcaRiaaiMdaca
% GGUaGaaG4maiaaiAdacaWGPbaabaGaaGynaiaaicdaaeaacaaIYaGa
% aGymaiaaikdacaGGUaGaaGioaiaaiAdaaaaa!55FE!
\begin{equation}
\begin{gathered}
   - 310.79 \hfill \\
   - 50.00 \hfill \\
  48.96 - 9.36i \hfill \\
  48.96 + 9.36i \hfill \\
  50 \hfill \\
  212.86 \hfill \\
\end{gathered}
\end{equation}
We modify $\kappa$ to $\kappa=500$, the pressure $p_2$ is now
$p_2=0.20352544\times 10^8$. The eigenvalues become real% MathType!MTEF!2!1!+-
% feaafiart1ev1aaatCvAUfeBSjuyZL2yd9gzLbvyNv2CaerbuLwBLn
% hiov2DGi1BTfMBaeXatLxBI9gBaerbd9wDYLwzYbItLDharqqtubsr
% 4rNCHbGeaGqiVu0Je9sqqrpepC0xbbL8F4rqqrFfpeea0xe9Lq-Jc9
% vqaqpepm0xbba9pwe9Q8fs0-yqaqpepae9pg0FirpepeKkFr0xfr-x
% fr-xb9adbaqaaeGaciGaaiaabeqaamaabaabaaGceaqabeaacaGGTa
% GaaG4maiaaigdacaaIYaGaaiOlaiaaiwdacaaI0aaabaGaaiylaiaa
% iwdacaaIWaGaaiOlaiaaicdacaaIWaGaaGimaaqaaiaaiodacaaIWa
% GaaiOlaiaaiwdacaaIWaGaaG4naaqaaiaaiwdacaaIWaGaaiOlaiaa
% icdacaaIWaGaaGimaaqaaiaaiAdacaaI3aGaaiOlaiaaisdacaaIZa
% GaaGioaaqaaiaaikdacaaIXaGaaGinaiaac6cacaaI1aGaaGyoaaaa
% aa!51AB!
\begin{equation}
\begin{gathered}
   - 312.54 \hfill \\
   - 50.000 \hfill \\
  30.507 \hfill \\
  50.000 \hfill \\
  67.438 \hfill \\
  214.59 \hfill \\
\end{gathered}
\end{equation}



\section{Appendix II: Associated entropy \label{assentrop}}
 It is also possible to compute an
entropy associated to the choice (\ref{nice}). For this, we
postulate the following form of the entropy of the solid phase (as
in Section \ref{num}, we omit the subscript)
% MathType!MTEF!2!1!+-
% feaafiart1ev1aaatCvAUfeBSjuyZL2yd9gzLbvyNv2CaerbuLwBLn
% hiov2DGi1BTfMBaeXatLxBI9gBaerbd9wDYLwzYbItLDharqqtubsr
% 4rNCHbGeaGqiVu0Je9sqqrpepC0xbbL8F4rqqrFfpeea0xe9Lq-Jc9
% vqaqpepm0xbba9pwe9Q8fs0-yqaqpepae9pg0FirpepeKkFr0xfr-x
% fr-xb9adbaqaaeGaciGaaiaabeqaamaabaabaaGcbaGaam4Caiabg2
% da9iaadUeacaGGOaGaeqySdeMaaiykaiaadwfacaGGOaGaaiikaiaa
% dwgacqGHsislcqaHapaCcqaHepaDcaGGPaGaeqiXdq3aaWbaaSqabe
% aacqaHZoWzcqGHsislcaaIXaaaaOGaaiykaiaac6caaaa!4A95!
\begin{equation}
s = K(\alpha )U((e - \pi \tau )\tau ^{\gamma  - 1} ).
\end{equation}

This choice is justified by the fact that when $K$ is constant, then
we recover the general entropy of a stiffened gas. We can verify
that our entropy and our stiffened gas law are compatible. Without
the subscripts, the equation (\ref{tds}) reads
% MathType!MTEF!2!1!+-
% feaafiart1ev1aaatCvAUfeBSjuyZL2yd9gzLbvyNv2CaerbuLwBLn
% hiov2DGi1BTfMBaeXatLxBI9gBaebbnrfifHhDYfgasaacH8srps0l
% bbf9q8WrFfeuY-Hhbbf9v8qqaqFr0xc9pk0xbba9q8WqFfea0-yr0R
% Yxir-Jbba9q8aq0-yq-He9q8qqQ8frFve9Fve9Ff0dmeaabaqaciGa
% caGaaeqabaqabeaadaaakeaacaWGubGaaGPaVlaadsgacaWGZbGaey
% ypa0JaamizaiaadwgacqGHRaWkcaWGWbGaaGPaVlaadsgacqaHepaD
% cqGHsislcqqHyoqucaaMc8Uaamizaiabeg7aHjaac6caaaa!4609!
\begin{equation}\label{tds2}
T\,ds = de + p\,d\tau  - \Theta \,d\alpha .
\end{equation}

The temperature is given by
% MathType!MTEF!2!1!+-
% feaafiart1ev1aaatCvAUfeBSjuyZL2yd9gzLbvyNv2CaerbuLwBLn
% hiov2DGi1BTfMBaeXatLxBI9gBaebbnrfifHhDYfgasaacH8srps0l
% bbf9q8WrFfeuY-Hhbbf9v8qqaqFr0xc9pk0xbba9q8WqFfea0-yr0R
% Yxir-Jbba9q8aq0-yq-He9q8qqQ8frFve9Fve9Ff0dmeaabaqaciGa
% caGaaeqabaqabeaadaaakeaacaWGZbWaaSbaaSqaaiaadwgaaeqaaO
% Gaeyypa0ZaaSaaaeaacaaIXaaabaGaamivaaaacqGH9aqpcaWGlbWa
% aeWaceaacqaHXoqyaiaawIcacaGLPaaacaaMc8UaeqiXdq3aaWbaaS
% qabeaacqaHZoWzcqGHsislcaaIXaaaaOGaaGPaVlaadwfacaGGNaWa
% aeWaceaadaqadiqaaiaadwgacaaMc8UaeyOeI0IaeqiWdaNaeqiXdq
% hacaGLOaGaayzkaaGaeqiXdq3aaWbaaSqabeaacqaHZoWzcqGHsisl
% caaIXaaaaaGccaGLOaGaayzkaaaaaa!549F!
\begin{equation}
s_e  = \frac{1}
{T} = K\left( \alpha  \right)\,\tau ^{\gamma  - 1} \,U'\left( {\left( {e\, - \pi \tau } \right)\tau ^{\gamma  - 1} } \right)
\end{equation}
In a similar way, we can deduce a relationship between $p$ and $T$
 % MathType!MTEF!2!1!+-
% feaafiart1ev1aaatCvAUfeBSjuyZL2yd9gzLbvyNv2CaerbuLwBLn
% hiov2DGi1BTfMBaeXatLxBI9gBaebbnrfifHhDYfgasaacH8srps0l
% bbf9q8WrFfeuY-Hhbbf9v8qqaqFr0xc9pk0xbba9q8WqFfea0-yr0R
% Yxir-Jbba9q8aq0-yq-He9q8qqQ8frFve9Fve9Ff0dmeaabaqaciGa
% caGaaeqabaqabeaadaaakeaacaWGZbWaaSbaaSqaaiabes8a0bqaba
% GccqGH9aqpdaWcaaqaaiaadchaaeaacaWGubaaaiabg2da9iaadcha
% caaMc8Uaam4CamaaBaaaleaacaWGLbaabeaaaaa!3D1D!
\begin{equation}
    s_\tau   = \frac{p}
    {T} = p\,s_e.
\end{equation}
This relation enables to compute the pressure
% MathType!MTEF!2!1!+-
% feaafiart1ev1aaatCvAUfeBSjuyZL2yd9gzLbvyNv2CaerbuLwBLn
% hiov2DGi1BTfMBaeXatLxBI9gBaebbnrfifHhDYfgasaacH8srps0l
% bbf9q8WrFfeuY-Hhbbf9v8qqaqFr0xc9pk0xbba9q8WqFfea0-yr0R
% Yxir-Jbba9q8aq0-yq-He9q8qqQ8frFve9Fve9Ff0dmeaabaqaciGa
% caGaaeqabaqabeaadaaakqaabeqaaiaadchacqGH9aqpdaWcaaqaai
% aadohadaWgaaWcbaGaeqiXdqhabeaaaOqaaiaadohadaWgaaWcbaGa
% amyzaaqabaaaaaGcbaGaeyypa0ZaaSaaaeaacaWGlbWaaeWaceaacq
% aHXoqyaiaawIcacaGLPaaacaaMc8+aaeWaceaadaqadiqaaiabeo7a
% NjabgkHiTiaaigdaaiaawIcacaGLPaaacaWGLbGaaGPaVlabes8a0n
% aaCaaaleqabaGaeq4SdCMaeyOeI0IaaGOmaaaakiabgkHiTiabeo7a
% NjaaykW7cqaHapaCcaaMc8UaeqiXdq3aaWbaaSqabeaacqaHZoWzcq
% GHsislcaaIXaaaaaGccaGLOaGaayzkaaGaaGPaVlaadwfacaGGNaWa
% aeWaceaadaqadiqaaiaadwgacaaMc8UaeyOeI0IaeqiWdaNaeqiXdq
% hacaGLOaGaayzkaaGaeqiXdq3aaWbaaSqabeaacqaHZoWzcqGHsisl
% caaIXaaaaaGccaGLOaGaayzkaaaabaGaam4samaabmGabaGaeqySde
% gacaGLOaGaayzkaaGaaGPaVlabes8a0naaCaaaleqabaGaeq4SdCMa
% eyOeI0IaaGymaaaakiaaykW7caWGvbGaai4jamaabmGabaWaaeWace
% aacaWGLbGaaGPaVlabgkHiTiabec8aWjabes8a0bGaayjkaiaawMca
% aiabes8a0naaCaaaleqabaGaeq4SdCMaeyOeI0IaaGymaaaaaOGaay
% jkaiaawMcaaaaaaeaacqGH9aqpdaqadiqaaiabeo7aNjabgkHiTiaa
% igdaaiaawIcacaGLPaaacaaMc8+aaSaaaeaacaWGLbaabaGaeqiXdq
% haaiabgkHiTiabeo7aNjaaykW7cqaHapaCaaaa!99C3!
\begin{equation}
\begin{gathered}
  p = \frac{{s_\tau  }}
{{s_e }} \hfill \\
   = \frac{{K\left( \alpha  \right)\,\left( {\left( {\gamma  - 1} \right)e\,\tau ^{\gamma  - 2}  - \gamma \,\pi \,\tau ^{\gamma  - 1} } \right)\,U'\left( {\left( {e\, - \pi \tau } \right)\tau ^{\gamma  - 1} } \right)}}
{{K\left( \alpha  \right)\,\tau ^{\gamma  - 1} \,U'\left( {\left( {e\, - \pi \tau } \right)\tau ^{\gamma  - 1} } \right)}} \hfill \\
   = \left( {\gamma  - 1} \right)\,\frac{e}
{\tau } - \gamma \,\pi  \hfill \\
\end{gathered}
\end{equation}
and we indeed recover the stiffened gas equation of state.







We try now to find an expression for the function $U \left( x
\right)$. From (\ref{tds2}) we can write
% MathType!MTEF!2!1!+-
% feaafiart1ev1aaatCvAUfeBSjuyZL2yd9gzLbvyNv2CaerbuLwBLn
% hiov2DGi1BTfMBaeXatLxBI9gBaebbnrfifHhDYfgasaacH8srps0l
% bbf9q8WrFfeuY-Hhbbf9v8qqaqFr0xc9pk0xbba9q8WqFfea0-yr0R
% Yxir-Jbba9q8aq0-yq-He9q8qqQ8frFve9Fve9Ff0dmeaabaqaciGa
% caGaaeqabaqabeaadaaakeaacaWGZbWaaSbaaSqaaiabeg7aHbqaba
% GccqGH9aqpcqGHsisldaWcaaqaaiabfI5arbqaaiaadsfaaaaaaa!38D2!
\begin{equation}
s_\alpha   =  - \frac{\Theta } {T}
\end{equation}

and thus
% MathType!MTEF!2!1!+-
% feaafiart1ev1aaatCvAUfeBSjuyZL2yd9gzLbvyNv2CaerbuLwBLn
% hiov2DGi1BTfMBaeXatLxBI9gBaebbnrfifHhDYfgasaacH8srps0l
% bbf9q8WrFfeuY-Hhbbf9v8qqaqFr0xc9pk0xbba9q8WqFfea0-yr0R
% Yxir-Jbba9q8aq0-yq-He9q8qqQ8frFve9Fve9Ff0dmeaabaqaciGa
% caGaaeqabaqabeaadaaakeaacqqHyoqucqGH9aqpcqGHsisldaWcaa
% qaaiaadohadaWgaaWcbaGaeqySdegabeaaaOqaaiaadohadaWgaaWc
% baGaamyzaaqabaaaaOGaeyypa0JaeyOeI0YaaSaaaeaaceWGlbGbau
% aadaqadiqaaiabeg7aHbGaayjkaiaawMcaaaqaaiaadUeadaqadiqa
% aiabeg7aHbGaayjkaiaawMcaaaaadaWcaaqaaiaadwfadaqadiqaam
% aabmGabaGaamyzaiaaykW7cqGHsislcqaHapaCcqaHepaDaiaawIca
% caGLPaaacqaHepaDdaahaaWcbeqaaiabeo7aNjabgkHiTiaaigdaaa
% aakiaawIcacaGLPaaaaeaacaWGvbGaai4jamaabmGabaWaaeWaceaa
% caWGLbGaaGPaVlabgkHiTiabec8aWjabes8a0bGaayjkaiaawMcaai
% abes8a0naaCaaaleqabaGaeq4SdCMaeyOeI0IaaGymaaaaaOGaayjk
% aiaawMcaaaaacqaHbpGCdaahaaWcbeqaaiabeo7aNjabgkHiTiaaig
% daaaaaaa!6A49!
\begin{equation}
\Theta  =  - \frac{{s_\alpha  }}
{{s_e }} =  - \frac{{K'\left( \alpha  \right)}}
{{K\left( \alpha  \right)}}\frac{{U\left( {\left( {e\, - \pi \tau } \right)\tau ^{\gamma  - 1} } \right)}}
{{U'\left( {\left( {e\, - \pi \tau } \right)\tau ^{\gamma  - 1} } \right)}}\rho ^{\gamma  - 1}
\end{equation}


But $\Theta$ has also to be of the form (\ref{genegranu}). It implies
that% MathType!MTEF!2!1!+-
% faaafaart1ev1aaat0uyJj1BTfMBaerbuLwBLnhiov2DGi1BTfMBae
% XatLxBI9gBaebbnrfifHhDYfgasaacH8srps0lbbf9q8WrFfeuY-Hh
% bbf9v8qqaqFr0xc9pk0xbba9q8WqFfea0-yr0RYxir-Jbba9q8aq0-
% yq-He9q8qqQ8frFve9Fve9Ff0dmeaabaqaciGacaGaaeqabaqabeaa
% daaakqaabeqaaiaadUeacaGGOaGaeqySdeMaaiykaiabg2da9iaadk
% eaciGGLbGaaiiEaiaacchadaqadaqaamaapedabaGaeqiUdeNaaiik
% aiaadwhacaGGPaGaamizaiaadwhaaSqaaiaaicdaaeaacqaHXoqya0
% Gaey4kIipaaOGaayjkaiaawMcaaiaacYcaaeaacaWGvbGaaiikaiaa
% dIhacaGGPaGaeyypa0JaeyOeI0IaamyqaiGacwgacaGG4bGaaiiCai
% aacIcacqGHsislcaWGcbGaamiEaiaacMcacaGGUaaaaaa!4FD1!
\begin{equation}\label{}
\begin{gathered}
  K(\alpha ) = B\exp \left( {\int_0^\alpha  {\theta (u)du} } \right), \hfill \\
  U(x) =  - A\exp ( - Bx). \hfill \\
\end{gathered}
\end{equation}



We choose now the sign of the constants $A$ and $B$ in such way that the
temperature is positive and that the function $U$ is concave. It
implies that $A$ and $B$ are $>0$. The positivity of $T$ and the concavity of $U$ are required for obtaining real sound speeds in the pure phases (see \cite{raviart96}).



\bibliography{granu}




\end{document}
\section{Thermodynamics of stiffened gases}

We define the sound speed $c_k$ in phase k by
% MathType!MTEF!2!1!+-
% feaafiart1ev1aaatCvAUfeBSjuyZL2yd9gzLbvyNv2CaerbuLwBLn
% hiov2DGi1BTfMBaeXatLxBI9gBaerbd9wDYLwzYbItLDharqqtubsr
% 4rNCHbGeaGqiVu0Je9sqqrpepC0xbbL8qbVhbbf9v8qqaqFr0xc9pk
% 0xbba9q8WqFfea0-yr0RYxir-Jbba9q8aq0-yq-He9q8qqQ8frFve9
% Fve9Ff0dmeaabaqaciGacaGaaeqabaWaaeaaeaaakeaacaWGJbWaa0
% baaSqaaiaadUgaaeaacaaIYaaaaOGaeyypa0ZaaeWaaeaadaWcaaqa
% aiabgkGi2kaaykW7caWGWbWaaSbaaSqaaiaadUgaaeqaaaGcbaGaey
% OaIyRaaGPaVlabeg8aYnaaBaaaleaacaWGRbaabeaaaaaakiaawIca
% caGLPaaadaWgaaWcbaGaam4CamaaBaaameaacaWGRbaabeaaaSqaba
% GccqGH9aqpdaWcaaqaaiabeo7aNnaaBaaaleaacaWGRbaabeaakiaa
% ykW7daqadaqaaiaadchadaWgaaWcbaGaam4AaaqabaGccqGHRaWkcq
% aHapaCdaWgaaWcbaGaam4AaaqabaaakiaawIcacaGLPaaaaeaacqaH
% bpGCdaWgaaWcbaGaam4Aaaqabaaaaaaa!5964!
\begin{equation}\label{soundspeed}
c_k^2  = \left( {\frac{{\partial \,p_k }} {{\partial \,\rho _k }}}
\right)_{s_k }  = \frac{{\gamma _k \,\left( {p_k  + \pi _k }
\right)}} {{\rho _k }}
\end{equation}
where $s_k$ is the entropy of the phase. This can be demonstrated by
differentiating (\ref{EOS})
% MathType!MTEF!2!1!+-
% feaafiart1ev1aaatCvAUfeBSjuyZL2yd9gzLbvyNv2CaerbuLwBLn
% hiov2DGi1BTfMBaeXatLxBI9gBaerbd9wDYLwzYbItLDharqqtubsr
% 4rNCHbGeaGqiVu0Je9sqqrpepC0xbbL8qbVhbbf9v8qqaqFr0xc9pk
% 0xbba9q8WqFfea0-yr0RYxir-Jbba9q8aq0-yq-He9q8qqQ8frFve9
% Fve9Ff0dmeaabaqaciGacaGaaeqabaWaaeaaeaaakeaacaWGKbGaam
% iCamaaBaaaleaacaWGRbaabeaakiabg2da9maabmaabaGaeq4SdC2a
% aSbaaSqaaiaadUgaaeqaaOGaeyOeI0IaaGymaaGaayjkaiaawMcaai
% abeg8aYnaaBaaaleaacaWGRbaabeaakiaaykW7caWGKbGaamyzamaa
% BaaaleaacaWGRbaabeaakiabgUcaRmaabmaabaGaeq4SdC2aaSbaaS
% qaaiaadUgaaeqaaOGaeyOeI0IaaGymaaGaayjkaiaawMcaaiaadwga
% daWgaaWcbaGaam4AaaqabaGccaaMc8Uaamizaiabeg8aYnaaBaaale
% aacaWGRbaabeaaaaa!56C0!
\begin{equation}\label{dp}
    dp_k  = \left( {\gamma _k  - 1} \right)\rho _k \,de_k  + \left( {\gamma _k  - 1} \right)e_k \,d\rho _k
\end{equation}
>From (\ref{EOS}), one can express
% MathType!MTEF!2!1!+-
% feaafiart1ev1aaatCvAUfeBSjuyZL2yd9gzLbvyNv2CaerbuLwBLn
% hiov2DGi1BTfMBaeXatLxBI9gBaerbd9wDYLwzYbItLDharqqtubsr
% 4rNCHbGeaGqiVu0Je9sqqrpepC0xbbL8qbVhbbf9v8qqaqFr0xc9pk
% 0xbba9q8WqFfea0-yr0RYxir-Jbba9q8aq0-yq-He9q8qqQ8frFve9
% Fve9Ff0dmeaabaqaciGacaGaaeqabaWaaeaaeaaakeaadaqadaqaai
% abeo7aNnaaBaaaleaacaWGRbaabeaakiabgkHiTiaaigdaaiaawIca
% caGLPaaacaaMc8UaamyzamaaBaaaleaacaWGRbaabeaakiabg2da9m
% aalaaabaGaamiCamaaBaaaleaacaWGRbaabeaakiabgUcaRiabeo7a
% NnaaBaaaleaacaWGRbaabeaakiaaykW7cqaHapaCdaWgaaWcbaGaam
% 4AaaqabaaakeaacqaHbpGCdaWgaaWcbaGaam4Aaaqabaaaaaaa!4ED1!
\begin{equation}\label{gm1e}
    \left( {\gamma _k  - 1} \right)\,e_k  = \frac{{p_k  + \gamma _k \,\pi _k }}
    {{\rho _k }}
\end{equation}
and by using the second principle of thermodynamics, we obtain
% MathType!MTEF!2!1!+-
% feaafiart1ev1aaatCvAUfeBSjuyZL2yd9gzLbvyNv2CaerbuLwBLn
% hiov2DGi1BTfMBaeXatLxBI9gBaebbnrfifHhDYfgasaacH8srps0l
% bbf9q8WrFfeuY-Hhbbf9v8qqaqFr0xc9pk0xbba9q8WqFfea0-yr0R
% Yxir-Jbba9q8aq0-yq-He9q8qqQ8frFve9Fve9Ff0dmeaabaqaciGa
% caGaaeqabaqabeaadaaakeaacaWGKbGaamyzamaaBaaaleaacaWGRb
% aabeaakiabg2da9iabgkHiTiaadchadaWgaaWcbaGaam4AaaqabaGc
% caaMc8Uaamizaiabes8a0naaBaaaleaacaWGRbaabeaakiabgUcaRi
% abeI7aXnaaBaaaleaacaWGRbaabeaakiaaykW7caWGKbGaeqySde2a
% aSbaaSqaaiaadUgaaeqaaaaa!4705!
\begin{equation}\label{dek}
    de_k  =  - p_k \,d\tau _k  + \theta _k \,d\alpha _k
\end{equation}
To find $d \alpha_k$, we differentiate the entropy expression
(\ref{entropy}) with respect to $d s_k =0$
% MathType!MTEF!2!1!+-
% feaafiart1ev1aaatCvAUfeBSjuyZL2yd9gzLbvyNv2CaerbuLwBLn
% hiov2DGi1BTfMBaeXatLxBI9gBaebbnrfifHhDYfgasaacH8srps0l
% bbf9q8WrFfeuY-Hhbbf9v8qqaqFr0xc9pk0xbba9q8WqFfea0-yr0R
% Yxir-Jbba9q8aq0-yq-He9q8qqQ8frFve9Fve9Ff0dmeaabaqaciGa
% caGaaeqabaqabeaadaaakeaacaWGMbWaaeWaceaacaWG4baacaGLOa
% GaayzkaaGaaGPaVlqadUeagaqbamaabmGabaGaeqySde2aaSbaaSqa
% aiaadUgaaeqaaaGccaGLOaGaayzkaaGaaGPaVlaadsgacqaHXoqyda
% WgaaWcbaGaam4AaaqabaGccqGHRaWkcaWGlbWaaeWaceaacqaHXoqy
% daWgaaWcbaGaam4AaaqabaaakiaawIcacaGLPaaacaaMc8UaeqiXdq
% 3aa0baaSqaaiaadUgaaeaacqaHZoWzdaWgaaadbaGaam4AaaqabaWc
% cqGHsislcaaIXaaaaOGabmOzayaafaWaaeWaceaacaWG4baacaGLOa
% GaayzkaaGaaGPaVlaadsgacaWGLbWaaSbaaSqaaiaadUgaaeqaaOGa
% ey4kaSIaam4samaabmGabaGaeqySde2aaSbaaSqaaiaadUgaaeqaaa
% GccaGLOaGaayzkaaGaaGPaVpaabmGabaWaaeWaceaacqaHZoWzdaWg
% aaWcbaGaam4AaaqabaGccqGHsislcaaIXaaacaGLOaGaayzkaaGaam
% yzamaaBaaaleaacaWGRbaabeaakiaaykW7cqaHepaDdaqhaaWcbaGa
% am4Aaaqaaiabeo7aNnaaBaaameaacaWGRbaabeaaliabgkHiTiaaik
% daaaGccqGHsislcqaHZoWzdaWgaaWcbaGaam4AaaqabaGccaaMc8Ua
% eqiWda3aaSbaaSqaaiaadUgaaeqaaOGaaGPaVlabes8a0naaDaaale
% aacaWGRbaabaGaeq4SdC2aaSbaaWqaaiaadUgaaeqaaSGaeyOeI0Ia
% aGymaaaaaOGaayjkaiaawMcaaiaaykW7ceWGMbGbauaadaqadiqaai
% aadIhaaiaawIcacaGLPaaacqGH9aqpcaaIWaaaaa!89FF!
\begin{equation}
    f\left( x \right)\,K'\left( {\alpha _k } \right)\,d\alpha _k  + K\left( {\alpha _k } \right)\,\tau _k^{\gamma _k  - 1} f'\left( x \right)\,de_k  + K\left( {\alpha _k } \right)\,\left( {\left( {\gamma _k  - 1} \right)e_k \,\tau _k^{\gamma _k  - 2}  - \gamma _k \,\pi _k \,\tau _k^{\gamma _k  - 1} } \right)\,f'\left( x \right) = 0
\end{equation}
and thus, with the notation previously defined,
% MathType!MTEF!2!1!+-
% feaafiart1ev1aaatCvAUfeBSjuyZL2yd9gzLbvyNv2CaerbuLwBLn
% hiov2DGi1BTfMBaeXatLxBI9gBaebbnrfifHhDYfgasaacH8srps0l
% bbf9q8WrFfeuY-Hhbbf9v8qqaqFr0xc9pk0xbba9q8WqFfea0-yr0R
% Yxir-Jbba9q8aq0-yq-He9q8qqQ8frFve9Fve9Ff0dmeaabaqaciGa
% caGaaeqabaqabeaadaaakeaacaWGKbGaeqySde2aaSbaaSqaaiaadU
% gaaeqaaOGaeyypa0JaeyOeI0YaaSaaaeaacqaHepaDdaqhaaWcbaGa
% am4Aaaqaaiabeo7aNnaaBaaameaacaWGRbaabeaaliabgkHiTiaaig
% daaaaakeaacaWGNbWaaeWaceaacqaHXoqydaWgaaWcbaGaam4Aaaqa
% baaakiaawIcacaGLPaaacaaMc8UaamOqamaaBaaaleaacaWGRbaabe
% aaaaGccaWGKbGaamyzamaaBaaaleaacaWGRbaabeaakiabgkHiTmaa
% laaabaGaaGymaaqaaiaadEgadaqadiqaaiabeg7aHnaaBaaaleaaca
% WGRbaabeaaaOGaayjkaiaawMcaaiaaykW7caWGcbWaaSbaaSqaaiaa
% dUgaaeqaaaaakmaabmGabaWaaeWaceaacqaHZoWzdaWgaaWcbaGaam
% 4AaaqabaGccqGHsislcaaIXaaacaGLOaGaayzkaaGaamyzamaaBaaa
% leaacaWGRbaabeaakiaaykW7cqaHepaDdaqhaaWcbaGaam4Aaaqaai
% abeo7aNjabgkHiTiaaikdaaaGccqGHsislcqaHZoWzdaWgaaWcbaGa
% am4AaaqabaGccaaMc8UaeqiWda3aaSbaaSqaaiaadUgaaeqaaOGaaG
% PaVlabes8a0naaDaaaleaacaWGRbaabaGaeq4SdC2aaSbaaWqaaiaa
% dUgaaeqaaSGaeyOeI0IaaGymaaaaaOGaayjkaiaawMcaaiaaykW7ca
% WGKbGaeqiXdq3aaSbaaSqaaiaadUgaaeqaaaaa!7CAC!
\begin{equation}\label{dak}
    d\alpha _k  =  - \frac{{\tau _k^{\gamma _k  - 1} }}
    {{g\left( {\alpha _k } \right)\,B_k }}de_k  - \frac{1}
    {{g\left( {\alpha _k } \right)\,B_k }}\left( {\left( {\gamma _k  - 1} \right)e_k \,\tau _k^{\gamma  - 2}  - \gamma _k \,\pi _k \,\tau _k^{\gamma _k  - 1} } \right)\,d\tau _k
\end{equation}
By injecting (\ref{dek}), (\ref{dak}) and (\ref{gm1e}) in
(\ref{dp}), we obtain the expression of the sound speed
(\ref{soundspeed}).
