%% LyX 2.3.7 created this file.  For more info, see http://www.lyx.org/.
%% Do not edit unless you really know what you are doing.
\documentclass[french]{article}
\usepackage[T1]{fontenc}
\usepackage[utf8]{inputenc}
\usepackage[a4paper]{geometry}
\geometry{verbose,tmargin=3cm,bmargin=3cm,lmargin=3cm,rmargin=3cm}
\setlength{\parskip}{\smallskipamount}
\setlength{\parindent}{0pt}
\usepackage{float}
\usepackage{url}
\usepackage{amsmath}
\usepackage{amssymb}
\usepackage{graphicx}

\synctex=1

\makeatletter

%%%%%%%%%%%%%%%%%%%%%%%%%%%%%% LyX specific LaTeX commands.
%% Because html converters don't know tabularnewline
\providecommand{\tabularnewline}{\\}

\makeatother

\usepackage{babel}
\makeatletter
\addto\extrasfrench{%
   \providecommand{\og}{\leavevmode\flqq~}%
   \providecommand{\fg}{\ifdim\lastskip>\z@\unskip\fi~\frqq}%
}

\makeatother
\begin{document}
\title{Analyse et optimisation d'un modèle thermohydraulique liquide-vapeur.}
\author{Philippe Helluy\thanks{IRMA CNRS, Inria Tonus, Strasbourg}, Gauthier
Lazare\thanks{EDF R\&D, Chatou}\thanks{IRMA, Unistra, Strasbourg}}
\maketitle
\begin{abstract}
Ce travail présente un modèle simplifié d'écoulement de fluide multiphasique
compressible dans un milieu poreux chauffant. Nous présentons d'abord
le modèle et son approximation numérique par un schéma de type volumes
finis implicite linéaire. Nous proposons ensuite une technique d'accélération
du schéma implicite par une approche d'apprentissage automatique.
\end{abstract}

\section{Introduction}

Très souvent, les techniques de production d'électricité passent par
le chauffage d'un fluide caloporteur (souvent de l'eau). Ce chauffage
est réalisé par un échange thermique entre le fluide et un milieu
chaud (four solaire, charbon, gaz, combustible nucléaire, etc.). Afin
de maximiser le transfert de chaleur, il est important que la surface
d'échange soit grande pour un volume donné. Pour cela, le fluide circule
au contact d'une surface d'échange le plus souvent dans un milieu
encombré en solide. Cette surface peut être complexe à modéliser précisément
et il est commode pour les simulations d'assimiler la zone d'échange
à un milieu poreux dans lequel le fluide circule. Dans ce travail
nous considérons donc un modèle compressible homogénéisé pour le fluide
caloporteur (de l'eau), susceptible de se vaporiser localement. Les
deux phases, à cause de la force d'Archimède, peuvent avoir des vitesses
différentes. L'écart des vitesses est pris en compte par un modèle
de drift. En revanche, les deux phases sont à l'équilibre de pression.

Cette modélisation conduit à un modèle plus simple à simuler numériquement,
mais plus complexe du point de vue mathématique. Il n'est pas clair,
par exemple que ce modèle proposé, à deux vitesses et une pression,
est hyperbolique. De nombreux travaux ont porté sur ce type de modèles,
voir (par exemple) \cite{baer1986two,saurel1999multiphase,gallouet2004numerical,gallouet2010hyperbolic,hurisse2022simulations}
et les références incluses. Pour une introduction générale au sujet
des fluides multiphasiques compressible on renvoie à \cite{ishii2010thermo}.

Le modèle mathématique est approché par un schéma semi-implicite en
temps sur grille décalée proposé dans \cite{THYC}. Le schéma est
écrit dans les variables non conservatives suivantes: entropie, fraction
de vapeur, débit et pression. Il est bien adapté aux calculs d'écoulements
à faible Mach. Les régimes observés ne présentent pas d'ondes de choc,
ce qui justifie ce choix.

Malgré les simplifications adoptés, l'obtention de l'état stationnaire
à partir d'un schéma instationnaire peut-être coûteux en temps de
calcul, voire instable pour des initialisations ou des pas de temps
mal choisis. 

Le premier objectif de ce travail est de décrire un modèle, le plus
simple possible, monodimensionnel, mais qui contient la plupart des
caractéristiques du modèle industriel complet présenté dans \cite{THYC}.
Le second objectif est d'évaluer une technique d'accélération du calcul
de la solution stationnaire, basée sur une approche de prédiction
par apprentissage automatique.

\section{Équations}

Le mélange diphasique (eau-vapeur) s'écoule dans un milieu occupé
partiellement par des barres ou tubes solides chauffants.

Le but de cette section est d'établir un système de lois de bilan
de la forme
\[
\partial_{t}W+\nabla\cdot F(W)+\nabla\cdot\Big(D(\nabla W)\Big)=S(W),
\]
où:
\begin{itemize}
\item le vecteur d'inconnues $W(x,t)$ dépend du temps $t$ et de l'espace
$x$. Ce vecteur inconnu est issu des bilans de masse, de quantité
de mouvement et d'énergie, par conséquent $W\in\mathbb{R}^{3}$ (modèle
homoène à 3 équations). Le bilan sur le titre massique en vapeur peut
aussi être considéré pour représenter certains phénomènes physiques
hors équilibre comme l'ébullition sous-saturée, alors $W\in\mathbb{R}^{4}$
(modèle déséquilibré à 4 équations),
\item le flux est noté $F(W)$,
\item le flux du second ordre est noté $D(\nabla W)$,
\item le terme source d'ordre $0$ est noté $S(W)$.
\end{itemize}
Dans les paragraphes suivants, nous détaillons successivement les
notations, les lois de bilans générales, le modèle thermodynamique
liquide-vapeur. Nous détaillerons en priorité les termes d'ordre 0
et d'ordre 1. Les termes d'ordre 2 sont négligés ici (pas de frottements
avec le solide, pas de viscosité, pas de diffusion de chaleur, pas
de turbulence). Le modèle présenté ici est une version simplifiée
du modèle THYC, code de calculs thermohydrauliques diphasique à l'échelle
composant, développé par EDF (voir\cite{THYC}).

\subsection{Quelques notations}

Les barres de combustibles ont une géométrie complexe et sont donc
modélisées par un milieu poreux, comme expliqué précédemment. Sur
un petit volume $V$ comportant du fluide et du solide, on considère
la porosité $\varepsilon$, définie comme le rapport du volume de
fluide $V_{f}$ sur le volume total (fluide + solide) $V=V_{s}+V_{f}$.
C'est une donnée géométrique qui dépend de l'espace $x$ mais pas
du temps et on a toujours $0\leq\varepsilon\leq1$. Le volume fluide
$V_{f}$ est lui même décomposé en un volume de vapeur $V_{v}$ et
un volume de liquide $V_{l}$. Les deux phases sont immiscibles de
sorte que 
\[
V_{f}=V_{l}+V_{v}.
\]

La fraction de volume de vapeur (appelé aussi taux de vide) est notée
$\alpha=\alpha(x,t)$. Il s’agit de 
\[
\alpha=\frac{V_{v}}{V_{l}+V_{v}}.
\]

La masse volumique du mélange est définie par 
\[
\rho=\frac{M_{l}+M_{v}}{V_{l}+V_{v}},\quad M_{l}:\text{ masse de liquide},\quad M_{v}:\text{ masse de gaz.}
\]
On a donc
\[
\rho=\alpha\rho_{v}+(1-\alpha)\rho_{l}
\]
avec les masses volumiques de chaque phase
\[
\rho_{l}=\frac{M_{l}}{V_{l}}\quad\text{et}\quad\rho_{v}=\frac{M_{v}}{V_{v}}
\]
où $M_{l}$ et $M_{v}$ sont, respectivement, la masse de liquide
et de gaz. Le titre massique y (ou fraction de masse) de vapeur est
alors défini par
\begin{equation}
y=\frac{\alpha\rho_{v}}{\rho}=\frac{M_{v}}{M_{l}+M_{v}}\quad\text{et}\quad1-y=\frac{(1-\alpha)}{\rho}\rho_{l}.\label{eq:formule_y}
\end{equation}
On définit aussi la quantité de mouvement volumique $q$ du mélange
\[
q=\frac{M_{v}u_{v}+M_{l}u_{l}}{V_{v}+V_{l}}=\rho u,
\]
ce qui définit la vitesse du mélange $u$ à partir des vitesses de
chaque phase $u_{l}$ et $u_{v}$. On définit aussi $u_{r}$ la vitesse
relative entre les deux phases
\[
u_{r}=u_{v}-u_{l}.
\]
De ces relations, on en déduit les relations suivantes
\[
\rho u=\alpha_{l}\rho_{l}u_{l}+\alpha_{v}\rho_{v}u_{v},
\]
\begin{equation}
u_{l}=u-yu_{r},\quad u_{v}=u+(1-y)u_{r}.\label{eq:ul_uv}
\end{equation}

La tension superficielle est négligée et on considère un équilibre
des pressions entre les phases c'est à dire que $p=p_{v}=p_{l}$.
On définit l'énergie interne massique du mélange $e$ et l'enthalpie
massique du mélange $h=e+\frac{p}{\rho}$ à partir de l'énergie interne
massique et de l'enthalpie massique de chaque phase

\[
e=ye_{v}+(1-y)e_{l},
\]

\[
h=yh_{v}+(1-y)h_{l}.
\]

Enfin, on définit aussi l'écart enthalpique
\begin{equation}
L=h_{v}-h_{l}.\label{eq:ecart_enthalp}
\end{equation}


\subsection{Lois de bilan}

Dans cette partie, chaque loi de bilan est déduite de principes de
la mécanique. Nous utilisons l'approche de \cite{pironneau1988methodes},
qui a l'avantage d'être particulièrement concise. Pour chaque bilan,
on considère un domaine $\Omega$ fixe dans l'espace. La normale unitaire
sortante à $\Omega$ sur $\partial\Omega$ est notée $n$.

\subsubsection{Bilan de masse}

La conservation de la masse totale s'écrit de façon identique pour
le modèle multiphasique et un modèle compressible à un seul fluide:
\begin{equation}
\partial_{t}(\varepsilon\rho)+\nabla\cdot(\varepsilon q)=0.\label{eq:cons_masse}
\end{equation}
Utilisant la méthode de \cite{pironneau1988methodes}, pendant un
temps $dt$, la masse de la phase $k$ qui traverse un élément de
surface $ds$ est donnée par $\varepsilon\alpha_{k}\rho_{k}u_{k}\cdot ndsdt$.
On obtient donc l'équation
\[
\frac{d}{dt}\int_{\Omega}\varepsilon\rho=-\int_{\partial\Omega}\left(\varepsilon\alpha_{l}\rho_{l}u_{l}\cdot n+\varepsilon\alpha_{v}\rho_{v}u_{v}\cdot n\right)=\int_{\partial\Omega}\varepsilon\rho u\cdot n,
\]
ce qui donne bien (\ref{eq:cons_masse}), grâce à la formule de Stokes.

\subsubsection{Bilan de quantité de mouvement}

Le bilan de quantité de mouvement est plus complexe car il faut tenir
compte de la vitesse relative entre les deux phases, qui n'est \textit{a
priori} pas nulle. L'équation s'écrit
\begin{equation}
\partial_{t}(\varepsilon q)+\nabla\cdot(\varepsilon\rho u\otimes u+\varepsilon pI)+\nabla\cdot(\varepsilon\rho y(1-y)u_{r}\otimes u_{r})=\varepsilon\rho g+I_{fs}+I_{t}\label{eq:qdm}
\end{equation}

où $g$ est l'accélération de la pesanteur, $I_{t}$ la dissipation
turbulente et $I_{fs}$ la dissipation par perte de charge (qui correspond
aux frottements entre le solide et le fluide). Comme nous l'avons
précisé précédemment, les frottements sont ici négligés, ainsi $I_{t}=0$
et $I_{fs}=0$.

L'équation (\ref{eq:qdm}) devient donc
\begin{equation}
\partial_{t}(\varepsilon q)+\nabla\cdot(\varepsilon\rho u\otimes u+\varepsilon pI)+\nabla\cdot(\varepsilon\rho y(1-y)u_{r}\otimes u_{r})=\varepsilon\rho g.\label{eq:qdm-simple}
\end{equation}
Afin d'établir cette équation de conservation, on peut également raisonner
comme dans \cite{pironneau1988methodes}. La variation de quantité
de mouvement dans le domaine $\Omega$ est égale à la somme de la
quantité de mouvement qui traverse la frontière par convection, des
forces de pression (toujours en négligeant la tension de surface)
qui s'exercent sur la frontière et des forces volumiques. On trouve
\begin{align*}
\frac{d}{dt}\int_{\Omega}\varepsilon\rho u & =-\int_{\partial\Omega}\varepsilon\alpha_{l}\rho_{l}u_{l}u_{l}\cdot n+\varepsilon\alpha_{v}\rho_{v}u_{v}u_{v}\cdot n\\
 & -\int_{\partial\Omega}\varepsilon pn\\
 & +\int_{\Omega}\varepsilon\rho g.
\end{align*}
On obtient, avec les relations (\ref{eq:formule_y}),
\begin{align*}
\frac{d}{dt}\int_{\Omega}\varepsilon\rho u & =-\int_{\partial\Omega}\left(\varepsilon(1-y)\rho u_{l}u_{l}\cdot n+\varepsilon y\rho u_{v}u_{v}\cdot n\right)\\
 & -\int_{\partial\Omega}\varepsilon pn\\
 & +\int_{\Omega}\varepsilon\rho g.
\end{align*}
On utilise ensuite les relations (\ref{eq:ul_uv}) et la formule de
Stokes pour conclure.

\subsubsection{Bilan d'énergie\label{subsec:Bilan-d'=0000E9nergie}}

Pour le bilan d'énergie, plusieurs termes sont à considérer pour prendre
en compte l'écart de vitesse (vitesse relative) et la chaleur latente
de vaporisation. Nous commençons par établir la formulation conservative
du bilan d'énergie. Pour cela, on introduit l'énergie totale massique
de chaque phase $k$
\[
\eta_{k}=e_{k}+\frac{u_{k}^{2}}{2}.
\]
L'énergie totale massique du mélange est alors donnée par
\begin{align*}
\rho\eta & =\sum_{k}\alpha_{k}\rho_{k}e_{k}+\alpha_{k}\rho_{k}\frac{u_{k}^{2}}{2},\\
\eta & =ye_{v}+(1-y)e_{l}+y\frac{u_{v}^{2}}{2}+(1-y)\frac{u_{l}^{2}}{2}.
\end{align*}
En utilisant les relations (\ref{eq:ul_uv}), l'énergie totale se
réécrit 
\[
\eta=ye_{v}+(1-y)e_{l}+\frac{u^{2}}{2}+y(1-y)\frac{u_{r}^{2}}{2}.
\]

\noindent
Attention, we do not have the equality $\eta=e+\frac{u^{2}}{2}$ in the case where the relative velocity is non-zero. To write the energy balance, we still use the technique presented in \cite{pironneau1988methodes}. The temporal variation of energy in $\Omega$
\[
\frac{d}{dt}\int_{\text{\ensuremath{\Omega}}}\varepsilon\rho\eta
\]
is given by the balance of several terms:
\begin{itemize}
\item the quantity of energy that crosses the boundary by convection
\[
W_{1}=-\int_{\partial\Omega}\left(\varepsilon\alpha_{l}\rho_{l}\eta_{l}u_{l}\cdot n+\varepsilon\alpha_{v}\rho_{v}\eta_{v}u_{v}\cdot n\right),
\]
\item the work of the pressure force at the boundary
\[
W_{2}=-\int_{\partial\Omega}\left(\varepsilon\alpha_{l}pu_{l}\cdot n+\varepsilon\alpha_{v}pu_{v}\cdot n\right),
\]
\item the work of the forces in the volume
\[
W_{3}=\int_{\Omega}\varepsilon\rho g\cdot u,
\]
\item the heat sources, with $\phi$ the heat source from the solid, $\phi_{dif}$ the diffusion heat flux and $\phi_{t}$ the turbulent heat flux
\[
W_{4}=\int_{\Omega}\phi+\int_{\partial\Omega}\varepsilon(\phi_{dif}+\phi_{t})\cdot n.
\]
\end{itemize}
After calculations, the terms $W_{i}$ can be simplified. For $W_{1}$ we find
\begin{align*}
W_{1} & =-\int_{\partial\Omega}\varepsilon\rho\eta u\cdot n\\
 & -\int_{\partial\Omega}\varepsilon\rho y(1-y)(\eta_{v}-\eta_{l})u_{r}\cdot n,
\end{align*}
For $W_{2}$
\[
W_{2}=-\int_{\partial\Omega}\varepsilon pu\cdot n=-\int_{\partial\Omega}\varepsilon p(\alpha-y)u_{r}\cdot n.
\]
We can also write (after calculations)
\[
W_{2}=-\int_{\partial\Omega}\varepsilon pu\cdot n=-\int_{\partial\Omega}\varepsilon p\rho y(1-y)(\tau_{v}-\tau_{l})u_{r}\cdot n.
\]
The conservative form of the equations is therefore
\begin{align*}
\partial_{t}\varepsilon\rho\eta & +\nabla\cdot\left(\varepsilon(\rho\eta+p)u\right)\\
 & +\nabla\cdot\left(\varepsilon\rho y(1-y)(\eta_{v}+\tau_{v}p-(\eta_{l}+\tau_{l}p))u_{r}\right)\\
 & =\varepsilon\rho g\cdot u+\phi+\nabla\cdot\varepsilon(\phi_{t}+\phi_{dif}).
\end{align*}

Since in this work we neglect the diffusion terms (order 2) $\phi_{t}+\phi_{dif}=0$.

Afin d'obtenir l'équation sur l'énergie interne uniquement, il faut
écrire l'équation d'énergie cinétique et l'équation sur la quantité
$y(1-y)\frac{u_{r}^{2}}{2}$ et les soustraire à l'équation d'énergie
totale.

L'équation d'énergie cinétique est obtenue en multipliant l'équation
de quantité de mouvement (\ref{eq:qdm-simple}) par $u$ et en utilisant
la conservation de la masse (\ref{eq:cons_masse})

\[
\partial_{t}\big(\varepsilon\rho\frac{u^{2}}{2}\big)+\nabla\cdot\left(\varepsilon\rho u\frac{u^{2}}{2}\right)+\varepsilon u\cdot\nabla p+u\cdot\Big(\nabla\cdot\left(\varepsilon\rho y(1-y)u_{r}^{2}\right)\Big)=\varepsilon\rho g\cdot u.
\]

Pour l'équation sur la quantité $y(1-y)\frac{u_{r}^{2}}{2}$, l'équation
de quantité de mouvement de la phase vapeur est multipliée par $(1-y)u_{r}$
et celle de la phase liquide par $yu_{r}$. L'équation de la phase
vapeur est soustraite à celle de la phase liquide puis l'équation
du bilan de masse vapeur multipliée par la quantité $(1-2y)\frac{u_{r}^{2}}{2}$
est ajoutée. On obtient alors l'équation sur la quantité $y(1-y)\frac{u_{r}^{2}}{2}$

\begin{align*}
\partial_{t}\big(\varepsilon\rho y(1-y)\frac{u_{r}^{2}}{2}\big)+ & \nabla\cdot\left(\varepsilon\rho y(1-y)u\frac{u_{r}^{2}}{2}\right)+\Big(\rho y(1-y)\varepsilon\frac{u_{r}^{2}}{2}\partial_{x}u+\partial_{x}\big(\rho y(1-y)\varepsilon u_{r}(1-2y)\frac{u_{r}^{2}}{2}\big)\bigg)=\\
 & -\rho y(1-y)(\tau_{v}-\tau_{l})\varepsilon\partial_{x}p-pu_{r}\varepsilon\partial_{x}\alpha+I_{p}u_{r}-\Gamma\Big(\frac{u_{g}^{2}}{2}-\frac{u_{l}^{2}}{2}\Big)
\end{align*}

avec $I_{p}$ le terme d'échange de quantité de mouvement entre la
phase vapeur et la phase liquide et $\Gamma$ le terme d'échange de
masse entre la phase vapeur et la phase liquide. Après calculs, on
obtient finalement l'équation sur l'énergie interne suivante

\begin{align}
\partial_{t}\varepsilon\rho e+\nabla\cdot\left(\varepsilon\rho ue\right)+\varepsilon\big(pI+\rho y(1-y)u_{r}^{2}\big)\cdot\nabla u & +\nabla\cdot\left(\varepsilon\rho y(1-y)\Big(h_{v}-h_{l}\Big)u_{r}\right)=\nonumber \\
 & \phi+\Big(\Gamma(\frac{u_{g}^{2}}{2}-\frac{u_{l}^{2}}{2})-u_{r}I\Big)+pu_{r}\varepsilon\partial_{x}\alpha.\label{eq:eq_energie_interne}
\end{align}

Dans les applications considérées, le flux de chaleur est très important,
ce qui permet de négliger les termes à droite sur l'équation ci-dessus
$\Big(\Gamma(\frac{u_{g}^{2}}{2}-\frac{u_{l}^{2}}{2})-u_{r}I\Big)+pu_{r}\varepsilon\partial_{x}\alpha$.
En utilisant la formule $h=e+p\tau$, le bilan d'enthalpie du mélange
s'écrit finalement 
\begin{align}
\partial_{t}(\varepsilon\rho h-\varepsilon p)+\nabla\cdot(\varepsilon hq) & +\nabla\cdot(\varepsilon\rho y(1-y)Lu_{r})=\nonumber \\
 & \varepsilon\left(u+y(1-y)\rho\left(\frac{1}{\rho_{v}}-\frac{1}{\rho_{l}}\right)u_{r}\right)\cdot\nabla p+\phi.\label{eq:eq_enthalpie}
\end{align}
L'écart enthalpique $L$ est donné par la relation (\ref{eq:ecart_enthalp}).
Pour notre application on peut se contenter aussi, en première approximation,
d'approcher le terme source de chauffage $\phi$ par une constante.

\subsubsection{Bilan de masse de gaz \label{subsec:Bilan-de-masse}}

Dans le cas où le mélange n'est pas à l'équilibre thermodynamique,
la température du liquide peut être, en moyenne, inférieure à la température
saturation, tout en observant de la production de vapeur. Cela s'explique
par le fait que la température de saturation est atteinte localement.
Ce phénomène est appelé ébullition sous-saturée. Dans ce cas, la quatrième
équation traduit le bilan de masse de la vapeur
\begin{equation}
\partial_{t}\left(\varepsilon\rho y\right)+\nabla\cdot(\varepsilon yq)+\nabla\cdot\left(\varepsilon y(1-y)\rho u_{r}\right)=\Gamma,\label{eq:titre}
\end{equation}

où $\Gamma$ le terme de production est divisé en deux termes

\[
\Gamma=\Gamma_{\phi}+\Gamma_{p},
\]
avec $\Gamma_{\phi}$ la production de vapeur due au flux de chaleur
(production directe de vapeur) et $\Gamma_{p}$ la production de vapeur
par effet de pression (terme de retour à l'équilibre thermodynamique).
Nous introduisons cette équation par soucis d'exhaustivité et parce
que nous aurons besoin de tenir compte de ces termes dans des travaux
ultérieurs. Cependant, dans les simulations numériques de cet article,
la vapeur est toujours supposée à l'équilibre. L'équation (\ref{eq:titre})
ne sera donc pas prise en compte. Le titre massique est toujours pris
à l'équilibre $y_{eq}$.

\subsection{Modélisation}

Les équations proposées précédemment reposent, dans leur ensemble,
sur des grands principes: transport géométrique, conservation, lois
de Newton, thermodynamique. Il s'agit maintenant de proposer une thermodynamique
pour le mélange diphasique liquide-vapeur. De plus, la vitesse relative
doit être modélisée par une corrélation pour fermer les équations.

\subsubsection{Thermodynamique du mélange}

On suppose connues les deux lois d'état complètes du liquide et de
la vapeur. Des exemples de ces lois de comportement, basés sur la
loi d'état des gaz raides, sont donnés en Annexe 1. Il y a plusieurs
façons d'écrire ces lois d'état complètes, on peut par exemple se
donner l'énergie spécifique de chaque phase $k$. Pour chaque phase
$k$, l'énergie spécifique $e_{k}$ est une fonction du volume spécifique
$\tau_{k}=\frac{1}{\rho_{k}}$ et de l'entropie spécifique $s_{k}$
\[
e_{k}=e_{k}(\tau_{k},s_{k}).
\]
Les autres variables thermodynamiques s'obtiennent toutes à partir
de $e_{k}$
\begin{itemize}
\item température
\[
\theta_{k}=\frac{\partial}{\partial s_{k}}e_{k}(\tau_{k},s_{k}),
\]
\item pression
\begin{equation}
p_{k}=-\frac{\partial}{\partial\tau_{k}}e_{k}(\tau_{k},s_{k}),\label{eq:def_pres}
\end{equation}
\item potentiel chimique
\[
\mu_{k}=e_{k}+p_{k}\tau_{k}-\theta_{k}s_{k},
\]
\item enthalpie
\[
h_{k}=e_{k}+p_{k}\tau_{k}.
\]
\end{itemize}
À partir des lois d'état de chaque phase, on peut établir une loi
d'état du mélange, qui est cohérente avec les lois thermodynamiques.
Pour cela, on commence par définir l'énergie massique du mélange avec
la formule 
\[
e=ye_{v}(\tau_{v},s_{v})+(1-y)e_{\ell}(\tau_{\ell},s_{\ell}).
\]
Mais par ailleurs $\tau_{k}=\frac{V_{k}}{M_{k}}=\frac{V_{k}}{V}\cdot\frac{V}{M}\cdot\frac{M}{M_{k}}$.
Donc
\[
\tau_{v}=\frac{\alpha}{y}\tau,\quad\tau_{\ell}=\frac{1-\alpha}{1-y}\tau.
\]
De même
\begin{equation}
s_{v}=\frac{z}{y}s,\quad s_{\ell}=\frac{1-z}{1-y}s,\label{eq:frac_entrop}
\end{equation}
où $z$ est la fraction d'entropie de la vapeur. Donc
\begin{equation}
e=ye_{v}(\frac{\alpha}{y}\tau,\frac{z}{y}s)+(1-y)e_{\ell}(\frac{1-\alpha}{1-y}\tau,\frac{1-z}{1-y}s).\label{eq:energ_mix_ph0}
\end{equation}
Pour une discussion plus longue sur ce choix, on renvoit (par exemple)
à \cite{callen1991thermodynamics,barberon2005finite,helluy2006relaxation,helluy2011pressure}.
\textit{A priori}, l'énergie du mélange est une fonction de $(\tau,s,y,z,\alpha).$
Mais certains équilibres vont permettre d'éliminer la fraction de
volume $\alpha$ et la fraction d'entropie $z$. L'équilibre est déterminé
en maximisant l'énergie. Calculons donc la différentielle de $e$
\begin{align}
de= & -\left(\alpha p_{v}+(1-\alpha)p_{\ell}\right)d\tau\nonumber \\
 & +\left(z\theta_{v}+(1-z)\theta_{\ell}\right)ds\nonumber \\
 & +(\mu_{v}-\mu_{\ell})dy\nonumber \\
 & +s(\theta_{v}-\theta_{\ell})dz\nonumber \\
 & -\tau(p_{v}-p_{\ell})d\alpha.\label{eq:variation_de}
\end{align}

Comme expliqué à la fin de la section \ref{subsec:Bilan-de-masse},
nous donnons les détails des calculs dans le cas hors équilibre, avec
ébullition sous-saturée, par soucis d'exhaustivité. Mais dans les
résultats numériques, nous supposerons que la vapeur est à l'équilibre.

\paragraph{Équilibre mécanique}

Pour obtenir l'équilibre mécanique, on maximise l’énergie de mélange
par rapport à la fraction de volume $\alpha$. D'après l'équation
(\ref{eq:variation_de}), on a donc équilibre des pressions des deux
phases
\[
p=p_{v}=p_{\ell}.
\]


\paragraph{Équilibre thermique}

De façon analogue, on peut établir la condition d'équilibre, en maximisant
l'énergie par rapport à la fraction d'entropie $z$. L'équilibre thermique
se traduit donc par l'égalité des températures des deux phases
\[
\theta_{v}=\theta_{\ell}.
\]

Mais supposer l'équilibre thermique n'est pas réaliste dans le modèle
moyenné lorsque l'on considère le modèle déséquilibré à 4 équations.
Cette hypothèse est cependant vraie dans le cadre 3 équations où les
deux phases sont à la même température, la température de saturation.

\textbf{Équilibre chimique:} on suppose que si de la vapeur apparaît
localement, celle-ci est à l'équilibre chimique avec le liquide environnant
\[
\mu_{v}(p,\theta_{v})=\mu_{\ell}(p,\theta_{*}),
\]
où $\theta_{*}$ est la température du liquide immédiatement en contact
avec la vapeur. En général, la vapeur apparaît à proximité des solides
chauffants. Cependant un volume de contrôle peut englober une zone
plus large, où la température liquide n'est pas à saturation en moyenne,
donc $\theta_{l}\neq\theta^{*}$.

Il est plus réaliste de supposer qu'il y a équilibre thermique local
entre la vapeur et le liquide directement au contact de la vapeur.
Dans ce cas la vapeur reste à saturation (pas de surchauffage) et
\[
\theta_{*}=\theta_{v}.
\]
 On obtient alors
\begin{equation}
\mu_{v}(p,\theta_{v})=\mu_{\ell}(p,\theta_{v}).\label{eq:def_sat}
\end{equation}
C'est exactement la définition de la température de saturation
\[
\mu_{v}(p,\theta_{sat})=\mu_{\ell}(p,\theta_{sat}).
\]
L'équilibre chimique peut donc aussi s'écrire 
\begin{equation}
\theta_{v}=\theta_{sat}(p)=\theta_{v}(p).\label{eq:temp_sat}
\end{equation}
Il permet d'éliminer la fraction d'entropie $z$ des calculs. En effet,
d'après l'équation (\ref{eq:frac_entrop})
\[
z=\frac{ys_{v}(p)}{s}.
\]
Compte tenu de la condition de saturation (\ref{eq:temp_sat}), l'entropie
de la vapeur ne dépend plus que de la pression. Ce qui donne
\begin{equation}
dz=\frac{s_{v}}{s}dy+\frac{y}{s}s'_{v}(p)dp-\frac{ys_{v}}{s^{2}}ds,\label{eq:dz}
\end{equation}

où $s'_{v}(p)$ est la dérivée totale de l'entropie de la vapeur par
rapport à la pression.

\paragraph{Bilan}

En conclusion des calculs précédents, nous pouvons reformuler la différentielle
de l'énergie de mélange (\ref{eq:variation_de}) en:
\begin{align*}
de= & -pd\tau\\
 & +\left(z\theta_{v}(p)+(1-z)\theta_{\ell}\right)ds\\
 & +(\mu_{v}(p,\theta_{v}(p))-\mu_{\ell}(p,\theta_{\ell}))dy\\
 & +(\theta_{v}(p)-\theta_{\ell})sdz.
\end{align*}
Avec (\ref{eq:dz}), $z=ys_{v}/s$, on obtient
\begin{align*}
de= & -pd\tau\\
 & +\left(z\theta_{v}(p)+(1-z)\theta_{\ell}\right)ds\\
 & +(\mu_{v}(p,\theta_{v}(p))-\mu_{\ell}(p,\theta_{\ell}))dy\\
 & +(\theta_{v}(p)-\theta_{\ell})\left(s_{v}dy+ys_{v}'(p)dp-zds\right).
\end{align*}
En passant en enthalpie $h=e+p\tau$
\begin{align}
dh= & \left(\tau+y(\theta_{v}(p)-\theta_{\ell})s'_{v}(p)\right)dp\nonumber \\
 & +\theta_{\ell}ds\nonumber \\
 & +\left((\theta_{v}(p))-\theta_{\ell})s_{v}+\mu_{v}(p,\theta_{v}(p))-\mu_{\ell}(p,\theta_{\ell})\right)dy,\label{eq:diff_enthalpie}
\end{align}

ce qui donne
\begin{align}
dh= & \left(\tau+y(\theta_{v}(p)-\theta_{\ell})s'_{v}(p)\right)dp\nonumber \\
 & +\theta_{\ell}ds\nonumber \\
 & \left(h_{v}-h_{\ell}+\theta_{\ell}(s_{\ell}-s_{v})\right)dy.\label{eq:diff_enthalpie-2}
\end{align}
En rappelant que 
\[
L=h_{v}-h_{\ell},
\]
et en notant
\[
\overline{L}=s_{v}(p)-s_{\text{\ensuremath{\ell}}},
\]
on a aussi
\begin{align}
dh= & \left(\tau+y(\theta_{v}(p)-\theta_{\ell})s'_{v}(p)\right)dp\nonumber \\
 & +\theta_{\ell}ds\nonumber \\
 & +(L-\theta_{\ell}\overline{L})dy.\label{eq:diff_enthalpie-1}
\end{align}

\textbf{Déséquilibre thermique:} comme expliqué ci-dessus, on ne suppose
pas toujours l'équilibre thermique. Dans le modèle retenu, en général,
\[
\theta_{\ell}\neq\theta_{v}.
\]

Ce déséquilibre thermique a des conséquences sur le calcul des variables
de mélanges. En particulier, comme $\partial_{s}h=\theta$, on voit
avec l'équation (\ref{eq:diff_enthalpie}) que la contrainte $\theta_{v}=\theta_{sat}(p)$
impose que la température du mélange est la température du liquide
\begin{equation}
\theta=\theta_{\ell}.\label{eq:temperature_mix}
\end{equation}


\paragraph{Calcul des autres variables}

Le modèle numérique sera résolu dans les variables suivantes: débit,
entropie, pression et titre, $(q,s,p,y)$. Il faut donc savoir calculer
toutes les autres variables thermodynamiques à partir de $y,s,p$.
On sait que
\[
s=ys_{v}+(1-y)s_{\ell}
\]
où $s_{\ell}$ et $s_{v}$ désignent respectivement l'entropie massique
du liquide et de la vapeur. La vapeur est toujours à saturation, donc
\[
\theta_{v}=\theta_{sat}(p).
\]
On peut alors en déduire l'entropie massique de la vapeur (ici exprimée
en fonction de la pression et de la température)
\[
s_{v}=s_{v}(\theta_{v},p),
\]
ce qui nous donne $s_{\ell}$
\begin{align*}
s_{\ell} & =\frac{s-ys_{v}}{1-y}\text{ si }y<1,\\
 & =0\text{ sinon. Ce régime ne se rencontre jamais en pratique.}
\end{align*}
 On en déduit alors toutes les autres quantités thermodynamiques pour
le liquide, puisqu'on connaît deux variables (ici la pression $p$
et l'entropie $s_{\ell}$).


\subsubsection{Vitesse relative}

La vitesse relative doit être modélisée par une loi de fermeture pour
avoir un système d'équations fermé. Elle est décomposée en deux termes

\[
u_{r}=v_{r}-\frac{D}{y(1-y)}\nabla y.
\]
Le terme en gradient de titre est lié à la diffusion turbulente. Le
coefficient de diffusion $D$ est souvent donné par un modèle mais
ce terme d'ordre 2 est négligé dans cet article. Le terme de drift-flux
$v_{r}$ selon l'axe vertical est pris sous la forme issue de la corrélation
de Bestion (initialement développé pour CATHARE, voir par exemple
\cite{coddington2002study})
\[
v_{r}=0.188\frac{1+y(\delta-1)}{1-y}\sqrt{gd_{h}(\delta-1)},
\]
où $d_{h}$ est une longueur caractéristique, dépendant de la géométrie
du solide, et 
\[
\delta=\frac{\rho_{l}}{\rho_{v}}.
\]


\subsection{Formulation vectorielle\label{subsec:Formulation-vectorielle}}

Nous pouvons regrouper le modèle dans un système de lois de conservation
de la forme
\[
\partial_{t}W+\nabla\cdot Q(W,x)=S(W).
\]
Le vecteur des variables de bilan est
\[
W=\left(\begin{array}{c}
\varepsilon\rho\\
\varepsilon q\\
\varepsilon\rho\eta\\
\varepsilon\rho y
\end{array}\right).
\]
Le flux $Q$ contient les termes du premier ordre en $W$ et le flux
$R$ les termes du second ordre. D'après ce qui précède, nous avons
\[
Q(W)\cdot n=\left(\begin{array}{c}
\varepsilon q\cdot n\\
\varepsilon\left(q\cdot nu+pn+\rho y(1-y)u_{r}\cdot nu_{r}\right)\\
\varepsilon\left(q\cdot n\eta+pu\cdot n+\rho y(1-y)(\eta_{v}+\tau_{v}p-(\eta_{l}+\tau_{l}p))u_{r}\cdot n\right)\\
\varepsilon yq\cdot n+\varepsilon\rho y(1-y)u_{r}\cdot n
\end{array}\right),
\]

et
\[
S(W)=\left(\begin{array}{c}
0\\
\varepsilon\rho g\\
\phi\\
\Gamma_{p}+\Gamma_{\phi}
\end{array}\right).
\]


\section{Formulation primitive des équations}

L'objectif est d'obtenir une formulation primitive pour les quatre
équations en inconnues pression, débit, entropie et titre $(p,q,s,y)$.
C'est cette formulation, non conservative, qui sera utilisée numériquement.
Les applications considérées ne font pas apparaître de choc. Cela
ne pose donc pas de difficulté particulière liée à la définition des
relations de saut de Rankine-Hugoniot.

\subsection{Équation en pression}

La différentielle de la masse volumique par rapport aux variables
$(p,q,s)$ est donnée par:

\begin{equation}
d\rho=\frac{1}{c_{s}^{2}}dp+\beta ds+\gamma dy,\label{eq:rho_linearisation}
\end{equation}

où les expressions des coefficients $\frac{1}{c_{s}^{2}},\beta$ et
$\gamma$ sont détaillées en Annexe 1. Dans le cas où l'on considère
un modèle homogène à 3 équations, c'est-à-dire lorsque le titre vapeur
est toujours à l'équilibre, $y=y^{eq}$, le coefficient $\gamma=0$.

En utilisant cette différentielle sur l'équation de conservation de
la masse (\ref{eq:cons_masse}) et en rappelant que la porosité $\varepsilon$
ne dépend pas du temps, l'équation sur la pression s'écrit

\begin{equation}
\varepsilon\partial_{t}p+c_{s}^{2}\nabla\cdot(\varepsilon q)=-\varepsilon\beta c_{s}^{2}\partial_{t}s-\varepsilon\gamma c_{s}^{2}\partial_{t}y\label{eq:eq_prim_pression}
\end{equation}


\subsection{Équation en entropie}

En utilisant la différentielle de l'enthalpie (\ref{eq:diff_enthalpie-1}),
on peut en déduire une expression simple de la différentielle de l'entropie
pour le modèle homogène. En effet, dans ce cadre, la saturation est
atteinte $\theta_{l}=\theta_{v}=\theta_{sat}$ et alors

\[
\theta_{l}ds=dh-\tau dp.
\]

L'hypothèse de saturation permet aussi d'écrire

\[
L=\theta_{l}\bar{L}
\]

et

\[
\theta_{l}d\bar{L}=dL-(\tau_{v}-\tau_{l})dp.
\]

En réécrivant l'équation d'énergie sous forme non conservative (en
utilisant la conservation de la masse)

\[
\rho\varepsilon\Big(\partial_{t}h-\tau\partial_{t}p+u\cdot\big(\nabla h-\tau\nabla p\big)\Big)+\rho\varepsilon y(1-y)u_{r}\cdot\Big(\nabla L-(\tau_{v}-\tau_{l})\nabla p\Big)+L\nabla\cdot(\varepsilon\rho y(1-y)u_{r})=\phi,
\]

et en utilisant les différentielles précédentes, l'équation sur l'entropie
est

\[
\theta_{l}\big(\varepsilon\rho\partial_{t}s+\varepsilon\rho u\cdot\nabla s+\nabla\cdot(\varepsilon\rho y(1-y)u_{r}\bar{L})\big)=\phi.
\]

Cette équation admet aussi une forme conservative

\begin{equation}
\partial(\varepsilon\rho s)+\nabla\cdot(\varepsilon\rho us+\varepsilon\rho y(1-y)u_{r}\bar{L})=\frac{\phi}{\theta_{l}}.\label{eq:entropie_eq}
\end{equation}


\subsection{Système d'équations final}

L'équation de conservation sur la quantité de mouvement et sur le
titre massique vapeur sont écrites sous forme primitive. Le modèle
final obtenu est donc

\begin{equation}
\varepsilon\rho\partial_{t}s+\varepsilon q\cdot\nabla s+\nabla\cdot(\varepsilon\rho y(1-y)u_{r}\bar{L})=\frac{\phi}{\theta_{l}},\label{eq:eq_s_num}
\end{equation}

\begin{equation}
\varepsilon\rho\partial_{t}y+\varepsilon q\cdot\nabla y+\nabla\cdot\left(\varepsilon\rho y(1-y)u_{r}\right)=\Gamma_{p}+\Gamma_{\phi},\label{eq:eq_c_num}
\end{equation}

\begin{equation}
\varepsilon\partial_{t}p+c_{s}^{2}\nabla\cdot(\varepsilon q)=-\varepsilon\beta c_{s}^{2}\partial_{t}s-\varepsilon\gamma c_{s}^{2}\partial_{t}y,\label{eq:eq_p_num}
\end{equation}

\begin{equation}
\varepsilon\partial_{t}q+\nabla\cdot(\varepsilon u\otimes q)+\nabla(\varepsilon p)+\nabla\big(\varepsilon\rho y(1-y)u_{r}\otimes u_{r}\big)=\varepsilon\rho g.\label{eq:eq_q_num}
\end{equation}

\textbf{Remarque }: Les termes non conservatifs $\varepsilon q\cdot\nabla f$
sont traités numériquement en les séparant en deux parties telles
que $\varepsilon q\cdot\nabla f=\nabla\cdot(\varepsilon fq)-f\nabla\cdot(\varepsilon q)$.

\section{Résolution numérique\label{sec:Code-num=0000E9rique}}

Dans cette section, nous décrivons le code ThermoTorch développé pour
approcher le modèle diphasique précédemment décrit considéré sur un
domaine monodimensionnel.

\subsection{Méthode de résolution}

L'algorithme de résolution pour un pas de temps se déroule de la manière
suivante
\begin{enumerate}
\item Actualisation de la thermodynamique à partir des champs $(p,s)$ ($(p,s,y)$
dans le modèle à 4 équations) obtenus au pas de temps précédent. Durant
cette étape, le titre à l'équilibre thermodynamique $y^{eq}$ est
lui aussi évalué.
\item Résolution de l'équation d'entropie (\ref{eq:eq_s_num}). En considérant
seulement l'entropie en implicite, l'équation devient indépendante
des autres inconnues et peut être traitée seule. Par ailleurs, ce
système implicite est un système linéaire.
\item \textit{Pour le modèle 4 équations} : Résolution de l'équation de
titre (\ref{eq:eq_c_num}). Cette étape se simplifie pour le modèle
homogène à 3 équations car le titre est à l'équilibre $y=y^{eq}$.
\item Résolution du système couplé débit-pression $(p,q)$ (\ref{eq:eq_p_num}-\ref{eq:eq_q_num}).
Le débit et la pression sont calculé par un schéma implicite. On utilise
l'entropie et le titre obtenus aux étapes 3 et 4 dans l'équation sur
la pression.
\end{enumerate}

\subsection{Schéma numérique}

\subsubsection{Discrétisation}

La résolution est faite avec un schéma 1D en grilles décalées, c'est-à-dire
une grille pour la pression $p$, l'entropie $s$ et le titre $y$
et une deuxième grille pour le débit $q$. Ce choix est justifié par
le fait que ce type de schéma a un bon comportement dans la limite
de faible Mach (voir par exemple \cite{herbin2021low}). Pour cela,
on considère un maillage de l'intervalle $[0,L]$ avec des cellules
$C_{i}$, $0\leq i<N+1$. Le pas d'espace est constant et égal à 
\[
\Delta x=\frac{2L}{2N-1}
\]
et les cellules sont alors
\[
C_{i}=](i-1)\Delta x,i\Delta x[.
\]
On note $p_{i}$ la pression au centre de la cellule $C_{i}$, au
point $x_{i}=(i-\frac{1}{2})\Delta x,$ $0\leq i<N+1$. On utilise
la même notation pour l'entropie $s_{i}$ et pour le titre $y_{i}$.
La notation $q_{i-1/2}$ est utilisé pour le débit aux bords des cellules,
aux points $x_{i-1/2}=(i-1)\Delta x$, $1\leq i<N+1$. L'écoulement
est supposé s'effectuer de la gauche vers la droite, c'est--à-dire
$q_{i-1/2}>0,\forall i$.

La cellule $C_{0}$ permet d'imposer la condition limite en entropie
(elle est dite cellule fantôme) en entrée du domaine. On fixe

\[
s_{0}=s_{in}.
\]

Pour le modèle 4 équation, la condition limite sur le titre est elle-aussi
ajoutée

\[
y_{0}=y_{in}.
\]

Le débit en entrée est imposé quant à lui sur la première cellule
$C_{1}$ avec

\[
q_{1/2}=q_{in}
\]
La cellule $C_{N}$ permet d'imposer la condition limite pour la pression
en sortie:

\[
p_{N}=p_{out}
\]

Un flux de chaleur constant est imposé dans le domaine $\Big[\frac{L}{8},\frac{7L}{8}\Big]$
:

\[
\phi\bigg(x\in\Big[\frac{L}{8},\frac{7L}{8}\Big]\bigg)=\phi_{0},\quad\phi(x)=0\text{ sinon}.
\]

\textbf{Remarque} : Pour chaque calcul, les seuls paramètres d'entrée
sont donc $(\theta_{in},q_{in},p_{out},\phi_{0})$ pour le modèle
3 équations. Ce nombre de paramètres réduits va permettre de prédire
la solution au moyen d'un réseau de neurones très simple.

\subsubsection{Schémas numériques}

Dans la suite, on note $\lambda$ le rapport du pas de temps $\Delta t$
et du pas de maillage $\Delta x$ 
\[
\lambda=\frac{\Delta t}{\Delta x}.
\]
Pour l'équation d'entropie et débit, la convection est traitée par
un schéma décentré en considérant un débit positif. On considère toutes
les grandeurs connues au pas de temps $n$. La discrétisation type
volumes finis des équations permet d'obtenir les inconnues au pas
de temps $n+1.$

La discrétisation de l'équation de titre n'est pas détaillée ici.
Nous considérons seulement le modèle 3 équations dans les applications
numériques présentées. De plus, dans le cadre considéré ici nous supposons
la porosité est constante et une vitesse relative nulle.

\subsection{Discrétisation de l'équation d'entropie}

La discrétisation de l'entropie pour la cellule $i$ s'écrit en figeant
toutes les grandeurs au pas de temps $n$ (y compris le débit) sauf
l'entropie afin d'obtenir un système linéaire indépendant des autres
inconnues $(p_{i}^{n+1},q_{i}^{n+1},y_{i}^{n+1})$.
\[
a_{i}^{s}s_{i}^{n+1}+m_{i}^{s}s_{i-1}^{n+1}=b_{i}^{s},
\]
avec

\[
a_{i}^{s}=1+\frac{\lambda}{\rho_{i}^{n}}q_{i-1/2}^{n},
\]
\[
m_{i}^{s}=-\frac{\lambda}{\rho_{i}^{n}}q_{i-1/2}^{n},
\]
\[
b_{i}^{s}=s_{i}^{n}+\frac{\phi_{i}^{n}\Delta t}{\rho_{i}^{n}\theta_{l,i}^{n}}.
\]


\subsection{Discrétisation du système couplé pression-débit}

Les équations sur la pression et le débit sont résolues de manière
couplée. Pour l'équation de pression, l'incrément d'entropie (et de
titre pour le modèle 4 équations) est nécessaire. Il est déterminé
par la résolution de l'équation indépendante de l'entropie présenté
précédemment. Pour l'équation de débit, la vitesse de convection est
figée au pas de temps $n$ afin d'obtenir un système linéaire. La
vitesse sur le maillage pression $u_{p,i}$ est nécessaire dans l'équation
de débit. Elle est déterminée par la formule

\[
u_{p,i}^{n}=\frac{q_{i-1/2}^{n}+q_{i+1/2}^{n}}{2\rho_{i}^{n}}
\]

Le système linéaire s'écrit

\[
p_{i}^{n+1}+\lambda(c_{s,i}^{n})^{2}\big(q_{i+1}^{n+1}-q_{i}^{n+1}\big)=p_{i}^{n}-\beta_{i}^{n}(c_{s,i}^{n})^{2}(s_{i}^{n+1}-s_{i}^{n})-\gamma_{i}^{n}(c_{s,i}^{n})^{2}(y_{i}^{n+1}-y_{i}^{n}),
\]

\[
\left(1+\lambda u_{p,i+1}^{n}\right)q_{i+1/2}^{n+1}-\lambda u_{p,i}^{n}q_{i-1/2}^{n+1}+\lambda\big(p_{i+1}^{n+1}-p_{i}^{n+1}\big)=q_{i+1/2}^{n}.
\]


\section{Applications numériques}

\subsection{Cas-test simplifié}

Afin de valider l'implémentation du code, on reprend une solution
analytique stationnaire des équations (\ref{eq:eq_s_num}-\ref{eq:eq_c_num}-\ref{eq:eq_p_num}-\ref{eq:eq_q_num}).
La technique de calcul de cette solution est donnée dans \cite{hurisse2017numerical}.
Nous considérons ici le modèle homogène à 3 équations sans vitesse
relative. La phase vapeur est donc à saturation.

Cette solution est évaluée pour deux cas typiques: cas (1) de calcul
monophasique (débit élevé) et cas (2) diphasique (débit faible). Les
paramètres physiques retenus pour les deux cas sont résumés dans le
Tableau \ref{tab:num_param_1_2}. La longueur du domaine est fixée
à $L=4.6\text{m}.$

\begin{table}[h]
\begin{centering}
\begin{tabular}{|c|c|c|c|}
\hline 
param. & cas (1) & cas (2) & unité\tabularnewline
\hline 
\hline 
débit d'entrée $q_{in}$ & $3500$ & $1500$ & $kg/s$\tabularnewline
\hline 
température d'entrée $\theta_{in}$ & $320$ & $320$ & ${^\circ}C$\tabularnewline
\hline 
pression de sortie $p_{out}$ & $155$ & $155$ & $bar$\tabularnewline
\hline 
flux de chaleur imposé $\phi_{0}$ & $10^{8}$ & $10^{8}$ & $J/s/m^{3}$\tabularnewline
\hline 
\end{tabular}
\par\end{centering}
\caption{Paramètres numériques pour les cas-test (1) et (2).\label{tab:num_param_1_2}}
\end{table}

Dans la Figure \ref{fig:cas1_values} (resp. \ref{fig:cas2_values}),
on représente les variations spatiales de l'état stationnaire (masse
volumique, pression, température, titre) dans le cas (1) (resp. (2)).
On constate la bonne correspondance entre la solution numérique et
la solution analytique. Sur la Figure \ref{fig:convergence_analytique},
on constate la convergence de la méthode numérique vers la solution
analytique pour les deux cas considérés. La convergence est d'ordre
un comme attendu.

\begin{figure}[H]
\begin{centering}
\includegraphics[width=0.49\textwidth]{images/cas1_rho}\includegraphics[width=0.49\textwidth]{images/cas1_pression}
\par\end{centering}
\begin{centering}
\includegraphics[width=0.49\textwidth]{images/cas1_temperature}\includegraphics[width=0.49\textwidth]{images/cas1_titre}
\par\end{centering}
\caption{De gauche à droite et de haut en bas: masse volumique, pression, température
et titre vapeur pour l'état stationnaire du cas (1). La solution numérique
(ligne continue) est comparée avec la solution analytique (ligne pointillée)
pour un maillage de taille $N=100$.\label{fig:cas1_values}}

\end{figure}

\begin{figure}[H]
\begin{centering}
\includegraphics[width=0.49\textwidth]{images/cas2_rho}\includegraphics[width=0.49\textwidth]{images/cas2_pression}
\par\end{centering}
\begin{centering}
\includegraphics[width=0.49\textwidth]{images/cas2_temperature}\includegraphics[width=0.49\textwidth]{images/cas2_titre}
\par\end{centering}
\caption{De gauche à droite et de haut en bas: masse volumique, pression, température
et titre vapeur pour l'état stationnaire du cas (2). La solution numérique
(ligne continue) est comparée avec la solution analytique (ligne pointillée)
pour un maillage de taille $N=100$.\label{fig:cas2_values}}
\end{figure}

\begin{figure}[H]
\begin{centering}
\includegraphics[width=0.49\textwidth]{images/cas1_convergence}\includegraphics[width=0.49\textwidth]{images/cas2_convergence}
\par\end{centering}
\caption{Test de convergence du schéma pour les cas-test (1) (à gauche) et
(2) (à droite) - résidus pour différentes variables: pression, débit,
entropie et titre vapeur (si diphasique).\label{fig:convergence_analytique}}
\end{figure}


\section{Motivation}

Pour atteindre l'état stationnaire, on passe par la résolution de
l'écoulement instationnaire en partant d'un état initial. Si cet état
initial est arbitraire (par exemple un état uniforme), la convergence
vers l'état stationnaire peut être coûteux en temps de calcul. De
plus, le calcul peut ne pas aboutir, ou nécessiter des réductions
drastiques du pas de temps $\Delta t$ pour éviter l'apparition de
valeurs non physiques.

L'initialisation actuelle (modèle 3 équations) de ThermoTorch consiste
à initialiser les champs solutions $(p,s,q)$ à partir de grandeurs
constantes de l'eau liquide hors saturation :

\[
p(x,t=0)=p_{out}\quad;\quad s(x,t=0)=s_{l}(p_{out},\theta_{in})\quad;\quad q(x,t=0)=q_{in}.
\]

La simulation d'un instationnaire fictif est utilisé pour arriver
à la solution stationnaire $(\bar{p},\bar{s},\bar{q})$. Cet instationnaire
nécessite $N_{iter}^{ic}$ itérations lorsque les champs sont initialisés
par des champs constants.

Afin d'accélérer la convergence, nous allons entraîner un réseau de
neurones pour prédire la solution stationnaire à partir des paramètres
d'entrée $(q_{in},\theta_{in},\phi,p_{out})$. Nous utiliserons cette
prédiction $(p^{ml},s^{ml})$ comme initialisation pour le calcul
ThermoTorch. La prédiction peut alors s'écrire

\[
p^{ml}(x)=\bar{p}(x)+\nu_{p}(x)\Delta p,
\]

\[
s^{ml}(x)=\bar{s}(x)+\nu_{s}(x)\Delta s
\]

avec $\Delta p={\displaystyle \max_{x}}(\bar{p}(x)-p_{out})$ et $\Delta s={\displaystyle \max_{x}}(\bar{s}(x)-s_{l}(p_{out},\theta_{in}))$.
Les fonctions $\nu_{s}$ et $\nu{}_{p}$ sont des fonctions évaluant
l'erreur commise par la prédiction par rapport à la solution stationnaire.
Si on note $N_{it}^{ML}$ le nombre d'itérations nécessaires pour
atteindre la convergence avec cette prédiction comme initialisation,
l'objectif est de faire en sorte que le réseau de neurones prédise
la solution avec des erreurs $\nu_{p}$ et $\nu_{s}$ suffisamment
faibles afin d'obtenir $N_{it}^{ML}<N_{it}^{ic}$.

\textbf{Remarque : }Dans ce cadre 1D simple, le débit n'est pour l'instant
pas utilisé, car la solution stationnaire est constante ($\bar{q}(x)=q_{in}$).

Afin de valider le concept et d'évaluer les erreurs $(\nu_{s},\nu_{p})$
nécessaires à une accélération conséquente, un cas-test est réalisé
avec une perturbation générée aléatoirement du type

\[
\nu_{s,p}(x)=P_{s,p}(x)\bar{\nu}
\]

où $P_{s}(x)$ et $P_{p}(x)$ sont des variables aléatoire suivant
une loi uniforme sur l'intervalle $[-1,1]$ et $\bar{\nu}$ est une
constante définissant l'amplitude de la perturbation.

Les cas-tests du Tableau \ref{tab:num_param_1_2} sont utilisés à
nouveau. Le cas-test (1) monophasique nécessite $N_{it}^{ic}=31$
itérations avant convergence avec une initialisation constante et
le cas-test (2) diphasique nécessite $N_{it}^{ic}=58$. Sur la figure
\ref{fig:perturbation_alea}, on représente le nombre d'itérations
nécessaire en fonction de l'amplitude de la perturbation pour les
deux cas-tests. Le calcul a été réalisé 100 fois avec un tirage aléatoire
différent afin d'obtenir des valeurs moyennes. On voit que le nombre
d'itération pour atteindre la convergence diminue avec l'amplitude
de la perturbation comme attendu. A partir de $\bar{\nu}=10^{-2}$,
la diminution du nombre d'itérations est assuré par rapport à l'initialisation
constante pour les deux cas-tests.

\begin{figure}[h]
\begin{centering}
\includegraphics[width=0.48\textwidth]{images/modwav_perturbation_casmono}\includegraphics[width=0.48\columnwidth]{images/modwav_perturbation_casdiph}
\par\end{centering}
\caption{Nombre d'itérations avant convergence en fonction de l'amplitude de
la perturbation à l'initialisation (100 calculs aléatoires moyennés)
pour le cas-test (1) (à gauche) et le cas-test (2) (à droite).\label{fig:perturbation_alea}}
\end{figure}


\section{Optimisation de l'état initial par apprentissage automatique}

Nous proposons dans cette section une technique de prédiction de la
solution stationnaire au moyen d'outils d'apprentissage automatique,
basés sur des réseaux de neurones. L'approche, classique, est la suivante:
\begin{itemize}
\item dans un premier temps, une base de données contenant des ensembles
de paramètres d'entrée $(q_{in},\theta_{in},\phi,p_{out})$ et de
solutions stationnaires numériques correspondantes est obtenue au
moyen du code ThermoTorch avec une initialisation constante.
\item cette base de données est exploitée pour entraîner un réseau de neurone
à prédire la solution stationnaire numérique à partir de paramètre
d'entrée quelconque.
\item le réseau est ensuite utilisé pour initialiser le code dans l'espoir
de diminuer le nombre d'itérations nécessaire à l'obtention de l'état
stationnaire.\\
\end{itemize}
Afin d'évaluer les apports de cette approche, on se place dans un
cadre très simplifié :
\begin{itemize}
\item La taille du maillage est fixée à $N=50$ cellules.
\item Le modèle homogène à 3 équations, sans vitesse relative $u_{r}=0$,
est toujours considéré.\\
\end{itemize}
Le réseau de neurones est programmé grâce à la bibliothèque PyTorch\footnote{\url{https://pytorch.org/}}
de Python. En se basant sur des propositions faites dans la littérature
\cite{mishra2018machine,obiols2020cfdnet,silva2021machine} et après
quelques tests, nous avons retenu les caractéristiques suivantes pour
l'architecture de réseau:
\begin{itemize}
\item Réseau de neurones linéaire dense à 4 couches dont deux couches cachées.
Les quatre couches ont respectivement comme taille 4 (paramètres d'entrée),
200, 200 et 150 (valeurs de sorties car il faut prédire 3 grandeurs
$(p,s,q)$ par cellule sur 50 cellules, les cellules intérieures du
domaine).
\item Les fonctions d'activation sont de type ReLU.
\item La fonction coût (\og loss\fg ) est de type moindre carré (MSELoss
dans PyTorch). L'algorithme de minimisation est l'algorithme de gradient
d'Adam avec les paramètres par défaut de PyTorch (dans sa version
1.13.1). Le taux d'apprentissage initial est fixé à 0.001.\\
\end{itemize}
Afin d'entraîner ce réseau, nous avons généré un dataset avec 10000
résultats de simulations ThermoTorch. Pour cela on a d'abord généré
10000 quadruplets de paramètres $(q_{in},\theta_{in},\phi,p_{out})$
dans l'hypercube $[1000,5000]\times[553.15,593.15]\times[1\times10^{7},5\times10^{8}]\times[140,170]$
au moyen de suites pseudo-aléatoires à discrépance faible. Certaines
contraintes sont ajoutées à la génération afin de considérer toujours
un écoulement liquide en entrée qui ne se vaporise jamais intégralement
dans le milieu. La température est donc limitée à la température de
saturation (à la pression de sortie considérée) et le flux de chaleur
est limité au flux de vaporisation totale (déterminé à partir des
autres conditions limites). Pour chacun de ces jeux de paramètres,
la solution stationnaire est calculée au moyen du code ThermoTorch.
Les paramètres et le jeu de données sont adimensionnés, puis utilisés
pour l’entraînement du réseau. Le réseau est entraîné sur la base
de données avec des batchs de taille 50 et sur 2000 \og epochs\fg{}
afin d'atteindre la convergence de la loss. Cette convergence est
illustrée sur la figure \ref{fig:loss_epoch} qui représente la loss
en fonction de l'epoch.

\begin{figure}[H]
\begin{centering}
\includegraphics[scale=0.5]{images/loss_epoch}
\par\end{centering}
\caption{Fonction de perte loss en fonction de l'epoch.\label{fig:loss_epoch}}
\end{figure}

Sur les Figures \ref{fig:comp_tht_neurones-1} (resp. \ref{fig:comp_tht_neurones-2}),
on compare la solution stationnaire (calculé avec ThermoTorch) avec
la solution prédite avec le réseau de neurones pour le cas (1) (resp.
le cas (2)) dont les paramètres sont indiqués dans le tableau \ref{tab:num_param_1_2}.
Le réseau de neurones parvient à prédire relativement bien la solution.

\begin{figure}[H]
\begin{centering}
\includegraphics[width=0.49\textwidth]{images/casmono_comp_entropy}\includegraphics[width=0.49\textwidth]{images/casmono_comp_pressure}
\par\end{centering}
\begin{centering}
\includegraphics[width=0.49\textwidth]{images/casmono_comp_velocity}
\par\end{centering}
\caption{Comparaison entre le stationnaire ThermoTorch et la prédiction par
réseau de neurones pour le cas (1) - $(q_{in},\theta_{in},\phi,p_{out})=(3500\,kg/s,320{^\circ}C,10^{8}\,J/s/m^{3},155\,bar)$.\label{fig:comp_tht_neurones-1}}
\end{figure}

\begin{figure}[H]
\begin{centering}
\includegraphics[width=0.49\textwidth]{images/casdiph_comp_entropy}\includegraphics[width=0.49\textwidth]{images/casdiph_comp_pressure}
\par\end{centering}
\begin{centering}
\includegraphics[width=0.49\textwidth]{images/casdiph_comp_velocity}
\par\end{centering}
\caption{Comparaison entre le stationnaire ThermoTorch et la prédiction par
réseau de neurones pour le cas (2) - $(q_{in},\theta_{in},\phi,p_{out})=(1500\,kg/s,320{^\circ}C,10^{8}\,J/s/m^{3},155\,bar)$.\label{fig:comp_tht_neurones-2}}
\end{figure}

Nous évaluons maintenant le gain de cette approche pour accélérer
la convergence vers l'état stationnaire. Sur une nouvelle base de
données de paramètres de taille 1000, nous avons évalués le nombre
d'itérations avant convergence pour une initialisation avec la prédiction
du réseau de neurones et une initialisation constante afin de comparer
les deux valeurs. Nous observons sur la Figure \ref{fig:gain_cases}
un gain de $21.6\pm6.7\%$ en nombre d'itérations $\big(g_{it}=\frac{N_{it}^{ic}-N_{it}^{ML}}{N_{it}^{ic}}\big)$,
et cela malgré la simplicité du réseau de neurone utilisé. La figure
\ref{fig:err_cases} représente la répartition du nombre de cas selon
l'amplitude maximale $\max_{x}\Big[\nu_{s,p}(x)\Big]$ de l'erreur
commise par la prédiction par rapport aux champs d'entropie stationnaire
et de pression stationnaire. On remarque que cette précision pourrait
éventuellement être amélioré en optimisant le réseau de neurones.

\begin{figure}[H]
\begin{centering}
\includegraphics[width=1\textwidth]{images/gain_nb_cases_psu}
\par\end{centering}
\caption{Répartition du nombre de cas par gain relatif en nombre d'itérations
sur une base test de 1000 cas.\label{fig:gain_cases}}
\end{figure}

\begin{figure}
\begin{centering}
\includegraphics[width=0.48\textwidth]{images/err_p_nb_cases_psu}\includegraphics[width=0.48\textwidth]{images/err_s_nb_cases_psu}
\par\end{centering}
\caption{Répartition de l'erreur maximale sur la pression et l'entropie sur
une base test de 1000 cas.\label{fig:err_cases}}
\end{figure}

En plus de la réduction du nombre d'itérations, le réseau de neurones
permet de prédire des états proches de la solution stationnaire de
telle sorte que les champs oscillent beaucoup moins lors de la convergence.
On peut observer cet effet sur la Figure \ref{fig:pressure_evolution}
qui représente la pression en 3 points du maillage ($x_{1}=0.88\text{m ; }x_{2}=1.81\text{m ; }x_{3}=2.74\text{ m}$)
au cours du temps pour le cas (1). Le cas avec initialisation constante
est représenté en pointillés alors que le cas avec l'initialisation
prédite est représenté en ligne continue. Cela permettrait éventuellement
d'augmenter le pas de temps et/ou de ne pas explorer des domaines
physiques instables durant la convergence.

\begin{figure}
\begin{centering}
\includegraphics[width=0.6\columnwidth]{images/pressure_variation_diph}
\par\end{centering}
\centering{}\caption{Evolution de la pression en 3 points du maillage ($x_{1}=0.88\text{m ; }x_{2}=1.81\text{m ; }x_{3}=2.74\text{ m}$)
au cours du temps pour le cas-test (1). \label{fig:pressure_evolution}}
\end{figure}


\section{Conclusion}

Dans ce travail, nous avons présenté un modèle 1D simplifié d’un écoulement
thermohydraulique compressible diphasique eau-vapeur avec changement
de phase. Ce modèle est résolu numériquement par un schéma semi-implicite
en (entropie, débit-pression). La formulation du schéma numérique
permet de résoudre séparément l'équation d'entropie des autres inconnues,
ce qui accélère l'algorithme. Le schéma est écrit sur une grille décalée
afin de garantir une bonne précision dans la limite de faible Mach.

Afin d'accélérer la convergence numérique vers l'état stationnaire,
nous avons réalisé des tests préliminaires sur une méthode d'initialisation
par apprentissage automatique.

Les premiers résultats sur un modèle très simplifié (modèle homogène
à 3 équations, sans vitesse relative) sont prometteurs avec un premier
réseau de neurones proposé. Nos perspectives sont d'améliorer la réseau
de neurones pour avoir une prédiction plus précise puis de complexifier
le modèle physique (vitesse relative non nulle, ajout de l'équation
de bilan de masse vapeur afin de prendre en compte l'ébullition sous-saturée)
afin de valider cette méthode d’accélération sur une palette plus
large de cas et en dimension supérieure.

\section{Annexes}

\subsection{Annexe 1: loi d'état thermodynamique}

Dans cette annexe, on détaille le calcul pratique des coefficients
des lois thermodynamiques. Pour chaque phase (vapeur et liquide),
une entropie de gaz raide est utilisée

\[
s(\tau,e)=s_{0}+c_{v}\ln\Big(\big(e-h_{0}-p_{\infty}\tau\big)\tau^{\gamma-1}\Big).
\]
Les paramètres ($\gamma,c_{v},p_{\infty},h_{0},s_{0}$) sont pour
l'instant inconnus. Il y a donc 10 paramètres (5 par phase) à choisir
pour établir le modèle thermodynamique complet. Il est impossible
de construire un modèle précis sur une vaste plage de températures
et de pressions voir \cite{hurisse2022simulations}. Il faut se contenter
d'un domaine dans un voisinage du régime de fonctionnement de la centrale,
c'est à dire $\theta\simeq310{^\circ}C$ et $p\simeq155$ bar.

Des calculs (qui sont détaillés par exemple dans \cite{barberon2005finite})
permettent de déterminer la température
\[
c_{v}\theta=e-h_{0}-p_{\infty}\tau,
\]
la pression
\[
p=(\gamma-1)\rho c_{v}\theta-p_{\infty},
\]
et l'enthalpie
\[
h=h_{0}+\gamma c_{v}\theta=h_{0}+c_{p}\theta.
\]
Par conséquent dans ce modèle la capacité thermique à volume constant
$c_{v}$ et la capacité thermique à pression constante $c_{p}$ sont
reliées par
\[
c_{p}=\gamma c_{v}.
\]
La vitesse du son est donnée par
\[
c_{s}^{2}=\gamma(p+p_{\infty})/\rho.
\]
Il est alors possible de calculer le potentiel chimique
\[
\mu(p,\theta)=h-\theta s.
\]
 Cette expression définit la température de saturation $\theta_{s}(p)$,
qui est donnée par
\begin{equation}
\mu_{v}(p,\theta_{s}(p))=\mu_{l}(p,\theta_{s}(p)).\label{eq:def_saturation}
\end{equation}

sudo apt install network-manager-fortisslvpn network- manager-fortisslvpn-gnome

\subsubsection{Calcul des coefficients}

L'organisation IAPWS (International Association for the Properties
of Water and Steam) \footnote{\url{http://www.iapws.org/}} propose
des formulations numériques précises des lois de comportement de l'eau
liquide ou vapeur. Ces formulations décrites dans \cite{wagner2008iapws}
sont considérées comme une référence pour établir d'autres lois de
comportement plus simples. Il existe des packages Python \footnote{voir par exemple \url{https://pypi.org/project/iapws/}}
qui implémentent ces formulations et qui permettent de réaliser le
calage numérique des coefficients.

Deux points (a) et (b) de températures différentes sont considérés
à pression $p=155$ bar:
\begin{itemize}
\item Point (a) à température $\theta_{l}=310{^\circ}C$ pour caler les
grandeurs liquide.
\item Point (b) à température $\theta_{g}=350{^\circ}C$ pour caler les
grandeurs vapeur.
\end{itemize}
On considère les pressions et températures suivantes
\[
p^{(a)}=p^{(b)}=155\text{ bar},\quad\theta^{(a)}=310{^\circ}C,\quad\theta^{(b)}=350{^\circ}C.
\]
Par ailleurs, pour fixer la dérivée de la température de saturation,
nous aurons besoin de la pression au point (c) :
\[
p^{(c)}=150\text{ bar}.
\]


\paragraph{Eau liquide}

Les paramètres à définir sont $\gamma_{l},c_{v,l},p_{\infty,l},h_{0,l},s_{0,l}$.
Les relations suivantes sont utilisées :
\begin{itemize}
\item Les paramètres $(\gamma_{l},p_{\infty,l})$ sont choisis pour imposer
la vitesse du son $c_{s}^{(a)}$ et la masse volumique $\rho^{(a)}$
au point (a), ce qui donne
\end{itemize}
\[
p_{\infty,l}=\frac{\rho^{(a)}(c_{s}^{(a)})^{2}}{\gamma_{l}}-p^{(a)},\gamma_{l}=1+\frac{p_{\infty,l}}{\rho^{(a)}c_{v}^{(a)}\theta^{(a)}}.
\]

\begin{itemize}
\item La capacité thermique est fixée à $c_{v,l}=c_{v}^{(a)}$.
\item L'enthalpie de référence est $h_{0,l}=h^{(a)}-\gamma_{l}c_{v,l}\theta^{(a)}$.
\item L'entropie de référence ${\displaystyle s_{0,l}=s^{(a)}-c_{v,l}\ln\Big(\frac{p^{(a)}+p_{\infty,l}}{\gamma_{l}-1}\cdot(\rho^{(a)})^{-\gamma_{l}}\Big)}$.
\end{itemize}

\paragraph{Eau vapeur}

Les paramètres à définir sont $\gamma_{g},c_{v,g},p_{\infty,g},h_{0,g},s_{0,g}$.
Les relations suivantes sont utilisées :
\begin{itemize}
\item Les paramètres $(\gamma_{v},p_{\infty,v})$ sont choisis pour imposer
la vitesse du son $c_{s}^{(b)}$ et la masse volumique $\rho^{(b)}$
au point (b), ce qui donne
\end{itemize}
\[
p_{\infty,v}=\frac{\rho^{(b)}(c_{s}^{(b)})^{2}}{\gamma_{v}}-p^{(b)},\gamma_{v}=1+\frac{p_{\infty,v}}{\rho^{(b)}c_{v}^{(b)}\theta^{(b)}}.
\]

\begin{itemize}
\item La capacité thermique est fixée à $c_{v,l}=c_{v}^{(a)}$.
\item Contrairement au cas liquide, les enthalpies et entropie de référence
$s_{0,v}$ et $h_{0,v}$ sont calculées afin de respecter certains
points de la courbe de saturation $\theta_{s}(p)$ définie par (\ref{eq:def_saturation}).
Précisément, on impose
\end{itemize}
\[
\theta_{s}(p^{(a)})=\theta_{s}^{(a)},\quad\theta_{s}(p^{(c)})=\theta_{s}^{(c)}.
\]
Les paramètres obtenus numériquement sont détaillés dans le tableau
\ref{tab:param_gaz_raide}.

\begin{table}
\centering{}%
\begin{tabular}{|c|c|c|c|}
\hline 
Paramètre & Liquide & Vapeur & Unité\tabularnewline
\hline 
$p_{\infty}$ & 433027888.73886645 & 2265618.76 & Pa\tabularnewline
\hline 
$\gamma$ & 1.347721005 & 1.092548 & -\tabularnewline
\hline 
$c_{v}$ & 3030.1475144213396 & 3273.937158624049 & J/kg/K\tabularnewline
\hline 
$h_{0}$ & -987900.1770384915 & 377353.1095 & J/kg\tabularnewline
\hline 
$s_{0}$ & -33462.74606510723 & -41040.96603 & J/kg/K\tabularnewline
\hline 
\end{tabular}\caption{Paramètres obtenus pour les lois de gaz raide (phase liquide et phase
gazeuse). \label{tab:param_gaz_raide}}
\end{table}


\subsubsection{Coefficient de la linéarisation de la masse volumique}

\global\long\def\pderc#1#2#3{{\displaystyle \left(\frac{\partial#1}{\partial#2}\right)_{#3}}}%
 
\global\long\def\der#1#2{\frac{d#1}{d#2}}%

Afin d'obtenir l'équation de pression à partir de l'équation de masse
volumique, la linéarisation \ref{eq:rho_linearisation} est utilisée.
Les coefficients de cette linéarisation sont décrits dans la suite.
Ces coefficients utilisent les dérivées de la masse volumique de chaque
phase par rapport à la pression ou à l'entropie de cette même phase.
Ces dérivées sont obtenues en utilisant la loi d'état des gaz raide
de chaque phase tel que

\[
\pderc{\rho_{k}}p{s_{k}}=\frac{\rho_{k}}{\gamma_{k}(p+p_{\infty,k})},
\]

\[
\pderc{\rho_{k}}{s_{k}}p=-\frac{\rho_{k}}{\gamma_{k}c_{v,k}}.
\]

Ces formules utilisent aussi la dérivée totale de l'entropie vapeur
par rapport à la pression qui est obtenue à partir de la pression
et de la température de saturation

\[
\frac{ds_{v}}{dp}=c_{v,v}\Big(\frac{\gamma_{v}}{\theta_{sat}(p)}\frac{d\theta_{sat}}{dp}(p)-\frac{\gamma_{v}-1}{p+p_{\infty,v}}\Big)
\]


\paragraph{Coefficient $\frac{1}{c^{2}}$ :}

\[
\frac{1}{c_{s}^{2}}=\rho^{2}\cdot\Bigg[\frac{y}{\rho_{v}^{2}}\left(\pderc{\rho_{v}}p{s_{v}}+\pderc{\rho_{v}}{s_{v}}p\cdot\der{s_{v}}p\right)+\frac{1-y}{\rho_{l}^{2}}\left(\pderc{\rho_{l}}p{s_{l}}+\pderc{\rho_{l}}{s_{l}}p\cdot\frac{y}{1-y}\cdot\der{s_{g}}p\right)\Bigg].
\]


\paragraph{Coefficient $\beta$ :}

\[
\beta=\frac{\rho^{2}}{\rho_{l}^{2}}\pderc{\rho_{l}}{s_{l}}p.
\]


\paragraph{Coefficient $\gamma$ :}

\[
\gamma=\rho^{2}\left(\tau_{l}-\tau_{v}+\frac{s_{l}-s_{v}}{\rho_{l}^{2}}\pderc{\rho_{l}}{s_{l}}p\right).
\]

\newpage

\bibliographystyle{plain}
\bibliography{thermo_torch_fr}


\end{document}
