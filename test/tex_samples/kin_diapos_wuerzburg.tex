%translatex_start_here


\documentclass[english]{beamer}
\usepackage[T1]{fontenc}
\usepackage[utf8]{inputenc}
\setcounter{secnumdepth}{3}
\setcounter{tocdepth}{3}
\usepackage{textcomp}
\usepackage{amstext}
\usepackage{amssymb}
\usepackage{mathtools}
\usepackage{graphicx}

\makeatletter

%%%%%%%%%%%%%%%%%%%%%%%%%%%%%% LyX specific LaTeX commands.
%% A simple dot to overcome graphicx limitations
\newcommand{\lyxdot}{.}

%%%%%%%%%%%%%%%%%%%%%%%%%%%%%% Textclass specific LaTeX commands.
% this default might be overridden by plain title style
\newcommand\makebeamertitle{\frame{\maketitle}}%
% (ERT) argument for the TOC
\AtBeginDocument{%
  \let\origtableofcontents=\tableofcontents
  \def\tableofcontents{\@ifnextchar[{\origtableofcontents}{\gobbletableofcontents}}
  \def\gobbletableofcontents#1{\origtableofcontents}
}


%%%%%%%%%%%%%%%%%%%%%%%%%%%%%% User specified LaTeX commands.
\AtBeginSection[]{
  \begin{frame}
  \vfill
  \centering
  \begin{beamercolorbox}[sep=8pt,center,shadow=true,rounded=true]{title}
    \usebeamerfont{title}\insertsectionhead\par%
  \end{beamercolorbox}
  \vfill
  \end{frame}
}

\setbeamertemplate{navigation symbols}{}

\addtobeamertemplate{navigation symbols}{}{%
    \usebeamerfont{footline}%
    \usebeamercolor[fg]{footline}%
    \hspace{1em}%
    \insertframenumber/\inserttotalframenumber
}

\usepackage{tikz}

\setbeamertemplate{bibliography item}{\insertbiblabel}

\makeatother

\usepackage{babel}
\makeatletter
\addto\extrasfrench{%
   \providecommand{\og}{\leavevmode\flqq~}%
   \providecommand{\fg}{\ifdim\lastskip>\z@\unskip\fi~\frqq}%
}

\makeatother

\synctex=1

 \title{Kinetic Approximations}
% Remove the duplicate \author command
\author{Philippe Helluy}
\institute{University of Strasbourg, IRMA CNRS, Inria Tonus}
\date{Würzburg, February 2024}
\begin{document}
\makebeamertitle
\begin{frame}{Plan}

\tableofcontents{}
\end{frame}

\section{Convection-Diffusion Equation}
\begin{frame}{Diffusion Equation}

Consider the diffusion (or heat) equation 
\[
w_{t}-\mu w_{xx}=0,
\]
where: 
\begin{itemize}
\item the unknown $w(x,t)$ is a function of $x \in \mathbb{R}$ and time
$t$,
\item $w_{t}=\frac{\partial w}{\partial t}$, $w_{x}=\frac{\partial w}{\partial x}$,
\end{itemize}
with an initial condition
\[
w(x,0)=w_{0}(x).
\]
The parameter $\mu$ is the diffusion coefficient.
\end{frame}
%
\begin{frame}{Fourier Transform}

The Fourier Transform on $\mathbb{R}$ is defined by ($i^2=-1$)
\[
\hat{w}(\xi)=\int_{x=-\infty}^{x=+\infty}w(x)\exp(-ix\xi)dx.
\]
Convolution is defined by
\[
(f*g)(x)=\int_{x=-\infty}^{x=+\infty}f(x-y)g(y)dy.
\]
Some properties: 
\begin{itemize}
\item $w(x)=\frac{1}{2\pi}\int_{\xi=-\infty}^{\xi=+\infty}\hat{w}(\xi)\exp(+ix\xi)d\xi$
(Inverse Fourier Transform)
\item $\int\left|w\right|^{2}=\int\left|\hat{w}\right|^{2}$ (Parseval's equality, the Fourier Transform is an isometry of $L^{2}(\mathbb{R})$)
\item $(f*g)\hat{\,}=\hat{f}\hat{g}$ (transform of the convolution into a product)
\end{itemize}
\end{frame}
%
\begin{frame}{Exact Solution}

Fourier Transform in $x$ ($\partial_x \rightarrow i\xi$)
\[
\hat{w}_{t}=-\mu\xi^{2}\hat{w}
\]
so
\[
\hat{w}(\xi,t)=\exp(-\mu\xi^{2}t)\hat{w}_{0}.
\]

Remarks:
\begin{itemize}
\item Energy decreases if $\mu>0$ (increases otherwise)
\item Convolution in $x$
\[
w=E(\cdot,t)*w_{0},\text{ with }E(x,t)=\frac{1}{2\sqrt{\pi\mu t}}\exp(\frac{-x^{2}}{4\mu t}).
\]
\item Smoothing effect when $\mu>0$.
\end{itemize}
Issue if $\mu<0$. Suppose that the spectrum of $w_{0}$ (the support of $\hat{w}_{0})$ is bounded, included in the interval $[-\phi,\phi]$,
then
\[
\left\Vert w(\cdot,t)\right\Vert _{L^{2}}\leq\exp(\left|\mu\right|\phi^{2}t)\left\Vert w(\cdot,0)\right\Vert _{L^{2}},
\]
but this estimate cannot be improved: the solution 'explodes' in time. 
\end{frame}
%translatex_split_here
 %
\begin{frame}{Convection Equation}

Convection (or transport) equation with velocity $c$
\[
w_{t}+cw_{x}=0,\quad w(\cdot,0)=w_{0}(\cdot).
\]
Fourier Transform
\[
\hat{w}_{t}=-ic\xi\hat{w}.
\]
We find
\[
\hat{w}(\xi,t)=\exp(-ic\xi t)\hat{w}(\xi,0).
\]
Hence (Fourier shift)
\[
w(x,t)=w_{0}(x-ct).
\]

\end{frame}

%

\begin{frame}{Convection-Diffusion}

For the convection-diffusion equation
\[
w_{t}+cw_{x}-\mu w_{xx}=0,\quad w(x,0)=w_{0}(x),
\]
where $\mu$ is the viscosity coefficient, we find
\[
w=E(\cdot,t)*w_{0},\text{ with }E(x,t)=\frac{1}{2\sqrt{\pi\mu t}}\exp(\frac{-(x-ct)^{2}}{4\mu t}).
\]

$$w(x,t)=\int_{y}E(y,t)w_{0}(x+y)dy,$$
and
$$E\geq0,\quad\int_{x}E(x,t)dx=1.$$
This is a convex combination of values of $w_{0}$.

\end{frame}

%translatexsplithere
 %
\begin{frame}{Stability}

From the previous formula, we can deduce:
\begin{itemize}

\item Maximum principle: if $0\leq w_{0}\leq M$ then $0\leq w(\cdot,t)\leq M,$
$t>0$.



\item Decay of energy $\mathcal{E}(t)=\int_{x}w(x,t)^{2}dx$: $\mathcal{E}(t)\leq\mathcal{E}(0)$, $t>0$.
\end{itemize}
\end{frame}




\begin{frame}{Upwind scheme}

We consider the transport equation
\[
w_{t}+cw_{x}=0,\quad x\in\mathbb{R},\quad t\geq0,
\]
with initial condition $w(x,0)=w_{0}(x)$ and $c>0$.

Time step $\tau$, space step $h$. Discretization at points
$x_{i}=ih$, $t_{n}=n\tau$, $w_{i}^{n}\simeq w(x_{i},t_{n})$. Upwind scheme, $w_{i}^{0}=w(x_{i},0)$ and
\[
\frac{w_{i}^{n+1}-w_{i}^{n}}{\tau}+c\frac{w_{i}^{n}-w_{i-1}^{n}}{h}=0.
\]
Very natural: information comes from the left. 
\end{frame}
%

\begin{frame}{Maximum principle}

We introduce the CFL number $\beta=c\tau/h$. Then:
\[
w_{i}^{n+1}=(1-\beta)w_{i}^{n}+\beta w_{i-1}^{n}.
\]
Under the condition $\beta\leq1$ we have the discrete maximum principle.
If for all $i$, $0\leq w_{i}^{0}\leq M$ then for all $i$ and
$n>0$, $0\leq w_{i}^{n}\leq M$.
\end{frame}


\begin{frame}{Equivalent Equation}

  We can construct a continuous version of the previous scheme. We seek
  a function $\tilde{w}(x,t)$ (which we still denote $w$) that solves
  the difference equation
  \[
  \frac{w(x,t+\tau)-w(x,t)}{\tau}+c\frac{w(x,t)-w(x-h,t)}{h}=0.
  \]
  This solution coincides with the discrete solution at the points $(x,t)=(x_{i},t_{n})$.
  What does $w$ satisfy formally when $h$ and $\tau$ tend to 0
  with $c\tau/h=\beta$ fixed?
  \end{frame}
%
\begin{frame}{Energy stability}

Shift operator (notation: $I^{2}=-1)$
\[
(\mathcal{D}_{h}w)(x)=w(x-h),\quad(\mathcal{D}_{h}w)^{\wedge}(\xi)=\exp(-Ih\xi)\hat{w}(\xi).
\]
The finite difference equation becomes, with $c\tau/h=\beta,$
\[
\hat{w}(\xi,t+\tau)=A(\xi,h)\hat{w}(\xi,t),
\]
with $A(\xi,h)=(1-\beta+\beta e^{-Ih\xi})$, the amplification coefficient.
The scheme is stable in $L^{2}$ iff $A$ is in the unit disk
for all frequencies $\xi$. We retrieve the condition
\[
\beta\leq1.
\]

\end{frame}
%translatex_split_here
 %
%
\begin{frame}{Using Fourier}

Shift operator (notation: $I^{2}=-1)$
\[
(\mathcal{D}_{h}w)(x)=w(x-h),\quad(\mathcal{D}_{h}w)^{\wedge}(\xi)=\exp(-Ih\xi)\hat{w}(\xi).
\]
The difference equation becomes, with $c\tau/h=\beta,$
\[
\hat{w}(\xi,t+\tau)=A(\xi,\tau)\hat{w}(\xi,t),
\]
with $A(\xi,h)=(1-\beta+\beta e^{-Ih\xi})$. So we have
\[
\frac{\hat{w}(\xi,t+\tau)-\hat{w}(\xi,t-\tau)}{2\tau}+\frac{1}{2\tau}\left(\frac{1}{A(\xi,-\tau)}-A(\xi,\tau)\right)\hat{w}(\xi,\tau)=0.
\]
With a Taylor expansion at $\tau=0$ and inverse Fourier transform, we find
\[
w_{t}+cw_{x}-\frac{c}{2}(1-\beta)hw_{xx}=0+O(h^{2}).
\]
The upwind scheme introduces a numerical viscosity $\mu=\frac{c}{2}(1-\beta)h.$
The consistency is therefore of order 1. We recover the CFL stability condition.
\end{frame}
%
\begin{frame}{Remark on the equivalent equation}

The equivalent equation often provides information on the CFL stability, but not always \cite{dhaouadi2021stability}. Example: heat equation
\[
w_{t}-w_{xx}=0,
\]
discretized by the classical explicit scheme
\[
\frac{u(x,t+\tau)-u(x,\tau)}{\tau}+\frac{-u(x-h,\tau)+2u(x,\tau)-u(x+h,\tau)}{h^{2}}=0.
\]
The equivalent equation is
\[
u_{t}-u_{xx}-\frac{1}{12}(1-6\beta)h^{2}u_{xxxx}=O(h^{4}),
\]
which is stable under the condition $\beta>1/6$ while the scheme is stable if $\beta<1/2$!
\end{frame}
%
%translatex_split_here
 \section{Hyperbolic Systems}
\begin{frame}{Conservation Laws}

First-order conservation laws system (CLS). Notation convention:  vectors and matrices with
capital letters, scalars with lowercase letters.
\[
W_{t}+\sum_{i=1}^{d}\partial_{i}Q^{i}(W)=0,
\]

\begin{itemize}
\item Unknown vector: $W(X,t)\in\mathbb{R}^{m}$, $X=(x^{1},\ldots,x^{d})\in\mathbb{R}^{d}$ space variable, $t\geq0,$ time variable;
\item $\partial_{i}=\frac{\partial}{\partial x^{i}}$. If $d=1$ we note
$w=W$, $x^{1}=x$, $Q^{1}(W)=q(w)$ and $\partial_{1}Q^{1}(W)=q(w)_{x}$.

\item $Q^{i}(W)$: flux in the direction $i$. If $Q^{i}(W,\nabla_{X}W)$: second-order system...
\end{itemize}
For a spatial vector $N\in\mathbb{R}^{d}$ we can also define
the flux in the direction $N$ by
\[
Q(W,N)=\sum_{i=1}^{d}Q^{i}(W)\cdot N_{i}(W)=Q(W)\cdot N(W).
\]

\end{frame}
%
\begin{frame}{Conservation ?}

Integrate the CLS over a space domain $\Omega$ and note the "mass"
contained in this domain at time $t$
\[
M(t)=\int_{X\in\Omega}W(X,t).
\]
The Stokes formula leads to
\[
\frac{d}{dt}M(t)=\int_{X\in\partial\Omega}Q(W(X,t),N(X)),
\]
where $N(X)$ is the outward normal vector to $\Omega$ at point $X$
on the boundary $\partial\Omega$.

In other words, the variation of the mass in the domain over time is given by
the integral of the flux on the boundary.

\end{frame}
%

\begin{frame}{Hyperbolicity}

The CLS is hyperbolic if for all directions $N$ and all vector
$W$ the Jacobian matrix of the flux
\[
A(W,N)=D_{W}Q(W,N)
\]
is diagonalizable with real eigenvalues. We note $\lambda_{i}(W,N)$
the eigenvalues (often arranged in ascending order) and $R_{i}(W)$
the corresponding eigenvectors.

Note that in the scalar case $m=1$ the system is necessarily hyperbolic.
\end{frame}
%
%translatex_split_here

\begin{frame}{Hyperbolicity?}

Consider the linear CLS $W=(a,b)^{\intercal}$
\[
\partial_{t}\left(\begin{array}{c}
a\\
b
\end{array}\right)+\partial_{x}\left(\left(\begin{array}{cc}
0 & \epsilon\\
1 & 0
\end{array}\right)\left(\begin{array}{c}
a\\
b
\end{array}\right)\right)=0,\quad\epsilon=\pm1.
\]
In Fourier space ($I^2=-1$)
\[
IM(\xi,\tau)\left(\begin{array}{c}
\hat{a}(\xi,\tau)\\
\hat{b}(\xi,\tau)
\end{array}\right)=0,\quad M(\xi,\tau)=\left(\begin{array}{cc}
\tau & \epsilon\xi\\
\xi & \tau
\end{array}\right).
\]
There are non-trivial solutions if and only if $\det M(\xi,\tau)=0$ which gives
\[
\tau{{}^2}-\epsilon\xi^{2}=0.
\]
If $\epsilon=1$, this resembles the equation of a hyperbola and the system is said to be hyperbolic. If $\epsilon=-1$, the system is said to be elliptic.
\end{frame}
%
\begin{frame}{Examples: transport, Burgers}

Consider $d=1$, $m=1$, and $q(w)=cw$. This gives the 1D transport equation
\[
w_{t}+cw_{x}=0.
\]
The eigenvalue $\lambda_{1}=c$.

The Burgers equation is obtained by choosing $q(w)=w^{2}/2$. This yields
\[
w_{t}+\left(\frac{w^{2}}{2}\right)_{x}=0.
\]
For smooth solutions, the Burgers equation can also be written
\[
w_{t}+ww_{x}=0.
\]
Here, 
\[
\lambda_{1}(w)=w.
\]
In the Burgers equation, the wave speed is also the unknown conservative quantity $w$.
\end{frame}
%

%translatex_split_here
 %
\begin{frame}{Example: Traffic Flow}

Vehicle density on a highway lane $w(x,t)\geq0$. Vehicle speed $v=v(w)$. Conservation law of vehicles
\[
w_{t}+(v(w)w)_{x}=0.
\]
The flux is therefore
\[
q(w)=wv(w).
\]
Vehicle driver behavior law. For a maximum density $w=w_{
m ax}$, the speed $v(w_{
m ax})=0$. For a very fluid traffic, drivers travel at the maximum allowed speed $v(0)=v_{
m ax}$. Therefore, we can take
\[
v(w)=(1-\frac{w}{w_{
m ax}})v_{
m ax}.
\]
Here the wave speed is therefore
\[
\lambda(w)=q'(w)=(1-\frac{2w}{w_{
m ax}})v_{
m ax}\in\left[-v_{
m ax},v_{
m ax}\right].
\]

\end{frame}

%
\begin{frame}{Other Examples}
\begin{itemize}
\item Saint-Venant Model (or shallow water): $m=2$, $d=1$, water height $h(x,t)$, mean horizontal  velocity $u(x,t)$, gravity $g=9.81$m/s².
$$
W=\left(\begin{array}{c}
  h\\
  hu
  \end{array}\right),\quad Q^{1}(W)=\left(\begin{array}{c}
  hu\\
  hu^{2}+gh^{2}/2
  \end{array}\right),
$$
$$
\partial_{t}W+\partial_{x}Q^1(W)=0.
$$
\item Compressible Gas;
\item Maxwell's Equations;
\item Multiphase Fluid;
\item MHD Equations;
\item \textsl{etc.}
\end{itemize}
\end{frame}
%
\begin{frame}{Method of Characteristics}

Consider a scalar 1D conservation law ($m=1$, $d=1$)
\[
w_{t}+q(w)_{x}=0.
\]
Characteristic curve: parameterized curve $t	\mapsto(x(t),t)$ in the $(x,t)$ plane along which $w$ is constant
\[
\frac{d}{dt}w(x(t),t)=0.
\]
We find that $x'(t)=q'(w(x(t),t)=q'(w(x(0),0)$ is constant. The characteristics are therefore straight lines. This allows to compute the solutions (strong solutions).
\end{frame}
%translatex_split_here
 %
\begin{frame}{Critical Time}
\begin{itemize}
\item Transport: if $q(w)=cw$ then $x(t)=ct+x_{0}.$ Therefore $w(x,t)=w(x(0),0)=w(x-ct,0).$
\item Burgers: if $q(w)=w^{2}/2$ then $x(t)=w(x_{0},0)t+x_{0}$. If the initial condition is decreasing and $q$ convex, one can see that the characteristics intersect while transporting different values of $w$. The strong solution ceases to exist after a certain time that can be calculated as:
\[
t=\frac{-1}{\inf_{x}q'(w_{0}(x))}.
\]
The concept of a strong solution is not sufficient. It will be necessary to generalize.
\end{itemize}
\end{frame}
%
\begin{frame}{Hyperbolicity and Transport}

Hyperbolicity is a necessary condition for stability. Example: a one-dimensional ($d=1$) linear CLS with constant coefficients:
\[
W_{t}+Q(W)_{x}=0,\quad Q(W)=AW.
\]
If $A$ is diagonalizable with real eigenvalues
\[
\text{diag}(\lambda_{1},\ldots,\lambda_{m})=\Lambda=R^{-1}AR,
\]
where the columns of $R$ are the eigenvectors $R_{i}$. Positing $W=PY$, we have
\[
Y_{t}+\Lambda Y_{x}=0
\]
and each component $Y^{i}$ of $Y$ is a solution to a transport equation with velocity $\lambda_{i}$. The eigenvalues can be interpreted as wave speeds.
\end{frame}
%
\begin{frame}{Hyperbolicity and Stability}

If an eigenvalue $\lambda_{i}$ is not real, that is $\lambda_{i}=a+Ib$, $b\neq0$. $y=Y^{i}$ is a solution to the transport equation
\[
y_{t}+(a+Ib)y_{x}=0.
\]
In Fourier space:
\[
\hat{y}_{t}+(a+Ib)I\xi\hat{y}=0.
\]
This implies that
\[
\hat{y}(\xi,t)=e^{-Ia\xi t}e^{b\xi t}\hat{y}(\xi,0).
\]
High-frequency modes are exponentially unstable...
\end{frame}
%
%translatex_split_here

\begin{frame}{Weak Solution}

Definition: $W(X,t)$ is a weak solution of $W_{t}+\nabla_{X}\cdot Q(W)=0,$
$W(X,0)=W_{0}(X)$ if for any regular test function $\varphi(X,t)$ with bounded support,
\[
\int_{X,t\geq0}\left(W\varphi_{t}+Q(W)\cdot\nabla_{X}\varphi\right)=\int_{X}W_{0}\varphi(\cdot,0).
\]

By integration by parts: strong $\Rightarrow$ weak and weak
+ regular $\Rightarrow$ strong.

What happens in the weak + discontinuous case?
\end{frame}
%
\begin{frame}{Rankine-Hugoniot}

Weak solution with discontinuity on a surface $\Sigma$ of the space-time
(``shock''). Normal vector $(N,n_{t})$ to this surface, oriented
from side $L$ to side $R$. We note $[a]=a_{R}-a_{L}$ the jump
of the quantity $a$ across the discontinuity.

Rankine-Hugoniot relations:
\[
n_{t}[W]+N\cdot[Q(W)]=0.
\]
If $N$ is a unit spatial vector then $n_{t}=-\sigma$ where
$\sigma$ is the normal speed of the discontinuity. We find
\[
\sigma[W]=N\cdot[Q(W)].
\]
 Caution: some calculations are no longer valid for weak solutions.
For example, if $w$ is a weak solution of $w_{t}+(w^{2}/2)_{x}=0$,
$w$ is not necessarily a weak solution of $(w^{2}/2)_{t}+(w^{3}/3)_{x}=0.$
\end{frame}
%
\begin{frame}{Loss of Uniqueness}

There is no uniqueness of weak solutions for the Cauchy problem.
Example (with Burgers $q(w)=w^{2}/2$):
\[
w_{t}+q(w)_{x}=0,
\]
\[
w(x,0)=\begin{cases}
0 & \text{if }x<0,\\
1 & \text{otherwise.}
\end{cases}
\]
At least two weak solutions:
\[
w_{1}(x,t)=\begin{cases}
0 & \text{if }x<t/2,\\
1 & \text{otherwise.}
\end{cases}
\]
\[
w_{2}(x,t)=\begin{cases}
0 & \text{if }x<0,\\
1 & \text{if }x>t,\\
x/t & \text{otherwise.}
\end{cases}
\]
We only keep the second solution (as it is less ``discontinuous'').
\end{frame}
%
%translatex_split_here

\begin{frame}{Lax Characteristic Criterion}

There is no need to introduce a shock when the characteristics do not intersect. A shock of velocity $\sigma$ satisfies the Lax characteristic criterion ($m=1$, $d=1$) if
\[
q'(w_{L})>\sigma>q'(w_{R}).
\]
In the case $m>1$, $d>1$, the Lax characteristic criterion becomes:
there exists an index $i$ such that
\[
\lambda_{i}(w_{L},N)>\sigma>\lambda_{i}(w_{R},N).
\]
Here, $N$ is the normal vector to the discontinuity surface, unitary, and oriented from $L$ to $R$.
\end{frame}
%
\begin{frame}{Entropy}

The characteristic criterion is geometric. Not practical for numerics.
We seek an integral criterion.

An entropy $s(W)$ associated with the entropy flux $G(W)$ is a function
that satisfies an additional conservation law
\[
s(W)_{t}+\sum_{i}\partial_{i}G^{i}(W)=0
\]
when $W$ is a strong solution.

Then, setting $A^{i}(W)=D_{W}Q^{i}(W)$,
\[
D_{W}s(W)A^{i}(W)=D_{W}G^{i}(W).
\]
For $m=1$ any function is an entropy. It is more complicated if $m>1$.
\end{frame}
%
\begin{frame}{Practical Calculation}

As we work with strong solutions, we can change variables.
If $W=W(Y)$
\[
D_{Y}WY_{t}+A^{i}D_{Y}W\partial_{i}Y=0,\quad A^{i}=D_{W}Q^{i},
\]
which implies
\[
Y_{t}+B^{i}(Y)\partial_{i}Y=0,\quad B^{i}=P^{-1}A^{i}P,\quad P=D_{Y}W.
\]
With $s(W)=u(Y)$ and $G^{i}(W)=H^{i}(Y)$, we have
\[
D_{Y}uB^{i}=D_{Y}H^{i}.
\]

\end{frame}
%



\begin{frame}{Example: Saint-Venant}

Saint-Venant equations, $m=2$, $d=1$, water height $h$, velocity $u$, gravity $g=9.81$m/s².
\[
W=\begin{pmatrix}
h\\
hu
\end{pmatrix},\quad Q(W)=\begin{pmatrix}
hu\\
hu^{2}+gh^{2}/2
\end{pmatrix}.
\]
By performing calculations in variables $Y=(h,u)^{\top}$, we find (non-unique solution)
\[
s(W)=h\frac{u^{2}}{2}+\frac{gh^{2}}{2},\quad G(W)=h\frac{u^{3}}{2}+ugh^{2}.
\]

\end{frame}
%
\begin{frame}{Lax Entropy}

An entropy $s(W)$ is a Lax entropy if $s$ is strictly convex with respect to $W$. A weak solution is a Lax solution if, in the weak sense,
\[
s(W)_{t}+\partial_{i}G^{i}(W)\leq0.
\]
Lax entropy criterion for shocks
\[
n_{t}[s(W)]+N\cdot[G(W)]\leq0,
\]
or with shock velocity $\sigma$
\[
\sigma[s(W)]\geq N\cdot[G(W)].
\]
Often, but not always, Lax entropy criterion $\Leftrightarrow$ Lax characteristic criterion \cite{lax72}.
\end{frame}
%
\begin{frame}{Legendre Transform}

An important tool: the Legendre transformation. Consider a function $s$ from $\mathcal{R}\subset\mathbb{R}^{m}$ to $\mathbb{R}$. Assume that the gradient of $s$, $\nabla_W s(W)$ from $\mathcal{R}$ to $\mathcal{S}=\nabla s(\mathcal{C})$ is invertible.

This is the case if $s$ is strictly convex, for example. The Legendre transformation $s^{*}$ of $s$ is defined for $V\in\mathcal{S}$ by
\[
s^{\ast}(V)=V\cdot W-s(W),\quad V=\nabla s(W).
\]

Examples: $s(x)=x^{2}/2$, $s(x)=x^{3}/3$, $s(x,y)=y^{2}/2/x+x^{2}/2$.

When $s$ is strictly convex, the Legendre transformation coincides with the Fenchel transformation
\[
s^{*}(V)=\sup_{W}\left(V\cdot W-s(W)\right).
\]
In the general case, $\nabla s(W)$ is multivalued, it requires differential geometry... 
\end{frame}
%
\begin{frame}{Useful General Properties}
\begin{itemize}
\item $V=\nabla s(W)\Leftrightarrow W=\nabla s^{*}(V)$.
\item $s^{**}=s$
%\item  $s$ convex $\Leftrightarrow s^{*}$ convex.
\item $ds(W)=\nabla s(W)\cdot dW=V\cdot dW$. And $ds^{*}(V)=\nabla s^{*}(V)\cdot dV=W\cdot dV$. Exchange of variables and derivatives. Justifies the term conjugate or dual function. Useful in thermodynamics.
\end{itemize}
\end{frame}
%

%translatex_split_here
 \begin{frame}{Convex Case}

If $s$ is strictly convex.
\begin{itemize}
\item $s^{*}$ is strictly convex
\item the Hessian matrices of $s$ and $s^{*}$ are symmetric and positive definite.
\item The inf-convolution
$$
s_{1}\square s_{2}(W)\coloneqq \inf_{W=W_{1}+W_{2}}s_{1}(W_{1})+s_{2}(W_{2})
$$
is changed into an addition:
$$
s^{*}(V)=s_{1}^{*}(V)+s_{2}^{*}(V).
$$
\end{itemize}
\end{frame}
%
\begin{frame}{Duality and Lax Entropy}

If $s$ is a Lax entropy, we can calculate its Legendre transform $s^{*}$. Entropic variables:
\[
V=\nabla s(W)\Leftrightarrow W=\nabla s^{*}(V).
\]
We then define the dual entropy flux:
\[
G^{i,\star}(V)=V\cdot Q^{i}(W)-G^{i}(W).
\]
(Note: this is not a Legendre transformation, hence the symbol
``$\star$'' is different from ``$\ast$''). Property:
\[
\nabla G^{i,\star}(V)=Q^{i}(W).
\]
In other words: the gradient of the dual entropy is the conservative variables. The gradient of the dual entropy flux, is the flux of the CLS.

The scalar functions $(s^{*},G^{i,\star})$ contain all the information on the CLS. It can be seen that the existence of a Lax entropy is a strong property: one reconstructs $d+1$ vectorial functions from only $d+1$ scalar functions!
\end{frame}
%
\begin{frame}{Mock's Theorem}
\begin{theorem}
A system is symmetrizable if and only if it admits a Lax entropy \cite{mock80,mazet89,harten98}.
\end{theorem}

\begin{proof}
$\Leftarrow$: $\partial_{t}W+\partial_{i}Q^{i}(W)=0$ can also be written as $\partial_{t}\nabla s^{*}(V)+\partial_{i}\nabla G^{i,\star}(V)=0.$ Therefore,
\[
D^{2}s^{*}(V)\partial_{t}V+D^{2}G^{i,\star}(V)\partial_{i}V=0.
\]
The Hessian matrices are symmetric and $s^{*}$ is strictly convex, therefore $D^{2}s^{*}(V)$ is positive definite.

$\Rightarrow$: if there exists a change of variables that symmetrizes the CLS, then $\partial_{t}W(V)+\partial_{i}W(V)Q^{i}(W)=0$ with $W(V)$ symmetric and positive definite and $W(V)Q^{i}(W)$ symmetric. By Poincaré lemma, these are the Hessians of $s^{*}$ and $G^{i,\star}.$ Thus, $s=s^{**}$ and $G^{i}=G^{i,\star\star}$. 
\end{proof}

\end{frame}
%translatex_split_here
 \begin{frame}{Example: Saint-Venant}

Calculate $s$, $G^{i}$, $s^{*}$, $G^{i,\star}$. See \cite{guillon2023stability}
\end{frame}
%
\begin{frame}{Vanishing Viscosity}

Entropic solutions are limits of viscous solutions:
\[
\partial_{t}W^{\epsilon}+\partial_{x}Q(W^{\epsilon})-\epsilon\partial_{xx}W^{\epsilon}=0.
\]
The viscosity $\epsilon>0$ ensures that $W^{\epsilon}$ is regular.
It is assumed that $W^{\epsilon}\to W$ (in a suitable sense). By integration by parts and passing to the limit, $W$ is a weak solution. Multiply by $ Ds(W^{\epsilon})$:
\[
\partial_{t}s(W^{\epsilon})+\partial_{x}g(W^{\epsilon})-\epsilon \nabla s\partial_{xx}W^{\epsilon}=0,
\]
or, since $ DsDQ=Dg$,
\[
\partial_{t}s(W^{\epsilon})+\partial_{x}g(W^{\epsilon})=\epsilon Ds\partial_{xx}W^{\epsilon}=\epsilon\partial_{x}Ds\partial_{x}W-\epsilon D^{2}s\partial_{x}W\cdot\partial_{x}W,
\]
As $s$ is convex $D^{2}s\partial_{x}W\cdot\partial_{x}W\geq0$.
Then we multiply by a test function $\varphi\geq0$ and we integrate by parts
\[
\int_{x,t}\left(-s(W^{\epsilon})\partial_{t}\varphi-g(W^{\epsilon})\partial_{x}\varphi\right)\leq\epsilon\int_{x,t}W^{\epsilon}\partial_{x}Ds\partial_{x}\varphi.
\]
Thus, when $\epsilon\to0$, we have in the weak sense
\[
\partial_{t}s(W)+\partial_{x}g(W)\leq0.
\]

\end{frame}
%translatex_split_here
 \section{Kinetic Approximation}
\begin{frame}{Kinetic Representation}

System of Conservation Laws (CSL)
\begin{equation}
\partial_{t}W+\partial_{i}Q^{i}(W)=0.\label{eq:CLS}
\end{equation}
Kinetic vectors $F_{k}$
\[
W=\sum_{k=1}^{n_{v}}F_{k}.
\]
Global kinetic vector $F$, made of all the $F_{k}$ stacked together:
\[
F=(F_{1}^\intercal,\ldots,F_{n_{nv}}^\intercal)^\intercal.
\]
Or
\[
W=PF,
\]
with $P$ a constant matrix, called the \textbf{projection matrix}.
\end{frame}
%
\begin{frame}{BGK Model}

Kinetic velocities $V_{k}$ \textbf{constants}, $k=1\ldots n_{v}$.
Transport with BGK-type relaxation
\[
\partial_{t}F_{k}+V_{k}\cdot\nabla F_{k}=\frac{1}{\varepsilon}(F_{k}^{eq}-F_{k}),\quad k=1\ldots n_{v}.
\]
Kinetic equilibrium $F_{k}^{eq}=F_{k}^{eq}(W)$.

Noting $1_{m}$ the identity matrix of size $m	\times m$ and $V^{i}$
the diagonal matrices
\[
V^{i}=\left(\begin{array}{ccc}
V_{1}^{i}1_{m}\\
 & \ddots\\
 &  & V_{n_{v}}^{i}1_{m}
\end{array}\right),
\]
the BGK system can also be written in the full vector form
\[
\partial_{t}F+\sum_{i=1}^{d}\partial_{i}\left(V^{i}F\right)=\frac{1}{\varepsilon}(F^{\text{eq}}(W)-F).
\]

\end{frame}
%
\begin{frame}{Consistency}

As $\varepsilon\to0$, we expect $F_{k}\simeq F_{k}^{eq}$.
The kinetic system is therefore an approximation of the CLS (\ref{eq:CLS})
if
\begin{equation}
W=\sum_{k}F_{k}^{\text{eq}}(W),\quad Q^{i}(W)=\sum_{k=1}^{n_{v}}V_{k}^{i}F_{k}^{\text{eq}}(W),\label{eq:algebraic_consistence}
\end{equation}

\end{frame}
%translatex_split_here
 %
\begin{frame}{Kinetic Scheme}

BGK relaxation: nonlinear coupling between all kinetic vectors $F_{k}$. To decouple, a decomposition scheme (\emph{splitting}) is used. Each time step $\Delta t$ is divided into:
\begin{itemize}
\item Transport: computation of $F_{k}(\cdot,t{{}^-})$ from $F_{k}(\cdot,t-\Delta t^{+})$ by solving
\[
\partial_{t}F+\sum_{i=1}^{d}\partial_{i}\left(V^{i}F\right)=0.
\]
\item Get the conservative variables
\[
W(\cdot,t)=\sum_{k}F_{k}(\cdot,t^{-}).
\]
\item Relaxation: computation of $F_{k}(\cdot,t^{+})$
\[
F_{k}(\cdot,t^{+})=\omega F_{k}^{eq}(W(\cdot,t))+(1-\omega)F_{k}(\cdot,t^{-}).
\]
\end{itemize}
Note: $\omega\in[1,2]$ is the relaxation parameter. First-order scheme if $\omega=1$, second-order scheme if $\omega=2$ (over-relaxation). $W$ is continuous in time, but not $F_{k}$.
\end{frame}
%
\begin{frame}{Kinetic Entropy}

A kinetic Lax-Mock theory can be developed. Suppose we find functions $s_{k}^{*}(V)$ such that
\[
\sum_{k=1}^{n_{v}}s_{k}^{*}=s^{*},\quad\sum_{k}V_{k}^{i}s_{k}^{*}=G^{i,\star}.
\]
Let
\[
F_{k}^{eq}(W(V))=\nabla_{V}s_{k}^{*}(V).
\]
Then, by taking the gradient:
\begin{itemize}
\item $\sum_{k}F_{k}^{eq}=\nabla_{V}s^{*}=W$, 
\item $\sum_{k}V_{k}^{i}F_{k}^{eq}=\nabla_{V}G^{i,\star}=Q^{i}$.
\end{itemize}
Moreover, if the $s_{k}^{*}$ are convex, the equilibrium is also a minimum of the kinetic entropy:
\[
s(W)=\min_{W=\sum_{k}F_{k}}\sum_{k}s_{k}(F_{k})=\sum_{k}s_{k}(F_{k}^{eq}).
\]
\end{frame}
%
\begin{frame}{Entropic Stability}

It then becomes easy to prove the entropic stability of the kinetic scheme. The total entropy ($x$ is assumed to be in an infinite or periodic domain)
\[
\mathcal{S}(t)=\int_{x}\sum_{k}s_{k}(F_{k})
\]
is conserved during the transport step. It is sufficient to show that
\[
\sum_{k}s_{k}(F_{k}(\cdot,t^{+}))\leq\sum_{k}s_{k}(F_{k}(\cdot,t^{-})).
\]
This is the case (proof) when $\omega=1$ and also (diagram) for $\omega\simeq2$.
\end{frame}
%
\begin{frame}{Sub-characteristic Condition}

The above proof works as long as the $s_{k}$ are convex, which is equivalent to $s_{k}^{*}$ being convex. Taking the case $d=1$ and $n_{v}=2$, we have
\[
s_{1,2}^{*}=\frac{s^{*}}{2}\pm\frac{g^{\star}}{2\lambda}.
\]
Since $s^{*}$ is strictly convex, if $\lambda$ is large enough, we expect $s_{k}^{*}$ to also be strictly convex, at least locally. The condition of $s_{k}^{*}$ being strictly convex leads to the sub-characteristic condition. Examples: transport, Burgers, Saint-Venant. 
\end{frame}
%
%translatex_split_here

\begin{frame}{Approximate Flux}

Another way to study stability: equivalent equation. The projection matrix $P$ is a matrix with $m$ rows and $mn_{v}$ columns. It is extended to an invertible matrix
\[
M=\left(\begin{array}{c}
P\\
R
\end{array}\right),
\]
called the moment matrix, such that
\[
\left(\begin{array}{c}
W\\
Z
\end{array}\right)=MF.
\]
The vector $Z=RF$ is called the ``approximate flux''. The ``flux error'' is also defined as
\[
Y=R(F-F^{eq}).
\]
It is enlightening to find the PDE satisfied by the couple $(W,Y)$.
\end{frame}
%
\begin{frame}{Equivalent PDE Algorithm}

The kinetic scheme is a functional operator that computes $F(\cdot,t+\Delta t^{+})$ from $F(\cdot,t^{+}).$ With the previous change of variables, we have thus a well-defined operator $\mathcal{M}(\Delta t),$ such that
\[
\left(\begin{array}{c}
W\\
Y
\end{array}\right)(\cdot,t+\Delta t^{+})=\mathcal{M}(\Delta t)\left(\begin{array}{c}
W\\
Y
\end{array}\right)(\cdot,t^{+}).
\]
To find the equivalent PDE, we perform a Taylor expansion in $\Delta t$ of
\[
\frac{\mathcal{M}(\Delta t/2)-\mathcal{M}(-\Delta t/2)}{\Delta t}\left(\begin{array}{c}
W\\
Y
\end{array}\right)=\partial_{t}\left(\begin{array}{c}
W\\
Y
\end{array}\right)+O(\Delta t^{2}).
\]
This expansion can be automated with Maple or SymPy for instance.
\end{frame}
%
\begin{frame}{Flux Error Oscillations}

In the set of variables $(W,Y)$, the relaxation step
\[
F_{k}(\cdot,t^{+})=\omega F_{k}^{eq}(W(\cdot,t))+(1-\omega)F_{k}(\cdot,t^{-}),
\]
becomes simply
\[
\left(\begin{array}{c}
W\\
Y
\end{array}\right)(\cdot,t^{+})=\left(\begin{array}{c}
W\\
(1-\omega)Y
\end{array}\right)(\cdot,t^{-}).
\]
In particular, if $\omega=2$, the flux error $Y$ is changed to $-Y$. To remove this rapid oscillation of frequency $1/\Delta t$, we can replace $\mathcal{M}(\Delta t)$  by $\mathcal{M}(\Delta t/2)\circ\mathcal{M}(\Delta t/2)$ in the analysis.
\end{frame}
%
\begin{frame}{Form of the Equivalent PDE}

The operator $\mathcal{M}$ is composed of shifts and nonlinear relaxations. In the asymptotic development, the shifts produce partial derivatives. The result is a system of nonlinear PDEs of the form
\[
\partial_{t}\left(\begin{array}{c}
W\\
Y
\end{array}\right)+\frac{r(\omega)}{\Delta t}\left(\begin{array}{c}
0\\
Y
\end{array}\right)+\sum_{i=1}^{d}A^{i}\partial_{i}\left(\begin{array}{c}
W\\
Y
\end{array}\right)
\]
\begin{equation}
+\Delta t\sum_{1\leq i,j\leq d}B^{i,j}\partial_{i,j}\left(\begin{array}{c}
W\\
Y
\end{array}\right)=O(\Delta t^{2}).\label{eq:forme_EDP_eq}
\end{equation}

\end{frame}
%
%translatex_split_here


%translatex_split_here
 \section{Examples}
\begin{frame}{Jin-Xin Model \cite{jin1995relaxation}}

We apply the previous theory to the Xin-Jin model for $d=1$,
$n_{v}=2$,
\[
V^{1}=\left(\begin{array}{cc}
\lambda & 0\\
0 & -\lambda
\end{array}\right),\quad F=\left(\begin{array}{c}
F^{+}\\
F^{-}
\end{array}\right),\quad M=\left(\begin{array}{cc}
1 & 1\\
\lambda & -\lambda
\end{array}\right).
\]
\[
F^{eq}=\left(\begin{array}{c}
\frac{W}{2}+\frac{Q(W)}{2\lambda}\\
\frac{W}{2}-\frac{Q(W)}{2\lambda}
\end{array}\right),\quad F=\left(\begin{array}{c}
\frac{W}{2}+\frac{Q(W)}{2\lambda}+\frac{Y}{2\lambda}\\
\frac{W}{2}-\frac{Q(W)}{2\lambda}-\frac{Y}{2\lambda}
\end{array}\right).
\]

\end{frame}
%
\begin{frame}{Jin-Xin, Equivalent System}

With $\delta=\omega-1$, we find
\[
O(\Delta t^{2})=\partial_{t}\left(\begin{array}{c}
W\\
Y
\end{array}\right)-\frac{1}{\Delta t}\frac{\delta^{4}-1}{2\delta^{2}}\left(\begin{array}{c}
0\\
Y
\end{array}\right)
\]
\[
+\left(\begin{array}{cc}
Q'(W) & \gamma_{1}\\
\gamma_{1}(\lambda^{2}-Q'(W)^{2}) & -\gamma_{2}Q'(W)
\end{array}\right)\partial_{x}\left(\begin{array}{c}
W\\
Y
\end{array}\right)
\]
\[
\Delta t\frac{\delta^{2}-1}{32\delta^{2}}\left(\begin{array}{cc}
(\lambda^{2}-v^{2})(-\delta^{2}+4\delta-1) & 3(\delta^{2}+1)Q'W)\\
3(\delta^{2}+1)(\lambda^{2}-v^{2})Q'W) & \gamma_{3}
\end{array}\right)\partial_{xx}\left(\begin{array}{c}
W\\
Y
\end{array}\right).
\]
\[
\gamma_{1}=\frac{\left(\delta-1\right)^{2}\left(\delta^{2}+1\right)}{\delta^{2}},\quad\gamma_{2}=\frac{\delta^{4}+1}{2\delta^{2}}
\]
\[
\gamma_{3}=-(5Q'(W)^{2}+3\lambda{{}^2})(\delta^{2}+1)+4(\lambda^{2}-Q'(W)^{2})\delta
\]

\end{frame}
%
\begin{frame}{Jin-Xin, Equivalent Equation}

Under the assumption that $Y=O(\Delta t)$, we obtain, to order 2 in
$\Delta t$:
\[
\partial_{t}W+\partial_{x}Q(W)=\frac{1}{2}(\frac{1}{\omega}-\frac{1}{2})\Delta t\partial_{x}(\lambda^{2}-Q'(W)^{2})\partial_{x}W.
\]

\end{frame}
%
\begin{frame}{Jin-Xin Stability}
\begin{itemize}
\item The terms of the first order of the equivalent system are symmetrizable (thus hyperbolic) if 
\[
\lambda>\left|Q'(W)\right|.
\]
\item Under the same condition, the equivalent equation is stable.
\end{itemize}
In this case, the two stability conditions are equivalent.
\end{frame}
%
%translatex_split_here


\begin{frame}{D2Q4 Model \cite{guillon2023stability,aregba2000discrete}}

We apply the previous theory to the D2Q4 model for transport
($W=w$, $Q(W)\cdot N=aN{{}^1}+bN^{2}$), $d=2$, $n_{v}=4$,
\[
V^{1}=\left(\begin{array}{cccc}
\lambda\\
 & -\lambda\\
 &  & 0\\
 &  &  & 0
\end{array}\right),\quad V^{2}=\left(\begin{array}{cccc}
0\\
 & 0\\
 &  & \lambda\\
 &  &  & -\lambda
\end{array}\right),
\]
\[
  \quad M=\left(\begin{array}{cccc}
    1 & 1 & 1 & 1\\
    \lambda & -\lambda & 0 & 0\\
    0 & 0 & \lambda & -\lambda\\
    \lambda^{2} & \lambda^{2} & -\lambda^{2} & -\lambda^{2}
    \end{array}\right).\]
\[
F^{eq}=\frac{1}{4}\left(\begin{array}{c}
w+2aw/\lambda\\
w-2aw/\lambda\\
w+2bw/\lambda\\
w-2bw/\lambda
\end{array}\right).
\]

\end{frame}

\begin{frame}{D2Q4, equivalent system}

We recover the form (\ref{eq:forme_EDP_eq}) 
\[
\partial_{t}\left(\begin{array}{c}
W\\
Y
\end{array}\right)+\frac{r(\omega)}{\Delta t}\left(\begin{array}{c}
0\\
Y
\end{array}\right)+\sum_{i=1}^{d}A^{i}\partial_{i}\left(\begin{array}{c}
W\\
Y
\end{array}\right)
\]
\[
+\Delta t\sum_{1\leq i,j\leq d}B^{i,j}\partial_{i,j}\left(\begin{array}{c}
W\\
Y
\end{array}\right)=O(\Delta t^{2}).
\]
Details in \cite{guillon2023stability}.
\end{frame}
%
\begin{frame}{D2Q4, equivalent equation}

Under the assumption that $Y=O(\Delta t)$, we obtain to order 2 in
$\Delta t$:
\[
\partial_{t}w+a\partial_{x}w+b\partial_{y}w=\frac{\Delta t}{2}(\frac{1}{\omega}-\frac{1}{2})\nabla\cdot(D\nabla w),
\]
with
\[
D=\left(\begin{array}{cc}
\lambda^{2}/2-a^{2} & -ab\\
-ab & \lambda^{2}/2-b^{2}
\end{array}\right).
\]

\end{frame}
%
\begin{frame}{D2Q4 stability}
\begin{itemize}
\item The first order terms  of the equivalent system are symmetrizable (hence hyperbolic) if and only if
\[
\lambda>\sqrt{2}\sqrt{a^{2}+b^{2}}.
\]
\item The equivalent equation is stable if and only if
\[
\lambda>2\max(\left|a\right|,\left|b\right|).
\]
\end{itemize}
\begin{minipage}[c]{0.5\textwidth}%
The hyperbolicity condition is more restrictive than the diffusion condition.%
\end{minipage}%
\begin{minipage}[c]{0.5\textwidth}%
\begin{center}
\includegraphics[width=0.6\textwidth]{images/conditions_D2Q4}
\par\end{center}%
\end{minipage}
\end{frame}



%
\begin{frame}{D2Q4 Numerical Results}

Transport of a Gaussian with $\omega=1.6$, $(a,b)=(1,0)$ on
the unit square, $N_{x}=200$ cells in $x$ and in $y$. Left $\lambda=1.6$
(stable diffusion), right $\lambda=2.2$ (stable entropy)
\begin{center}
\includegraphics[width=0.48\textwidth]{images/om_1\lyxdot 6_lambda_1\lyxdot 6_Nt_80_Nx_200}\includegraphics[width=0.48\textwidth]{images/om_1\lyxdot 6_lambda_2\lyxdot 2_Nt_110_Nx_200}
\par\end{center}

The most constraining condition appears to be necessary.
\end{frame}
%
\begin{frame}{Lattice-Boltzmann \cite{frapolli2015entropic}}

The standard D2Q9 model could be analyzed using this approach. TODO !
\end{frame}
%

\section{Schemes without CFL}
\begin{frame}{DG Approximation}

In the LBM, the transport equation
\begin{equation}
\partial_{t}f+V\cdot\nabla f=0\label{eq:transport}
\end{equation}
is solved by shifting. It no longer works on unstructured meshes (since it is non-conservative). It can be solved with a DG (Discontinuous Galerkin) scheme. Computational domain: $\Omega$. Triangulation of $\Omega$: $\mathtt{T}=(L_{i})$ in open cells $L_{i}$
such that
\[
\overline{\Omega}=\bigcup_{i}\overline{L_{i}},\quad L_{i}\cap L_{j}=\emptyset\text{ if }i\neq j.
\]
At time $t_{n}=n\Delta t$, on cell $L\in\mathtt{T}$, the solution is approximated by the discontinuous function $f^{n}$.
\[
f(X,n\Delta t)\simeq f^{n}(X)=\sum_{k=1}^{p}f_{L}^{n,k}(t)\phi_{L}^{k}(X),\quad X\in L,
\]
where the $\phi_{L}^{k}$ are DG basis functions on the cell $L$.
\end{frame}
%
\begin{frame}{Implicit DG Scheme}

A DG scheme, implicit, first-order in time, is given by:
\[
\forall(L,k)\quad\int_{L}\frac{f^{n}-f^{n-1}}{\Delta t}\phi_{L}^{k}-\int_{L}f^{n}V\cdot\nabla\phi_{L}^{k}
\]
\[
+\int_{\partial L}\left(V\cdot N^{+}f_{L}+V\cdot N^{-}f_{R}\right)\phi_{L}^{k}=0.
\]

\begin{minipage}[c]{0.5\textwidth}%
\begin{itemize}
\item The outward normal to $L$ on $\partial L$ is noted $N$.
\item We use the upwind flux ($a^{+}=\max(a,0)$, $a^-=\min(a,0)$).
\item $R$ denotes the neighbor of $L$ along $\partial L$.
\end{itemize}
%
\end{minipage}%
\begin{minipage}[c]{0.5\textwidth}%
\begin{center}
\includegraphics[width=0.6\textwidth]{images/cell_geom}
\par\end{center}%
\end{minipage}
\end{frame}
%


%translatex_split_here
 \begin{frame}{Explicit Algorithm}

The "implicit" scheme is actually {\bf explicit}, thanks to the upwind flux. Cell $R$ is "upstream" of cell $L$ if $V\cdot N_{RL}>0$.
Construction of the dependency graph: oriented arc $R\rightarrow L$ if $R$ is upstream of $L$. The time step can then be solved explicitly by traversing the graph in a topological order.
\begin{center}
\includegraphics[width=0.9\textwidth]{images/downwind_graph}
\par\end{center}

\end{frame}
%
\begin{frame}{Application: Antenna Simulation}
\begin{itemize}
\item Maxwell's equations: $W=(E^{T},H^{T})^{T},$ electric field $E\in\mathbb{R}^{3}$, magnetic field $H\in\mathbb{R}^{3}$.
\item Maxwell's flux: 
\[
Q(W,N)=\left(\begin{array}{c}
-N\times H\\
N\times E
\end{array}\right).
\]
\item Source term, conductivity $\sigma$, Ohm's law
\[
S(W)=\left(\begin{array}{c}
-\sigma E\\
0
\end{array}\right).
\]
\end{itemize}
\[
\partial_{t}W+\nabla\cdot Q(W)=S(W).
\]

\end{frame}
%
\begin{frame}{Numerical Results}
\begin{itemize}
\item Unstructured mesh of an electrical wire in a cube. Sending a plane pulse.
\item Second order DG-LBM solver in time (implicit Euler replaced by Crank-Nicolson).
\item CFL=7.
\end{itemize}
\begin{center}
\includegraphics[width=0.5\textwidth]{images/mesh-fil}
\par\end{center}

\begin{center}
\includegraphics[width=0.8\textwidth]{images/resu-fil}
\par\end{center}

\end{frame}
%
\begin{frame}{Comparison of FDTD and DG}

It is possible to make $\sigma=+\infty$ in the scheme while remaining explicit. The source term is resolved in the relaxation step. This is equivalent to doing $E\leftarrow-E$ in this step. Comparison with a finite difference code (Yee's FDTD scheme) on a uniform mesh.
\begin{center}
\includegraphics[width=0.45\textwidth]{images/plot_Hx_sigma_inf}\includegraphics[width=0.45\textwidth]{images/plot_Hy_sigma_inf}
\par\end{center}

\end{frame}
%
%translatex_split_here
 \section{Boundary Conditions}
\begin{frame}{Boundary Conditions}

  \begin{itemize}
\item A fundamental challenge with numerical schemes: stable and precise handling of boundary conditions.
\item Still an open problem for LBM.
\item We present an attempt for stabilizing a second order boundary condition.
\end{itemize}
\end{frame}
%
\begin{frame}{Transport Equation}

For $\omega=2$, the LBM is second-order. In practice, the application of boundary conditions can reduce the order or stability.

Consider the 1D transport equation with speed $c>0$ and a boundary condition on the left, $W=w$, $Q^{1}(W)=cw$,
\begin{align*}
\partial_{t}w+c\partial_{x}w & =0,\quad x\in[L,R]\\
w(x,0) & =0,\\
w(0,t) & =w_{0}(x).
\end{align*}

\begin{itemize}
\item Grid points: $x_{i}=L+ih+h/2$, $0\leq i<N$, with $h=(R-L)/N$.
\item Time step: $\Delta t=h/\lambda$. Time $t_{n}=n\Delta t$.
\end{itemize}
\end{frame}
%
\begin{frame}{LBM}

\[
F=\left(\begin{array}{c}
F_{1}\\
F_{2}
\end{array}\right),\quad W=F_{1}+F_{2}.
\]
We denote $F_{i}^{n,-}$ the value of $F(x_{i},t_{n}^{-})$ before relaxation. In the shifting step
\[
F_{1,i}^{n,-}=F_{1,i+1}^{n-1},\quad F_{2,i}^{n,-}=F_{2,i-1}^{n-1},
\]
the values $F_{2,-1}^{n-1}$ (left boundary) and $F_{1,N}^{n-1}$ (right boundary) are missing. Ghost cell method
\[
F_{2,-1}^{n-1}=b_{L}(F_{1,0}^{n-1},F_{2,0}^{n-1}),\quad F_{1,N}^{n-1}=b_{R}(F_{1,N-1}^{n-1},F_{2,N-1}^{n-1}).
\]

\end{frame}
%
\begin{frame}{Entropic Stability \cite{audounet84,dubois88,aregba2004kinetic}}

The incoming kinetic entropy must be smaller than the outgoing one:
\begin{equation}
s_{2}(b_{L}(F_{1},F_{2}))\leq s_{1}(F_{1}),\quad s_{1}(b_{R}(F_{1},F_{2}))\leq s_{2}(F_{2}).\label{eq:stab_entrop}
\end{equation}
Application to the D1Q2 model. We impose $W=F_{1}+F_{2}=0$ on the left and $Y=0=\lambda(F_{2}-F_{1})-c(F_{1}+F_{2})$ on the right. Thus:
\[
b_{L}(F_{1},F_{2})=-F_{1},\quad b_{R}(F_{1},F_{2})=\frac{\lambda-c}{\lambda+c}F_{2}
\]
Simple calculations show that (\ref{eq:stab_entrop}) is satisfied.
The scheme is stable, but even when $\omega=2$, it is experimentally only first-order.
\end{frame}
%
%translatex_split_here
 \begin{frame}{Scheme of Order 2}

It is more accurate to apply a Neumann condition\cite{drui2019analysis}
$\partial_{x}Y=0$ on the right. This extends the stencil of the ghost function
to the right as
\begin{equation}
F_{1,N}^{n-1}=b_{R}(F_{1,N-1}^{n-1},F_{2,N-1}^{n-1},F_{1,N-2}^{n-1},F_{2,N-2}^{n-1}).\label{eq:big_stencil}
\end{equation}
To prevent an increase in entropy, the following scheme is applied:
\begin{itemize}
\item Calculate $F_{1,N}^{n-1}$ with (\ref{eq:big_stencil});
\item If the entropy condition is not satisfied, i.e. if $s_{1}(F_{1,N}^{n-1})>s_{2}(F_{2,N-1}^{n-1})$
then replace $F_{1,N}^{n-1}$ with the closest value $\widetilde{F_{1,N}^{n-1}}$
such that $s_{1}(\widetilde{F_{1,N}^{n-1}})=s_{2}(F_{2,N-1}^{n-1})$.
\end{itemize}
\end{frame}
%
\begin{frame}{Extension to D2Q4}
\begin{itemize}
\item For $d>1$, on a boundary point, in general, the number of incoming characteristics
of the kinetic model and the equivalent system are different.
\item This phenomenon leads to unstable or inaccurate results when $\omega\simeq2$.
\item Entropy limiter improves the results.
\end{itemize}
Transport equation in 2D with velocity $c=(a,b)$ on the square $\Omega=]0,1[\times]0,1[$.
\[
\partial_{t}W+\sum_{i=1}^{2}\partial_{i}Q^{i}(W)=0,\quad Q^{1}(W)=aW,\quad Q^{2}(W)=bW.
\]

\end{frame}

%
\begin{frame}{Second Order Boundary Conditions\cite{helie:tel-04034510,helluy_bc2023}}
\begin{itemize}
\item Transport equation in 2D with velocity $c=(a,b)$ on the square $\Omega=]0,1[\times]0,1[$. 
\item Normal vector $(n_{1},n_{2})$ on $\partial\Omega$.
\item Test of two boundary condition strategies.\medskip{}
\end{itemize}
{\scriptsize{}
\begin{tabular}{|c|l|l|} \hline  Boundary conditions & Entropy stable & Second order accurate\tabularnewline \hline  Inflow border & $\begin{array}{l} \text{Exact solution on }w\\ y_{3}=0 \end{array}$ & $\begin{array}{l} \text{Exact solution on }w\end{array}$\tabularnewline \hline  Outflow border & $\begin{array}{l} y_{1}n_{1}+y_{2}n_{2}=0\\ y_{3}=0 \end{array}$ & $\begin{array}{l} \text{Neumann on }v_{1}y_{1}+v_{2}y_{2}\end{array}$\tabularnewline \hline  Corner inflow/inflow & $\begin{array}{l} \text{Exact solution on }w\\ y_{3}=0 \end{array}$ & $\begin{array}{l} \text{Exact solution on }w\\ y_{3}=0 \end{array}$\tabularnewline \hline  Corner inflow/outflow & $\begin{array}{l} \text{Exact solution on }w\\ n_{1}y_{1}+n_{2}y_{2}=0\\ y_{3}=0 \end{array}$ & $\begin{array}{l} \text{Exact solution on }w\\ \text{Neumann on }v_{1}y_{1}+v_{2}y_{2} \end{array}$\tabularnewline \hline  Corner outflow/outflow & $\begin{array}{l} y_{1}=0\\ y_{2}=0\\ y_{3}=0 \end{array}$ & $\begin{array}{l} \text{Neumann on }v_{1}y_{1}+v_{2}y_{2}\\ y_{3}=0 \end{array}$\tabularnewline \hline  \end{tabular} }{\scriptsize\par}
\end{frame}
\begin{frame}{Entropy Evolution}

\includegraphics[width=0.49\textwidth]{images/D2Q4_decroissance_entropy_BC_order2}\includegraphics[width=0.49\textwidth]{images/D2Q4_decroissance_entropy_BC_proj}

Left without entropy limitation, Right with limitation.
\end{frame}

%translatex_split_here%
\begin{frame}{Order}

\includegraphics[width=0.49\textwidth]{images/convergence_order_BC_stable}\includegraphics[width=0.49\textwidth]{images/convergence_order_BC_proj}

Left: First-order stable boundary condition (CL), Right: Second-order boundary condition with entropy stabilization
\end{frame}
%
\begin{frame}{Conclusion}
\begin{itemize}
\item Systems of conservation laws provide a very rich class of models for physics.
\item The kinetic approach is a general and highly effective method for building numerical approximations.
\item The numerical viscosity intuition is useful but not always correct.
\item Entropic theory allows for the mathematical study of stability and consistence of these schemes.
\end{itemize}
\end{frame}

\section{Bibliography}
\begin{frame}[allowframebreaks]{Bibliography}

\tiny\bibliographystyle{plain}
\bibliography{kin_diapos}

\end{frame}
%

\end{document}



